\documentclass{article}
\usepackage[utf8]{inputenc}

\usepackage{amsmath,amsthm, amssymb}
\usepackage[margin=3cm]{geometry}
\usepackage{mathtools}
\usepackage{dsfont}
\usepackage{xcolor}
\usepackage{algorithm,algpseudocode}
\usepackage{todonotes}
\usepackage{nicefrac}
\usepackage{mathrsfs}
\usepackage{tikz}
\usepackage{thm-restate}


%%%%%%%%    THEOREM DEFINITIONS AND RESTATABLE
\newtheorem{theorem}{Theorem}
\newtheorem{lemma}[theorem]{Lemma}
\newtheorem{corollary}[theorem]{Corollary}

\usepackage{todonotes}

\newcommand{\matt}[1]{\todo[color=red!50, prepend, caption={Matt}, tickmarkheight=0.25cm]{#1}}
\newcommand{\note}[1]{\emph{Note: #1}}
\newcommand{\conjecture}[1]{ \noindent\emph{\textbf{Conjecture:}} \emph{ #1 }}




%%%%%%%%    NOTATION DEFINITIONS FOR EASIER WRITING
\newcommand{\ket}[1]{|#1\rangle}
\newcommand{\bra}[1]{\langle #1|}
\newcommand{\braket}[2]{\langle #1|#2\rangle}
\newcommand{\ketbra}[2]{| #1\rangle\! \langle #2|}
\newcommand{\parens}[1]{\left( #1 \right)}
\newcommand{\brackets}[1]{\left[ #1 \right]}
\newcommand{\abs}[1]{\left| #1 \right|}
\newcommand{\norm}[1]{\left| \left| #1 \right| \right|}
\newcommand{\diamondnorm}[1]{\left| \left| #1 \right| \right|_\diamond}
\newcommand{\anglebrackets}[1]{\left< #1 \right>}
\newcommand{\overlap}[2]{\anglebrackets{#1 , #2 }}
\newcommand{\set}[1]{\left\{ #1 \right\}}
\newcommand{\ceil}[1]{\left\lceil #1 \right\rceil}
\newcommand{\openone}{\mathds{1}}
\newcommand{\expect}[1]{\mathbb{E}\brackets{#1}}
\newcommand{\variance}[1]{\textit{Var} \brackets{ #1 }}
\newcommand{\prob}[1]{\text{Pr}\left[ #1 \right]}
\newcommand{\bigo}[1]{O\left( #1 \right)}
\newcommand{\bigotilde}[1]{\widetilde{O} \left( #1 \right)}
\newcommand{\ts}{\textsuperscript}

\DeclareMathOperator{\Tr}{Tr}
\newcommand{\trace}[1]{\Tr \brackets{ #1 }}
\newcommand{\partrace}[2]{\Tr_{#1} \brackets{ #2 }}
\newcommand{\complex}{\mathbb{C}}

%%%%% COMMONLY USED OBJECTS
\newcommand{\hilb}{\mathcal{H}}
\newcommand{\partfun}{\mathcal{Z}}
\newcommand{\identity}{\mathds{1}}
\newcommand{\gue}{\rm GUE}
\DeclareMathOperator{\hermMathOp}{Herm}
\newcommand{\herm}[1]{\hermMathOp\parens{#1}}


\title{Thermal State Prep}
\author{Nathan Wiebe, Matthew Hagan}
\date{May 2022}

\begin{document}

\maketitle

%%%%%%%%%%%%%%%%%%%%%%%%%%%%%%%%%%%%%%%%%%%%%%%%%%%%%%%%%%%%%%%%%%%%%%%%%%%%%%%%%%%%%%%%%%%%%%%%%%%%%%%%%%%%%%%%%%%%%%%%%%%%%%%%%%%%%%%%%%%%%%%%
%%%%%%%%%%%%%%%%%%%%%%%%%%%%%%%%%%%%%%%%%%%%%%%%%%%%%%%%%%%%%%%%%%%%%%%%%%%%%%%%%%%%%%%%%%%%%%%%%%%%%%%%%%%%%%%%%%%%%%%%%%%%%%%%%%%%%%%%%%%%%%%%
%%%%%%%%%%%%%%%%%%%%%%%%%%%%%%%%%%%%%%%%%%%%%%%%%%%%%%%%%%%%%%%%%%%%%%%%%%%%%%%%%%%%%%%%%%%%%%%%%%%%%%%%%%%%%%%%%%%%%%%%%%%%%%%%%%%%%%%%%%%%%%%%
\section{Introduction}
Going to leave this blank for now. \cite{shiraishi_undecidability_2021}

%%%%%%%%%%%%%%%%%%%%%%%%%%%%%%%%%%%%%%%%%%%%%%%%%%%%%%%%%%%%%%%%%%%%%%%%%%%%%%%%%%%%%%%%%%%%%%%%%%%%%%%%%%%%%%%%%%%%%%%%%%%%%%%%%%%%%%%%%%%%%%%%
%%%%%%%%%%%%%%%%%%%%%%%%%%%%%%%%%%%%%%%%%%%%%%%%%%%%%%%%%%%%%%%%%%%%%%%%%%%%%%%%%%%%%%%%%%%%%%%%%%%%%%%%%%%%%%%%%%%%%%%%%%%%%%%%%%%%%%%%%%%%%%%%
%%%%%%%%%%%%%%%%%%%%%%%%%%%%%%%%%%%%%%%%%%%%%%%%%%%%%%%%%%%%%%%%%%%%%%%%%%%%%%%%%%%%%%%%%%%%%%%%%%%%%%%%%%%%%%%%%%%%%%%%%%%%%%%%%%%%%%%%%%%%%%%%
\section{Preliminaries}
We denote the Hilbert space of the system as $\hilb_{S}$ and the environment as $\hilb_{E}$, with the Hamiltonians governing each as $H_{S}$ and $H_{E}$. We will assume without loss of generality that the system's Hilbert space can be encoded with $n$ qubits, giving $\dim_S = 2^{n}$, and the environment's Hilbert space can be encoded with $m$ qubits giving $\dim_E = 2^{m}$. The Hamiltonian for the joint system on $\hilb_{S} \otimes \hilb_{E}$ is then $H = H_{S} \otimes \identity + \identity \otimes H_{E}$. The Hilbert space of the combined system and environment is of dimension $\dim = \dim_E \cdot \dim_S = 2^{n + m}$. 

We will primarily work in the eigenbasis for each Hamiltonian:
\begin{equation}
    H_{S} = \sum_{i = 0}^{2^n - 1} \lambda_S(i) \ketbra{s_i}{s_i} ~,~ H_{E} = \sum_{j=0}^{2^m - 1} \lambda_E(j) \ketbra{e_j}{e_j} ~,~ H = \sum_{i=0}^{2^n - 1} \sum_{j=0}^{2^m - 1} \lambda(i,j) (\ket{s_i} \otimes \ket{e_j})(\bra{s_i} \otimes \bra{e_j}),
\end{equation}
for convenience we will denote the tensor product of eigenvectors simply by their indices $\ket{i,j} \coloneqq \ket{s_i} \otimes \ket{e_j}$. For convenience we define $\lambda(i,j) \coloneqq \lambda_S(i) + \lambda_E(j)$. For many proofs we will simply use a single index over the total system-environment Hilbert space, in which case we will represent $\Delta_{ij} \coloneqq \lambda(i) - \lambda(j)$ as the energy difference and $\eta(i) = \abs{\set{j : \lambda(j) = \lambda(i)}}$ as the degeneracy of the $i$\ts{th} subspace (with 1 denoting a unique eigenvector).

We study the effects of simulating the time dynamics of the system-environment space with randomized interactions. Our goal is to produce a channel that can reduce the entropy of a system thermal state by utilizing easily prepared thermal states of the environment. We consider using the Haar measure over eigenvectors for the randomized interaction and i.i.d eigenvalues from a zero mean $\sigma^2$ variance distribution. 

For input states we will typically assume thermal states of the form $\rho_S(\beta) = \frac{e^{-\beta H_S}}{\partfun_S}$, where $\partfun_S = \trace{e^{-\beta H_S}}$, where the inverse temperature $\beta$ of the partition function will typically be assumed but written explicitly if need be. We will assume environment states of the form $\rho_E(\beta) = \frac{e^{-\beta H_E}}{\partfun_E}$ and we will typically denote the tensor product of the system and environment states as $\rho(\beta_S, \beta_E) = \rho_S(\beta_S) \otimes \rho_E(\beta_E)$. The inputs $\beta_S, \beta_E$ will typically be surpressed in most cases as well.


Overall one application of our channel is represented as
\begin{equation}
    \Phi(\rho) := \int \partrace{\hilb_E}{e^{+i(H + \alpha G)t} \rho e^{-i(H + \alpha G) t}} dG.
\end{equation}
For simplicity we will denote the time evolution channel for a specific random interaction $G$ as
\begin{equation}
    \Phi_G(\rho) := e^{+i (H+ \alpha G) t} \rho e^{-i (H + \alpha G) t}. \label{eq:phi_g_definition}
\end{equation}
Clearly then $\Phi = \int \partrace{env}{\Phi_G} dG$. We use $G$ to denote the randomized interaction term, where $G = U_G D U_G^\dagger$. The measure we choose for the eigenbasis of $G$ is $U_G \sim Haar$ and the eigenvalues are i.i.d with mean 0 and variance $\sigma^2$. This gives the overall measure decomposition $dG = dD ~ dU_G$. 

\begin{restatable}{lemma}{haar_two_moment} \label{lem:haar_two_moment}
    Let $U$ be a unitary matrix over $\dim$ dimensions that is distributed according to the Haar measure. Then the following average is
    \begin{align}
        \int \bra{i_1} U \ket{j_1} \bra{i_2} U \ket{j_2} \bra{k_1} U^\dagger \ket{l_1} ~ \bra{k_2} U^\dagger \ket{l_2} dU =& ~\frac{1}{\dim^2 - 1} \parens{\delta_{i_1, l_1} \delta_{j_1, k_1} \delta_{i_2, l_2} \delta_{j_2, k_2} + \delta_{i_1, l_2} \delta_{j_1, k_2} \delta_{i_2, l_1} \delta_{j_2, k_1}} \nonumber \\
        &- \frac{1}{\dim(\dim^2 - 1)} \parens{\delta_{i_1, l_2} \delta_{j_1, k_1} \delta_{i_2, l_1} \delta_{j_2, k_2} + \delta_{i_1, l_1} \delta_{j_1, k_2} \delta_{i_2, l_2} \delta_{j_2, k_1}}. \label{eq:haar_two_moment_integral}
    \end{align}
    \end{restatable}


%%%%%%%%%%%%%%%%%%%%%%%%%%%%%%%%%%%%%%%%%%%%%%%%%%%%%%%%%%%%%%%%%%%%%%%%%%%%%%%%%%%%%%%%%%%%%%%%%%%%%%%%%%%%%%%%%%%%%%%%%%%%%%%%%%%%%%%%%%%%%%%%
%%%%%%%%%%%%%%%%%%%%%%%%%%%%%%%%%%%%%%%%%%%%%%%%%%%%%%%%%%%%%%%%%%%%%%%%%%%%%%%%%%%%%%%%%%%%%%%%%%%%%%%%%%%%%%%%%%%%%%%%%%%%%%%%%%%%%%%%%%%%%%%%
%%%%%%%%%%%%%%%%%%%%%%%%%%%%%%%%%%%%%%%%%%%%%%%%%%%%%%%%%%%%%%%%%%%%%%%%%%%%%%%%%%%%%%%%%%%%%%%%%%%%%%%%%%%%%%%%%%%%%%%%%%%%%%%%%%%%%%%%%%%%%%%%
\section{Fixed Points}

\begin{lemma}[First Order $\alpha$ Correction to $\Phi$]
   Given a randomized environment interaction channel $\Phi$ with coupling coefficient $\alpha$, the first order correction is
   \begin{equation}
        \frac{\partial}{\partial \alpha} \Phi(\rho_S) \bigg|_{\alpha = 0} = 0.
   \end{equation}
\end{lemma}
\begin{proof}
    We start by using linearity of derivatives, integration, and partial trace to pull the $\alpha$ derivative to act on $\Phi_G$ as
    \begin{align}
        \frac{\partial}{\partial \alpha} \Phi(\rho_S) \bigg|_{\alpha = 0} &= \frac{\partial}{\partial \alpha} \partrace{\mathcal{H}_E}{\int \Phi_G(\rho_s) dG} \bigg|_{\alpha = 0} \\
         &= \partrace{\mathcal{H}_E}{\int \frac{\partial}{\partial \alpha} \Phi_G(\rho_S) dG \bigg|_{\alpha = 0} } .
    \end{align}
    Now we use Eq. \eqref{eq:phi_g_definition} to compute the derivatives, we remind the reader that $\rho = \rho_S \otimes \rho_E$,
    \begin{align}
        \frac{\partial}{\partial \alpha} \Phi_G (\rho_S) &= \parens{\frac{\partial}{\partial \alpha} e^{+ i (H + \alpha G)t}} \rho e^{-i (H + \alpha G) t} + e^{+i (H + \alpha G)t} \rho \parens{\frac{\partial}{\partial \alpha} e^{- i (H + \alpha G)t}} \\
        &= \parens{\int_{0}^{1} e^{i s (H+\alpha G)t} (i t G) e^{i (1-s) (H+\alpha G)t} ds} \rho e^{-i(H+\alpha G)t} \nonumber \\
    &~ ~+ e^{i(H+\alpha G)t} \rho \parens{\int_{0}^1 e^{-i s (H+\alpha G) t} (- i t G) e^{-i (1-s) (H+\alpha G)t} ds}. \label{eq:first_order_alpha_derivative}
    \end{align}
    We can further simplify this by bringing in the evaluation of $\alpha = 0$ through the partial trace and integration, as they are uniformly convergent over $\alpha$ (is that the correct notion that allows us to switch orders?)
    \begin{align}
        \frac{\partial}{\partial \alpha} \Phi_G(\rho_S) \bigg|_{\alpha = 0} &= i t \int_0^1 e^{i s H t} G e^{-i s H t} ds e^{i H t} \rho e^{-i H t} - i t e^{+i H t} \rho \int_0^1 e^{-is H t} G e^{-i(1-s) Ht} ds \\
        &= i t \parens{\int_0^1 G(s t) ds} \rho(t) - it \rho(t) \parens{\int_0^1 G(s t) ds} \\
        &= i t \int_0^1 [G(s t), \rho(t)] ds,
    \end{align}
    where we have used the Heisenberg picture $\rho(t) = e^{i H t} \rho e^{-i H t}$ to simplify the notation.

    This expression is now amenable to computing the correction to the total channel. We do so by performing the integration over the randomized interactions. We take advantage of the structure of our interaction measure, that is $G = U_G D U_G^\dagger$ and $dG = dU_G dD$, which allows us to write
    \begin{align}
        \int \frac{\partial}{\partial \alpha} \Phi_G(\rho_S) \bigg|_{\alpha = 0} dG &= it \int \int_0^1 \left[ e^{i H s t} G e^{-i H s t}, \rho(t) \right] ds ~dG \\
        &= it \int_0^1 \left[ e^{i H s t} \parens{\int \int U_G D U_G^\dagger ~dU_G ~ dD} e^{-i H s t}, \rho(t)  \right] ds \\
        &= i t \int_0^1 \left[ e^{i H s t} \parens{\int U_G \parens{\int D ~ dD} ~ U_G^\dagger dU_G } e^{-i H s t}, \rho(t) \right] ds \\
        &= 0.
    \end{align}
    This last step relies on the use of random eigenvalues with mean 0, implying $\int D ~dD = 0$ which shows that $\frac{\partial}{\partial \alpha} \Phi(\rho_S) \big|_{\alpha = 0 } = 0$.
\end{proof}

\begin{lemma} \label{lem:sandwiched_interaction}
    Our goal is to compute $\int G(x) \ketbra{i}{j} G(y) dG$.
\end{lemma}
\begin{proof}
    \begin{align}
        \int G(x) \ketbra{i}{j} G(y) dG &=  \int e^{i H x} U_G D U_G^{\dagger} e^{-i H x} \ketbra{i}{j} e^{i H y} U_G D U_G^\dagger e^{-i H y} ~dG \\
        &= \sum_{a, b,c, d} e^{i (\lambda(a) - \lambda(i))x} e^{i (\lambda(j) - \lambda(d))y} \nonumber \\
        &\quad \times \int \ketbra{a}{a} U_G D(b) \ketbra{b}{b} U_G^\dagger \ketbra{i}{j} U_G D(c) \ketbra{c}{c} U_G^\dagger \ketbra{d}{d} dG \\
        &= \sum_{a, b,c, d}  e^{i (\lambda(a) - \lambda(i))x} e^{i (\lambda(j) - \lambda(d))y} \ketbra{a}{d} \nonumber \\
        &\quad \times \int D(b) D(c) dD \int \bra{a} U_G \ket{b} \bra{j} U_G \ket{c} \bra{b} U_G^\dagger \ket{i} \bra{c} U_G^\dagger \ket{d} dU_G \\
        &= \sigma^2 \sum_{a,b,d} e^{i (\lambda(a) - \lambda(i))x} e^{i (\lambda(j) - \lambda(d))y} \ketbra{a}{d} \nonumber \\ 
        &\quad \times \int \bra{a} U_G \ket{b} \bra{j} U_G \ket{b} \bra{i} \overline{U_G} \ket{b} \bra{d} \overline{U_G} \ket{b} dU_G \\
        &= \frac{\sigma^2}{\dim^2 - 1} \sum_{a,b,d} e^{i (\lambda(a) - \lambda(i))x} e^{i (\lambda(j) - \lambda(d))y} \ketbra{a}{d} (\delta_{ai} \delta_{jd} + \delta_{ad}\delta_{ij})\parens{1 - \frac{1}{\dim}} \\
        &= \frac{\sigma^2}{\dim + 1} \sum_{a,d} e^{i (\lambda(a) - \lambda(i))x} e^{i (\lambda(j) - \lambda(d))y} \ketbra{a}{d} (\delta_{ai} \delta_{jd} + \delta_{ad}\delta_{ij}) \\
        &= \frac{\sigma^2}{\dim + 1} \parens{\ketbra{i}{j} + \delta_{ij} \sum_{a} e^{i(\lambda(a) - \lambda(i))(x-y)} \ketbra{a}{a} }
    \end{align}
\end{proof}

\begin{lemma} \label{lem:two_heisenberg_interactions}
    The product of two Heisenberg evolved random interactions $G = U_G D U_G^\dagger$, where $U_G \sim Haar$ and $D = \sum_a D(a) \ketbra{a}{a}$ is diagonal in the total Hamiltonian $H$ basis with each eigenvalue $D(a)$ sampled i.i.d from a distribution with $\expect{D(a)} = 0$ and $\variance{D(a)} = \sigma^2$ can be integrated as
    \begin{equation}
        \int G(x) G(y) dG = \frac{\sigma^2}{\dim + 1} \parens{\sum_{a,c} e^{i (\lambda(a) - \lambda(c))(x-y)} \ketbra{a}{a} + \identity}.
    \end{equation}
\end{lemma}
\begin{proof}
    \begin{align}
        \int G(x) G(y) dG &= \int e^{+i H x} U_G D U_G^\dagger e^{-i H x} e^{+i H y} U_G D U_G^\dagger e^{-i H y} dU_G ~dD \\
        &= \int \bigg[\sum_a e^{+i \lambda(a)x}\ketbra{a}{a}  U_G \sum_b D(b)\ketbra{b}{b} U_G^\dagger \nonumber \\
        &\quad \sum_c e^{-i \lambda(c) (x - y)} \ketbra{c}{c} U_G \sum_d D(d)\ketbra{d}{d} U_G^\dagger \sum_e e^{-i \lambda(e) y} \ketbra{e}{e} \bigg] dU_G ~dD\\
        &=\sum_{a,b,c,d,e} \ketbra{a}{e} e^{-i (\lambda(c) - \lambda(a))x} e^{-i (\lambda(e) - \lambda(c))y} \nonumber \\
        &\quad \times \int \bra{a} U_G \ket{b} \bra{c} U_G \ket{d} \bra{b} U_G^{\dagger} \ket{c} \bra{d} U_G^\dagger \ket{e} dU_G \int D(b) D(d) dD \\
        &= \sigma^2 \sum_{a, b, c, d, e} \delta_{bd} \ketbra{a}{e} e^{-i (\lambda(c) - \lambda(a))x} e^{-i (\lambda(e) - \lambda(c))y} \nonumber \\
        &\quad \times \int \bra{a} U_G \ket{b} \bra{c} U_G \ket{d} \bra{b} U_G^{\dagger} \ket{c} \bra{d} U_G^\dagger \ket{e} dU_G. \\
    \end{align}
    Now the summation over $d$ fixes $d=b$ and we use Lemma \ref{lem:haar_two_moment} to compute the Haar integral, which simplifies greatly due to the repeated $b$ index. Plugging the result into the above yields the following
    \begin{align}
        &= \frac{\sigma^2}{\dim^2 - 1} \sum_{a, b, c, e} \ketbra{a}{e} e^{-i (\lambda(c) - \lambda(a))x} e^{-i (\lambda(e) - \lambda(c))y} \parens{\delta_{ac} \delta_{ce} + \delta_{ae} - \frac{1}{\dim} \parens{\delta_{ac} \delta_{ce} + \delta_{ae}}}  \\
        &= \frac{\sigma^2}{\dim^2 - 1} \parens{1 - \frac{1}{\dim}} \sum_{a, b, c, e} \ketbra{a}{e} e^{-i (\lambda(c) - \lambda(a))x} e^{-i (\lambda(e) - \lambda(c))y} \delta_{ae} (1 + \delta_{ac}) \\
        &= \frac{\sigma^2}{\dim^2 - 1} \parens{1 - \frac{1}{\dim}} \sum_{a, b, c} \ketbra{a}{a} e^{i (\lambda(a) - \lambda(c))(x-y)} (1 + \delta_{ac}) \\
        &= \frac{\sigma^2 \parens{\dim - 1}}{\dim^2 - 1} \sum_{a,c} \ketbra{a}{a} e^{i (\lambda(a) - \lambda(c))(x - y)} (1 + \delta_{ac}) \\
        &= \frac{\sigma^2}{\dim + 1} \parens{\sum_{a,c} e^{i (\lambda(a) - \lambda(c))(x-y)} \ketbra{a}{a} + \identity}.
    \end{align}
\end{proof}

\begin{lemma}[Second Order Correction] \label{lem:the_double_duhamel}
    Given a system Hamiltonian $H_{S}$, an environment Hamiltonian $H_{E}$, a simulation time $t$, and coupling coefficient $\alpha$, let $\Phi_G : \hilb_S \otimes \hilb_E \to \hilb_S \otimes \hilb_E$ denote the fixed interaction channel 
    \begin{equation}
        \Phi_G(\rho) = e^{+i (H + \alpha G)t} \rho e^{-i (H + \alpha G)t},
    \end{equation}
    where $H = H_S \otimes \identity + \identity \otimes H_E$. We compute the output of the averaged channel at $\alpha = 0$ for the basis $\ketbra{a}{b}$ of linear operators as:
 \begin{align}
     &\int \frac{\partial^2}{\partial \alpha^2} \Phi_G(\ketbra{a}{b})\bigg|_{\alpha = 0} dG \\
     &= -\frac{2 \sigma^2 e^{i \Delta_{ab} t}}{\dim + 1} \bigg(\sum_{j: \Delta_{aj} \neq 0} \frac{1 - i \Delta_{aj}t - e^{-i \Delta_{aj} t}}{\Delta_{aj}^2} + \sum_{j: \Delta_{bj} \neq 0} \frac{1 + i \Delta_{bj} t - e^{i \Delta_{bj} t}}{\Delta_{bj}^2} + \frac{t^2}{2}(\eta(a) + \eta(b)) \bigg) \ketbra{a}{b} \nonumber \\
    &~ +\delta_{ab} \frac{2\sigma^2 e^{i \Delta_{ab}t}}{\dim+1} \parens{ \sum_{i: \Delta_{ai} \neq 0 } \frac{2(1- \cos (\Delta_{ai}t))}{\Delta_{ai}^2} \ketbra{i}{i} + t^2 \sum_{i : \Delta_{ai} = 0} \ketbra{i}{i}}
 \end{align}
\end{lemma}
\begin{proof}
We start from the expression for the first derivative of the channel $\frac{\partial}{\partial \alpha} \Phi_G(\rho_S)$ given by Eq. \eqref{eq:first_order_alpha_derivative}. To take the second derivative there are six factors involving $\alpha$, so we will end up with six terms. We repeat Eq. \eqref{eq:first_order_alpha_derivative} below, add a derivative, and label each factor containing an $\alpha$ for easier computation
\begin{align}
    \frac{\partial^2}{\partial \alpha^2} \Phi_G(\rho_S) =& \frac{\partial}{\partial \alpha} \parens{\int_{0}^{1} \underset{\substack{\downarrow \\ (A)}}{e^{i s (H+\alpha G)t}} (i t G) \underset{\substack{\downarrow \\ (B)}}{e^{i (1-s) (H+\alpha G)t}} ds ~ \rho \underset{\substack{\downarrow \\ (C)}}{e^{-i(H+\alpha G)t}} } \nonumber \\
    &~ ~+\frac{\partial}{\partial \alpha} \parens{ \underset{\substack{\downarrow \\ (D)} }{e^{i(H+\alpha G)t}} \rho \int_{0}^1 \underset{\substack{\downarrow \\ (E)} }{e^{-i s (H+\alpha G) t} } (- i t G) \underset{\substack{\downarrow \\ (F)}}{e^{-i (1-s) (H+\alpha G)t}} ds }.
\end{align}
We will work through the first term in detail and will simply state the remaining terms. We work in order from left to right and top to bottom. The first term can be computed as
\begin{align}
    (A) &=i t\int_0^1 \parens{\frac{\partial}{\partial \alpha} e^{i s_1 (H+ \alpha G)t}} G e^{i(1-s_1)(H+\alpha G)t} ds_1 \rho e^{-i (H+\alpha G)t} \bigg|_{\alpha=0} \\
    &= (it)^2 \int_0^1 \parens{\int_0^1 e^{i s_1 s_2 (H+\alpha G)t} s_1 G e^{i s_1 (1-s_2) (H+\alpha G)t} ds_2} G e^{i(1-s_1) (H+\alpha G)t} ds_1 \rho e^{-i(H+\alpha G) t} \bigg|_{\alpha=0} \\
    &= -t^2 \int_0^1 \int_0^1 e^{i s_1 s_2 H t} G e^{-i s_1 s_2 H t} e^{i s_1 H t} G e^{-i s_1 H t} s_1 ds_1 ds_2 e^{i H t} \rho e^{-i H t} \\
    &= -t^2 \int_0^1 \int_0^1 G(s_1 s_2 t) G(s_1 t) s_1 ds_1 ds_2 \rho(t). \label{eq:second_deriv_alpha_first_term}
\end{align}

\begin{align}
    (B) &= it \int_0^1 e^{i s_1 (H + \alpha G)t} G \frac{\partial}{\partial \alpha}\parens{e^{i(1-s_1)(H + \alpha G)t}} ds_1 \rho e^{-i(H + \alpha G) t} \bigg|_{\alpha = 0} \\
    &= (it)^2 \int_0^1 e^{i s_1 (H + \alpha G)t} G \parens{\int_0^1 e^{i(1-s_1)s_2 (H + \alpha G)t} (1-s_1) G e^{i(1 - s_1)(1 - s_2)(H + \alpha G)t} ds_2} ds_1 ~ \rho e^{-i ( H + \alpha G)t} \bigg|_{\alpha = 0} \\
    &= -t^2 \int_0^1 \int_0^1 e^{i s_1 H t} G e^{i(1-s_1)s_2 H t} G e^{i(1-s_1)(1-s_2) H t} (1-s_1) ds_1 ds_2 ~ \rho e^{-i H t}\\ 
    &= -t^2 \int_0^1 \int_0^1 e^{i s_1 H t} G e^{-i s_1 H t} e^{i(s_1 + s_2 - s_1 s_2) H t} G e^{-i (s_1 + s_2 - s_1 s_2) H t} (1-s_1) ds_1 ds_2 ~ \rho(t) \\
    &= -t^2 \int_0^1 \int_0^1 G(s_1 t) G((s_1 + s_2 - s_1 s_2)t) (1-s_1) ds_1 ds_2 ~ \rho(t)
\end{align}

\begin{align}
    (C) &= it \int_0^1 e^{i s (H + \alpha G)t} G e^{i(1-s) (H + \alpha G) t} ds ~\rho ~ \frac{\partial}{\partial \alpha} \parens{ e^{-i (H + \alpha G) t} } \bigg|_{\alpha = 0} \\
    &= (i t) (-it) \int_0^1 e^{i s (H + \alpha G)t} G e^{i (1 - s) (H + \alpha G)t} ds ~ \rho ~ \parens{ \int_0^1 e^{-i s (H + \alpha G)t} G e^{-i (1- s) ( H + \alpha G)t } ds}\bigg|_{\alpha = 0} \\
    &= + t^2 \parens{\int_0^1 e^{i s H t} G e^{-i s H t} ds} e^{i H t} \rho e^{-i H t} \parens{\int_0^1 e^{i (1-s) H t} G e^{-i (1-s) H t} ds} \\
    &= + t^2 \int_0^1 G(st) ds ~ \rho(t) \int_0^1 G((1-s)t) ds
\end{align}

\begin{align}
    (D) &= (-it) \frac{\partial}{\partial \alpha} \parens{e^{i(H + \alpha G)t}} \rho \int_0^1 e^{-i s (H + \alpha G)t} G e^{-i (1-s)(H + \alpha G)t} ds \bigg|_{\alpha = 0} \\
    &= t^2 \parens{\int_0^1 e^{i s (H+ \alpha G)t} G e^{i (1-s) (H + \alpha G)t}ds} \rho \int_0^1 e^{-i s (H + \alpha G)t} G e^{-i (1-s)(H + \alpha G)t} ds \bigg|_{\alpha = 0} \\
    &=  t^2 \int_0^1 e^{i s H t} G e^{-i s H t} ds ~\rho(t) \int_0^1 e^{i (1-s) H t} G e^{-i (1-s) H t} ds \\
    &= t^2 \int_0^1 G(st) ds ~ \rho(t) ~ \int_0^1 G((1-s)t) ds
\end{align}

\begin{align}
    (E) &= (-it) e^{i (H+ \alpha G) t} ~ \rho ~\int_0^1 \frac{\partial}{\partial \alpha} \parens{e^{-i s_1 (H + \alpha G)t}} G e^{-i (1-s_1)(H + \alpha G)t} ds_1 \bigg|_{\alpha = 0} \\
    &= - t^2 e^{i(H + \alpha G)t} ~ \rho ~\int_0^1 \parens{\int_0^1 e^{-i s_1 s_2 (H + \alpha G) t} (s_1 G) e^{-i s_1 (1-s_2) (H + \alpha G)t} ds_2} G e^{-i(1-s_1)(H + \alpha G)t} ds_1 \bigg|_{\alpha = 0} \\
    &= -t^2 e^{i H t} \rho e^{-i H t} \int_0^1 \int_0^1 e^{i (1 - s_1 s_2) H t} G e^{-i (s_1 - s_1 s_2)H t} G e^{-i (1-s_1)H t} s_1 ds_1 ds_2 \\
    &= -t^2 \rho(t) \int_0^1 \int_0^1 G((1- s_1 s_2) t) G((1-s_1)t) s_1 ds_1 ds_2
\end{align}

\begin{align}
    (F) &= (-it) e^{i(H + \alpha G) t} \rho \int_0^1 e^{-i s_1 ( H + \alpha G) t} G \frac{\partial}{\partial \alpha} \parens{ e^{-i (1-s_1) ( H +\alpha G)t}} ds_1 \bigg|_{\alpha = 0} \\
    &= (-it)^2 e^{i (H + \alpha G)t} \rho \int_0^1 e^{-i s_1 (H + \alpha G)t} G \parens{\int_0^1 e^{-i(1-s_1) s_2 (H + \alpha G)t} (1-s_1) G e^{-i(1-s_1) (1-s_2) (H + \alpha G) t} ds_2} ds_1 \bigg|_{\alpha = 0} \\
    &= -t^2 e^{-i H t} \rho e^{-i H t} \int_0^1 \int_0^1 e^{i (1- s_1) H t} G e^{-i (1-s_1) H t} e^{i (1-s_1)(1-s_2) H t} G e^{-i(1-s_1)(1-s_2) H t} (1-s_1) ds_1 ds_2 \\
    &= -t^2 \rho(t) \int_0^1 \int_0^1 G((1-s_1)t) G((1-s_1)(1 - s_2) t) (1-s_1)ds_1 ds_2
\end{align}

Now our goal is to compute the effects of averaging over the interaction $G$ on the above terms, starting with $(A)$. We will make heavy use of Lemma \ref{lem:two_heisenberg_interactions}.
\begin{align}
    \int (A) dG &= -t^2 \int_0^1 \int_0^1 \int G(s_1 s_2 t) G(s_1 t) dG s_1 ds_1 ds_2 \rho(t) \\
    &= \frac{-t^2 \sigma^2}{\dim + 1} \int_0^1 \int_0^1 \parens{\sum_{i,j} e^{i (\lambda(i) - \lambda(j)) (s_1 s_2 t - s_1 t)} \ketbra{i}{i} + \identity} s_1 ds_1 ds_2 \rho(t) \\
    &= \frac{- t^2 \sigma^2}{\dim + 1} \parens{\sum_{i} \sum_{j : \lambda(i) \neq \lambda(j)} \int_0^1 \int_0^1 e^{i(\lambda(i) - \lambda(j))t (s_1 s_2 - s_1)} s_1 ds_1 ds_2 \ketbra{i}{i} + \sum_{i} \sum_{j : \lambda(i) = \lambda(j)}\frac{1}{2} \ketbra{i}{i} + \frac{1}{2} \identity} \rho(t) \\
    &= \frac{- t^2 \sigma^2}{\dim + 1} \parens{\sum_i \sum_{j : \lambda(i) \neq \lambda(j)} \frac{1 - i (\lambda(i) - \lambda(j))t - e^{-i (\lambda(i) - \lambda(j))t}}{t^2 (\lambda(i) - \lambda(j))^2} \ketbra{i}{i} + \frac{1}{2} \sum_{i} (\eta(i) + 1) \ketbra{i}{i} } \rho(t) \\
    &= \frac{- \sigma^2}{\dim + 1}\parens{\sum_{i} \sum_{j: \Delta_{ij} \neq 0} \frac{1 - i \Delta_{ij}t - e^{-i \Delta_{ij} t}}{\Delta_{ij}^2} \ketbra{i}{i} + \frac{t^2}{2} \sum_{i} (\eta(i) + 1)\ketbra{i}{i} } \rho(t)
\end{align}

We can similarly compute the averaged $(B)$ term:
\begin{align}
    \int (B) dG &= -t^2 \int_0^1 \int_0^1 \int G(s_1 t) G((s_1 + s_2 - s_1 s_2) t) dG (1-s_1) ds_1 ds_2 ~ \rho(t) \\
    &= \frac{- t^2 \sigma^2}{\dim + 1} \int_0^1 \int_0^1 \parens{\sum_{i,j} e^{i (\lambda(i) - \lambda(j))(s_1 s_2 - s_2) t} \ketbra{i}{i} + \identity} (1 -s_1) ds_1 ds_2 \rho \\
    &= \frac{- t^2 \sigma^2}{\dim + 1} \parens{\sum_{i} \sum_{j : \lambda(i) \neq \lambda(j)} \int_0^1 \int_0^1 e^{i(\lambda(i) - \lambda(j))t (s_1 s_2 - s_2)} (1 - s_1) ds_1 ds_2 \ketbra{i}{i} + \sum_{i} \sum_{j : \lambda(i) = \lambda(j)}\frac{1}{2} \ketbra{i}{i} + \frac{1}{2} \identity} \rho(t) \\
    &= \frac{- t^2 \sigma^2}{\dim + 1} \parens{\sum_i \sum_{j : \lambda(i) \neq \lambda(j)} \frac{1 - i (\lambda(i) - \lambda(j))t - e^{-i (\lambda(i) - \lambda(j))t}}{t^2 (\lambda(i) - \lambda(j))^2} \ketbra{i}{i} + \frac{1}{2} \sum_{i} (\eta(i) + 1) \ketbra{i}{i} } \rho(t) \\
    &= \frac{- \sigma^2}{\dim + 1}\parens{\sum_{i} \sum_{j: \Delta_{ij} \neq 0} \frac{1 - i \Delta_{ij}t - e^{-i \Delta_{ij} t}}{\Delta_{ij}^2} \ketbra{i}{i} + \frac{t^2}{2} \sum_{i} (\eta(i) + 1)\ketbra{i}{i} } \rho(t),
\end{align}
which we note is identical to $\int (A) dG$. As terms $(C)$ and $(D)$ involve a different method of computation we skip them for now and compute $(E)$ and $(F)$. 
\begin{align}
    \int (E) dG &= -t^2 \rho(t) \int_0^1 \int_0^1 \int G((1- s_1 s_2) t) G((1-s_1)t) dG s_1 ds_1 ds_2 \\
    &= \frac{- t^2 \sigma^2}{\dim + 1} \rho(t) \int_0^1 \int_0^1 \parens{\sum_{i,j} e^{i(\lambda(i) - \lambda(j)) t (s_1 - s_1 s_2)} \ketbra{i}{i} + \identity } s_1 ds_1 ds_2 \\
    &= \frac{- t^2 \sigma^2}{\dim + 1} \rho(t) \parens{\sum_i \sum_{j : \lambda(i) \neq \lambda(j)} \frac{1 + i (\lambda(i) - \lambda(j))t - e^{i(\lambda(i) - \lambda(j))t}}{t^2 (\lambda(i) - \lambda(j))^2}\ketbra{i}{i} + \frac{1}{2} \sum_{i} (\eta(i) + 1 )\ketbra{i}{i}} \\
    &= \frac{- \sigma^2}{\dim + 1} \rho(t) \parens{\sum_i \sum_{j: (\Delta_{ij} \neq 0)} \frac{1 + i \Delta_{ij}t - e^{i\Delta_{ij}t}}{\Delta_{ij}^2} \ketbra{i}{i} + \frac{t^2}{2}\sum_i (\eta(i) + 1) \ketbra{i}{i}}.
\end{align}
Computing $(F)$ yields
\begin{align}
    \int (F) dG &= -t^2 \rho(t) \int_0^1 \int_0^1 \int G((1-s_1)t) G((1-s_1)(1 - s_2) t) dG (1-s_1)ds_1 ds_2 \\
    &= \frac{- t^2 \sigma ^2}{\dim + 1} \rho(t) \int_0^1 \int_0^1 \parens{\sum_{i,j} e^{i(\lambda(i) - \lambda(j))t (s_2 - s_1 s_2)}\ketbra{i}{i} + \identity} (1-s_1) ds_1 ds_2 \\
    &= \frac{- t^2 \sigma^2}{\dim + 1} \rho(t) \parens{\sum_{i} \sum_{j : \lambda(i) \neq \lambda(j)} \frac{1 + i (\lambda(i) - \lambda(j))t - e^{i (\lambda(i) - \lambda(j))t}}{t^2 (\lambda(i) - \lambda(j))^2} \ketbra{i}{i} +\frac{1}{2} \sum_{i} (\eta(i) + 1) \ketbra{i}{i}} \\
    &= \frac{- \sigma^2}{\dim + 1} \rho(t) \parens{\sum_i \sum_{j: (\Delta_{ij} \neq 0)} \frac{1 + i \Delta_{ij}t - e^{i\Delta_{ij}t}}{\Delta_{ij}^2} \ketbra{i}{i} + \frac{t^2}{2}\sum_i (\eta(i) + 1) \ketbra{i}{i}}
\end{align}
 which is identical to $\int (E) dG$.

 The last two terms $(C) = (D)$ are computed as follows:
 \begin{align}
     \int (C) dG &= t^2 \int_0^1 \int_0^1 \int G(s_1 t) \rho(t) G((1-s_2)t) ~dG ~ ds_1 ds_2 \\
     &= t^2 \sum_{i,j} \rho_{ij} e^{i(\lambda(i) - \lambda(j))t} \int_0^1 \int_0^1 \int G(s_1 t) \ketbra{i}{j} G((1-s_2)t) ~ dG ~ ds_1 ds_2 \\
     &= \frac{\sigma^2 t^2}{\dim + 1} \sum_{i,j} \rho_{ij} e^{i(\lambda(i) - \lambda(j))t} \parens{ \ketbra{i}{j} + \delta_{ij} \sum_{a} \int_0^1 \int_0^1 e^{i(\lambda(a) - \lambda(i))(s_1 + s_2 - 1)t} ds_1 ds_2 \ketbra{a}{a}} \\
     &= \frac{\sigma^2 t^2}{\dim + 1} \sum_{i,j} \rho_{ij} e^{i \Delta_{ij} t} \parens{\ketbra{i}{j} + \delta_{ij} \sum_{a : \Delta_{ai} \neq 0} \frac{2( 1- \cos (\Delta_{ai} t))}{\Delta_{ai}^2 t^2} \ketbra{a}{a} + \delta_{ij} \sum_{a : \Delta_{ai} = 0} \ketbra{a}{a}}
 \end{align}

 We can now combine each of these terms to offer the full picture of the output of the channel to second order. For simplicity we will let $\rho = \ketbra{a}{b}$ and change summation indices accordingly. 
 \begin{align}
     &\int \frac{\partial^2}{\partial \alpha^2} \Phi_G(\ketbra{a}{b})\bigg|_{\alpha = 0} dG \\
     &= -\frac{2 \sigma^2 e^{i \Delta_{ab} t}}{\dim + 1} \bigg(\sum_{j: \Delta_{aj} \neq 0} \frac{1 - i \Delta_{aj}t - e^{-i \Delta_{aj} t}}{\Delta_{aj}^2} + \sum_{j: \Delta_{bj} \neq 0} \frac{1 + i \Delta_{bj} t - e^{i \Delta_{bj} t}}{\Delta_{bj}^2} + \frac{t^2}{2}(\eta(a) + \eta(b)) \bigg) \ketbra{a}{b} \nonumber \\
    &~ +\frac{2\sigma^2 e^{i \Delta_{ab}t}}{\dim+1} \parens{\delta_{ab} \sum_{i: \Delta_{ai} \neq 0 } \frac{2(1- \cos (\Delta_{ai}t))}{\Delta_{ai}^2} \ketbra{i}{i} + t^2 \delta_{ab} \sum_{i : \Delta_{ai} = 0} \ketbra{i}{i}} \label{eq:second_order_output}
 \end{align}

\end{proof}

We can now leverage this lemma to show transition probabilities among diagonal matrix elements and to argue that off-diagonal transitions are suppressed to order $\bigo{\alpha^2}$.
\begin{theorem}
    Given the setup in Lemma \ref{lem:the_double_duhamel}, let $\prob{a \to b}$ denote the transition probability from $a$ to $b$ for the joint system-environment Hilbert space be given to second order in alpha $\bigo{\alpha^2}$ as:
    \begin{equation}
    \prob{a \to b} \coloneqq \trace{\ketbra{b}{b} \parens{\ketbra{a}{a} + \frac{\alpha^2}{2} \int \frac{\partial^2}{\partial \alpha^2} \Phi_G(\ketbra{a}{a}) \bigg|_{\alpha = 0} dG}}
    \end{equation}
    we can compute the following transition probabilities among diagonal matrix elements
    \begin{equation}
        \prob{a \to b} = \begin{cases}
            1 -\frac{ \sigma^2 \alpha^2}{\dim + 1} \parens{\sum_{j: \Delta_{aj} \neq 0} \frac{2( 1- \cos (\Delta_{aj}t)) }{\Delta_{aj}^2} + t^2 (\eta(a) - 1)} & a = b \\
            \frac{\sigma^2 \alpha^2 t^2}{\dim + 1} & a \neq b, \lambda(a) = \lambda(b) \\
            \frac{2\sigma^2 \alpha^2}{\dim + 1} \frac{1 - \cos( \Delta_{ab} t)}{\Delta_{ab}^2} & \lambda(a) \neq \lambda(b). \label{eq:transition_rule}
        \end{cases}
    \end{equation}
    We further note that in order for these to be valid transition probabilities we have to make the assumption that $t^2 \leq \frac{\dim + 1}{\alpha^2 \sigma^2}$. 
\end{theorem}
\begin{proof}
    We focus exclusively on diagonal inputs and outputs for density matrices, $\ketbra{a}{a}$ and $\ketbra{b}{b}$. We use Eq. \eqref{eq:second_order_output}, starting with transitions within the degenerate subspace of $a$. For the following we assume $a \neq b$ and $\Delta_{ab} = 0$: 
    \begin{align}
        &\prob{a \to b | a \neq b, \Delta_{ab} = 0} = \trace{\ketbra{b}{b} \int \Phi_G(\ketbra{a}{a}) dG} \\
        &= \braket{b}{a} \braket{a}{b} + \frac{\alpha^2}{2} \bra{b} \int \frac{\partial^2}{\partial \alpha^2} \Phi_G(\ketbra{a}{a})\bigg|_{\alpha = 0} dG \ket{b} \\
        &= -\frac{\sigma^2 \alpha^2}{\dim + 1} \bigg(\sum_{j: \Delta_{aj} \neq 0} \frac{1 - i \Delta_{aj}t - e^{-i \Delta_{aj} t}}{\Delta_{aj}^2} + \sum_{j: \Delta_{aj} \neq 0} \frac{1 + i \Delta_{aj} t - e^{i \Delta_{aj} t}}{\Delta_{aj}^2} + t^2 \eta(a) \bigg) \braket{b}{a} \braket{a}{b} \nonumber \\
        &~+ \frac{\alpha^2 \sigma^2}{\dim + 1} \parens{\sum_{i: \Delta_{ai} \neq 0} \frac{2(1- \cos(\Delta_{ai} t)}{\Delta_{ai}^2} \braket{b}{i}\braket{i}{b} + t^2 \sum_{i: \Delta_{ai} = 0 } \braket{b}{i} \braket{i}{b}} \label{eq:transition_intermediate} \\
        &= \frac{\sigma^2 \alpha^2 t^2}{\dim + 1}. 
    \end{align}
    We now proceed to the case that $b \neq a$ and $\Delta_{ab} \neq 0$, where we can start from Eq. \eqref{eq:transition_intermediate} and we get
    \begin{equation}
        \prob{a \to b | a \neq b, \Delta_{ab} \neq 0} = \frac{2 \sigma^2 \alpha^2 }{\dim + 1} \frac{1 - \cos (\Delta_{ab} t)}{\Delta_{ab}^2}.
    \end{equation}

    The remaining case to consider is when $b = a$. This involves simplifying the summations in Eq. \eqref{eq:second_order_output} and including the zeroth order $\bigo{\alpha^0}$ term which simply contributes a 1. 
    \begin{align}
        \prob{a \to a} &= 1 -\frac{\sigma^2 \alpha^2 }{\dim + 1} \bigg(\sum_{j: \Delta_{aj} \neq 0} \frac{1 - i \Delta_{aj}t - e^{-i \Delta_{aj} t}}{\Delta_{aj}^2} + \sum_{j: \Delta_{aj} \neq 0} \frac{1 + i \Delta_{aj} t - e^{i \Delta_{aj} t}}{\Delta_{aj}^2} + t^2 \eta(a) \bigg) \nonumber \\
    &~ +\frac{\sigma^2 \alpha^2 }{\dim+1} \parens{ \sum_{i: \Delta_{ai} \neq 0 } \frac{2(1- \cos (\Delta_{ai}t))}{\Delta_{ai}^2} \braket{a}{i} \braket{i}{a} + t^2  \sum_{i : \Delta_{ai} = 0} \braket{a}{i} \braket{i}{a}} \\
    &= 1 - \frac{\sigma^2 \alpha^2}{\dim + 1} \parens{\sum_{i: \Delta_{ai} \neq 0} \frac{2(1 - \cos(\Delta_{ai}t))}{\Delta_{ai}^2} + t^2 (\eta(a) - 1)}.
    \end{align}
    This completes the transition probability computation. It is straightforward to see that $\sum_{b} \prob{a \to b} = 1$ and that $0 \leq \prob{a \to b} \leq 1$ given that $t^2 \leq \frac{\dim + 1}{\alpha^2 \sigma^2}$.
\end{proof}

Note that we can also easily show that a diagonal entry does not transition to an off-diagonal entry at this order in $\alpha$, so we assume $c \neq d$ in the following:
\begin{align}
    \bra{c} \int \frac{\partial^2}{\partial \alpha^2} \Phi_G(\ketbra{a}{a}) \bigg|_{\alpha = 0} dG \ket{d} &= \frac{2 \sigma^2}{\dim + 1} \parens{\sum_{j : \Delta_{aj} \neq 0} \frac{2(\cos(\Delta_{aj}t) - 1)}{\Delta_{aj}^2} - t^2 \eta(a)} \delta_{ac} \delta_{ad} \nonumber \\
    & + \frac{2 \sigma^2}{\dim + 1}  \parens{\sum_{i: \Delta_{ai \neq 0} }\frac{2(1 - \cos(\Delta_{ai}t))}{\Delta_{ai}^2} \delta_{ci} \delta_{id} + \sum_{i : \Delta_{ai} = 0} t^2 \delta_{ci} \delta_{id}} \\
    &= 0.
\end{align}
It is also evident that off-diagonal elements do not transition to other off diagonal elements based on the only off-diagonal output state of Eq. \eqref{eq:second_order_output} being the exact same as the off-diagonal input. 

We now give the following result showing the transition amplitudes for the system register when it is coupled with a single qubit environment.
\begin{corollary}
    Let $H_E = \begin{bmatrix} 0 & 0 \\ 0 & \gamma \end{bmatrix}$ be a single qubit environment and $\rho_E = e^{-\beta H_E} / \partfun_E(\beta)$ be the state of the environment for application of $\Phi$. Then we show that if $\lambda_S(a) > \lambda_S(b)$, then $\prob{a \to b} > \prob{b \to a}$ after time-averaging, indicating that our channel prefers to move probability mass from high energy to lower energy. 
\end{corollary}
\begin{proof}
Our goal is to look at the system transitions at this order. This involves tracing out the environment, so we switch back to double index notation. Further for now we assume $a \neq b$ and compute the self-transition later.
\begin{align}
    \bra{b} \Phi(\ketbra{a}{a}) \ket{b} &= \sum_{i} \bra{b, i} \int \Phi_G(\ketbra{a}{a} \otimes \rho_E) dG \ket{b,i} \\
    &= \sum_i \bra{b,i} \parens{\ketbra{a}{a} \otimes \rho_E + \frac{\alpha^2}{2} \int \frac{\partial^2}{\partial \alpha^2} \Phi_G(\ketbra{a}{a} \otimes \rho_E) \bigg|_{\alpha =0} ~dG } \ket{b, i} + \bigo{\alpha^3} \\
    &= \frac{\alpha^2}{2} \sum_i \bra{b,i}  \int \frac{\partial^2}{\partial \alpha^2} \Phi_G(\ketbra{a}{a} \otimes \rho_E) \bigg|_{\alpha =0} ~dG  \ket{b, i} + \bigo{\alpha^3}.
\end{align}
To further simplify, we let $\rho_E = e^{-\beta H_E} / \partfun_E(\beta) = \partfun_E(\beta)^{-1} \ketbra{0}{0}  +  e^{-\beta \gamma} \partfun_E(\beta)^{-1} \ketbra{1}{1}$ and we ignore all terms of order $\bigo{\alpha^3}$. Another further simplification we make is that $\lambda_S(a) \neq \lambda_S(b)$ and that $\lambda_S(a) - \lambda_S(b) \neq \gamma$. This allows us to assume that all system-environment transitions occur at different energy levels, aka $\Delta(a,j | b, i) \neq 0$ for all $i$ and $j$. Given that the transition amplitudes replicate the degenerate condition, we can always make this assumption and look at the resulting behavior as the energy differences approach zero.
\begin{align}
    \bra{b} \Phi(\ketbra{a}{a}) \ket{b} &= \sum_{i, j} \frac{e^{-\beta \lambda_E(j)}}{\partfun_E(\beta)} \frac{2 \alpha^2 \sigma^2}{\dim + 1} \frac{1 - \cos(\Delta(a,j | b,i)t}{\Delta(a,j | b, i)^2} \\
    \bra{b} \Phi(\ketbra{a}{a}) \ket{b} \frac{(\dim + 1) \partfun_E(\beta)}{2 \alpha^2 \sigma^2} &= \frac{1 - \cos ( \Delta(a, 0 | b, 0) t)}{\Delta(a, 0 | b, 0)^2} + \frac{1 - \cos ( \Delta(a, 0 | b, 1) t)}{\Delta(a, 0 | b, 1)^2} \nonumber \\
    &~ + e^{-\beta \gamma} \frac{1 - \cos ( \Delta(a, 1 | b, 0) t)}{\Delta(a, 1 | b, 0)^2} + e^{-\beta \gamma} \frac{1 - \cos ( \Delta(a, 1 | b, 1) t)}{\Delta(a, 1 | b, 1)^2} \\
    &= \frac{1 - \cos(\Delta_S(a,b)t)}{\Delta_S(a,b)^2}(1 + e^{-\beta \gamma}) \nonumber \\
    &+ \frac{1 - \cos((\Delta_S(a,b) - \gamma)t)}{(\Delta_S(a,b) - \gamma)^2} + e^{-\beta \gamma} \frac{1 - \cos((\Delta_S(a,b) + \gamma)t)}{(\Delta_S(a,b) + \gamma)^2} \label{eq:sys_transition_with_qubit_bath}.
\end{align}
Here we see that the transition is no longer symmetric about $a$ and $b$ as observed in the total system-environment transitions due to the changing of the factor of $e^{-\beta \gamma}$ about the two terms with $\Delta_S(a,b) \pm \gamma$ when $a \leftrightarrow b$. We now analyze this for $\Delta_S(a,b) = \lambda(a) - \lambda(b) > 0$. We also know that $e^{-\beta \gamma} < 1$ for $\beta$ and $\gamma$ larger than 0. This allows us to study the difference:
\begin{align}
    \bra{b} \Phi(\ketbra{a}{a}) \ket{b} - \bra{a} \Phi(\ketbra{b}{b}) \ket{a} &= \frac{2 \alpha^2 \sigma^2}{(\dim + 1) \partfun_E(\beta)} \bigg( \frac{1 - \cos((\Delta_S(a,b) - \gamma)t)}{(\Delta_S(a,b) - \gamma)^2}(1 - e^{-\beta \gamma}) +  \nonumber \\
    &+\frac{1 - \cos((\Delta_S(a,b) + \gamma)t)}{(\Delta_S(a,b) + \gamma)^2}(e^{-\beta \gamma} - 1) \bigg)
\end{align}
Now we further simplify by  time-averaging:
\begin{align}
    (\prob{a \to b} - \prob{b \to a}) \frac{(\dim + 1) \partfun_E(\beta) (1 - e^{-\beta \gamma})}{2 \alpha^2 \sigma^2} &= \frac{1}{(\Delta_S(a,b) - \gamma)^2} - \frac{1}{(\Delta_S(a,b) + \gamma)^2}.
\end{align}
Now we note that $\Delta_S(a,b) > 0$ and $\gamma > 0$ implies that $(\Delta_S(a,b) - \gamma)^2 < (\Delta_S(a,b) + \gamma)^2$, which follows from expanding both sides. This yields 
\begin{align}
(\prob{a \to b} - \prob{b \to a}) \frac{(\dim + 1) \partfun_E(\beta) (1 - e^{-\beta \gamma})}{2 \alpha^2 \sigma^2} &= \frac{1}{(\Delta_S(a,b) - \gamma)^2} - \frac{1}{(\Delta_S(a,b) + \gamma)^2}.
&\geq 0 \\
\implies \prob{a \to b} &\geq \prob{b \to a},
\end{align}
whenever $\lambda_S(a) > \lambda_S(b)$. 
\end{proof}

\begin{lemma}
    Our next goal is to look at the trace distance for a thermal state system input at the same inverse temperature $\beta$ as the environment, with the aim of showing that it can be made negligibly small with appropriate choice of $\alpha, t$. We consider an extremely simple system of 
    \begin{equation}
        H_S = \begin{bmatrix}
            0 & 0 \\ 0 & 1
        \end{bmatrix},
    \end{equation}
    just a single qubit. Similarly our environment is $H_E = \begin{bmatrix}
        0 & 0 \\ 0 & \gamma
    \end{bmatrix}$ incredibly simple. We want to study the trace distance of the thermal state before and after application of $\Phi$
    \begin{equation}
        \norm{\rho_S(\beta) - \Phi(\rho_S(\beta))}_1
    \end{equation}
    as a function of $\gamma$. 
\end{lemma}
\begin{proof}
its in the pudding    
\end{proof}
We first plug in the second order Taylor's Series for $\Phi$, noting that the zeroth order terms cancel
\begin{align}
    \norm{\rho_S(\beta) - \Phi(\rho_S(\beta))}_1 &= \norm{\partrace{\hilb_E}{ \frac{\alpha^2}{2} \int \frac{\partial^2}{\partial \alpha^2} \Phi_G(\rho_S(\beta) \otimes \rho_E(\beta) ) \bigg|_{\alpha = 0} dG }} \\
    &= \frac{\alpha^2}{2 \partfun(\beta)}\norm{\partrace{\hilb_E}{ \sum_{a, i} e^{-\beta \lambda(a, i)} \int \frac{\partial^2}{\partial \alpha^2} \Phi_G(\ketbra{a,i}{a,i}) \bigg|_{\alpha=0} dG}}
\end{align}
We further simplify by noting that the operator within the norm is diagonal in the energy eigenbasis. This can be seen from Eq. \eqref{eq:second_order_output} after letting $a = b$. This allows us to compute the trace norm by solely using the diagonal outputs
\begin{align}
    \norm{A}_1 &= \trace{\sqrt{A A^\dagger}} \\
    &= \trace{\sqrt{\sum_i |a_i|^2 \ketbra{i}{i}}} \\
    &= \trace{\sum_i |a_i| \ketbra{i}{i}} \\
    &= \sum_i |a_i|.
\end{align}
Therefore we only need to compute 
\begin{align}
\bra{b} \partrace{\hilb_E}{\int \frac{\partial^2}{\partial \alpha^2} \Phi_G(\ketbra{a,i}{a,i})\bigg|_{\alpha=0} dG} \ket{b} &= \sum_{j} \bra{b,j} \int \frac{\partial^2}{\partial \alpha^2} \Phi_G(\ketbra{a,i}{a,i})\bigg|_{\alpha=0} dG \ket{b,j}.
\end{align}
To do so we first assume that $a = 1$ and $b = 0$, meaning $\lambda_S(a) = 1$ and $\lambda_S(b) = 0$ and that $\gamma \neq 1$. We make this assumption on $\gamma$ so we can utilize the non-degenerate transitions from Eq. \eqref{eq:transition_rule}. The exact same computation was carried out before up until Eq. \eqref{eq:sys_transition_with_qubit_bath} so we use that intermediate result, noting that we have already accounted for the factors of $\partfun_E(\beta)$ and $\alpha^2$. 
\begin{align}
    \bra{b} \partrace{\hilb_E}{\int \frac{\partial^2}{\partial \alpha^2} \Phi_G(\ketbra{a,i}{a,i})\bigg|_{\alpha=0} dG} \ket{b} \parens{\frac{\dim + 1}{4 \sigma^2}} &= \frac{1 - \cos(\Delta_S(a,b)t)}{\Delta_S(a,b)^2}(1 + e^{-\beta \gamma}) \nonumber \\
    &+ \frac{1 - \cos((\Delta_S(a,b) - \gamma)t)}{(\Delta_S(a,b) - \gamma)^2} \nonumber \\
    &+ e^{-\beta \gamma} \frac{1 - \cos((\Delta_S(a,b) + \gamma)t)}{(\Delta_S(a,b) + \gamma)^2}
\end{align}


\newpage
\noindent\makebox[\linewidth]{\rule{\textwidth}{0.4pt}}
\noindent\makebox[\linewidth]{\rule{\textwidth}{0.4pt}}
We now investigate detailed balance to this order. We want to show that the distribution $e^{- \beta \lambda(i)} / \partfun (\beta)$ is a fixed point of the system and environment. We let the system and environment be in the thermal state of their respective systems at the same inverse temperature $\beta$. We consider the transition probability before taking the partial trace over the environment. 
We first show that for $\lambda(i) = \lambda(j)$ with $i \neq j$ that detailed balance is trivially satisfied
\begin{align}
    \frac{e^{-\beta \lambda(i)}}{\partfun(\beta)} \prob{\ketbra{i}{i} \to \ketbra{j}{j}} &= \frac{e^{-\beta \lambda(j)}}{\partfun(\beta)} \prob{\ketbra{j}{j} \to \ketbra{i}{i}} \\
    \bra{j} \int \Phi_G(\ketbra{i}{i}) dG \ket{j} &= \bra{i} \int \Phi_G(\ketbra{j}{j}) dG \ket{i} \\
    \delta_{ij} + \frac{\alpha^2 \sigma^2 t^2}{\dim + 1} + \bigo{\alpha^3} &= \delta_{ij} + \frac{\alpha^2 \sigma^2 t^2}{\dim + 1} + \bigo{\alpha^3},
\end{align}
which we see agrees up to order $\bigo{\alpha^2}$. We now investigate $\lambda(i) \neq \lambda(j)$:
\begin{align}
    \frac{e^{-\beta \lambda(i)}}{\partfun(\beta)} \prob{\ketbra{i}{i} \to \ketbra{j}{j}} &= \frac{e^{-\beta \lambda(j)}}{\partfun(\beta)} \prob{\ketbra{j}{j} \to \ketbra{i}{i}} \\
    e^{-\beta \lambda(i)} \parens{\delta_{ij} + \frac{2 \sigma^2 \alpha^2}{\dim + 1} \frac{1 - \cos(\Delta_{ij} t)}{\Delta_{ij}^2}} &= e^{-\beta \lambda(j)} \parens{\delta_{ij} + \frac{2 \sigma^2 \alpha^2}{\dim + 1} \frac{1 - \cos(\Delta_{ij} t)}{\Delta_{ij}^2}} + \bigo{\alpha^3}
\end{align}
\newpage


\subsection{Harmonic Oscillator Fixed Point}
Now consider the situation of a system and environment which are both harmonic oscillators of frequency $\omega$, $H_{sys} = \hbar \omega \sum_{i=0}^{N_{sys}} (0.5 + i) \ketbra{i}{i}$ and $H_{env} = \hbar \omega \sum_{i=0}^{N_{env}} (0.5 + i) \ketbra{i}{i}$, yeilding $H = \sum_{i,j} \hbar \omega (1 + i + j) \ketbra{i,j}{i,j}$. We set $\hbar = \omega = 1$ to avoid dimensionful quantities. 

Our hunch is that the second order correction of $\Phi(\rho)$ vanishes whenever $\rho$ is a thermal state of the joint system-environment hamiltonian, i.e. $\rho = \sum_{i,j} \ketbra{i}{i} \otimes \ketbra{j}{j}\rho_{i,j} = \frac{e^{-\beta (1 + i + j)}}{\sum_{e^{-\beta(1+  i + j)}}} \ketbra{i}{i} \otimes \ketbra{j}{j}$. Before we compute this second-order correction, note that the assumption of a thermal state input eliminates the sum over equal energy contributions in Eq. \eqref{eq:second_order_final}. Further, in the interest of simplifying the expression as much as possible, we will only consider the time-averaged contributions $\frac{1}{T} \int_0^T \Phi(\rho, t) dt$, this pretty much just eliminates the cosine term in Eq. \eqref{eq:second_order_final}.

First define $\rho(\gamma) := \frac{e^{-(\beta_E -\gamma) H_{sys}}}{\partfun_{sys}(\gamma)}$
Our main goal is to show $\norm{\Phi(\rho(\gamma)) - \rho(0)} \leq \norm{\rho(\gamma) - \rho(0)}$, in other words that a higher temperature state ($\beta_{sys} = \beta_{env} - \gamma$) gets closer to a lower temperature state after application of $\Phi$. In order to do this we first use our power series of $\Phi$ in terms of the coupling constant $\alpha$ and use the Schatten-1 norm:
\begin{align}
    \norm{\Phi(\rho(\gamma)) - \rho(0)}_1 &= \norm{\rho(\gamma) + T(\rho(\gamma)) + R(\rho(\gamma)) - \rho(0)}_1 \\
    &= \norm{ \sum_{i} \parens{ e^{-(\beta_{env} + \gamma) E_i} \partfun_{S}^{-1}(\gamma) + T_{i,i}(\rho(\gamma)) - e^{-\beta_{env} E_i} \partfun_{S}^{-1}(\gamma) }\ketbra{i}{i}} + \bigo{\alpha^{4}} \\
    &= \sum_i \abs{ e^{-(\beta_{env} + \gamma) E_i} \partfun_{S}^{-1}(\gamma) + T_{i,i}(\rho(\gamma)) - e^{-\beta_{env} E_i} \partfun_{S}^{-1}(\gamma) } + \bigo{\alpha^4},
\end{align}
where we can avoid using the triangle inequality as our operator is already diagonalized. \todo{Does the $\bigo{\alpha^4}$ term not pick up dimensional factors from the schatten-1 norm?} Now we have to use a linearization of the operator $\rho(\gamma)$ as well as the second order corrections $T_{i,i}$.
\begin{align}
    \rho(\gamma) = \frac{e^{-\beta_{env} H_{sys}}}{\partfun_{sys}(\gamma=0)} + \gamma \frac{\partial}{\partial \gamma} \frac{e^{-\beta_{env} H_{sys} + \gamma H_{sys}}}{\partfun_{sys}(\gamma)} \bigg|_{\gamma=0} + \bigo{\gamma^2}.
\end{align}
The derivative is easy to compute (I think I wrote this down on the overleaf) 
\begin{equation}
    \frac{\partial}{\partial \gamma} \frac{e^{-\beta_{env} H_{sys} + \gamma H_{sys}}}{\partfun_{sys}(\gamma)} = e^{-\beta_{env}}
\end{equation}




%%%%%%%%%%%%%%%%%%%%%%%%%%%%%%%%%%%%%%%%%%%%%%%%%%%%%%%%%%%%%%%%%%%%%%%%%%%%%%%%%%%%%%%%%%%%%%%%%%%%%%%%%%%%%%%%%%%%%%%%%%%%%%%%%%%%%%%%%%%%%%%%
%%%%%%%%%%%%%%%%%%%%%%%%%%%%%%%%%%%%%%%%%%%%%%%%%%%%%%%%%%%%%%%%%%%%%%%%%%%%%%%%%%%%%%%%%%%%%%%%%%%%%%%%%%%%%%%%%%%%%%%%%%%%%%%%%%%%%%%%%%%%%%%%
%%%%%%%%%%%%%%%%%%%%%%%%%%%%%%%%%%%%%%%%%%%%%%%%%%%%%%%%%%%%%%%%%%%%%%%%%%%%%%%%%%%%%%%%%%%%%%%%%%%%%%%%%%%%%%%%%%%%%%%%%%%%%%%%%%%%%%%%%%%%%%%%
\section{Energy and Entropic concerns}
One requirement that our construction should satisfy is that a system in a thermal state should be invariant under contact with an environment at the same temperature. Formally, we would like that $\Phi \parens{\frac{e^{-\beta H_{sys}}}{\partfun_{sys}} \otimes \frac{e^{- \beta H_{env}}}{\partfun_{env}}} = \frac{e^{-\beta H_{sys}}}{\partfun_{sys}}$. Exact equality may be too strong to enforce, so we instead consider that the energy of the system remains unchanged after coupling with the environment. 

\begin{equation}
    \trace{H_{sys} \frac{e^{-\beta H_{sys}}}{\partfun_{sys}}} = \trace{H_{sys} \Phi \parens{\frac{e^{-\beta H_{sys}}}{\partfun_{sys}} \otimes \frac{e^{- \beta H_{env}}}{\partfun_{env}}}}.
\end{equation}
Now we look at the RHS in detail
\begin{align}
    \trace{H_{sys} \Phi \parens{\rho} } &= \trace{ H_{sys} \partrace{env}{\int e^{i \widetilde{H} t} \rho e^{-i \widetilde{H} t} dG}} \\
    &= \int \trace{H_{sys} \otimes \openone_{env} e^{i \widetilde{H} t} \rho e^{-i \widetilde{H} t} } dG \\
    &=  \trace{\int e^{- i (H + \alpha G) t} H_{sys} \otimes \openone_{env} e^{+ i (H + \alpha G) t} dG ~ \rho } 
\end{align}


%%%%%%%%%%%%%%%%%%%%%%%%%%%%%%%%%%%%%%%%%%%%%%%%%%%%%%%%%%%%%%%%%%%%%%%%%%%%%%%%%%%%%%%%%%%%%%%%%%%%%%%%%%%%%%%%%%%%%%%%%%%%%%%%%%%%%%%%%%%%%%%%
%%%%%%%%%%%%%%%%%%%%%%%%%%%%%%%%%%%%%%%%%%%%%%%%%%%%%%%%%%%%%%%%%%%%%%%%%%%%%%%%%%%%%%%%%%%%%%%%%%%%%%%%%%%%%%%%%%%%%%%%%%%%%%%%%%%%%%%%%%%%%%%%
%%%%%%%%%%%%%%%%%%%%%%%%%%%%%%%%%%%%%%%%%%%%%%%%%%%%%%%%%%%%%%%%%%%%%%%%%%%%%%%%%%%%%%%%%%%%%%%%%%%%%%%%%%%%%%%%%%%%%%%%%%%%%%%%%%%%%%%%%%%%%%%%
\section{Numerics}



%%%%%%%%%%%%%%%%%%%%%%%%%%%%%%%%%%%%%%%%%%%%%%%%%%%%%%%%%%%%%%%%%%%%%%%%%%%%%%%%%%%%%%%%%%%%%%%%%%%%%%%%%%%%%%%%%%%%%%%%%%%%%%%%%%%%%%%%%%%%%%%%
%%%%%%%%%%%%%%%%%%%%%%%%%%%%%%%%%%%%%%%%%%%%%%%%%%%%%%%%%%%%%%%%%%%%%%%%%%%%%%%%%%%%%%%%%%%%%%%%%%%%%%%%%%%%%%%%%%%%%%%%%%%%%%%%%%%%%%%%%%%%%%%%
%%%%%%%%%%%%%%%%%%%%%%%%%%%%%%%%%%%%%%%%%%%%%%%%%%%%%%%%%%%%%%%%%%%%%%%%%%%%%%%%%%%%%%%%%%%%%%%%%%%%%%%%%%%%%%%%%%%%%%%%%%%%%%%%%%%%%%%%%%%%%%%%
\section{Approximating $e^{i (H + \alpha G) t}$ evolution}


\section{Leftovers}
%%%%%%%%%%%%%%%%%%%%%%%%%%%%%%%%%%%%%%%%%%%%%%%%%%%%%%%%%%%%%%%%%%%%%%%%%%%%%%%%%%%%%%%%%%%%%%%%%%%%%%%%%%%%%%%%%%%%%%%%%%%%%%%%%%%%%%%%%%%%%%%%
%%%%%%%%%%%%%%%%%%%%%%%%%%%%%%%%%%%%%%%%%%%%%%%%%%%%%%%%%%%%%%%%%%%%%%%%%%%%%%%%%%%%%%%%%%%%%%%%%%%%%%%%%%%%%%%%%%%%%%%%%%%%%%%%%%%%%%%%%%%%%%%%



%%%%%%%%%%%%%%%%%%%%%%%%%%%%%%%%%%%%%%%%%%%%%%%%%%%%%%%%%%%%%%%%%%%%%%%%%%%%%%%%%%%%%%%%%%%%%%%%%%%%%%%%%%%%%%%%%%%%%%%%%%%%%%%%%%%%%%%%%%%%%%%%
%%%%%%%%%%%%%%%%%%%%%%%%%%%%%%%%%%%%%%%%%%%%%%%%%%%%%%%%%%%%%%%%%%%%%%%%%%%%%%%%%%%%%%%%%%%%%%%%%%%%%%%%%%%%%%%%%%%%%%%%%%%%%%%%%%%%%%%%%%%%%%%%

%%%%%%%%%%%%%%%%%%%%%%%%%%%%%%%%%%%%%%%%%%%%%%%%%%%%%%%%%%%%%%%%%%%%%%%%%%%%%%%%%%%%%%%%%%%%%%%%%%%%%%%%%%%%%%%%%%%%%%%%%%%%%%%%%%%%%%%%%%%%%%%%
%%%%%%%%%%%%%%%%%%%%%%%%%%%%%%%%%%%%%%%%%%%%%%%%%%%%%%%%%%%%%%%%%%%%%%%%%%%%%%%%%%%%%%%%%%%%%%%%%%%%%%%%%%%%%%%%%%%%%%%%%%%%%%%%%%%%%%%%%%%%%%%%
%%%%%%%%%%%%%%%%%%%%%%%%%%%%%%%%%%%%%%%%%%%%%%%%%%%%%%%%%%%%%%%%%%%%%%%%%%%%%%%%%%%%%%%%%%%%%%%%%%%%%%%%%%%%%%%%%%%%%%%%%%%%%%%%%%%%%%%%%%%%%%%%
\section{Approximating $e^{i (H + \alpha G) t}$ evolution}

\bibliographystyle{unsrt}
\bibliography{bib}

\appendix
\section{Perturbation Theory Attempt}
The goal is to use perturbation theory to write the output of the channel in terms of powers of the coupling strength $\alpha$. It remains to be seen if first order perturbation theory will be enough. We also are specifically interested when $\rho$ is a product state of thermal states, in other words $\rho_{PS} = \frac{e^{-\beta_1 H_1}}{\partfun_1} \otimes \frac{e^{-\beta_2 H_2}}{\partfun_2}$. We also denote the action of our channel with respect to a particular choice of $G$ as $\Phi_G$, which combined with linearity of partial traces gives $\Phi = \int dG \Phi_G$. 
\begin{align}
    \Phi_G(\rho_{PS}) &= \partrace{2}{e^{+i \widetilde{H} t} \rho_{PS} e^{-i \widetilde{H} t}} \\
    &= \sum_{k} \openone \otimes v_k^* \parens{\sum_{i, j} e^{i \widetilde{\eta}_{i, j} t} \widetilde{\Pi}_{i,j}} \rho_{PS} \parens{ \sum_{m, n} e^{i \widetilde{\eta}_{m, n} t} \widetilde{\Pi}_{m,n} }
\end{align}
Our next goal is to analyze $e^{i \widetilde{\eta}_{i,j} t} \widetilde{\Pi}_{i,j}$ using time-independent perturbation theory. Our goal is to write $e^{i \widetilde{\eta}_{i,j} t}$ as a power series in terms of $\alpha$, which is given by perturbation theory for $\widetilde{H} = H + \alpha G$. We first write the perturbation series for $\widetilde{\eta}$ as 
\begin{equation}
    \widetilde{\eta}_{i,j} = \eta_{i,j} + \alpha (u_i \otimes v_j)^* G (u_i \otimes v_j) + \alpha^2 \sum_{(i',j') \neq (i,j)}  \frac{\abs{(u_i \otimes v_j)^* G (u_{i'} \otimes v_{j'})}^{2}}{\eta_{i,j} - \eta_{i',j'}} + \bigo{\alpha^3}.
\end{equation}
To simplify the notation, we introduce the indexing $G(i,j,k,l) := (u_i \otimes v_j)^* G (u_k \otimes v_l)$. $G(i,j)$ without a second pair of indices denotes the diagonal matrix element $(u_i \otimes v_j)^* G (u_i \otimes v_j)$. The Taylor's Series for $e^{i \widetilde{\eta}_{i,j} t}$ is
\begin{align}
    e^{i \widetilde{\eta}_{i,j} t} &= e^{i \eta_{i,j} t} \\
    &\quad + \frac{\partial \widetilde{\eta}_{i,j}}{\partial \alpha} e^{i \widetilde{\eta}_{i,j}t} \bigg|_{\alpha = 0} i \alpha t \\
    &\quad + \frac{ i\alpha^2 t}{2!} \frac{\partial^2 \widetilde{\eta}_{i,j}}{\partial \alpha^2} e^{i \widetilde{\eta}_{i,j}t}\bigg|_{\alpha=0} \\
    &\quad - \frac{\alpha^2 t^2}{2!} \parens{\frac{\partial \widetilde{\eta}_{i,j}}{\partial \alpha}}^2 e^{i \widetilde{\eta}_{i,j} t} \bigg|_{\alpha=0} + \bigo{\alpha^3} .
\end{align}
These are rather straightforward to compute and plug in, yielding
\begin{equation}
    e^{i \widetilde{\eta}_{ij} t} = e^{i \eta_{ij} t}\parens{1 + i \alpha t G(i,j) - \frac{\alpha^2 t^2}{2} G(i,j)^2 + i \frac{\alpha^2 t}{2} \sum_{(i',j') \neq (i,j)}\frac{\abs{G(i',j',i,j)}^2}{\eta_{ij} - \eta_{i' j'}} + \bigo{\alpha^3}}.
\end{equation}
This is sloppy, need to be able to bound the $\bigo{\alpha^3}$ terms better. Also note that $G(i,j)$ is a real Gaussian variable, the realness is due to the Hermiticity constraint on $G$. 

The other quantity we need to compute a perturbation series for is $\widetilde{\Pi}_{ij}$. We first need to compute the perturbation series for the eigenvectors of $H$, which we do to second order
\begin{align}
    \widetilde{u_i \otimes v_j} &= u_i \otimes v_j \\
    &\quad + \alpha \sum_{(i',j') \neq (i,j)} u_{i'} \otimes v_{j'} \parens{  \frac{G(i,j,i',j')}{\eta_{i,j} - \eta_{i',j'}} } \\
    &\quad + \alpha^2 \sum_{(i',j') \neq (i,j)} u_{i'} \otimes v_{j'} \parens{ \sum_{(i'',j'') \neq (i,j)} \frac{G(i,j,i'',j'') G(i'',j'',i',j')}{(\eta_{i,j} - \eta_{i'',j''})(\eta_{i,j} - \eta_{i'',j''})} -  \frac{G(i,j,i,j) G(i,j,i',j')}{(\eta_{i,j} - \eta_{i',j'})^2}} \\
    &\quad + \bigo{\alpha^3}.
\end{align}
Now we need to compute the perturbed eigenspace projectors
\begin{align}
    \widetilde{\Pi}_{i,j} &= \Pi_{i,j} \\
    & \quad + \alpha \sum_{(i',j') \neq (i,j)} u_{i'} u_i^* \otimes v_{j'} v_j^* \parens{  \frac{G(i,j,i',j')}{\eta_{i,j} - \eta_{i',j'}} } + \sum_{(i',j') \neq (i,j)} u_i u_{i'}^* \otimes v_j v_{j'}^* \parens{  \frac{G(i,j,i',j')^*}{\eta_{i,j} - \eta_{i',j'}} } \\
    & \quad + \alpha^2 \sum_{(i',j') \neq (i,j)} u_{i'} u_i^* \otimes v_{j'} v_j^* \parens{ \sum_{(i'',j'') \neq (i,j)} \frac{G(i,j,i'',j'') G(i'',j'',i',j')}{(\eta_{i,j} - \eta_{i'',j''})(\eta_{i,j} - \eta_{i'',j''})} -  \frac{G(i,j,i,j) G(i,j,i',j')}{(\eta_{i,j} - \eta_{i',j'})^2}} \\
    & \quad + \alpha^2 \sum_{(i',j') \neq (i,j)} u_i u_{i'}^* \otimes v_j v_{j'}^* \parens{ \sum_{(i'',j'') \neq (i,j)} \frac{G(i,j,i'',j'')^* G(i'',j'',i',j')^*}{(\eta_{i,j} - \eta_{i'',j''})(\eta_{i,j} - \eta_{i'',j''})} -  \frac{G(i,j,i,j)^* G(i,j,i',j')^*}{(\eta_{i,j} - \eta_{i',j'})^2}} \\
    & \quad + \alpha^2 \sum_{(i',j') \neq (i,j)} \sum_{(i'',j'') \neq (i,j)} u_{i'}u_{i''}^* \otimes v_{j'} v_{j''}^* \frac{G(i,j,i',j') G(i,j,i'',j'')^*}{(\eta_{i,j} - \eta_{i',j'})(\eta_{i,j} - \eta_{i'',j''})} \\
    & \quad + \bigo{\alpha^3}.
\end{align}

This allows us to write the output of the channel $\Phi$ as a power series in $\alpha$. The expressions can become quite cumbersome, so we introduce the following notation:
\begin{align}
    M_{i,j,i',j'} &:=  u_{i'} u_i^* \otimes v_{j'} v_j^* \parens{  \frac{G(i,j,i',j')}{\eta_{i,j} - \eta_{i',j'}} } +  u_i u_{i'}^* \otimes v_j v_{j'}^* \parens{  \frac{G(i,j,i',j')^*}{\eta_{i,j} - \eta_{i',j'}} } \\
    N_{i,j,i',j'} &:= u_{i'} u_i^* \otimes v_{j'} v_j^* \parens{ \sum_{(i'',j'') \neq (i,j)} \frac{G(i,j,i'',j'') G(i'',j'',i',j')}{(\eta_{i,j} - \eta_{i'',j''})(\eta_{i,j} - \eta_{i'',j''})} -  \frac{G(i,j,i,j) G(i,j,i',j')}{(\eta_{i,j} - \eta_{i',j'})^2}} \\
    & \quad + \sum_{(i'',j'') \neq (i,j)} u_{i'}u_{i''}^* \otimes v_{j'} v_{j''}^* \frac{G(i,j,i',j') G(i,j,i'',j'')^*}{(\eta_{i,j} - \eta_{i',j'})(\eta_{i,j} - \eta_{i'',j''})}
\end{align}
which captures the first and second order corrections to the eigenspace projectors. We can then simplify
\begin{align}
    \widetilde{\Pi}_{i,j} &= \Pi_{i,j} + \alpha \sum_{(i',j') \neq (i,j)} M_{i,j,i',j'} + \alpha^2 \sum_{(i',j') \neq (i,j)} N_{i,j,i',j'} + \bigo{\alpha^3}
\end{align}
The second order corrections have 10 possible sources for terms, however many are hermitian conjugates of priors, so if herm. conj. appears it refers to the hermitian conjugate of the entire previous line.
\newpage
\begin{equation}
    e^{+i(H + \alpha G)t} \rho e^{-i (H + \alpha G) t}
\end{equation}
\begin{align}
\alpha^0 :& e^{+i H t} \rho e^{-i H t} \\
 \alpha^1 :& \sum_{i,j} e^{i\eta_{i,j}t} \parens{i t G(i,j) \Pi_{i,j} + \sum_{(i',j') \neq (i,j)} M_{i,j,i',j'}} \rho e^{- i H t} \\
 &+ \text{herm. conj.} \\
 \alpha^2 :& \sum_{i,j} i t G(i,j) e^{i \eta_{i,j}} \Pi_{i,j} \rho \sum_{k,l} (-i t) G(k,l)^* e^{-i \eta_{k,l}t} \Pi_{k,l} \\
 &+ \sum_{i,j} e^{i \eta_{i,j} t} \sum_{(i',j') \neq (i,j)} M_{i,j,i',j'} \rho \sum_{k,l} (-i t) G(k,l) e^{-i \eta_{k,l} t} \Pi_{k,l} \\
 &+ \text{herm. conj.} \\
 &+ \sum_{i,j} e^{i \eta_{i,j} t} \sum_{(i',j') \neq (i,j)} M_{i,j,i',j'} \rho \sum_{k,l} e^{-i \eta_{k,l} t} \sum_{(k',l') \neq (k,l)} M_{k,l,k',l'}\\
 &+  \sum_{i,j} e^{i \eta_{i,j} t} \parens{- \frac{t^2 G(i,j)^2}{2} + i \frac{t}{2} \sum_{(i',j') \neq (i,j)} \frac{\abs{G(i,j,i',j')}^2}{\eta_{i,j} - \eta_{i',j'}} } \rho e^{- i H t} \\
 &+ \text{herm. conj.} \\
 &+ \sum_{i,j} e^{i \eta_{i,j} t} \sum_{(i',j') \neq (i,j)} N_{i,j,i',j'} \rho e^{-i H t} \\
 &+ \text{herm. conj.} \\
 &+ \sum_{i,j} i t G(i,j) e^{i \eta_{i,j} t} \sum_{(i',j') \neq (i,j)} M_{i,j,i',j'} \rho e^{-i H t} \\
 &+ \text{herm. conj.}.
\end{align}

Two potential issues. 1) How can we normalize the output state? This may depend on how we approach the second. 2) How do we bound the remaining terms in the perturbation series? Are there any dumb bounds I can work out? 

\end{document}
