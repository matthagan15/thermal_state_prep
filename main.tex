\documentclass{article}
\usepackage[utf8]{inputenc}

\usepackage{amsmath,amsthm, amssymb}
\usepackage[margin=1in]{geometry}
\usepackage{mathtools}
\usepackage{dsfont}
\usepackage{xcolor}
\usepackage{algorithm,algpseudocode}
\usepackage{todonotes}
\usepackage{nicefrac}
\usepackage{mathrsfs}
\usepackage{tikz}

%%%%%%%%    THEOREM DEFINITIONS AND RESTATABLE
\newtheorem{theorem}{Theorem}
\newtheorem{lemma}[theorem]{Lemma}

\usepackage{todonotes}

\newcommand{\matt}[1]{\todo[color=red!50, prepend, caption={Matt}, tickmarkheight=0.25cm]{#1}}
\newcommand{\note}[1]{\emph{Note: #1}}
\newcommand{\conjecture}[1]{ \noindent\emph{\textbf{Conjecture:}} \emph{ #1 }}




%%%%%%%%    NOTATION DEFINITIONS FOR EASIER WRITING
\newcommand{\ket}[1]{|#1\rangle}
\newcommand{\bra}[1]{\langle #1|}
\newcommand{\braket}[2]{\langle #1|#2\rangle}
\newcommand{\ketbra}[2]{| #1\rangle\! \langle #2|}
\newcommand{\parens}[1]{\left( #1 \right)}
\newcommand{\brackets}[1]{\left[ #1 \right]}
\newcommand{\abs}[1]{\left| #1 \right|}
\newcommand{\norm}[1]{\left| \left| #1 \right| \right|}
\newcommand{\diamondnorm}[1]{\left| \left| #1 \right| \right|_\diamond}
\newcommand{\anglebrackets}[1]{\left< #1 \right>}
\newcommand{\overlap}[2]{\anglebrackets{#1 , #2 }}
\newcommand{\set}[1]{\left\{ #1 \right\}}
\newcommand{\ceil}[1]{\left\lceil #1 \right\rceil}
\newcommand{\openone}{\mathds{1}}
\newcommand{\expect}[1]{\mathbb{E}\brackets{#1}}
\newcommand{\prob}[1]{\text{Pr}\left[ #1 \right]}
\newcommand{\bigo}[1]{\mathcal{O}\left( #1 \right)}
\newcommand{\bigotilde}[1]{\widetilde{\mathcal{O}} \left( #1 \right)}
\newcommand{\ts}{\textsuperscript}

\DeclareMathOperator{\Tr}{Tr}
\newcommand{\trace}[1]{\Tr \brackets{ #1 }}
\newcommand{\partrace}[2]{\Tr_{#1} \brackets{ #2 }}
\newcommand{\complex}{\mathbb{C}}

%%%%% COMMONLY USED OBJECTS
\newcommand{\hilb}{\mathscr{H}}
\newcommand{\partfun}{\mathcal{Z}}
\newcommand{\gue}{\rm GUE}
\DeclareMathOperator{\hermMathOp}{Herm}
\newcommand{\herm}[1]{\hermMathOp\parens{#1}}


\title{Thermal State Prep}
\author{Nathan Wiebe, Matthew Hagan}
\date{May 2022}

\begin{document}

\maketitle

\section{Introduction}
The computational task of preparing states of the form $\frac{e^{-\beta H}}{\partfun}$ in a usable approximation on a quantum computer is both incredibly useful and difficult. The closely related problem of estimating the free energy, defined as $F = -(1/\beta) \log \partfun$, up to additive error is known to be QMA-Hard for 2-local Hamiltonians \cite{bravyi_complexity_2021}. Intuitively, finding the ground state of a $k$-local Hamiltonian is known to be QMA-Hard \cite{} and the thermal state $e^{-\beta H} \partfun^{-1}$ can be constructed with arbitrarily high overlap with the ground state as $\beta \to \infty$ or the temperature approaches 0. 

Preparing thermal states for arbitrary Hamiltonians for arbitrary temperatures is $QMA$-Hard, however thermal states are incredibly useful inputs whenever low energy states are needed, such as when training Boltzmann machines \cite{}, preparing ground states for quantum error correcting codes \cite{}, or starting states for chemistry simulations \cite{}. 

Want to include here the main methods for preparing thermal states that currently exist, mainly the Poulin and Wocjan paper and Quantum-Quantum Metropolis Hastings. Where should more niche/advanced methods go?

\section{Preliminaries}
We denote the Hilbert space of the system as $\hilb_{sys}$ and the environment as $\hilb_{env}$. The algorithm we propose for preparing thermal states for a Hamiltonian $H_{sys} \in \herm{\hilb_{sys}}$ involves direct time simulation of the system coupled to an environment governed by $H_{env} \in \herm{\hilb_{env}}$. We denote the interaction $H_{int} \in \herm{\hilb_{sys} \otimes \hilb_{env}}$. At a broad level, the algorithm starts from some initial state (what is the distribution over inputs?) we denote as $\rho$, undergoes time evolution with $H = H_{sys} + H_{env} + H_{int}$ for some time $t$, after which we trace out the environment leaving
\begin{equation}
    \mathcal{P}(\rho) = \partrace{Env}{e^{i H t} \rho e^{-i H t}}.
\end{equation}
The goal of this paper is to provide subsets of possible systems, environments, and interactions that can lead to efficient thermalization. 

Can thermal state preparation be written as a SDP? Brandao and Svore showed that thermal states can be used as a resource for a quantum Aurora-Kale SDP solver, but what about the inverse problem? We know that the thermal state satisfies the following:
\begin{align}
    \rho_G &= \arg \max_{\rho} \trace{- \rho \log \rho} \quad \text{such that} \\
    \trace{\rho H} &= E ?
\end{align}


\bibliographystyle{unsrt}
\bibliography{bib}

\end{document}
