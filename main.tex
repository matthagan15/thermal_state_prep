\documentclass{article}
\usepackage[utf8]{inputenc}

\usepackage{amsmath,amsthm, amssymb}
\usepackage[margin=3cm]{geometry}
\usepackage{mathtools}
\usepackage{dsfont}
\usepackage{xcolor}
\usepackage{algorithm,algpseudocode}
\usepackage{todonotes}
\usepackage{nicefrac}
\usepackage{mathrsfs}
\usepackage{tikz}
\usepackage{thm-restate}


%%%%%%%%    THEOREM DEFINITIONS AND RESTATABLE
\newtheorem{theorem}{Theorem}
\newtheorem{lemma}[theorem]{Lemma}

\usepackage{todonotes}

\newcommand{\matt}[1]{\todo[color=red!50, prepend, caption={Matt}, tickmarkheight=0.25cm]{#1}}
\newcommand{\note}[1]{\emph{Note: #1}}
\newcommand{\conjecture}[1]{ \noindent\emph{\textbf{Conjecture:}} \emph{ #1 }}




%%%%%%%%    NOTATION DEFINITIONS FOR EASIER WRITING
\newcommand{\ket}[1]{|#1\rangle}
\newcommand{\bra}[1]{\langle #1|}
\newcommand{\braket}[2]{\langle #1|#2\rangle}
\newcommand{\ketbra}[2]{| #1\rangle\! \langle #2|}
\newcommand{\parens}[1]{\left( #1 \right)}
\newcommand{\brackets}[1]{\left[ #1 \right]}
\newcommand{\abs}[1]{\left| #1 \right|}
\newcommand{\norm}[1]{\left| \left| #1 \right| \right|}
\newcommand{\diamondnorm}[1]{\left| \left| #1 \right| \right|_\diamond}
\newcommand{\anglebrackets}[1]{\left< #1 \right>}
\newcommand{\overlap}[2]{\anglebrackets{#1 , #2 }}
\newcommand{\set}[1]{\left\{ #1 \right\}}
\newcommand{\ceil}[1]{\left\lceil #1 \right\rceil}
\newcommand{\openone}{\mathds{1}}
\newcommand{\expect}[1]{\mathbb{E}\brackets{#1}}
\newcommand{\prob}[1]{\text{Pr}\left[ #1 \right]}
\newcommand{\bigo}[1]{O\left( #1 \right)}
\newcommand{\bigotilde}[1]{\widetilde{O} \left( #1 \right)}
\newcommand{\ts}{\textsuperscript}

\DeclareMathOperator{\Tr}{Tr}
\newcommand{\trace}[1]{\Tr \brackets{ #1 }}
\newcommand{\partrace}[2]{\Tr_{#1} \brackets{ #2 }}
\newcommand{\complex}{\mathbb{C}}

%%%%% COMMONLY USED OBJECTS
\newcommand{\hilb}{\mathcal{H}}
\newcommand{\partfun}{\mathcal{Z}}
\newcommand{\identity}{\mathds{1}}
\newcommand{\gue}{\rm GUE}
\DeclareMathOperator{\hermMathOp}{Herm}
\newcommand{\herm}[1]{\hermMathOp\parens{#1}}


\title{Thermal State Prep}
\author{Nathan Wiebe, Matthew Hagan}
\date{May 2022}

\begin{document}

\maketitle

%%%%%%%%%%%%%%%%%%%%%%%%%%%%%%%%%%%%%%%%%%%%%%%%%%%%%%%%%%%%%%%%%%%%%%%%%%%%%%%%%%%%%%%%%%%%%%%%%%%%%%%%%%%%%%%%%%%%%%%%%%%%%%%%%%%%%%%%%%%%%%%%
%%%%%%%%%%%%%%%%%%%%%%%%%%%%%%%%%%%%%%%%%%%%%%%%%%%%%%%%%%%%%%%%%%%%%%%%%%%%%%%%%%%%%%%%%%%%%%%%%%%%%%%%%%%%%%%%%%%%%%%%%%%%%%%%%%%%%%%%%%%%%%%%
%%%%%%%%%%%%%%%%%%%%%%%%%%%%%%%%%%%%%%%%%%%%%%%%%%%%%%%%%%%%%%%%%%%%%%%%%%%%%%%%%%%%%%%%%%%%%%%%%%%%%%%%%%%%%%%%%%%%%%%%%%%%%%%%%%%%%%%%%%%%%%%%
\section{Introduction}
Going to leave this blank for now. \cite{shiraishi_undecidability_2021}

%%%%%%%%%%%%%%%%%%%%%%%%%%%%%%%%%%%%%%%%%%%%%%%%%%%%%%%%%%%%%%%%%%%%%%%%%%%%%%%%%%%%%%%%%%%%%%%%%%%%%%%%%%%%%%%%%%%%%%%%%%%%%%%%%%%%%%%%%%%%%%%%
%%%%%%%%%%%%%%%%%%%%%%%%%%%%%%%%%%%%%%%%%%%%%%%%%%%%%%%%%%%%%%%%%%%%%%%%%%%%%%%%%%%%%%%%%%%%%%%%%%%%%%%%%%%%%%%%%%%%%%%%%%%%%%%%%%%%%%%%%%%%%%%%
%%%%%%%%%%%%%%%%%%%%%%%%%%%%%%%%%%%%%%%%%%%%%%%%%%%%%%%%%%%%%%%%%%%%%%%%%%%%%%%%%%%%%%%%%%%%%%%%%%%%%%%%%%%%%%%%%%%%%%%%%%%%%%%%%%%%%%%%%%%%%%%%
\section{Preliminaries}
We denote the Hilbert space of the system as $\hilb_{S}$ and the environment as $\hilb_{E}$, with the Hamiltonians governing each as $H_{S}$ and $H_{E}$. We will assume without loss of generality that the system's Hilbert space can be encoded with $n$ qubits, giving $\dim_S = 2^{n}$, and the environment's Hilbert space can be encoded with $m$ qubits giving $\dim_E = 2^{m}$. The Hamiltonian for the joint system on $\hilb_{S} \otimes \hilb_{E}$ is then $H = H_{S} \otimes \identity + \identity \otimes H_{E}$. The Hilbert space of the combined system and environment is of dimension $\dim = \dim_E \cdot \dim_S = 2^{n + m}$. 

We will primarily work in the eigenbasis for each Hamiltonian:
\begin{equation}
    H_{S} = \sum_{i = 0}^{2^n - 1} \lambda_S(i) \ketbra{s_i}{s_i} ~,~ H_{E} = \sum_{j=0}^{2^m - 1} \lambda_E(j) \ketbra{e_j}{e_j} ~,~ H = \sum_{i=0}^{2^n - 1} \sum_{j=0}^{2^m - 1} \lambda(i,j) (\ket{s_i} \otimes \ket{e_j})(\bra{s_i} \otimes \bra{e_j}),
\end{equation}
for convenience we will denote the tensor product of eigenvectors simply by their indices $\ket{i,j} \coloneqq \ket{s_i} \otimes \ket{e_j}$. For convenience we define $\lambda(i,j) \coloneqq \lambda_S(i) + \lambda_E(j)$.

We study the effects of simulating the time dynamics of the system-environment space with randomized interactions. Our goal is to produce a channel that can reduce the entropy of a system thermal state by utilizing easily prepared thermal states of the environment. We consider using the Haar measure over eigenvectors for the randomized interaction and i.i.d eigenvalues from a zero mean $\sigma^2$ variance distribution. 

For input states we will typically assume thermal states of the form $\rho_S(\beta) = \frac{e^{-\beta H_S}}{\partfun_S}$, where $\partfun_S = \trace{e^{-\beta H_S}}$, where the inverse temperature $\beta$ of the partition function will typically be assumed but written explicitly if need be. We will assume environment states of the form $\rho_E(\beta) = \frac{e^{-\beta H_E}}{\partfun_E}$ and we will typically denote the tensor product of the system and environment states as $\rho(\beta_S, \beta_E) = \rho_S(\beta_S) \otimes \rho_E(\beta_E)$. The inputs $\beta_S, \beta_E$ will typically be surpressed in most cases as well.


Overall one application of our channel is represented as
\begin{equation}
    \Phi(\rho) := \int \partrace{\hilb_E}{e^{+i(H + \alpha G)t} \rho e^{-i(H + \alpha G) t}} dG.
\end{equation}
For simplicity we will denote the time evolution channel for a specific random interaction $G$ as
\begin{equation}
    \Phi_G(\rho) := e^{+i (H+ \alpha G) t} \rho e^{-i (H + \alpha G) t}. \label{eq:phi_g_definition}
\end{equation}
Clearly then $\Phi = \int \partrace{env}{\Phi_G} dG$. We use $G$ to denote the randomized interaction term, where $G = U_G D U_G^\dagger$. The measure we choose for the eigenbasis of $G$ is $U_G \sim Haar$ and the eigenvalues are i.i.d with mean 0 and variance $\sigma^2$. This gives the overall measure decomposition $dG = dD ~ dU_G$. 

\begin{restatable}{lemma}{haar_two_moment} \label{lem:haar_two_moment}
    Let $U$ be a unitary matrix over $\dim$ dimensions that is distributed according to the Haar measure. Then the following average is
    \begin{align}
        \int \bra{i_1} U \ket{j_1} \bra{i_2} U \ket{j_2} \bra{k_1} U^\dagger \ket{l_1} ~ \bra{k_2} U^\dagger \ket{l_2} dU =& ~\frac{1}{\dim^2 - 1} \parens{\delta_{i_1, l_1} \delta_{j_1, k_1} \delta_{i_2, l_2} \delta_{j_2, k_2} + \delta_{i_1, l_2} \delta_{j_1, k_2} \delta_{i_2, l_1} \delta_{j_2, k_1}} \nonumber \\
        &- \frac{1}{\dim(\dim^2 - 1)} \parens{\delta_{i_1, l_2} \delta_{j_1, k_1} \delta_{i_2, l_1} \delta_{j_2, k_2} + \delta_{i_1, l_1} \delta_{j_1, k_2} \delta_{i_2, l_2} \delta_{j_2, k_1}}. \label{eq:haar_two_moment_integral}
    \end{align}
    \end{restatable}


\begin{align}
    \trace{\ketbra{j}{j} \Phi(\ketbra{i}{i})}&= \bra{j} \Phi(\ketbra{i}{i}) \ket{j} \\
    &= \int \bra{j} \partrace{\hilb_E}{e^{-i(H + \alpha G)t} \ketbra{i}{i} \otimes \rho_E e^{+i(H + \alpha G)t}} \ket{j} \\
    &= \int \sum_{k} \bra{j,k} e^{-i (H + \alpha G)t} \ketbra{i,0}{i,0} e^{+i(H + \alpha G)t} \ket{j,k}
\end{align}

%%%%%%%%%%%%%%%%%%%%%%%%%%%%%%%%%%%%%%%%%%%%%%%%%%%%%%%%%%%%%%%%%%%%%%%%%%%%%%%%%%%%%%%%%%%%%%%%%%%%%%%%%%%%%%%%%%%%%%%%%%%%%%%%%%%%%%%%%%%%%%%%
%%%%%%%%%%%%%%%%%%%%%%%%%%%%%%%%%%%%%%%%%%%%%%%%%%%%%%%%%%%%%%%%%%%%%%%%%%%%%%%%%%%%%%%%%%%%%%%%%%%%%%%%%%%%%%%%%%%%%%%%%%%%%%%%%%%%%%%%%%%%%%%%
%%%%%%%%%%%%%%%%%%%%%%%%%%%%%%%%%%%%%%%%%%%%%%%%%%%%%%%%%%%%%%%%%%%%%%%%%%%%%%%%%%%%%%%%%%%%%%%%%%%%%%%%%%%%%%%%%%%%%%%%%%%%%%%%%%%%%%%%%%%%%%%%
\section{Fixed Points}

\begin{lemma}[First Order $\alpha$ Correction to $\Phi$]
   Given a randomized environment interaction channel $\Phi$ with coupling coefficient $\alpha$, the first order correction is
   \begin{equation}
        \frac{\partial}{\partial \alpha} \Phi(\rho_S) \bigg|_{\alpha = 0} = 0.
   \end{equation}
\end{lemma}
\begin{proof}
    We start by using linearity of derivatives, integration, and partial trace to pull the $\alpha$ derivative to act on $\Phi_G$ as
    \begin{align}
        \frac{\partial}{\partial \alpha} \Phi(\rho_S) \bigg|_{\alpha = 0} &= \frac{\partial}{\partial \alpha} \partrace{\mathcal{H}_E}{\int \Phi_G(\rho_s) dG} \bigg|_{\alpha = 0} \\
         &= \partrace{\mathcal{H}_E}{\int \frac{\partial}{\partial \alpha} \Phi_G(\rho_S) dG \bigg|_{\alpha = 0} } .
    \end{align}
    Now we use Eq. \eqref{eq:phi_g_definition} to compute the derivatives, we remind the reader that $\rho = \rho_S \otimes \rho_E$,
    \begin{align}
        \frac{\partial}{\partial \alpha} \Phi_G (\rho_S) &= \parens{\frac{\partial}{\partial \alpha} e^{+ i (H + \alpha G)t}} \rho e^{-i (H + \alpha G) t} + e^{+i (H + \alpha G)t} \rho \parens{\frac{\partial}{\partial \alpha} e^{- i (H + \alpha G)t}} \\
        &= \parens{\int_{0}^{1} e^{i s (H+\alpha G)t} (i t G) e^{i (1-s) (H+\alpha G)t} ds} \rho e^{-i(H+\alpha G)t} \nonumber \\
    &~ ~+ e^{i(H+\alpha G)t} \rho \parens{\int_{0}^1 e^{-i s (H+\alpha G) t} (- i t G) e^{-i (1-s) (H+\alpha G)t} ds}. \label{eq:first_order_alpha_derivative}
    \end{align}
    We can further simplify this by bringing in the evaluation of $\alpha = 0$ through the partial trace and integration, as they are uniformly convergent over $\alpha$ (is that the correct notion that allows us to switch orders?)
    \begin{align}
        \frac{\partial}{\partial \alpha} \Phi_G(\rho_S) \bigg|_{\alpha = 0} &= i t \int_0^1 e^{i s H t} G e^{-i s H t} ds e^{i H t} \rho e^{-i H t} - i t e^{+i H t} \rho \int_0^1 e^{-is H t} G e^{-i(1-s) Ht} ds \\
        &= i t \parens{\int_0^1 G(s t) ds} \rho(t) - it \rho(t) \parens{\int_0^1 G(s t) ds} \\
        &= i t \int_0^1 [G(s t), \rho(t)] ds,
    \end{align}
    where we have used the Heisenberg picture $\rho(t) = e^{i H t} \rho e^{-i H t}$ to simplify the notation.

    This expression is now amenable to computing the correction to the total channel. We do so by performing the integration over the randomized interactions. We take advantage of the structure of our interaction measure, that is $G = U_G D U_G^\dagger$ and $dG = dU_G dD$, which allows us to write
    \begin{align}
        \int \frac{\partial}{\partial \alpha} \Phi_G(\rho_S) \bigg|_{\alpha = 0} dG &= it \int \int_0^1 \left[ e^{i H s t} G e^{-i H s t}, \rho(t) \right] ds ~dG \\
        &= it \int_0^1 \left[ e^{i H s t} \parens{\int \int U_G D U_G^\dagger ~dU_G ~ dD} e^{-i H s t}, \rho(t)  \right] ds \\
        &= i t \int_0^1 \left[ e^{i H s t} \parens{\int U_G \parens{\int D ~ dD} ~ U_G^\dagger dU_G } e^{-i H s t}, \rho(t) \right] ds \\
        &= 0.
    \end{align}
    This last step relies on the use of random eigenvalues with mean 0, implying $\int D ~dD = 0$ which shows that $\frac{\partial}{\partial \alpha} \Phi(\rho_S) \big|_{\alpha = 0 } = 0$.
\end{proof}

\begin{lemma}[Second Order Correction]
    Given a system Hamiltonian $H_{S}$, an environment Hamiltonian $H_{E}$, a simulation time $t$, and coupling coefficient $\alpha$, let $\Phi(\rho_S)$ denote the fixed interaction system-environment channel 
    \begin{equation}
        \Phi_G(\rho_S) = e^{+i (H + \alpha G)t} (\rho_S \otimes \rho_E) e^{-i (H + \alpha G)t},
    \end{equation}
    where $H = H_S \otimes \identity + \identity \otimes H_E$. For simplicity we assume no degeneracies in the total Hamiltonian $H$, and therefore in $H_S$ and $G$ is our randomized interaction. Further we let $\rho \coloneqq \rho_S \otimes \rho_E = \sum_{i,j} B(i,j) \ketbra{i,j}{i,j}$. The long time average and random interaction average second order correction is given by the following
    \begin{align}
        \lim_{T \to \infty} \frac{1}{T} \int_0^T \int \frac{\partial^2}{\partial \alpha^2} \Phi_G(\rho_S)\bigg|_{\alpha = 0} dG ~dt &=  \frac{4 \sigma^2}{\dim^2 - 1} \sum_{i,j} \ketbra{i,j}{i,j} \sum_{(k,l) \neq (i,j)} \frac{B(i,j) - B(k,l)}{(\lambda(i,j) - \lambda(k,l))^2}.
    \end{align}
\end{lemma}
\begin{proof}
We start from the expression for the first derivative of the channel $\frac{\partial}{\partial \alpha} \Phi_G(\rho_S)$ given by Eq. \eqref{eq:first_order_alpha_derivative}. To take the second derivative there are six factors involving $\alpha$, so we will end up with six terms. We repeat Eq. \eqref{eq:first_order_alpha_derivative} below, add a derivative, and label each factor containing an $\alpha$ for easier computation
\begin{align}
    \frac{\partial^2}{\partial \alpha^2} \Phi_G(\rho_S) =& \frac{\partial}{\partial \alpha} \parens{\int_{0}^{1} \underset{\substack{\downarrow \\ (A)}}{e^{i s (H+\alpha G)t}} (i t G) \underset{\substack{\downarrow \\ (B)}}{e^{i (1-s) (H+\alpha G)t}} ds ~ \rho \underset{\substack{\downarrow \\ (C)}}{e^{-i(H+\alpha G)t}} } \nonumber \\
    &~ ~+\frac{\partial}{\partial \alpha} \parens{ \underset{\substack{\downarrow \\ (D)} }{e^{i(H+\alpha G)t}} \rho \int_{0}^1 \underset{\substack{\downarrow \\ (E)} }{e^{-i s (H+\alpha G) t} } (- i t G) \underset{\substack{\downarrow \\ (F)}}{e^{-i (1-s) (H+\alpha G)t}} ds }.
\end{align}
We will work through the first term in detail and will simply state the remaining terms. We work in order from left to right and top to bottom. The first term can be computed as
\begin{align}
    (A) &=i t\int_0^1 \parens{\frac{\partial}{\partial \alpha} e^{i s_1 (H+ \alpha G)t}} G e^{i(1-s_1)(H+\alpha G)t} ds_1 \rho e^{-i (H+\alpha G)t} \bigg|_{\alpha=0} \\
    &= (it)^2 \int_0^1 \parens{\int_0^1 e^{i s_1 s_2 (H+\alpha G)t} s_1 G e^{i s_1 (1-s_2) (H+\alpha G)t} ds_2} G e^{i(1-s_1) (H+\alpha G)t} ds_1 \rho e^{-i(H+\alpha G) t} \bigg|_{\alpha=0} \\
    &= -t^2 \int_0^1 \int_0^1 e^{i s_1 s_2 H t} G e^{-i s_1 s_2 H t} e^{i s_1 H t} G e^{-i s_1 H t} s_1 ds_1 ds_2 e^{i H t} \rho e^{-i H t} \\
    &= -t^2 \int_0^1 \int_0^1 G(s_1 s_2 t) G(s_1 t) s_1 ds_1 ds_2 \rho(t). \label{eq:second_deriv_alpha_first_term}
\end{align}

\begin{align}
    (B) &= it \int_0^1 e^{i s_1 (H + \alpha G)t} G \frac{\partial}{\partial \alpha}\parens{e^{i(1-s_1)(H + \alpha G)t}} ds_1 \rho e^{-i(H + \alpha G) t} \bigg|_{\alpha = 0} \\
    &= (it)^2 \int_0^1 e^{i s_1 (H + \alpha G)t} G \parens{\int_0^1 e^{i(1-s_1)s_2 (H + \alpha G)t} (1-s_1) G e^{i(1 - s_1)(1 - s_2)(H + \alpha G)t} ds_2} ds_1 ~ \rho e^{-i ( H + \alpha G)t} \bigg|_{\alpha = 0} \\
    &= -t^2 \int_0^1 \int_0^1 e^{i s_1 H t} G e^{i(1-s_1)s_2 H t} G e^{i(1-s_1)(1-s_2) H t} (1-s_1) ds_1 ds_2 ~ \rho e^{-i H t}\\ 
    &= -t^2 \int_0^1 \int_0^1 e^{i s_1 H t} G e^{-i s_1 H t} e^{i(s_1 + s_2 - s_1 s_2) H t} G e^{-i (s_1 + s_2 - s_1 s_2) H t} (1-s_1) ds_1 ds_2 ~ \rho(t) \\
    &= -t^2 \int_0^1 \int_0^1 G(s_1 t) G((s_1 + s_2 - s_1 s_2)t) (1-s_1) ds_1 ds_2 ~ \rho(t)
\end{align}

\begin{align}
    (C) &= it \int_0^1 e^{i s (H + \alpha G)t} G e^{i(1-s) (H + \alpha G) t} ds ~\rho ~ \frac{\partial}{\partial \alpha} \parens{ e^{-i (H + \alpha G) t} } \bigg|_{\alpha = 0} \\
    &= (i t) (-it) \int_0^1 e^{i s (H + \alpha G)t} G e^{i (1 - s) (H + \alpha G)t} ds ~ \rho ~ \parens{ \int_0^1 e^{-i s (H + \alpha G)t} G e^{-i (1- s) ( H + \alpha G)t } ds}\bigg|_{\alpha = 0} \\
    &= + t^2 \parens{\int_0^1 e^{i s H t} G e^{-i s H t} ds} e^{i H t} \rho e^{-i H t} \parens{\int_0^1 e^{i (1-s) H t} G e^{-i (1-s) H t} ds} \\
    &= + t^2 \int_0^1 G(st) ds ~ \rho(t) \int_0^1 G((1-s)t) ds
\end{align}

\begin{align}
    (D) &= (-it) \frac{\partial}{\partial \alpha} \parens{e^{i(H + \alpha G)t}} \rho \int_0^1 e^{-i s (H + \alpha G)t} G e^{-i (1-s)(H + \alpha G)t} ds \bigg|_{\alpha = 0} \\
    &= t^2 \parens{\int_0^1 e^{i s (H+ \alpha G)t} G e^{i (1-s) (H + \alpha G)t}ds} \rho \int_0^1 e^{-i s (H + \alpha G)t} G e^{-i (1-s)(H + \alpha G)t} ds \bigg|_{\alpha = 0} \\
    &=  t^2 \int_0^1 e^{i s H t} G e^{-i s H t} ds ~\rho(t) \int_0^1 e^{i (1-s) H t} G e^{-i (1-s) H t} ds \\
    &= t^2 \int_0^1 G(st) ds ~ \rho(t) ~ \int_0^1 G((1-s)t) ds
\end{align}

\begin{align}
    (E) &= (-it) e^{i (H+ \alpha G) t} ~ \rho ~\int_0^1 \frac{\partial}{\partial \alpha} \parens{e^{-i s_1 (H + \alpha G)t}} G e^{-i (1-s_1)(H + \alpha G)t} ds_1 \bigg|_{\alpha = 0} \\
    &= - t^2 e^{i(H + \alpha G)t} ~ \rho ~\int_0^1 \parens{\int_0^1 e^{-i s_1 s_2 (H + \alpha G) t} (s_1 G) e^{-i s_1 (1-s_2) (H + \alpha G)t} ds_2} G e^{-i(1-s_1)(H + \alpha G)t} ds_1 \bigg|_{\alpha = 0} \\
    &= -t^2 e^{i H t} \rho e^{-i H t} \int_0^1 \int_0^1 e^{i (1 - s_1 s_2) H t} G e^{-i (s_1 - s_1 s_2)H t} G e^{-i (1-s_1)H t} s_1 ds_1 ds_2 \\
    &= -t^2 \rho(t) \int_0^1 \int_0^1 G((1- s_1 s_2) t) G((1-s_1)t) s_1 ds_1 ds_2
\end{align}

\begin{align}
    (F) &= (-it) e^{i(H + \alpha G) t} \rho \int_0^1 e^{-i s_1 ( H + \alpha G) t} G \frac{\partial}{\partial \alpha} \parens{ e^{-i (1-s_1) ( H +\alpha G)t}} ds_1 \bigg|_{\alpha = 0} \\
    &= (-it)^2 e^{i (H + \alpha G)t} \rho \int_0^1 e^{-i s_1 (H + \alpha G)t} G \parens{\int_0^1 e^{-i(1-s_1) s_2 (H + \alpha G)t} (1-s_1) G e^{-i(1-s_1) (1-s_2) (H + \alpha G) t} ds_2} ds_1 \bigg|_{\alpha = 0} \\
    &= -t^2 e^{-i H t} \rho e^{-i H t} \int_0^1 \int_0^1 e^{i (1- s_1) H t} G e^{-i (1-s_1) H t} e^{i (1-s_1)(1-s_2) H t} G e^{-i(1-s_1)(1-s_2) H t} (1-s_1) ds_1 ds_2 \\
    &= -t^2 \rho(t) \int_0^1 \int_0^1 G((1-s_1)t) G((1-s_1)(1 - s_2) t) (1-s_1)ds_1 ds_2
\end{align}

We now combine terms $(A) - (F)$ with minor adjustments 
\begin{align}
    \frac{\partial^2}{\partial \alpha^2} \Phi_G(\rho_S) \bigg|_{\alpha = 0} &= -t^2 \int_0^1 \int_0^1 G(s_1 s_2 t) G(s_1 t) s_1 ds_1 ds_2 \rho(t) \nonumber \\
    &~ ~ -t^2 \int_0^1 \int_0^1 G(s_1 t) G((s_1 + s_2 - s_1 s_2)t) (1-s_1) ds_1 ds_2 ~ \rho(t) \nonumber \\
    &~ ~ + 2 t^2 \int_0^1 G(st) ds ~ \rho(t) \int_0^1 G((1-s)t) ds \nonumber \\
    & ~ ~ -t^2 \rho(t) \int_0^1 \int_0^1 G((1- s_1 s_2) t) G((1-s_1)t) s_1 ds_1 ds_2 \nonumber \\
    & ~ ~ -t^2 \rho(t) \int_0^1 \int_0^1 G(s_1 t) G(s_1 s_2 t) s_1 ds_1 ds_2 ~ .
\end{align}
Our last step is to decompose each factor into phase, eigenvalue, or Haar contributions. We start with the following breakdown of the Heisenberg evolution of $G$
\begin{align}
    G(x) &= e^{i H x} G e^{-i H x} \\
    &= \parens{\sum_{i,j} e^{i \lambda(i,j) x} \ketbra{i,j}{i, j}} G \parens{\sum_{k,l} e^{-i \lambda(k,l) x} \ketbra{k, l}{k, l}} \\
    &= \parens{\sum_{i,j} e^{i \lambda(i,j) x} \ketbra{i,j}{i, j}} U_G D U_G^\dagger \parens{\sum_{k,l} e^{-i \lambda(k,l) x} \ketbra{k, l}{k, l}} \\
    &= \parens{\sum_{i,j} e^{i \lambda(i,j) x} \ketbra{i,j}{i, j}} U_G \parens{ \sum_{m, n} D(m,n) \ketbra{m,n}{m,n}} U_G^\dagger \parens{\sum_{k,l} e^{-i \lambda(k,l) x} \ketbra{k, l}{k, l}} \\
    &= \sum_{i,j,k,l,m,n} e^{i (\lambda(i,j) - \lambda(k,l))x} D(m,n) \bra{i,j} U_G \ket{m,n} \bra{m,n} U_G^\dagger \ket{k,l} ~\ketbra{i,j}{k,l}.
\end{align}
The product of two time-evolved random interactions is then 
\begin{align}
    G(x) G(y) &= \sum_{i,j,k,l,m,n} e^{i (\lambda(i,j) - \lambda(k,l))x} D(m,n) \bra{i,j} U_G \ket{m,n} \bra{m,n} U_G^\dagger \ket{k,l} \ketbra{i,j}{k,l} \nonumber \\
    &~ \times \sum_{a,b,c,d,e,f} e^{i(\lambda(a,b) - \lambda(c,d))y} D(e,f) \bra{a,b} U_G \ket{e,f} \bra{e,f} U_G^\dagger \ket{c,d} \ketbra{a,b}{c,d} \\ 
    &= \sum_{i,j,k,l,m,n} \sum_{c,d,e,f} \bigg[ e^{i(\lambda(i,j) - \lambda(k,l))x} e^{i(\lambda(k,l) - \lambda(c,d))y} D(m,n) D(e,f) \nonumber \\
    & \quad \bra{i,j} U_G \ket{m,n} \bra{m,n} U_G^\dagger \ket{k,l} \bra{k,l} U_G \ket{e,f} \bra{e,f} U_G^\dagger \ket{c,d} \bigg] \ketbra{i,j}{c,d}.
\end{align}
Although this expression is verbose, we note that $x$ and $y$ only appear in the phase factors. This will allow us to pull out the eigenvalue contributions $D(m,n), D(e,f)$ of each term along with the Haar contributions. As these expressions are fairly verbose we will introduce the following notation:
\begin{align}
    C_\lambda(i,j,k,l,c,d; x, y) &\coloneqq e^{i(\lambda(i,j) - \lambda(k,l))x} e^{i(\lambda(k,l) - \lambda(c,d))y} \\
    C_D(m,n,e,f) &\coloneqq D(m,n) D(e,f) \\
    C_U(i,j,k,l,m,n,c,d,e,f) &\coloneqq \bra{i,j} U_G \ket{m,n} \bra{m,n} U_G^\dagger \ket{k,l} \bra{k,l} U_G \ket{e,f} \bra{e,f} U_G^\dagger \ket{c,d}.
\end{align}
Unfortunately these are not much shorter, so we will simply drop the indices and use the exact same indices for each term. Using implicit indices we get
\begin{align}
    G(x) G(y) &= \sum_{i,j,k,l,m,n} \sum_{c,d,e,f} C_\lambda(x,y) C_D C_U \ketbra{i,j}{c,d}.
\end{align}

We are now in a position to compute $\int \frac{\partial^2}{\partial \alpha^2} \Phi_G(\rho_S) \bigg|_{\alpha = 0} dG$. In order to do so we use our assumption that $\rho_S$ is a thermal state, implying that the state of the combined system-environment is $\rho = \frac{e^{-\beta_S H_s}}{\partfun_S} \otimes \frac{e^{-\beta_E H_E}}{\partfun_E}$. We use $\rho = \sum_{r,s} B(r,s) \ketbra{r,s}{r,s}$ to denote the diagonal elements. We note that this commutes with the overall time evolution, so the Heisenberg evolution does nothing: $\rho(t) = \rho$ for all $t$. 
\begin{align}
    \int (A) dG &= -t^2 \int \int_0^1 \int_0^1 G(s_1 s_2 t) G(s_1 t) s_1 ds_1 ds_2~ \rho \\
    &= -t^2 \sum_{i,j,k,l,m,n} \sum_{c,d,e,f} \bigg[ \ketbra{i,j}{c,d}\int C_D C_U dD dU \nonumber \\
    &\quad \int_0^1 \int_0^1 C_{\lambda}(s_1 s_2 t, s_1 t) s_1 ds_1 ds_2 \bigg] \sum_{r, s} B(r,s) \ketbra{r,s}{r,s} .
\end{align}
The first step we take to simplify is to compute $\int C_D dD$, which is straightforward.
\begin{align}
    \int C_D dD &= \int D(m,n) D(e,f) dD \\
    &= Cov(D(m,n), D(e,f) ) \\
    &= \sigma^2 \delta(m,e) \delta(n,f),
\end{align}
where we used the assumption of i.i.d random eigenvalues of G. We can further use Lemma \ref{lem:haar_two_moment} to compute the Haar contribution
\begin{align}
    \sum_{e,f}\int C_U \delta(m,e) \delta(n,f) dU &= \sum_{e,f} \delta_{m,e} \delta_{n,f} \int \bra{i,j} U_G \ket{m,n} \bra{m,n} U_G^\dagger \ket{k,l} \bra{k,l} U_G \ket{e,f} \bra{e,f} U_G^\dagger \ket{c,d} dU \\ 
    &= \int \bra{i,j} U_G \ket{m,n} \bra{k,l} U_G \ket{m,n} \bra{m,n} U_G^\dagger \ket{k,l} \bra{m,n} U_G^\dagger \ket{c,d} dU \\ 
    &= \frac{1}{\dim^2 - 1} \parens{\delta(i,j| k,l) \delta(k,l | c,d)} - \frac{1}{\dim(\dim^2 - 1)} \parens{\delta(i,j|c,d) + \delta(i,j| k,l) \delta(k,l|c,d)} \\
    &= \frac{\delta(i,j|k,l) \delta(k,l | c,d)}{\dim^2 - 1} \parens{1 - \frac{1}{\dim}} - \frac{\delta(i,j|c,d)}{\dim(\dim^2 - 1)} \\
    &= \frac{\delta(i,j|c,d)}{\dim^2 - 1} \parens{\delta(i,j|k,l)\parens{1 - \frac{1}{\dim}} - \frac{1}{\dim}}
\end{align}
Putting these all together gives a final result:
\begin{align}
    \int G(x) G(y) dG &= \sum_{i,j,k,l,m,n} C_{\lambda}(x,y) \sum_{c,d,e,f} \int C_D dD \int C_U dU_G \ketbra{i,j}{c,d} \\
    &= \sum_{i,j,k,l,m,n} C_{\lambda}(x,y) \sum_{c,d} \ketbra{i,j}{c,d} \sigma^2 \frac{\delta(i,j| c,d)}{\dim^2 - 1}\parens{\delta(i,j|k,l)(1 - 1/\dim) -1/\dim} \\
    &= \sigma^2 \sum_{i,j,k,l} \ketbra{i,j}{i,j} C_{\lambda}(x,y) \frac{\delta(i,j|k,l)(\dim -1) - 1}{\dim^2 - 1}
\end{align}

Plugging these into our computation for $(A)$ yields
\begin{align}
    \int (A) dG &= -\sigma^2 t^2 \sum_{i,j,k,l,m,n} \sum_{c,d, e, f} \ketbra{i,j}{c,d} B(c,d) \int C_D dD ~ \int C_U dU_G \int_0^1 \int_0^1 C_{\lambda}(s_1 s_2 t, s_1 t) s_1 ds_1 ds_2 \\
    &= -\sigma^2 t^2 \sum_{i,j,k,l,m,n} \sum_{c,d} \ketbra{i,j}{c,d} B(c,d)\int_0^1 \int_0^1 C_{\lambda}(s_1 s_2 t, s_1 t) s_1 ds_1 ds_2 \sum_{e,f}\delta(m,n|e,f) \int C_U dU_G \\
    &= -\sigma^2 t^2 \sum_{i,j,k,l,m,n} \sum_{c,d} B(c,d) \ketbra{i,j}{c,d} \int_0^1 \int_0^1 C_{\lambda}(s_1 s_2 t, s_1 t) s_1 ds_1 ds_2 \frac{\delta(i,j| c,d)}{\dim^2 - 1} \parens{\delta(i,j|k,l)(1-1/\dim) - 1/\dim} \\
    &= -\sigma^2 t^2 \sum_{i,j,k,l,m,n} \sum_{c,d} B(c,d) \ketbra{i,j}{c,d} \int_0^1 \int_0^1 e^{i(\lambda(i,j) - \lambda(k,l)) s_1 s_2 t} e^{i(\lambda(k,l) - \lambda(c,d))s_1t} s_1 ds_1 ds_2 \frac{\delta(i,j| c,d)}{\dim^2 - 1} \parens{\delta(i,j|k,l)(1-1/\dim) - 1/\dim} \\
    &= -\sigma^2 t^2 \sum_{i,j,k,l,m,n} \frac{B(i,j)}{\dim^2 - 1} \ketbra{i,j}{i,j} \int_0^1 \int_0^1 e^{i(\lambda(i,j) - \lambda(k,l))(s_1 s_2 - s_1)t} s_1 ds_1 ds_2 \parens{\delta(i,j|k,l)(1-1/\dim) - 1/\dim} \\
    &= -\sigma^2 t^2 \sum_{i,j,k,l} \frac{B(i,j) \dim}{\dim^2 - 1} \ketbra{i,j}{i,j} \int_0^1 \int_0^1 e^{i(\lambda(i,j) - \lambda(k,l))(s_1 s_2 - s_1)t} s_1 ds_1 ds_2 \parens{\delta(i,j|k,l)(1-1/\dim) - 1/\dim}.
\end{align}

We now proceed by pulling out the $i,j = k,l$ term from the $k,l$ summation and performing the phase integration. This yields
\begin{align}
    &\sum_{k,l} \int_0^1 \int_0^1 e^{i(\lambda(i,j) - \lambda(k,l))(s_1 s_2 - s_1)t} s_1 ds_1 ds_2 \parens{\delta(i,j|k,l)(\dim-1) - 1} \\
    &= \frac{\dim - 2}{2} - \sum_{(k,l) \neq (i,j)} \frac{1 - i t (\lambda(i,j) - \lambda(k,l)) - e^{-i(\lambda(i,j) - \lambda(k,l))t}}{t^2(\lambda(i,j) - \lambda(k,l))^2}
\end{align}
Combining all of the above gives the final computation of the $(A)$ term
\begin{align}
    \int (A) dG &= \frac{\sigma^2 t^2}{\dim^2 -1} \sum_{i,j} B(i,j) \ketbra{i,j}{i,j} \bigg[ -\frac{\dim - 2}{2} + \sum_{(k,l) \neq (i,j)} \frac{1 - i t (\lambda(i,j) - \lambda(k,l)) - e^{-i(\lambda(i,j) - \lambda(k,l))t}}{t^2(\lambda(i,j) - \lambda(k,l))^2} \bigg]
\end{align}

We now can compute the $(B)$ term in a similar fashion
\begin{align}
    \int (B) dG &= -t^2 \int \int_0^1 \int_0^1 G(s_1 t) G((s_1 + s_2 - s_1 s_2)t) (1-s_1) ds_1 ds_2 dG ~ \rho(t) \\
    &= -t^2 \sum_{i,j,k,l,m,n} \sum_{c,d,e,f} B(c,d) \ketbra{i,j}{c,d} \int C_D dD \int C_U dU_G \int_0^1 \int_0^1 C_{\lambda}(s_1 t, (s_1 + s_2 - s_1 s_2)t) (1-s_1) ds_1 ds_2 \\
    &= -\sigma^2 t^2 \sum_{i,j,k,l,m,n} \sum_{c,d} B(c,d) \ketbra{i,j}{c,d} \sum_{e,f} \parens{\delta(m,n | e,f) \int C_U dU_G }\int_0^1 \int_0^1 C_{\lambda}(s_1 t, (s_1 + s_2 - s_1 s_2)t) (1-s_1) ds_1 ds_2 \\
    &= -\sigma^2 t^2 \sum_{i,j,k,l,m,n} \sum_{c,d} B(c,d) \ketbra{i,j}{c,d} \frac{\delta(i,j| c,d)}{\dim^2 - 1} \parens{\delta(i,j|k,l)(1-1/\dim) - 1/\dim} \int_0^1 \int_0^1 C_{\lambda}(s_1 t, (s_1 + s_2 - s_1 s_2)t) (1-s_1) ds_1 ds_2 \\
    &= -\sigma^2 t^2 \sum_{i,j,k,l,m,n} B(i,j) \ketbra{i,j}{i,j} \frac{\delta(i,j | k,l)(1-1/\dim) - 1/\dim}{\dim^2 - 1} \int_0^1 \int_0^1 C_{\lambda}(s_1 t, (s_1 + s_2 - s_1 s_2)t) (1-s_1) ds_1 ds_2 \\
    &= -\sigma^2 t^2 \sum_{i,j} \sum_{k,l} B(i,j) \ketbra{i,j}{i,j} \frac{\delta(i,j | k,l)(\dim-1) - 1}{\dim^2 - 1} \int_0^1 \int_0^1 C_{\lambda}(s_1 t, (s_1 + s_2 - s_1 s_2)t) (1-s_1) ds_1 ds_2 .
\end{align}
Now we compute the phase contribution in a similar manner as $(A)$:
\begin{align}
    &\sum_{k,l} (\delta(i,j|k,l)(\dim - 1) - 1) \int_0^1 \int_0^1 e^{-i (\lambda(i,j) - \lambda(k,l))s_2 (1-s_1)t} (1 - s_1) ds_1 ds_2 \\
    &= \frac{\dim - 2}{2} - \sum_{(k,l) \neq (i,j)} \frac{1 - i t (\lambda(i,j) - \lambda(k,l)) - e^{-i (\lambda(i,j) - \lambda(k,l))t}}{t^2 (\lambda(i,j) - \lambda(k,l))^2},
\end{align}
where we note that all signs are the same as those in $(A)$. Plugging this back in to $(B)$ yields
\begin{align}
    \int (B) dG &= \frac{\sigma^2 t^2}{\dim^2-1} \sum_{i,j} B(i,j) \ketbra{i,j}{i,j} \bigg[ - \frac{\dim - 2}{2} + \sum_{(k,l) \neq (i,j)} \frac{1  -i t (\lambda(i,j) - \lambda(k,l)) - e^{-it(\lambda(i,j) - \lambda(k,l))}}{t^2 (\lambda(i,j) - \lambda(k,l))^2} \bigg].
\end{align}
We note that this term is exactly equal to the contribution from $(A)$.
Now we can compute the ``center" term where $\rho(t)$ is sandwiched between the two random interaction matrices
\begin{align}
    (C) &= t^2 \int_0^1 G(st) ds ~ \rho(t) ~ \int_0^1 G((1-s)t) ds \\
    &= t^2 \sum_{i,j,k,l,m,n} \ketbra{i,j}{k,l} \int_0^1 e^{i (\lambda(i,j) - \lambda(k,l)s t} ds D(m,n) \bra{i,j} U_G \ket{m,n} \bra{m,n} U_G^\dagger \ket{k,l} \parens{\sum_{o,p} B(o,p) \ketbra{o,p}{o,p}} \nonumber \\
    & ~\times \sum_{a,b,c,d,e,f} \ketbra{a,b}{c,d} \int_0^1 e^{i (\lambda(a,b) - \lambda(c,d)(1-s)t} ds D(e,f) \bra{a,b} U_G \ket{e,f} \bra{e,f} U_G^\dagger \ket{c,d}\\
    &= t^2 \sum_{i,j,k,l,m,n} \sum_{a,b,c,d,e,f} \sum_{o,p} B(o,p) \bigg[ \ketbra{i,j}{c,d} \delta(k,l|o,p) \delta(o,p | a,b) \nonumber \\
    &\quad D(m,n) D(e,f) \int_0^1 e^{i(\lambda(i,j) - \lambda(k,l)s t} ds \int_0^1 e^{i(\lambda(a,b) - \lambda(c,d))(1-s)t} ds \nonumber \\
    &\quad \bra{i,j} U_G \ket{m,n} \bra{m,n} U_G^\dagger \ket{k,l} \bra{a,b} U_G \ket{e,f} \bra{e,f} U_G^\dagger \ket{c,d} \bigg] \\
    &= t^2 \sum_{i,j,k,l,m,n} \sum_{c,d,e,f} \bigg[ B(k,l) \ketbra{i,j}{c,d} D(m,n) D(e,f) \int_0^1 e^{i(\lambda(i,j) - \lambda(k,l)s t} ds \int_0^1 e^{i(\lambda(k,l) - \lambda(c,d))(1-s)t} ds \nonumber \\
    &\quad \bra{i,j} U_G \ket{m,n} \bra{m,n} U_G^\dagger \ket{k,l} \bra{k,l} U_G \ket{e,f} \bra{e,f} U_G^\dagger \ket{c,d} \bigg].
\end{align}

Next we perform the random interaction eigenvalue integral $\int (C) dG = \int \parens{\int (C) dD} dU_G$. This is the same as prior computations and yields $\int D(m,n) D(e,f) dD = \delta(m,n | e,f) \sigma^2$. Plugging this into the last step and doing the $e,f$ summation yields
\begin{align}
    \int (C) dD &= \sigma^2 t^2 \sum_{i,j,k,l,m,n} \sum_{c,d} \bigg[ B(k,l) \ketbra{i,j}{c,d} \int_0^1 e^{i(\lambda(i,j) - \lambda(k,l)s t} ds \int_0^1 e^{i(\lambda(k,l) - \lambda(c,d))(1-s)t} ds \nonumber \\
    &\quad \bra{i,j} U_G \ket{m,n} \bra{m,n} U_G^\dagger \ket{k,l} \bra{k,l} U_G \ket{m,n} \bra{m,n} U_G^\dagger \ket{c,d} \bigg].
\end{align}
Now performing the Haar integration, which was done in prior terms $(A)$ and $(B)$ yields
\begin{align}
    \int (C) dG &= \sigma^2 t^2 \sum_{i,j,k,l,m,n} \sum_{c,d} \bigg[B(k,l) \ketbra{i,j}{c,d} \int_0^1 e^{i(\lambda(i,j) - \lambda(k,l)s t} ds \int_0^1 e^{i(\lambda(k,l) - \lambda(c,d))(1-s)t} ds \nonumber \\
    &\quad \frac{\delta(i,j|c,d)}{\dim^2 - 1} \parens{\delta(i,j|k,l)\parens{1 - \frac{1}{\dim}} - \frac{1}{\dim}} \bigg] \\
    &= \frac{\sigma^2 t^2}{\dim^2 - 1} \sum_{i,j,k,l} \bigg[ B(k,l) \ketbra{i,j}{i,j} \int_0^1 e^{i(\lambda(i,j) - \lambda(k,l)s t} ds \int_0^1 e^{i(\lambda(k,l) - \lambda(i,j))(1-s)t} ds \nonumber \\
    &\quad \parens{\delta(i,j | k,l)(\dim - 1) - 1} \bigg]
\end{align}

We now compute the phase contribution to simplify the $k,l$ summands
\begin{align}
    &\sum_{k,l} B(k,l) \parens{\delta(i,j | k,l)(\dim - 1) - 1} \int_0^1 e^{i (\lambda(i,j) - \lambda(k,l))st} ds \int_0^1 e^{-i (\lambda(i,j) - \lambda(k,l))(1-s)t} ds \\
    &= B(i,j) (\dim - 2) - 2 \sum_{(k,l) \neq (i,j)} B(k,l) \frac{1 - \cos((\lambda(i,j) - \lambda(k,l))t)}{t^2 (\lambda(i,j) - \lambda(k,l))^2}.
\end{align}
Plugging back in yields
\begin{align}
    \int (C) dG &= \frac{2 \sigma^2 t^2}{\dim^2 - 1} \sum_{i,j} B(i,j) \ketbra{i,j}{i,j} \bigg[ \frac{\dim -2 }{2} - \sum_{(k,l) \neq (i,j)} \frac{B(k,l)}{B(i,j)} \frac{1 - \cos((\lambda(i,j) - \lambda(k,l))t)}{t^2 (\lambda(i,j) - \lambda(k,l))^2} \bigg]
\end{align}
Since $(D) = (C)$ we can move on to the remaining two terms $(E)$ and $(F)$.
\begin{align}
    \int (E) dG &= -t^2 \int  \rho(t) \int_0^1 \int_0^1 G((1- s_1 s_2) t) G((1-s_1)t) s_1 ds_1 ds_2 dG \\
    &= -\sigma^2 t^2 \sum_{o,p} B(o,p) \ketbra{o,p}{o,p} \sum_{i,j,k,l} \ketbra{i,j}{i,j} \int_0^1 \int_0^1 C_{\lambda}((1-s_1 s_2)t, (1-s_1)t) s_1 ds_1 ds_2 \frac{\delta(i,j|k,l)(\dim -1) - 1}{\dim^2 - 1} \\
    &= -\sigma^2 t^2 \sum_{i,j,k,l} B(i,j) \ketbra{i,j}{i,j} \int_0^1 \int_0^1 e^{i(\lambda(i,j) - \lambda(k,l))(s_1 (1- s_2) )t} s_1 ds_1 ds_2 \frac{\delta(i,j|k,l)(\dim -1) - 1}{\dim^2 - 1}
\end{align} 
We now compute the phase contribution
\begin{align}
    &\sum_{k,l} (\delta(i,j | k,l) (\dim - 1) - 1) \int_0^1 \int_0^1 e^{i (\lambda(i,j) - \lambda(k,l))(s_1 (1 - s_2))t} s_1 ds_1 ds_2 \\
    &= \frac{\dim - 2}{2} - \sum_{(k,l) \neq (i,j)} \frac{1 + i t (\lambda(i,j) - \lambda(k,l)) - e^{i (\lambda(i,j) - \lambda(k,l))t}}{t^2 (\lambda(i,j) - \lambda(k,l))^2}
\end{align}
Plugging back in yields
\begin{align}
    \int (E) dG &= \frac{\sigma^2 t^2}{\dim^2 - 1} \sum_{i,j} B(i,j) \ketbra{i,j}{i,j} \bigg[ - \frac{\dim - 2}{2} + \sum_{(k,l) \neq (i,j)} \frac{1 + i t (\lambda(i,j) - \lambda(k,l)) - e^{i (\lambda(i,j) - \lambda(k,l))t}}{t^2 (\lambda(i,j) - \lambda(k,l))^2} \bigg].
\end{align}
We note that this expression is the complex conjugate of the expression derived for $(A)$ and $(B)$. 

The last term is computed similarly:
\begin{align}
    \int(F) dG &= -t^2 \rho(t) \int \int_0^1 \int_0^1 G((1-s_1)t) G((1 - s_1 )(1-s_2)t) (1-s_1) ds_1 ds_2 dG \\
    &= - \frac{\sigma^2 t^2}{\dim^2 - 1} \sum_{i,j,k,l} B(i,j) \ketbra{i,j}{i,j} \int_0^1 \int_0^1 \bigg[ \nonumber \\
    &\quad \quad e^{i(\lambda(i,j) - \lambda(k,l))s_1 t} e^{-i(\lambda(i,j) - \lambda(k,l))(s_1 s_2 )t} s_1 ds_1 ds_2 (\delta(i,j|k,l)(\dim - 1) - 1) \bigg] \\
    &= - \frac{\sigma^2 t^2}{\dim^2 -1 } \sum_{i,j} B(i,j) \ketbra{i,j}{i,j} \bigg[\frac{\dim - 2}{2} - \sum_{(k,l) \neq (i,j)}\int_0^1 \int_0^1 e^{i(\lambda(i,j) - \lambda(k,l))s_1 s_2 t} s_1 ds_1 ds_2 \bigg] \\
    &= \frac{\sigma^2 t^2}{\dim^2 - 1} \sum_{i,j} B(i,j) \ketbra{i,j}{i,j} \bigg[ - \frac{\dim - 2}{2} + \sum_{(k,l) \neq (i,j)} \frac{1 + i (\lambda(i,j) - \lambda(k,l))t - e^{i(\lambda(i,j) - \lambda(k,l))t} }{t^2 (\lambda(i,j) - \lambda(k,l))^2} \bigg],
\end{align}
which we note is equal to term $(E)$. 

Combining terms $(A)$ through $(F)$ yields
\begin{align}
    \int \frac{\partial^2}{\partial \alpha^2} \Phi_G(\rho_S) \bigg|_{\alpha = 0} dG &= \frac{\sigma^2 t^2}{\dim^2 - 1} \sum_{i,j} B(i,j) \ketbra{i,j}{i,j} \sum_{(k,l) \neq (i,j)} \frac{4 + 4 \cos((\lambda(i,j) - \lambda(k,l)t))}{t^2 (\lambda(i,j) - \lambda(k,l))^2} - 4 \frac{B(k,l)}{B(i,j)} \frac{1 - \cos((\lambda(i,j) - \lambda(k,l))t)}{t^2 (\lambda(i,j) - \lambda(k,l))^2} 
\end{align}
We note that this can be drastically simplified by time averaging:
\begin{align}
    \lim_{T \to \infty} \frac{1}{T} \int_0^T \int \frac{\partial^2}{\partial \alpha^2} \Phi_G(\rho_S) \bigg|_{\alpha = 0} dG ~ dt &= \frac{4 \sigma^2}{\dim^2 - 1} \sum_{i,j} \ketbra{i,j}{i,j} \sum_{(k,l) \neq (i,j)} \frac{B(i,j) - B(k,l)}{(\lambda(i,j) - \lambda(k,l))^2}
\end{align}
\end{proof}

Our next goal is to show that this procedure can be used to increase the overlap of our system state with the ground state of the provided Hamiltonian.
\begin{align}
    \overlap{\ketbra{0}{0}}{ \rho_S(\beta_S)} &< \overlap{\ketbra{0}{0}}{ \Phi(\rho_S(\beta_S))} \\
    &= \overlap{\ketbra{0}{0}}{ \rho_S(\beta_S)} + \overlap{\ketbra{0}{0}}{\partrace{\hilb_E}{\sum_{i,j} \ketbra{i,j}{i,j} T(i,j)}} \\
    0 &< \overlap{\ketbra{0}{0}}{ \partrace{\hilb_E}{\sum_{i,j} \ketbra{i,j}{i,j} T(i,j)}} \\
    0 &< \overlap{\ketbra{0}{0}}{ \sum_i \ketbra{i}{i} \sum_{j} T(i,j) } \\
    0 &< \sum_j T(0,j)
\end{align}
Now we will analyze this situation for a simple bath in which $H_E = \begin{bmatrix} 0 & 0 \\ 0 & \Delta \end{bmatrix}$. 

\begin{align}
    0 &< \frac{2 \sigma^2 \alpha^2}{\dim^2 - 1} \sum_{(k,l) \neq (0,j)} \frac{B(0, j) - B(k,l)}{(\lambda(0, j) - \lambda(k, l))^2} \\
    0 &< \sum_{(k,l) \neq (0, 0)} \frac{B(0, 0) - B(k,l)}{(\lambda(0, 0) - \lambda(k, l))^2} + \sum_{(k,l) \neq (0, 1)} \frac{B(0, 1) - B(k,l)}{(\lambda(0, 1) - \lambda(k, l))^2} \\
    &= \sum_{k \neq 0} \frac{B(0,0) - B(k, 0)}{ (\lambda(0,0) - \lambda(k, 0))^2} + \sum_{k} \frac{B(0,0) - B(k, 1)}{(\lambda(0,0) - \lambda(k, 1))^2} \nonumber \\
    &\quad + \sum_{k} \frac{B(0,1) - B(k, 0)}{(\lambda(0,1) - \lambda(k, 0))^2}  + \sum_{k \neq 0 }\frac{B(0,1) - B(k, 1)}{(\lambda(0,1) - \lambda(k, 1))^2}
\end{align}
We now use the fact that $B(i,j) = \frac{e^{-\beta_S \lambda_S(i)} e^{-\beta_E \lambda_E(j)}}{\partfun_{S}(\beta_S) \partfun_E (\beta_E)}$ to simplify the numerators. Because the partition functions are constant across $i,j$ we can simply multiply the whole expression by $\partfun_S(\beta_S) \partfun_E(\beta_E)$. To further simplify the analysis we will multiply through by $e^{\beta_S \lambda_S(0)}$. We also remind the reader that $\lambda(i,j) = \lambda_S(i) + \lambda_E(j)$, and $\lambda_E(0) = 0, \lambda_E(1) = \Delta$. 
\begin{align}
    0 &< \sum_{k \neq 0 } \frac{1 - e^{-\beta_S(\lambda_S(k) - \lambda_S(0))}}{(\lambda_S(0) - \lambda_S(k))^2} + \sum_{k} \frac{1 - e^{-\beta_S(\lambda_S(k) - \lambda_S(0))} e^{-\beta_E \Delta}}{(\lambda_S(0) - \lambda_S(k) - \Delta)^2} \nonumber \\
    &\quad + \sum_k \frac{e^{-\beta_E \Delta} - e^{-\beta_S (\lambda_S(k) - \lambda_S(0))}}{(\lambda_S(0) + \Delta - \lambda_S(k))^2} + \sum_{k \neq 0 } \frac{e^{-\beta_E \Delta}(1 - e^{-\beta_S(\lambda_S(k) - \lambda_S(0))})}{(\lambda_S(0) - \lambda_S(k))^2} .
\end{align}
To enforce positivity, we could split these sums into positive and negative contributions and determine conditions leading to the positive terms outweighing the negative terms. Or we could be brain dead and require that each term be positive, which leads to the conditions (working from left-to-right, top-to-bottom):
\begin{align}
    &\lambda_S(k) > \lambda_S(0) \\
    &\lambda_S(k) > \lambda_S(0) \text{ and } 0 < \Delta \\
    & \beta_E \Delta < \beta_S (\lambda_S(k) - \lambda_S(0)) \\
    & \lambda_S(k) > \lambda_S(0).
\end{align}
The only non-trivial requirement is that $\beta_E \Delta < \beta_S \delta_{S}$, where $\delta_{S}$ is the eigenvalue gap of the system $H_S$. 

We now move on to linearizing the second order corrections with respect to $\beta_S$. Let $\beta_S = \beta_E - \delta$:
\begin{align}
    \frac{\partial}{\partial \delta} B(i,j) \bigg|_{\delta = 0} &= \frac{\partial}{\partial \delta} \frac{e^{-\beta_S \lambda_S(i)}}{\partfun_S(\beta_S)} \frac{e^{-\beta_E \lambda_E(j)}}{\partfun_E(\beta_E)} \bigg|_{\delta = 0} \\
    &= \frac{e^{-\beta_E \lambda_E(j)} e^{-\beta_E \lambda_S(i)}}{\partfun_E(\beta_E)} \parens{\frac{\lambda_S(i) e^{\delta \lambda_S(i)}}{\partfun_S(\beta_E - \delta)} - \frac{e^{\delta \lambda_S(i)} \trace{H_S e^{(-\beta_E  + \delta) H_S}}}{\partfun_S(\beta_E - \delta)^2}} \bigg|_{\delta = 0} \\
    &= \frac{e^{-\beta_E \lambda(i,j)}}{\partfun_S(\beta_E) \partfun_E(\beta_E)} \parens{\lambda_S(i) - \trace{H_S \rho_S(\beta_E)}} \\
    &= B(i,j; \beta_E) \parens{\lambda_S(i) - \trace{H_S \rho_S(\beta_E)}}
\end{align}
which allows us to write 
\begin{align}
    B(i,j; \beta_E - \delta) &= B(i,j; \beta_E) + \delta B(i,j; \beta_E) \parens{\lambda_S(i) - \trace{H_S \rho_S (\beta_E)}} + \bigo{\delta^2}.
\end{align}
This gives the second order $\alpha$ correction as
\begin{align}
    T(i,j; \beta_S) &= \frac{2 \sigma^2 \alpha^2}{\dim^2 - 1} \sum_{(k,l) \neq (i,j)} \frac{B(i,j; \beta_E)(1 + \delta(\lambda_S(i) - \trace{H_S \rho_S(\beta_E)}) ) - B(k,l; \beta_E)(1 + \delta(\lambda_S(k) - \trace{H_S \rho_S(\beta_E)}) ) }{(\lambda(i,j) - \lambda(k,l))^2} + \bigo{\delta^2} 
\end{align}

Now we would like to show that the system state $\rho(\beta_E)$ is an approximate fixed point of the channel to $\bigo{\alpha^2}$.

\begin{align}
    \norm{\rho_S(\beta_E) - \Phi(\rho_S(\beta_E))}_1 &= \norm{\partrace{\hilb_E}{\frac{2 \sigma^2 \alpha^2}{\dim^2 - 1}\sum_{i,j} \ketbra{i,j}{i,j} \sum_{(k,l) \neq (i,j)}\frac{\bra{i,j} \rho \ket{i,j} - \bra{k,l} \rho \ket{k,l}}{(\lambda(i,j) - \lambda(k,l))^2} }} + \bigo{\alpha^4} \\
    &= \frac{2 \sigma^2 \alpha^2}{\dim^2 - 1} \sum_{i,j} \abs{\sum_{(k,l) \neq (i,j)} \frac{1}{\partfun(\beta_E)} \frac{e^{-\beta_E \lambda(i,j)} - e^{-\beta_E \lambda(k,l)}}{(\lambda(i,j) - \lambda(k,l))^2}} \\ 
    &\leq \frac{2 \sigma^2 \alpha^2}{\partfun(\beta_E) (\dim^2 - 1)} \sum_{i,j; (k,l) \neq (i,j)} \frac{\abs{e^{-\beta_E \lambda(i,j)} - e^{-\beta_E \lambda(k,l)}}}{(\lambda(i,j) - \lambda(k,l))^2} \\
    &\leq \frac{2 \sigma^2 \alpha^2}{\partfun(\beta_E) \Delta^2 (\dim^2 - 1)} \sum_{i,j; (k,l) \neq (i,j)} \abs{e^{-\beta_E \lambda(i,j)} - e^{-\beta_E \lambda(k,l)}} \\
    &\leq \frac{2 \sigma^2 \alpha^2}{\partfun(\beta_E) \Delta^2 (\dim^2 - 1)}  \sum_{i,j; (k,l) \neq (i,j)} e^{-\beta_E \min(\lambda(i,j), \lambda(k,l))} \\
    &\leq \frac{2 \sigma^2 \alpha^2 e^{-\beta_E \min_{i,j} \lambda(i,j)}}{\partfun(\beta_E) \Delta^2}
\end{align}
Requiring the output to be an $\epsilon_{fix}$ distance at most is satisfied by $\sigma \alpha \leq \frac{\epsilon_{fix} \Delta^2 \partfun(\beta_E) e^{\beta_E \lambda_{min}}}{2}$.

We now would like to show that the distance to $\rho_S(\beta_E)$ is reduced by the channel. $$\norm{\rho_S(\beta_E) - \Phi(\rho_S(\beta_S))}_1 \leq \norm{\rho_S(\beta_E) - \rho_S(\beta_S)}_1.$$ We do so by assuming $\beta_S = \beta_E - \delta$, where $0 \leq \delta$ is small.  
\begin{align}
    \bra{i,j} \rho_S(\beta_S) \otimes \rho_E(\beta_E) \ket{i,j} &= \frac{e^{-\beta_E \lambda_S (i) + \delta \lambda_S(i)} e^{-\beta_E \lambda_E(j)}}{\partfun_S(\beta_S) \partfun_E(\beta_E)} = \frac{e^{-\beta_E \lambda(i,j)}}{\partfun_S(\beta_E - \delta) \partfun_E(\beta_E) } e^{\delta \lambda_S(i)}
\end{align}

Our next goal is to show that any thermal state for $\delta$ above a certain value are not even approximate fixed points. We do so by lower bounding the distance 
\begin{align}
    \norm{\rho_S(\beta_S) - \Phi(\rho_S(\beta_S))}_1 &\geq \norm{\partrace{\hilb_E}{\frac{2 \sigma^2 \alpha^2}{\dim^2 - 1} \sum_{i,j} \ketbra{i,j}{i,j} \sum_{(k,l) \neq (i,j)} \frac{\bra{i,j} \rho \ket{i,j} - \bra{k,l} \rho \ket{k,l}}{(\lambda(i,j) - \lambda(k,l))^2}}}_1 \\
    &= \frac{2 \sigma^2 \alpha^2}{\dim^2 - 1} \sum_i \abs{\sum_j \sum_{(k,l) \neq (i,j)} \frac{\bra{i,j} \rho \ket{i,j} - \bra{k,l} \rho \ket{k,l}}{(\lambda(i,j) - \lambda(k,l))^2}} \\
    &\geq \frac{2 \sigma^2 \alpha^2}{\dim^2 - 1} \abs{\sum_{j} \sum_{(k,l) \neq (0,j)} \frac{\bra{0,j} \rho \ket{0,j} - \bra{k,l} \rho \ket{k,l}}{(\lambda(0,j) - \lambda(k,l))^2}} \label{eq:i_need_an_eq_label}
\end{align}
We now want to look at the condition $\bra{0,j} \rho \ket{0,j} > \bra{k,l} \rho \ket{k,l}$ for $k > 0$ (note the partition function cancels). We can see that if $j=0$ this is satisfied trivially (due to monotonicity of eigenvalues). In order to push this to it's limit we can set $\lambda_E(j)$ to the largest eigenvalue $\norm{H_E}$. This then means that we can push $k$ to the second smallest eigenvalue and $l$ to the ground state of the environment. Also note without loss of generality we can require that $\lambda_E(0) = 0$, as we control the environment.
\begin{align}
    e^{-\beta_E \lambda_S (0) + \delta \lambda_S(0)} e^{-\beta_E \norm{H_E}} &> e^{-\beta_E \lambda_S(1) + \delta \lambda_S(1)} e^{-\beta_E \lambda_E(0)} \\
    e^{(\beta_E - \delta) \Delta_S} &> e^{\beta_E(\norm{H_E} - \lambda_E(0))} \\
    \beta_S \Delta_S &> \beta_E \norm{H_E} \\
    \norm{H_E} &< \frac{\beta_S \Delta_S}{\beta_E} = \parens{1 - \frac{\delta}{\beta_E}} \Delta_S
\end{align}

\begin{align}
    e^{-\beta_S \lambda_S(0) - \beta_E \norm{H_E}} - e^{-\beta_S \lambda_S(1)} &\geq \epsilon_1 \\
    e^{-\beta_S \lambda_S(0) - \beta_E \norm{H_E}} (1 - e^{-\beta_S (\lambda_S(1) - \lambda_S(0)) + \beta_E \norm{H_E} }) & \geq 
\end{align}
We now add this to our growing list of random assumptions. 

Given the above assumption, we can safely assume that each summand in Eq. \eqref{eq:i_need_an_eq_label} is positive. This allows us to drop the absolute value without concern.
\begin{align}
    \norm{\rho_S(\beta_S) - \Phi(\rho_S(\beta_S))}_1 &\geq \frac{2 \sigma^2 \alpha^2}{\dim^2 - 1} \sum_j \sum_{(k,l) \neq (0,j)} \frac{\bra{0,j} \rho \ket{0,j} - \bra{k,l} \rho \ket{k,l}}{(\lambda(0,j) - \lambda(k,l))^2} \\
    & \geq \frac{2 \sigma^2 \alpha^2}{\norm{H}^2 \partfun_S(\beta_S) \partfun_E(\beta_E) (\dim^2 - 1)} \sum_{j} \sum_{(k,l) \neq (0,j)} e^{-\beta_S \lambda_S(0) - \beta_E \lambda_E(j)} - e^{-\beta_S \lambda_S(k) - \beta_E \lambda_E(l)} \\
    &\geq \frac{2 \sigma^2 \alpha^2 \dim_E (\dim - 1) x}{\norm{H}^2 \partfun_S(\beta_S) \partfun_E(\beta_E)(\dim + 1) (\dim-1)}
\end{align}
\newpage

\subsection{Harmonic Oscillator Fixed Point}
Now consider the situation of a system and environment which are both harmonic oscillators of frequency $\omega$, $H_{sys} = \hbar \omega \sum_{i=0}^{N_{sys}} (0.5 + i) \ketbra{i}{i}$ and $H_{env} = \hbar \omega \sum_{i=0}^{N_{env}} (0.5 + i) \ketbra{i}{i}$, yeilding $H = \sum_{i,j} \hbar \omega (1 + i + j) \ketbra{i,j}{i,j}$. We set $\hbar = \omega = 1$ to avoid dimensionful quantities. 

Our hunch is that the second order correction of $\Phi(\rho)$ vanishes whenever $\rho$ is a thermal state of the joint system-environment hamiltonian, i.e. $\rho = \sum_{i,j} \ketbra{i}{i} \otimes \ketbra{j}{j}\rho_{i,j} = \frac{e^{-\beta (1 + i + j)}}{\sum_{e^{-\beta(1+  i + j)}}} \ketbra{i}{i} \otimes \ketbra{j}{j}$. Before we compute this second-order correction, note that the assumption of a thermal state input eliminates the sum over equal energy contributions in Eq. \eqref{eq:second_order_final}. Further, in the interest of simplifying the expression as much as possible, we will only consider the time-averaged contributions $\frac{1}{T} \int_0^T \Phi(\rho, t) dt$, this pretty much just eliminates the cosine term in Eq. \eqref{eq:second_order_final}.

First define $\rho(\gamma) := \frac{e^{-(\beta_E -\gamma) H_{sys}}}{\partfun_{sys}(\gamma)}$
Our main goal is to show $\norm{\Phi(\rho(\gamma)) - \rho(0)} \leq \norm{\rho(\gamma) - \rho(0)}$, in other words that a higher temperature state ($\beta_{sys} = \beta_{env} - \gamma$) gets closer to a lower temperature state after application of $\Phi$. In order to do this we first use our power series of $\Phi$ in terms of the coupling constant $\alpha$ and use the Schatten-1 norm:
\begin{align}
    \norm{\Phi(\rho(\gamma)) - \rho(0)}_1 &= \norm{\rho(\gamma) + T(\rho(\gamma)) + R(\rho(\gamma)) - \rho(0)}_1 \\
    &= \norm{ \sum_{i} \parens{ e^{-(\beta_{env} + \gamma) E_i} \partfun_{S}^{-1}(\gamma) + T_{i,i}(\rho(\gamma)) - e^{-\beta_{env} E_i} \partfun_{S}^{-1}(\gamma) }\ketbra{i}{i}} + \bigo{\alpha^{4}} \\
    &= \sum_i \abs{ e^{-(\beta_{env} + \gamma) E_i} \partfun_{S}^{-1}(\gamma) + T_{i,i}(\rho(\gamma)) - e^{-\beta_{env} E_i} \partfun_{S}^{-1}(\gamma) } + \bigo{\alpha^4},
\end{align}
where we can avoid using the triangle inequality as our operator is already diagonalized. \todo{Does the $\bigo{\alpha^4}$ term not pick up dimensional factors from the schatten-1 norm?} Now we have to use a linearization of the operator $\rho(\gamma)$ as well as the second order corrections $T_{i,i}$.
\begin{align}
    \rho(\gamma) = \frac{e^{-\beta_{env} H_{sys}}}{\partfun_{sys}(\gamma=0)} + \gamma \frac{\partial}{\partial \gamma} \frac{e^{-\beta_{env} H_{sys} + \gamma H_{sys}}}{\partfun_{sys}(\gamma)} \bigg|_{\gamma=0} + \bigo{\gamma^2}.
\end{align}
The derivative is easy to compute (I think I wrote this down on the overleaf) 
\begin{equation}
    \frac{\partial}{\partial \gamma} \frac{e^{-\beta_{env} H_{sys} + \gamma H_{sys}}}{\partfun_{sys}(\gamma)} = e^{-\beta_{env}}
\end{equation}



% Now we compute the second term, Eq. \eqref{eq:second_order_duhamel_two}. We avoid duplicating similar steps to the computation of the first computation and simply state the results (aka I need to write this up in an appendix.)
% The second term, after writing everything in terms of sums over matrix elements, becomes
% \begin{align}
%     -t^2 \parens{\bra{i,j} \rho \ket{i,j} + \bra{k,l} \rho \ket{k,l}} \sum_{i',j',k',l',m,n} \int_0^1 \int_0^1 e^{i(E_{i,j} - E_{m,n})s_1 t} e^{i(E_{m,n} - E_{k,l})s_1 s_2 t} s_1 ds_1 ds_2 \\
%     \times \bra{i',j'} D \ket{i',j'} \bra{k',l'} D \ket{k',l'} \\
%     \times \bra{i,j} U_G \ket{i',j'} \bra{m,n} U_G \ket{k',l'} \bra{m,n} \overline{U_G} \ket{i',j'} \bra{k,l} \overline{U_G} \ket{k',l'}
% \end{align}
% Plugging in similar results and integrating over the randomized interaction yields 
% \begin{align}
%     \sum_{i',j',k',l',m,n}\frac{\bra{i,j} \rho \ket{i,j} \sigma^2 }{E_{m,n} - E_{k,l}} \delta_{i',k'} \delta_{j',l'} \delta_{i,k} \delta_{j,l} 
% \end{align}


%%%%%%%%%%%%%%%%%%%%%%%%%%%%%%%%%%%%%%%%%%%%%%%%%%%%%%%%%%%%%%%%%%%%%%%%%%%%%%%%%%%%%%%%%%%%%%%%%%%%%%%%%%%%%%%%%%%%%%%%%%%%%%%%%%%%%%%%%%%%%%%%
%%%%%%%%%%%%%%%%%%%%%%%%%%%%%%%%%%%%%%%%%%%%%%%%%%%%%%%%%%%%%%%%%%%%%%%%%%%%%%%%%%%%%%%%%%%%%%%%%%%%%%%%%%%%%%%%%%%%%%%%%%%%%%%%%%%%%%%%%%%%%%%%
%%%%%%%%%%%%%%%%%%%%%%%%%%%%%%%%%%%%%%%%%%%%%%%%%%%%%%%%%%%%%%%%%%%%%%%%%%%%%%%%%%%%%%%%%%%%%%%%%%%%%%%%%%%%%%%%%%%%%%%%%%%%%%%%%%%%%%%%%%%%%%%%
\section{Energy and Entropic concerns}
One requirement that our construction should satisfy is that a system in a thermal state should be invariant under contact with an environment at the same temperature. Formally, we would like that $\Phi \parens{\frac{e^{-\beta H_{sys}}}{\partfun_{sys}} \otimes \frac{e^{- \beta H_{env}}}{\partfun_{env}}} = \frac{e^{-\beta H_{sys}}}{\partfun_{sys}}$. Exact equality may be too strong to enforce, so we instead consider that the energy of the system remains unchanged after coupling with the environment. 

\begin{equation}
    \trace{H_{sys} \frac{e^{-\beta H_{sys}}}{\partfun_{sys}}} = \trace{H_{sys} \Phi \parens{\frac{e^{-\beta H_{sys}}}{\partfun_{sys}} \otimes \frac{e^{- \beta H_{env}}}{\partfun_{env}}}}.
\end{equation}
Now we look at the RHS in detail
\begin{align}
    \trace{H_{sys} \Phi \parens{\rho} } &= \trace{ H_{sys} \partrace{env}{\int e^{i \widetilde{H} t} \rho e^{-i \widetilde{H} t} dG}} \\
    &= \int \trace{H_{sys} \otimes \openone_{env} e^{i \widetilde{H} t} \rho e^{-i \widetilde{H} t} } dG \\
    &=  \trace{\int e^{- i (H + \alpha G) t} H_{sys} \otimes \openone_{env} e^{+ i (H + \alpha G) t} dG ~ \rho } 
\end{align}


%%%%%%%%%%%%%%%%%%%%%%%%%%%%%%%%%%%%%%%%%%%%%%%%%%%%%%%%%%%%%%%%%%%%%%%%%%%%%%%%%%%%%%%%%%%%%%%%%%%%%%%%%%%%%%%%%%%%%%%%%%%%%%%%%%%%%%%%%%%%%%%%
%%%%%%%%%%%%%%%%%%%%%%%%%%%%%%%%%%%%%%%%%%%%%%%%%%%%%%%%%%%%%%%%%%%%%%%%%%%%%%%%%%%%%%%%%%%%%%%%%%%%%%%%%%%%%%%%%%%%%%%%%%%%%%%%%%%%%%%%%%%%%%%%
%%%%%%%%%%%%%%%%%%%%%%%%%%%%%%%%%%%%%%%%%%%%%%%%%%%%%%%%%%%%%%%%%%%%%%%%%%%%%%%%%%%%%%%%%%%%%%%%%%%%%%%%%%%%%%%%%%%%%%%%%%%%%%%%%%%%%%%%%%%%%%%%
\section{Numerics}



%%%%%%%%%%%%%%%%%%%%%%%%%%%%%%%%%%%%%%%%%%%%%%%%%%%%%%%%%%%%%%%%%%%%%%%%%%%%%%%%%%%%%%%%%%%%%%%%%%%%%%%%%%%%%%%%%%%%%%%%%%%%%%%%%%%%%%%%%%%%%%%%
%%%%%%%%%%%%%%%%%%%%%%%%%%%%%%%%%%%%%%%%%%%%%%%%%%%%%%%%%%%%%%%%%%%%%%%%%%%%%%%%%%%%%%%%%%%%%%%%%%%%%%%%%%%%%%%%%%%%%%%%%%%%%%%%%%%%%%%%%%%%%%%%
%%%%%%%%%%%%%%%%%%%%%%%%%%%%%%%%%%%%%%%%%%%%%%%%%%%%%%%%%%%%%%%%%%%%%%%%%%%%%%%%%%%%%%%%%%%%%%%%%%%%%%%%%%%%%%%%%%%%%%%%%%%%%%%%%%%%%%%%%%%%%%%%
\section{Approximating $e^{i (H + \alpha G) t}$ evolution}


\section{Leftovers}
%%%%%%%%%%%%%%%%%%%%%%%%%%%%%%%%%%%%%%%%%%%%%%%%%%%%%%%%%%%%%%%%%%%%%%%%%%%%%%%%%%%%%%%%%%%%%%%%%%%%%%%%%%%%%%%%%%%%%%%%%%%%%%%%%%%%%%%%%%%%%%%%
%%%%%%%%%%%%%%%%%%%%%%%%%%%%%%%%%%%%%%%%%%%%%%%%%%%%%%%%%%%%%%%%%%%%%%%%%%%%%%%%%%%%%%%%%%%%%%%%%%%%%%%%%%%%%%%%%%%%%%%%%%%%%%%%%%%%%%%%%%%%%%%%



%%%%%%%%%%%%%%%%%%%%%%%%%%%%%%%%%%%%%%%%%%%%%%%%%%%%%%%%%%%%%%%%%%%%%%%%%%%%%%%%%%%%%%%%%%%%%%%%%%%%%%%%%%%%%%%%%%%%%%%%%%%%%%%%%%%%%%%%%%%%%%%%
%%%%%%%%%%%%%%%%%%%%%%%%%%%%%%%%%%%%%%%%%%%%%%%%%%%%%%%%%%%%%%%%%%%%%%%%%%%%%%%%%%%%%%%%%%%%%%%%%%%%%%%%%%%%%%%%%%%%%%%%%%%%%%%%%%%%%%%%%%%%%%%%
\subsection{Simplifying $\Phi(\rho)$ with Trotter}
The idea here is to try and see what very short time evolution does to our channel. At the very least this might give some intuition as to what is happening?
\begin{align}
    \Phi_G(\rho_{PS}) &= \partrace{2}{e^{+i\widetilde{H}t} \rho_{PS} e^{-i \widetilde{H} t}} \\
    &= \partrace{2}{\parens{e^{i G \alpha t} e^{i H t} + \bigo{t^2}} \rho_{PS} \parens{e^{-iHt} e^{-i G \alpha t} + \bigo{t^2}}} \\
    &= \partrace{2}{e^{i G \alpha t} \rho_{PS} e^{-i G \alpha t}} + \bigo{t^2} 
\end{align}

\section{Energy and Entropic concerns}
One requirement that our construction should satisfy is that a system in a thermal state should be invariant under contact with an environment at the same temperature. Formally, we would like that $\Phi \parens{\frac{e^{-\beta H_{sys}}}{\partfun_{sys}} \otimes \frac{e^{- \beta H_{env}}}{\partfun_{env}}} = \frac{e^{-\beta H_{sys}}}{\partfun_{sys}}$. Exact equality may be too strong to enforce, so we instead consider that the energy of the system remains unchanged after coupling with the environment. 

\begin{equation}
    \trace{H_{sys} \frac{e^{-\beta H_{sys}}}{\partfun_{sys}}} = \trace{H_{sys} \Phi \parens{\frac{e^{-\beta H_{sys}}}{\partfun_{sys}} \otimes \frac{e^{- \beta H_{env}}}{\partfun_{env}}}}.
\end{equation}
Now we look at the RHS in detail
\begin{align}
    \trace{H_{sys} \Phi \parens{\rho} } &= \trace{ H_{sys} \partrace{env}{\int e^{i \widetilde{H} t} \rho e^{-i \widetilde{H} t} dG}} \\
    &= \int \trace{H_{sys} \otimes \openone_{env} e^{i \widetilde{H} t} \rho e^{-i \widetilde{H} t} } dG \\
    &=  \trace{\int e^{- i (H + \alpha G) t} H_{sys} \otimes \openone_{env} e^{+ i (H + \alpha G) t} dG ~ \rho }
\end{align}

 \subsection{Phase Contribution complicated}
 Now we investigate the phase integrals $C_E$. 
 \begin{align}
     C_E &= \int_0^1 \int_0^1 \bigg(  2 \rho_{m,n} e^{i(E_{i,j} - E_{m,n})s_1t} e^{i(E_{m,n} - E_{k,l})s_2 t}  \\
     &- (\rho_{i,j} + \rho_{k,l}) e^{i(E_{i,j} - E_{m,n})s_1 t} e^{i(E_{m,n} - E_{k,l})s_1 s_2 t} s_1 \\
     &- (\rho_{i,j} + \rho_{k,l}) e^{i(E_{i,j} - E_{m,n})s_1 t} e^{i(E_{m,n} - E_{k,l})(s_1 + s_2 - s_1 s_2)t} (1-s_1) \bigg) ds_1 ds_2 \\
 \end{align}
 These can be solved analytically as
 \begin{align}
     \int_0^1 \int_0^1 e^{i(E_{i,j} - E_{m,n})s_1t} e^{i(E_{m,n} - E_{k,l})s_2 t} ds_1 ds_2 &= \frac{(e^{i(E_{i,j} - E_{m,n})t} - 1)(e^{i(E_{m,n} - E_{k,l})t} - 1)}{t^2 (E_{i,j} - E_{m,n})(E_{k,l} - E_{m,n})} \\
     \int_0^1 \int_0^1 e^{i(E_{i,j} - E_{m,n})s_1 t} e^{i(E_{m,n} - E_{k,l})s_1 s_2 t} s_1 ds_1 ds_2 &= \nonumber \\
     \frac{e^{i(E_{i,j} - E_{k,l})t}(E_{i,j} - E_{m,n}) + (E_{m,n} - E_{k,l}) - e^{i(E_{i,j} - E_{m,n})t}(E_{i,j} - E_{k,l})}{t^2(E_{i,j} - E_{m,n})(E_{k,l} - E_{m,n})(E_{i,j} - E_{k,l})} \\
     \int_0^1 \int_0^1 e^{i(E_{i,j} - E_{m,n})s_1 t} e^{i(E_{m,n} - E_{k,l})(s_1 + s_2 - s_1 s_2)t} (1-s_1) ds_1 ds_2 &= \nonumber \\
     \frac{-(E_{i,j} - E_{m,n}) + e^{i(E_{m,n} - E_{k,l})t}\parens{E_{i,j} - E_{k,l} - e^{i(E_{i,j} - E_{m,n})t}(E_{m,n} - E_{k,l})}}{t^2 (E_{i,j} - E_{m,n})(E_{m,n} - E_{k,l})(E_{i,j} - E_{k,l})}
 \end{align}
 Now we introduce some minor notation to help readability $\Delta_{ij,kl} := E_{i,j} - E_{k,l}$. This reduces the phase contribution to
 \begin{align}
      t^2 C_E(\rho) = &e^{i\Delta_{ij,kl}t} \parens{\frac{2 \rho_{m,n}}{\Delta_{ij,mn} \Delta_{kl,mn}} + \frac{(\rho_{ij} + \rho_{k,l}) \Delta_{ij,mn}}{\Delta_{ij,kl} \Delta_{ij,mn} \Delta_{kl,mn}}} \\
     +& e^{i \Delta_{ij,mn}t} \parens{\frac{- 2 \rho_{m,n}}{\Delta_{ij,mn} \Delta_{kl,mn}} - \frac{(\rho_{ij} + \rho_{k,l}) \Delta_{mn,kl}}{\Delta_{ij,mn} \Delta_{ij,kl} \Delta_{kl,mn}} - \frac{(\rho_{i,j} + \rho_{k,l})\Delta_{ij,kl}}{\Delta_{ij,mn} \Delta_{ij,kl} \Delta_{kl,mn}}} \\
     +& e^{i \Delta_{mn,kl}t} \parens{\frac{- 2 \rho_{m,n}}{\Delta_{ij,mn} \Delta_{kl,mn}} - \frac{(\rho_{i,j} + \rho_{k,l}) \Delta_{ij,kl}}{\Delta_{ij,mn} \Delta_{ij,kl} \Delta_{kl,mn}}} \\
     +& \parens{2 \rho_{m,n} + (\rho_{i,j} + \rho_{k,l}) \frac{\Delta_{mn,kl} + \Delta_{ij,mn}}{\Delta_{ij,kl} \Delta_{ij,mn} \Delta_{kl,mn}}}
 \end{align}
 
 Since we have the factor of $\delta_{i,j | k,l}$ from the Haar integral contribution we can further simplify these as
%%%%%%%%%%%%%%%%%%%%%%%%%%%%%%%%%%%%%%%%%%%%%%%%%%%%%%%%%%%%%%%%%%%%%%%%%%%%%%%%%%%%%%%%%%%%%%%%%%%%%%%%%%%%%%%%%%%%%%%%%%%%%%%%%%%%%%%%%%%%%%%%
%%%%%%%%%%%%%%%%%%%%%%%%%%%%%%%%%%%%%%%%%%%%%%%%%%%%%%%%%%%%%%%%%%%%%%%%%%%%%%%%%%%%%%%%%%%%%%%%%%%%%%%%%%%%%%%%%%%%%%%%%%%%%%%%%%%%%%%%%%%%%%%%
%%%%%%%%%%%%%%%%%%%%%%%%%%%%%%%%%%%%%%%%%%%%%%%%%%%%%%%%%%%%%%%%%%%%%%%%%%%%%%%%%%%%%%%%%%%%%%%%%%%%%%%%%%%%%%%%%%%%%%%%%%%%%%%%%%%%%%%%%%%%%%%%
\section{Approximating $e^{i (H + \alpha G) t}$ evolution}

\bibliographystyle{unsrt}
\bibliography{bib}

\appendix
\section{Perturbation Theory Attempt}
The goal is to use perturbation theory to write the output of the channel in terms of powers of the coupling strength $\alpha$. It remains to be seen if first order perturbation theory will be enough. We also are specifically interested when $\rho$ is a product state of thermal states, in other words $\rho_{PS} = \frac{e^{-\beta_1 H_1}}{\partfun_1} \otimes \frac{e^{-\beta_2 H_2}}{\partfun_2}$. We also denote the action of our channel with respect to a particular choice of $G$ as $\Phi_G$, which combined with linearity of partial traces gives $\Phi = \int dG \Phi_G$. 
\begin{align}
    \Phi_G(\rho_{PS}) &= \partrace{2}{e^{+i \widetilde{H} t} \rho_{PS} e^{-i \widetilde{H} t}} \\
    &= \sum_{k} \openone \otimes v_k^* \parens{\sum_{i, j} e^{i \widetilde{\eta}_{i, j} t} \widetilde{\Pi}_{i,j}} \rho_{PS} \parens{ \sum_{m, n} e^{i \widetilde{\eta}_{m, n} t} \widetilde{\Pi}_{m,n} }
\end{align}
Our next goal is to analyze $e^{i \widetilde{\eta}_{i,j} t} \widetilde{\Pi}_{i,j}$ using time-independent perturbation theory. Our goal is to write $e^{i \widetilde{\eta}_{i,j} t}$ as a power series in terms of $\alpha$, which is given by perturbation theory for $\widetilde{H} = H + \alpha G$. We first write the perturbation series for $\widetilde{\eta}$ as 
\begin{equation}
    \widetilde{\eta}_{i,j} = \eta_{i,j} + \alpha (u_i \otimes v_j)^* G (u_i \otimes v_j) + \alpha^2 \sum_{(i',j') \neq (i,j)}  \frac{\abs{(u_i \otimes v_j)^* G (u_{i'} \otimes v_{j'})}^{2}}{\eta_{i,j} - \eta_{i',j'}} + \bigo{\alpha^3}.
\end{equation}
To simplify the notation, we introduce the indexing $G(i,j,k,l) := (u_i \otimes v_j)^* G (u_k \otimes v_l)$. $G(i,j)$ without a second pair of indices denotes the diagonal matrix element $(u_i \otimes v_j)^* G (u_i \otimes v_j)$. The Taylor's Series for $e^{i \widetilde{\eta}_{i,j} t}$ is
\begin{align}
    e^{i \widetilde{\eta}_{i,j} t} &= e^{i \eta_{i,j} t} \\
    &\quad + \frac{\partial \widetilde{\eta}_{i,j}}{\partial \alpha} e^{i \widetilde{\eta}_{i,j}t} \bigg|_{\alpha = 0} i \alpha t \\
    &\quad + \frac{ i\alpha^2 t}{2!} \frac{\partial^2 \widetilde{\eta}_{i,j}}{\partial \alpha^2} e^{i \widetilde{\eta}_{i,j}t}\bigg|_{\alpha=0} \\
    &\quad - \frac{\alpha^2 t^2}{2!} \parens{\frac{\partial \widetilde{\eta}_{i,j}}{\partial \alpha}}^2 e^{i \widetilde{\eta}_{i,j} t} \bigg|_{\alpha=0} + \bigo{\alpha^3} .
\end{align}
These are rather straightforward to compute and plug in, yielding
\begin{equation}
    e^{i \widetilde{\eta}_{ij} t} = e^{i \eta_{ij} t}\parens{1 + i \alpha t G(i,j) - \frac{\alpha^2 t^2}{2} G(i,j)^2 + i \frac{\alpha^2 t}{2} \sum_{(i',j') \neq (i,j)}\frac{\abs{G(i',j',i,j)}^2}{\eta_{ij} - \eta_{i' j'}} + \bigo{\alpha^3}}.
\end{equation}
This is sloppy, need to be able to bound the $\bigo{\alpha^3}$ terms better. Also note that $G(i,j)$ is a real Gaussian variable, the realness is due to the Hermiticity constraint on $G$. 

The other quantity we need to compute a perturbation series for is $\widetilde{\Pi}_{ij}$. We first need to compute the perturbation series for the eigenvectors of $H$, which we do to second order
\begin{align}
    \widetilde{u_i \otimes v_j} &= u_i \otimes v_j \\
    &\quad + \alpha \sum_{(i',j') \neq (i,j)} u_{i'} \otimes v_{j'} \parens{  \frac{G(i,j,i',j')}{\eta_{i,j} - \eta_{i',j'}} } \\
    &\quad + \alpha^2 \sum_{(i',j') \neq (i,j)} u_{i'} \otimes v_{j'} \parens{ \sum_{(i'',j'') \neq (i,j)} \frac{G(i,j,i'',j'') G(i'',j'',i',j')}{(\eta_{i,j} - \eta_{i'',j''})(\eta_{i,j} - \eta_{i'',j''})} -  \frac{G(i,j,i,j) G(i,j,i',j')}{(\eta_{i,j} - \eta_{i',j'})^2}} \\
    &\quad + \bigo{\alpha^3}.
\end{align}
Now we need to compute the perturbed eigenspace projectors
\begin{align}
    \widetilde{\Pi}_{i,j} &= \Pi_{i,j} \\
    & \quad + \alpha \sum_{(i',j') \neq (i,j)} u_{i'} u_i^* \otimes v_{j'} v_j^* \parens{  \frac{G(i,j,i',j')}{\eta_{i,j} - \eta_{i',j'}} } + \sum_{(i',j') \neq (i,j)} u_i u_{i'}^* \otimes v_j v_{j'}^* \parens{  \frac{G(i,j,i',j')^*}{\eta_{i,j} - \eta_{i',j'}} } \\
    & \quad + \alpha^2 \sum_{(i',j') \neq (i,j)} u_{i'} u_i^* \otimes v_{j'} v_j^* \parens{ \sum_{(i'',j'') \neq (i,j)} \frac{G(i,j,i'',j'') G(i'',j'',i',j')}{(\eta_{i,j} - \eta_{i'',j''})(\eta_{i,j} - \eta_{i'',j''})} -  \frac{G(i,j,i,j) G(i,j,i',j')}{(\eta_{i,j} - \eta_{i',j'})^2}} \\
    & \quad + \alpha^2 \sum_{(i',j') \neq (i,j)} u_i u_{i'}^* \otimes v_j v_{j'}^* \parens{ \sum_{(i'',j'') \neq (i,j)} \frac{G(i,j,i'',j'')^* G(i'',j'',i',j')^*}{(\eta_{i,j} - \eta_{i'',j''})(\eta_{i,j} - \eta_{i'',j''})} -  \frac{G(i,j,i,j)^* G(i,j,i',j')^*}{(\eta_{i,j} - \eta_{i',j'})^2}} \\
    & \quad + \alpha^2 \sum_{(i',j') \neq (i,j)} \sum_{(i'',j'') \neq (i,j)} u_{i'}u_{i''}^* \otimes v_{j'} v_{j''}^* \frac{G(i,j,i',j') G(i,j,i'',j'')^*}{(\eta_{i,j} - \eta_{i',j'})(\eta_{i,j} - \eta_{i'',j''})} \\
    & \quad + \bigo{\alpha^3}.
\end{align}

This allows us to write the output of the channel $\Phi$ as a power series in $\alpha$. The expressions can become quite cumbersome, so we introduce the following notation:
\begin{align}
    M_{i,j,i',j'} &:=  u_{i'} u_i^* \otimes v_{j'} v_j^* \parens{  \frac{G(i,j,i',j')}{\eta_{i,j} - \eta_{i',j'}} } +  u_i u_{i'}^* \otimes v_j v_{j'}^* \parens{  \frac{G(i,j,i',j')^*}{\eta_{i,j} - \eta_{i',j'}} } \\
    N_{i,j,i',j'} &:= u_{i'} u_i^* \otimes v_{j'} v_j^* \parens{ \sum_{(i'',j'') \neq (i,j)} \frac{G(i,j,i'',j'') G(i'',j'',i',j')}{(\eta_{i,j} - \eta_{i'',j''})(\eta_{i,j} - \eta_{i'',j''})} -  \frac{G(i,j,i,j) G(i,j,i',j')}{(\eta_{i,j} - \eta_{i',j'})^2}} \\
    & \quad + \sum_{(i'',j'') \neq (i,j)} u_{i'}u_{i''}^* \otimes v_{j'} v_{j''}^* \frac{G(i,j,i',j') G(i,j,i'',j'')^*}{(\eta_{i,j} - \eta_{i',j'})(\eta_{i,j} - \eta_{i'',j''})}
\end{align}
which captures the first and second order corrections to the eigenspace projectors. We can then simplify
\begin{align}
    \widetilde{\Pi}_{i,j} &= \Pi_{i,j} + \alpha \sum_{(i',j') \neq (i,j)} M_{i,j,i',j'} + \alpha^2 \sum_{(i',j') \neq (i,j)} N_{i,j,i',j'} + \bigo{\alpha^3}
\end{align}
The second order corrections have 10 possible sources for terms, however many are hermitian conjugates of priors, so if herm. conj. appears it refers to the hermitian conjugate of the entire previous line.
\newpage
\begin{equation}
    e^{+i(H + \alpha G)t} \rho e^{-i (H + \alpha G) t}
\end{equation}
\begin{align}
\alpha^0 :& e^{+i H t} \rho e^{-i H t} \\
 \alpha^1 :& \sum_{i,j} e^{i\eta_{i,j}t} \parens{i t G(i,j) \Pi_{i,j} + \sum_{(i',j') \neq (i,j)} M_{i,j,i',j'}} \rho e^{- i H t} \\
 &+ \text{herm. conj.} \\
 \alpha^2 :& \sum_{i,j} i t G(i,j) e^{i \eta_{i,j}} \Pi_{i,j} \rho \sum_{k,l} (-i t) G(k,l)^* e^{-i \eta_{k,l}t} \Pi_{k,l} \\
 &+ \sum_{i,j} e^{i \eta_{i,j} t} \sum_{(i',j') \neq (i,j)} M_{i,j,i',j'} \rho \sum_{k,l} (-i t) G(k,l) e^{-i \eta_{k,l} t} \Pi_{k,l} \\
 &+ \text{herm. conj.} \\
 &+ \sum_{i,j} e^{i \eta_{i,j} t} \sum_{(i',j') \neq (i,j)} M_{i,j,i',j'} \rho \sum_{k,l} e^{-i \eta_{k,l} t} \sum_{(k',l') \neq (k,l)} M_{k,l,k',l'}\\
 &+  \sum_{i,j} e^{i \eta_{i,j} t} \parens{- \frac{t^2 G(i,j)^2}{2} + i \frac{t}{2} \sum_{(i',j') \neq (i,j)} \frac{\abs{G(i,j,i',j')}^2}{\eta_{i,j} - \eta_{i',j'}} } \rho e^{- i H t} \\
 &+ \text{herm. conj.} \\
 &+ \sum_{i,j} e^{i \eta_{i,j} t} \sum_{(i',j') \neq (i,j)} N_{i,j,i',j'} \rho e^{-i H t} \\
 &+ \text{herm. conj.} \\
 &+ \sum_{i,j} i t G(i,j) e^{i \eta_{i,j} t} \sum_{(i',j') \neq (i,j)} M_{i,j,i',j'} \rho e^{-i H t} \\
 &+ \text{herm. conj.}.
\end{align}

Two potential issues. 1) How can we normalize the output state? This may depend on how we approach the second. 2) How do we bound the remaining terms in the perturbation series? Are there any dumb bounds I can work out? 

\end{document}
