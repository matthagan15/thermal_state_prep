\documentclass{article}
\usepackage[utf8]{inputenc}

\usepackage{amsmath,amsthm, amssymb}
\usepackage[margin=3cm]{geometry}
\usepackage{mathtools}
\usepackage{dsfont}
\usepackage{xcolor}
\usepackage{algorithm,algpseudocode}
\usepackage{todonotes}
\usepackage{nicefrac}
\usepackage{mathrsfs}
\usepackage{tikz}
\usepackage{thm-restate}

\usepackage{etoc}

%%%%%%%%    THEOREM DEFINITIONS AND RESTATABLE
% \newcounter{claim}
% \setcounter{claim}{0}
\newtheorem{theorem}{Theorem}
\newtheorem{lemma}[theorem]{Lemma}
\newtheorem{corollary}[theorem]{Corollary}
\newtheorem{claim}{Claim}
\newtheorem{dependency}{Dependency}
\newtheorem{definition}{Definition}

\newcommand{\matt}[1]{\todo[color=red!50, prepend, caption={Matt}, tickmarkheight=0.25cm]{#1}}
\newcommand{\inlinetodo}[1]{\textcolor{red}{{\Large TODO:} #1}}




%%%%%%%%    NOTATION DEFINITIONS FOR EASIER WRITING
\newcommand{\ket}[1]{|#1\rangle}
\newcommand{\bra}[1]{\langle #1|}
\newcommand{\braket}[2]{\langle #1|#2\rangle}
\newcommand{\ketbra}[2]{| #1\rangle\! \langle #2|}
\newcommand{\parens}[1]{\left( #1 \right)}
\newcommand{\brackets}[1]{\left[ #1 \right]}
\newcommand{\abs}[1]{\left| #1 \right|}
\newcommand{\norm}[1]{\left| \left| #1 \right| \right|}
\newcommand{\diamondnorm}[1]{\left| \left| #1 \right| \right|_\diamond}
\newcommand{\anglebrackets}[1]{\left< #1 \right>}
\newcommand{\overlap}[2]{\anglebrackets{#1 , #2 }}
\newcommand{\set}[1]{\left\{ #1 \right\}}
\newcommand{\ceil}[1]{\left\lceil #1 \right\rceil}
\newcommand{\openone}{\mathds{1}}
\newcommand{\expect}[1]{\mathbb{E}\brackets{#1}}
\newcommand{\variance}[1]{\textit{Var} \brackets{ #1 }}
\newcommand{\prob}[1]{\text{Pr}\left[ #1 \right]}
\newcommand{\bigo}[1]{O\left( #1 \right)}
\newcommand{\bigotilde}[1]{\widetilde{O} \left( #1 \right)}
\newcommand{\ts}{\textsuperscript}

\DeclareMathOperator{\Tr}{Tr}
\newcommand{\trace}[1]{\Tr \brackets{ #1 }}
\newcommand{\partrace}[2]{\Tr_{#1} \brackets{ #2 }}
\newcommand{\complex}{\mathbb{C}}

%%%%% COMMONLY USED OBJECTS
\newcommand{\hilb}{\mathcal{H}}
\newcommand{\partfun}{\mathcal{Z}}
\newcommand{\identity}{\mathds{1}}
\newcommand{\gue}{\rm GUE}
\DeclareMathOperator{\sinc}{sinc}
\DeclareMathOperator{\hermMathOp}{Herm}
\DeclareMathOperator{\im}{Im}
\newcommand{\herm}[1]{\hermMathOp\parens{#1}}


\title{Thermal State Prep}
\author{Matthew Hagan, Nathan Wiebe}
\date{May 2022}

\begin{document}

\maketitle

\tableofcontents

%%%%%%%%%%%%%%%%%%%%%%%%%%%%%%%%%%%%%%%%%%%%%%%%%%%%%%%%%%%%%%%%%%%%%%%%%%%%%%%%%%%%%%%%%%%%%%%%%%%%%%%%%%%%%%%%%%%%%%%%%%%%%%%%%%%%%%%%%%%%%%%%
%%%%%%%%%%%%%%%%%%%%%%%%%%%%%%%%%%%%%%%%%%%%%%%%%%%%%%%%%%%%%%%%%%%%%%%%%%%%%%%%%%%%%%%%%%%%%%%%%%%%%%%%%%%%%%%%%%%%%%%%%%%%%%%%%%%%%%%%%%%%%%%%
%%%%%%%%%%%%%%%%%%%%%%%%%%%%%%%%%%%%%%%%%%%%%%%%%%%%%%%%%%%%%%%%%%%%%%%%%%%%%%%%%%%%%%%%%%%%%%%%%%%%%%%%%%%%%%%%%%%%%%%%%%%%%%%%%%%%%%%%%%%%%%%%
\section{Introduction}
Going to leave this blank for now. \cite{shiraishi_undecidability_2021}

%%%%%%%%%%%%%%%%%%%%%%%%%%%%%%%%%%%%%%%%%%%%%%%%%%%%%%%%%%%%%%%%%%%%%%%%%%%%%%%%%%%%%%%%%%%%%%%%%%%%%%%%%%%%%%%%%%%%%%%%%%%%%%%%%%%%%%%%%%%%%%%%
%%%%%%%%%%%%%%%%%%%%%%%%%%%%%%%%%%%%%%%%%%%%%%%%%%%%%%%%%%%%%%%%%%%%%%%%%%%%%%%%%%%%%%%%%%%%%%%%%%%%%%%%%%%%%%%%%%%%%%%%%%%%%%%%%%%%%%%%%%%%%%%%
%%%%%%%%%%%%%%%%%%%%%%%%%%%%%%%%%%%%%%%%%%%%%%%%%%%%%%%%%%%%%%%%%%%%%%%%%%%%%%%%%%%%%%%%%%%%%%%%%%%%%%%%%%%%%%%%%%%%%%%%%%%%%%%%%%%%%%%%%%%%%%%%
\section{Preliminaries}
We denote the Hilbert space of the system as $\hilb_{S}$ and the environment as $\hilb_{E}$, with the Hamiltonians governing each as $H_{S}$ and $H_{E}$. We will assume without loss of generality that the system's Hilbert space can be encoded with $n$ qubits, giving $\dim_S = 2^{n}$, and the environment's Hilbert space can be encoded with $m$ qubits giving $\dim_E = 2^{m}$. The Hamiltonian for the joint system on $\hilb_{S} \otimes \hilb_{E}$ is then $H = H_{S} \otimes \identity + \identity \otimes H_{E}$. The Hilbert space of the combined system and environment is of dimension $\dim = \dim_E \cdot \dim_S = 2^{n + m}$. 

We will primarily work in the eigenbasis for each Hamiltonian:
\begin{equation}
    H_{S} = \sum_{i = 0}^{2^n - 1} \lambda_S(i) \ketbra{s_i}{s_i} ~,~ H_{E} = \sum_{j=0}^{2^m - 1} \lambda_E(j) \ketbra{e_j}{e_j} ~,~ H = \sum_{i=0}^{2^n - 1} \sum_{j=0}^{2^m - 1} \lambda(i,j) (\ket{s_i} \otimes \ket{e_j})(\bra{s_i} \otimes \bra{e_j}),
\end{equation}
for convenience we will denote the tensor product of eigenvectors simply by their indices $\ket{i,j} \coloneqq \ket{s_i} \otimes \ket{e_j}$. For convenience we define $\lambda(i,j) \coloneqq \lambda_S(i) + \lambda_E(j)$. We also make use of the following notation for the energy differences of the system-environment Hamiltonian
$$\Delta(i,j|k,l) \coloneqq \lambda(i,j) - \lambda(k,l),$$
and will use $\Delta_S(i,j) \coloneqq \lambda_S(i) - \lambda_S(j)$ for just the system differences. To define the degeneracy of each eigenvalue we use the symbol
$$\eta(i,j) \coloneqq \sum_{(k,l) : \Delta(i,j|k,l) = 0} 1.$$ We use the notation $\delta(i,j|k,l)$ to denote the product of Kronecker delta functions $\delta(i,j|k,l) = \delta_{i,k} \delta_{j,l}$.

For input states we will typically assume thermal states of the form $\rho_S(\beta) = \frac{e^{-\beta H_S}}{\partfun_S}$, where $\partfun_S = \trace{e^{-\beta H_S}}$, where the inverse temperature $\beta$ of the partition function will typically be assumed but written explicitly if need be. We will assume environment states of the form $\rho_E(\beta) = \frac{e^{-\beta H_E}}{\partfun_E}$ and we will typically denote the tensor product of the system and environment states as $\rho(\beta_S, \beta_E) = \rho_S(\beta_S) \otimes \rho_E(\beta_E)$. The inputs $\beta_S, \beta_E$ will typically be surpressed in most cases as well.


Overall one application of our channel is represented as
\begin{equation}
    \Phi(\rho) := \int \partrace{\hilb_E}{e^{+i(H + \alpha G)t} \rho e^{-i(H + \alpha G) t}} dG.
\end{equation}
It will prove convenient to study the time evolution of a fixed interaction as a map from the total system-environment space to itself. We denote this channel for a specific random interaction $G$ as
\begin{equation}
    \Phi_G(\rho_S \otimes \rho_E) := e^{+i (H+ \alpha G) t} \rho_S \otimes \rho_E e^{-i (H + \alpha G) t}. \label{eq:phi_g_definition}
\end{equation}
Clearly then $\Phi(\rho_S) = \partrace{env}{\int \Phi_G (\rho_S \otimes \rho_E) dG}$. We use $G$ to denote the randomized interaction term, where $G = U_G D U_G^\dagger$. The measure we choose for the eigenbasis of $G$ is $U_G \sim Haar$ and the eigenvalues are i.i.d with mean 0 and variance $1$.  This gives the overall interaction measure as the decomposition $\int dG = \int \int dD ~ dU_G$. The interaction strength of $G$ will be controlled through the coupling coefficient $\alpha$.

\begin{restatable}{lemma}{haar_two_moment} \label{lem:haar_two_moment}
    Let $U$ be a unitary matrix over $\dim$ dimensions that is distributed according to the Haar measure. Then the following average is
    \begin{align}
        &\int \bra{i_1} U \ket{j_1} \bra{i_2} U \ket{j_2} \bra{k_1} U^\dagger \ket{l_1} ~ \bra{k_2} U^\dagger \ket{l_2} dU \nonumber \\
        &= ~\frac{1}{\dim^2 - 1} \parens{\delta_{i_1, l_1} \delta_{j_1, k_1} \delta_{i_2, l_2} \delta_{j_2, k_2} + \delta_{i_1, l_2} \delta_{j_1, k_2} \delta_{i_2, l_1} \delta_{j_2, k_1}} \nonumber \\
        &\quad - \frac{1}{\dim(\dim^2 - 1)} \parens{\delta_{i_1, l_2} \delta_{j_1, k_1} \delta_{i_2, l_1} \delta_{j_2, k_2} + \delta_{i_1, l_1} \delta_{j_1, k_2} \delta_{i_2, l_2} \delta_{j_2, k_1}}. \label{eq:haar_two_moment_integral}
    \end{align}
\end{restatable}

\begin{lemma}[Sinc Function Bounds] \label{lem:sinc_poly_approx}
    The following implications hold 
    \begin{align}
        |x| \leq \sqrt{10 \epsilon_{\sinc}/7} &\implies \sinc^2(x) \geq 1 - \epsilon_{\sinc} \label{eq:sinc_lower_bound}\\
        |x| \geq 1 / \sqrt{\epsilon_{\sinc}} &\implies \sinc^2(x) \leq \epsilon_{\sinc}. \label{eq:sinc_upper_bound}
    \end{align}
    % The constant approximation $f(x) = 1$ has error $|f(x) - 1| \leq \widetilde{\epsilon}_{\sinc}$ if $|x| \leq \sqrt{2 \widetilde{\epsilon}_{\sinc}}$. This leads to the observation that $\widetilde{\epsilon}_{\sinc}$ acts as a lower bound for $f(x)$, as $|x| \leq \sqrt{2 \widetilde{\epsilon}_{\sinc}}$ implies $f(x) \geq 1 - \widetilde{\epsilon}_{\sinc}$. We denote this upper bound with $\Delta_{\sinc} \coloneqq \sqrt{2 \widetilde{\epsilon}_{\sinc}}$. 
    % We also require a lower bound $\epsilon_{\sinc}$, such that $|x| \geq \Delta_{\min} \implies f(x) \leq \epsilon_{\sinc}$. Setting $\Delta_{\min} = 1 / \sqrt{\epsilon_{\sinc}}$ guarantees this to hold. 
\end{lemma}
\begin{proof}
    We start with a Taylor Series for $\sinc^2$, which we compute using the expression of $\sinc$ as $\sinc(x) = \frac{\sin x}{x} = \int_0^1 \cos(sx) ds$. We compute the first two derivatives as
    \begin{align}
        \frac{d \sinc^2(x)}{dx} &= -2 \int_0^1 \sin(sx) s ds \int_0^1 \cos(sx) ds \\
        \frac{d^2 \sinc^2(x)}{dx^2} &= -2 \int_0^1 \cos(sx)s^2 ds \int_0^1 \cos(sx) ds + 2\int_0^1 \sin(sx) s ~ds \int_0^1 \sin(sx) s ~ds.
    \end{align}
    We can evaluate each of these derivatives about the origin using continuity of the derivatives along with the limits $\lim_{x \to 0} \cos(sx) = 1$ and $\lim_{x \to 0} \sin(sx) = 0$. We can now compute the Maclaurin Series for some $x_{\star} \in [0,1]$ as
    \begin{align}
        f(x) &= f(0) + x \frac{df}{dx}\bigg|_{x = 0} + \frac{x^2}{2!} \frac{d^2f}{dx^2}\bigg|_{x = x_{\star}}.
    \end{align}
    Plugging in $\sinc^2(0) = 1$ and $\frac{d\sinc^2(x)}{dx}\big|_{x = 0} = 0$ then yields $|\sinc^2(x) - 1| = \frac{|x|^2}{2} \abs{\frac{d^2\sinc^2(x)}{dx^2}(x_{\star})}$. We make use of the rather simplistic bound
    \begin{align}
        \abs{\frac{d^2\sinc^2(x)}{dx^2}(x_{\star})} &\leq 2 \abs{\int_0^1 \cos(sx) s^2 ds \int_0^1 \cos(sx) ds} + 2\abs{\int_0^1 \sin(sx) s ds \int_0^1 \sin(sx) s ds} \\
        &\leq 2 \int_0^1 \abs{\cos(sx)} s^2 ds \int_0^1 \abs{\cos(sx)} ds + 2\parens{\int_0^1 \abs{\sin(sx)} |s| ds}^2 \\
        &\leq 2 \int_0^1 s^2 ds + 2\parens{\int_0^1 s ds}^2 \\
        &\leq 2/3 + 1/2 = 7/6.
    \end{align}
    This yields the final inequality $|\sinc^2(x) - 1| \leq \frac{7|x|^2}{10}$. We then see that $|x| \leq \sqrt{10 \widetilde{\epsilon}_{\sinc}/7}$ implies $|f(x) - 1| \leq \widetilde{\epsilon}_{\sinc}$. 

    The upper bound of $\epsilon_{\sinc}$ for large $|x|$ is relatively straightforward:
    \begin{align}
        f(x) &= \frac{\sin^2(x)}{x^2} \\
            &\leq \frac{1}{|x|^2},
    \end{align}
    where we see that $|x| \geq 1 / \sqrt{\epsilon_{\sinc}}$ implies $\sinc^2(x) \leq \epsilon_{\sinc}$.
\end{proof}

We will often rely on a particularly parametrized form of $f(x)$ which is worth investigating on it's own right. Note we can get a square root improvement of the dependence of $|\Delta_S(i,j) - \gamma|$ on $\epsilon_{\sinc}$ in the below Corallary if we only require $f(x) \geq 1 - \widetilde{\epsilon}_{\sinc}$. This then requires $|\Delta_S(i,j) - \gamma| \in \bigo{\sqrt{\epsilon_{\sinc} \widetilde{\epsilon}_{\sinc}}}$, however this will not prove significantly useful for us so we use the looser bound.
\begin{corollary} \label{cor:gamma_difference_reqs}
    The statements 
    $$|x| \geq \Delta_{\min} \implies \sinc^2\parens{xt / 2} \leq \epsilon_{\sinc}$$
    and
    $$|x| = |\Delta_S(i,j) - \gamma| \leq \sqrt{2} \Delta_{\min} \epsilon_{\sinc} \implies f(xt/2) \geq 1 - \epsilon_{\sinc}$$
    hold for $t = \frac{2}{\Delta_{\min} \sqrt{\epsilon_{\sinc}}}$. This gives $\epsilon_{\sinc} = \frac{4}{\Delta_{\min}^2 t^2}$. We denote the barrier $\Delta_{\sinc} = \sqrt{2} \Delta_{\min} \epsilon_{\sinc}$. 
\end{corollary}
\begin{proof}
    Throughout this proof we can think of $0 \leq \Delta_{\min} \leq \Delta_S(i,j)$ as a constant, so we avoid writing it as function arguments.
    We first want to provide a bound on $t$ such that $|x| \geq \Delta_{\min}$ implies $f(xt/2) \leq \epsilon_{\sinc}$. This is provided through Eq. \eqref{eq:sinc_upper_bound}
    \begin{align}
        \left| \frac{x t }{ 2} \right| = \frac{|x| t}{2} \geq \frac{\Delta_{\min}t}{2}.
    \end{align}
    We see that setting $t$ such that $\Delta_{\min} t / 2 = 1 /\sqrt{\epsilon_{\sinc}}$, which can be rewritten as $t = \frac{2}{\Delta_{min} \sqrt{\epsilon_{\sinc}}}$, yields the implication $|x| \geq \Delta_{\min} \implies f(xt/2) \leq \epsilon_{\sinc}$. 

    We now want to investigate what values of $x = \Delta_S(i,j) - \gamma$, for the given $t$ as above, yields $f(xt/2) \geq 1 - \epsilon_{\sinc}$. We see that the inequality required for this is
    \begin{align}
        \frac{|x| t}{2} &\leq \sqrt{2 \epsilon_{\sinc}} \\
        \iff  |\Delta_S(i,j) - \gamma| \frac{2}{2 \Delta_{\min} \sqrt{\epsilon_{\sinc}}} &\leq \sqrt{2 \epsilon_{\sinc}} \\
        \iff \abs{\Delta_S(i,j) - \gamma} &\leq \sqrt{2} \Delta_{\min} \epsilon_{\sinc}
    \end{align}
\end{proof}


We now see that if we want there to be unique $(i,j)$ such that $|\Delta_S(i,j) - \gamma| \leq \Delta_{\sinc}$ and for $(i',j') \neq (i,j) \implies |\Delta_S(i',j') - \gamma| \geq \Delta_{\min}$, then we require $|\Delta_S(i,j) - \Delta_S(k,l)| \geq \Delta_{\min} + \Delta_{\sinc}$. 

Suppose $|\Delta_S(i,j) - \gamma| \leq \Delta_{\sinc}$ and $|\Delta_S(i,j) - \Delta_S(k,l)| \geq \Delta_{\sinc} + \Delta_{\min}$ for $(k,l) \neq (i,j)$. We would like to show then that $|\Delta_S(k,l) - \gamma| \geq \Delta_{\min}$. We see that given three real numbers $\gamma, \Delta_S(i,j), \Delta_S(k,l)$ we have four relevant orderings:
\begin{align}
    \gamma \leq \Delta_S(i,j) \leq \Delta_S(k,l) \\
    \Delta_S(k,l) \leq \Delta_S(i,j) \leq \gamma \\
    \Delta_S(i,j) \leq \gamma \leq \Delta_S(k,l) \\
    \Delta_S(k,l) \leq \gamma \leq \Delta_S(i,j).
\end{align}
The scenario $\gamma \leq \Delta_S(i,j) \leq \Delta_S(k,l)$ yields
\begin{align}
    |\Delta_S(k,l) - \gamma| &= \Delta_S(k,l) - \gamma \\
    &= \Delta_S(k,l) - \Delta_S(i,j) + \Delta_S(i,j) - \gamma \\
    &\geq \Delta_S(k,l) - \Delta_S(i,j) \\
    &= |\Delta_S(k,l) - \Delta_S(i,j)| \\
    &\geq \Delta_{\min} + \Delta_{\sinc} \\
    &\geq \Delta_{\min}.
\end{align}
The other direction ($\Delta_S(k,l) \leq \Delta_S(i,j) \leq \gamma$) holds similarly. 

The scenario $\Delta_S(k,l) \leq \gamma \leq \Delta_S(i,j)$ holds through the following computation
\begin{align}
    |\Delta_S(k,l) - \gamma| &= \gamma - \Delta_S(k,l) \\
    &= \gamma + \Delta_S(i,j) - \Delta_S(i,j) - \Delta_S(k,l) \\
    &= \gamma - \Delta_S(i,j) + |\Delta_S(i,j) - \Delta_S(k,l)| \\
    &= -|\gamma - \Delta_S(i,j)| + |\Delta_S(i,j) - \Delta_S(k,l)| \\
    &\geq -\Delta_{\sinc} + \Delta_{\min} + \Delta_{\sinc} \\
    &= \Delta_{\min}.
\end{align}
The other direction ($\Delta_S(i,j) \leq \gamma \leq \Delta_S(k,l)$ ) holds similarly.


% \begin{lemma}[Sinc function parameters]
%     Let $f(x) = \frac{\sin^2(x)}{x^2}$. The constant approximation $f(x) = 1$ has error $|f(x) - 1| \leq \epsilon_{1}$ if $|x| \leq \sqrt{2 \epsilon_{1}}$. We denote $\epsilon_{\sinc} \coloneqq \frac{1}{\Delta_{\min}^2 t^2}$, which leads to the claim that if $\Delta \geq \Delta_{\min}$, then $f(\Delta) \leq \epsilon_{\sinc}$. Further, if $x = \Delta t$, with $\Delta \geq \Delta_{\min}$, we note that $t \geq (\Delta_{\min} \sqrt{\epsilon_{\sinc} } )^{-1}$ suffices for $f(\Delta t) \leq \epsilon_{\sinc}$.
% \end{lemma}
% \begin{proof}
%     We use the form of $\sinc$ as $\frac{\sin(x)}{x} = \int_0^1 \cos(sx) ds$. The first 3 derivatives are computed using straightforward calculus
%     \begin{align}
%         \frac{df}{dx} &= -2 \int_0^1 \sin(sx) s ds \int_0^1 \cos(sx) ds \\
%         \frac{d^2f}{dx^2} &= -2 \int_0^1 \cos(sx)s^2 ds \int_0^1 \cos(sx) ds + \int_0^1 \sin(sx) s ~ds \int_0^1 \sin(sx) s ~ds.
%         % \frac{d^3 f}{dx^3} &= 2 \int_0^1 \sin(sx)s^3 ds \int_0^1 \cos(sx) ds + 4 \int_0^1 \cos(sx) s^2 ds \int_0^1 \sin(sx) s ds .
%     \end{align}
%     By using the fact that $\cos(sx) \to 1$ and $\sin(sx) \to 0$ as $x \to 0$ we can evaluate the Taylor's series to $f(x)$ directly. We also make use of the inequality $\abs{\int_a^b f(x) dx} \leq \int_a^b \abs{f(x)} dx$ and that sine and cosine are bounded by 1.
%     \begin{align}
%         f(x) &= \frac{\sin^2(x)}{x^2} \bigg|_{x = 0} + x \frac{df}{dx}\bigg|_{x = 0} + \frac{x^2}{2!} \frac{d^2f}{dx^2}\bigg|_{x=c} \\
%         f(x) &= 1 + \frac{x^2}{2} \frac{d^2f}{dx^2}\bigg|_{x=c}  \\
%         \abs{f(x) - 1} &= \frac{\abs{x}^2}{2} \abs{\frac{d^2f}{dx^2}(x=c)} \\
%         &\leq \frac{|x|^2}{3}
%     \end{align}
%     Requiring $|x|^2/3 \leq \epsilon$ yields the statement.
    
%     We will also have a need for bounding $f(x)$ when $x$ is of the form $x = \Delta t$. We would like to choose $t$ large enough so that $\Delta \geq \Delta_{\text{min}}$ implies that $f(\Delta t) \leq \epsilon_{\sinc}$. This can be given using the fact that $\sin(x) \leq 1$:
%     \begin{align}
%         f(\Delta t) &= \frac{\sin^2(\Delta t)}{\Delta^2 t^2} \\
%         &\leq \frac{1}{\Delta^2 t^2} \\
%         \frac{1}{\Delta_{\text{min}}^2 t^2} &\leq \frac{1}{\Delta^2 t^2} \leq \epsilon_{\sinc} \\
%         \frac{1}{\Delta_{\text{min}} \sqrt{\epsilon_{\sinc}}} &\leq t.
%     \end{align}
%     We now use this bound on $t$ to investigate when the polynomial approximation given above holds for inputs $x = (\Delta - \gamma)t$. We assume $\gamma \geq 0 $.
%     \begin{align}
%         |(\Delta - \gamma)t| &\leq \sqrt[3]{6\epsilon} \\
%         \abs{\Delta - \gamma} &\leq \frac{\sqrt[3]{6 \epsilon}}{t} \\
%         &\leq \Delta_{\text{min}} \sqrt{\epsilon_{\sinc}}\sqrt[3]{6 \epsilon}.
%     \end{align}
%     We also would like to note the differences if one uses the constant term for $f(x)$ as opposed to a quadratic.
%     $$f(x) = 1 + x \frac{df}{dx}\bigg|_{x=c}$$
%     Using the fact that the first derivative is 0 at $x= 0$ and the second is bounded by $\abs{\frac{d^2f}{dx^2}} \leq 1$, we get
%     $$f(x) = 1 + R(x)$$
%     and $\abs{R(x)} \leq |x|^2 / 2$ implies that $|x| \leq \sqrt{2\epsilon}$ suffices for $\abs{f(x) - 1} \leq \epsilon$. Using this bound with $f((\Delta - \gamma)t)$, we get $\abs{\Delta - \gamma} \leq \Delta_{\text{min}} \sqrt{2 \epsilon_{\sinc} \epsilon} $.
% \end{proof}

%%%%%%%%%%%%%%%%%%%%%%%%%%%%%%%%%%%%%%%%%%%%%%%%%%%%%%%%%%%%%%%%%%%%%%%%%%%%%%%%%%%%%%%%%%%%%%%%%%%%%%%%%%%%%%%%%%%%%%%%%%%%%%%%%%%%%%%%%%%%%%%%
%%%%%%%%%%%%%%%%%%%%%%%%%%%%%%%%%%%%%%%%%%%%%%%%%%%%%%%%%%%%%%%%%%%%%%%%%%%%%%%%%%%%%%%%%%%%%%%%%%%%%%%%%%%%%%%%%%%%%%%%%%%%%%%%%%%%%%%%%%%%%%%%
%%%%%%%%%%%%%%%%%%%%%%%%%%%%%%%%%%%%%%%%%%%%%%%%%%%%%%%%%%%%%%%%%%%%%%%%%%%%%%%%%%%%%%%%%%%%%%%%%%%%%%%%%%%%%%%%%%%%%%%%%%%%%%%%%%%%%%%%%%%%%%%%
\section{Taylor's Series for $\Phi$}

We will work extensively with a polynomial approximation, with respect to $\alpha t$, of the overall channel $\Phi$. This section provides results about the structure of the first and second order terms in the Taylor's Series and provides a bound on the remainder term $R_{\Phi}$. As we will exclusively work with a second order approximation we leave the order of the remainder term implicit.

\begin{equation}
    \Phi(X, \alpha) = \Phi(X, 0) + \alpha \frac{\partial}{\partial \alpha} \Phi(X, \alpha) \bigg|_{\alpha = 0} + \frac{\alpha^2}{2!} \frac{\partial^2}{\partial \alpha^2} \Phi(X, \alpha) \bigg|_{\alpha = 0} + R_{\Phi}(X, \alpha_{\star}).
\end{equation}
We will denote the second order approximation as
\begin{equation}
    \Phi^{(2)}(X, \alpha) \coloneqq \Phi(X, 0) + \frac{\partial}{\partial \alpha} \Phi(X, \alpha) \bigg|_{\alpha = 0} + \frac{1}{2!} \frac{\partial^2}{\partial \alpha^2} \Phi(X, \alpha) \bigg|_{\alpha = 0}. \label{def:second_order_approx}
\end{equation}
We will also find it helpful to denote each of the terms as
\begin{equation}
    T^{(k)}(X, \alpha) \coloneqq \frac{\alpha^k}{k!} \frac{\partial^k}{\partial \alpha^k} \Phi(X, \alpha)\bigg|_{\alpha = 0} \label{def:taylor_series_terms}
\end{equation}

\begin{lemma}[First Order $\alpha$ Correction to $\Phi$]
   Given a randomized environment interaction channel $\Phi$ with coupling coefficient $\alpha$, the first order correction is
   \begin{equation}
        \frac{\partial}{\partial \alpha} \Phi(\rho_S) \bigg|_{\alpha = 0} = 0.
   \end{equation}
\end{lemma}
\begin{proof}
    We start by using linearity of derivatives, integration, and partial trace to pull the $\alpha$ derivative to act on $\Phi_G$ as
    \begin{align}
        \frac{\partial}{\partial \alpha} \Phi(\rho_S) \bigg|_{\alpha = 0} &= \frac{\partial}{\partial \alpha} \partrace{\mathcal{H}_E}{\int \Phi_G(\rho_s) dG} \bigg|_{\alpha = 0} \\
         &= \partrace{\mathcal{H}_E}{\int \frac{\partial}{\partial \alpha} \Phi_G(\rho_S) dG \bigg|_{\alpha = 0} } .
    \end{align}
    Now we use Eq. \eqref{eq:phi_g_definition} to compute the derivatives, we remind the reader that $\rho = \rho_S \otimes \rho_E$,
    \begin{align}
        \frac{\partial}{\partial \alpha} \Phi_G (\rho_S) &= \parens{\frac{\partial}{\partial \alpha} e^{+ i (H + \alpha G)t}} \rho e^{-i (H + \alpha G) t} + e^{+i (H + \alpha G)t} \rho \parens{\frac{\partial}{\partial \alpha} e^{- i (H + \alpha G)t}} \\
        &= \parens{\int_{0}^{1} e^{i s (H+\alpha G)t} (i t G) e^{i (1-s) (H+\alpha G)t} ds} \rho e^{-i(H+\alpha G)t} \nonumber \\
    &~ ~+ e^{i(H+\alpha G)t} \rho \parens{\int_{0}^1 e^{-i s (H+\alpha G) t} (- i t G) e^{-i (1-s) (H+\alpha G)t} ds}. \label{eq:first_order_alpha_derivative}
    \end{align}
    We can further simplify this by bringing in the evaluation of $\alpha = 0$ through the partial trace and integration, as they are uniformly convergent over $\alpha$ (is that the correct notion that allows us to switch orders?)
    \begin{align}
        \frac{\partial}{\partial \alpha} \Phi_G(\rho_S) \bigg|_{\alpha = 0} &= i t \int_0^1 e^{i s H t} G e^{-i s H t} ds e^{i H t} \rho e^{-i H t} - i t e^{+i H t} \rho \int_0^1 e^{-is H t} G e^{-i(1-s) Ht} ds \\
        &= i t \parens{\int_0^1 G(s t) ds} \rho(t) - it \rho(t) \parens{\int_0^1 G(s t) ds} \\
        &= i t \int_0^1 [G(s t), \rho(t)] ds,
    \end{align}
    where we have used the Heisenberg picture $\rho(t) = e^{i H t} \rho e^{-i H t}$ to simplify the notation.

    This expression is now amenable to computing the correction to the total channel. We do so by performing the integration over the randomized interactions. We take advantage of the structure of our interaction measure, that is $G = U_G D U_G^\dagger$ and $dG = dU_G dD$, which allows us to write
    \begin{align}
        \int \frac{\partial}{\partial \alpha} \Phi_G(\rho_S) \bigg|_{\alpha = 0} dG &= it \int \int_0^1 \left[ e^{i H s t} G e^{-i H s t}, \rho(t) \right] ds ~dG \\
        &= it \int_0^1 \left[ e^{i H s t} \parens{\int \int U_G D U_G^\dagger ~dU_G ~ dD} e^{-i H s t}, \rho(t)  \right] ds \\
        &= i t \int_0^1 \left[ e^{i H s t} \parens{\int U_G \parens{\int D ~ dD} ~ U_G^\dagger dU_G } e^{-i H s t}, \rho(t) \right] ds \\
        &= 0.
    \end{align}
    This last step relies on the use of random eigenvalues with mean 0, implying $\int D ~dD = 0$ which shows that $\frac{\partial}{\partial \alpha} \Phi(\rho_S) \big|_{\alpha = 0 } = 0$.
\end{proof}

\begin{lemma} \label{lem:two_heisenberg_interactions}
    Let $G(t)$ denote the Heisenberg evolved random interaction $G(t) = e^{iHt} G e^{-iHt}$ for a total Hamiltonian $H$. After averaging over the interaction measure the product $G(x) G(y)$ can be computed as
    \begin{equation}
        \int G(x) G(y) dG = \frac{1}{\dim + 1} \parens{\sum_{(i,j),(k,l)} e^{i \Delta(i,j|k,l) (x-y)} \ketbra{i,j}{i,j} + \identity}.
    \end{equation}
\end{lemma}
\begin{proof}
The overall structure of this proof is to evaluate the product in the Hamiltonian eigenbasis and split the product into three factors: a phase contribution from the time evolution, a Haar integral from the eigenvalues of the random interaction, and the eigenvalue contribution of the random interaction. Since this involves the use of multiple indices, it will greatly simplify the proof to use a single index over the total Hilbert space $\hilb$ as opposed to two indices over $\hilb_S \otimes \hilb_E$. For example, the index $a$ should be thought of as a pair $(a_s, a_e)$, and functions $\lambda(a)$ should be thought of as $\lambda(a_s, a_e)$. Once the final form of the expression is reached we will substitute in pairs of indices for easier use of the lemma in other places.
    \begin{align}
        \int G(x) G(y) dG &= \int e^{+i H x} U_G D U_G^\dagger e^{-i H x} e^{+i H y} U_G D U_G^\dagger e^{-i H y} dU_G ~dD \\
        &= \int \bigg[\sum_a e^{+i \lambda(a)x}\ketbra{a}{a}  U_G \sum_b D(b)\ketbra{b}{b} U_G^\dagger \nonumber \\
        &\quad \sum_c e^{-i \lambda(c) (x - y)} \ketbra{c}{c} U_G \sum_d D(d)\ketbra{d}{d} U_G^\dagger \sum_e e^{-i \lambda(e) y} \ketbra{e}{e} \bigg] dU_G ~dD\\
        &=\sum_{a,b,c,d,e} \ketbra{a}{e} e^{-i (\lambda(c) - \lambda(a))x} e^{-i (\lambda(e) - \lambda(c))y} \nonumber \\
        &\quad \times \int \bra{a} U_G \ket{b} \bra{c} U_G \ket{d} \bra{b} U_G^{\dagger} \ket{c} \bra{d} U_G^\dagger \ket{e} dU_G \int D(b) D(d) dD \\
        &=  \sum_{a, b, c, d, e} \delta_{bd} \ketbra{a}{e} e^{-i (\lambda(c) - \lambda(a))x} e^{-i (\lambda(e) - \lambda(c))y} \nonumber \\
        &\quad \times \int \bra{a} U_G \ket{b} \bra{c} U_G \ket{d} \bra{b} U_G^{\dagger} \ket{c} \bra{d} U_G^\dagger \ket{e} dU_G. \\
    \end{align}
    Now the summation over $d$ fixes $d=b$ and we use Lemma \ref{lem:haar_two_moment} to compute the Haar integral, which simplifies greatly due to the repeated $b$ index. Plugging the result into the above yields the following
    \begin{align}
        &= \frac{1}{\dim^2 - 1} \sum_{a, b, c, e} \ketbra{a}{e} e^{-i (\lambda(c) - \lambda(a))x} e^{-i (\lambda(e) - \lambda(c))y} \parens{\delta_{ac} \delta_{ce} + \delta_{ae} - \frac{1}{\dim} \parens{\delta_{ac} \delta_{ce} + \delta_{ae}}}  \\
        &= \frac{1}{\dim^2 - 1} \parens{1 - \frac{1}{\dim}} \sum_{a, b, c, e} \ketbra{a}{e} e^{-i (\lambda(c) - \lambda(a))x} e^{-i (\lambda(e) - \lambda(c))y} \delta_{ae} (1 + \delta_{ac}) \\
        &= \frac{1}{\dim^2 - 1} \parens{1 - \frac{1}{\dim}} \sum_{a, b, c} \ketbra{a}{a} e^{i (\lambda(a) - \lambda(c))(x-y)} (1 + \delta_{ac}) \\
        &= \frac{1 \parens{\dim - 1}}{\dim^2 - 1} \sum_{a,c} \ketbra{a}{a} e^{i (\lambda(a) - \lambda(c))(x - y)} (1 + \delta_{ac}) \\
        &= \frac{1}{\dim + 1} \parens{\sum_{a,c} e^{i (\lambda(a) - \lambda(c))(x-y)} \ketbra{a}{a} + \identity}.
    \end{align}
    Reindexing by $a \mapsto i,j$, $c \mapsto k,l$, and plugging in the definition of $\Delta$ yields the statement of the lemma.
\end{proof}


\begin{lemma} \label{lem:sandwiched_interaction}
    Given two Heisenberg evolved random interactions, $G(x)$ and $G(y)$, we compute their action on an outer-product $\ketbra{i,j}{k,l}$ as
    \begin{equation}
        \int G(x) \ketbra{i,j}{k,l} G(y) ~dG = \frac{1}{\dim + 1} \parens{\ketbra{i,j}{k,l} + \delta(i,j|k,l) \sum_{m,n} e^{i \Delta(m,n | i,j) (x-y)} \ketbra{m,n}{m,n}}
    \end{equation}
\end{lemma}
\begin{proof}
This proof is structured the same as Lemma \ref{lem:two_heisenberg_interactions} and similarly we will use a single index of the total Hilbert space $\hilb$ and switch to two indices to match the rest of the exposition.
    \begin{align}
        \int G(x) \ketbra{a}{b} G(y) dG &=  \int e^{i H x} U_G D U_G^{\dagger} e^{-i H x} \ketbra{a}{b} e^{i H y} U_G D U_G^\dagger e^{-i H y} ~dG \\
        &= \sum_{c, d, e, f} e^{i (\lambda(c) - \lambda(a))x} e^{i (\lambda(b) - \lambda(f))y} \nonumber \\
        &\quad \times \int \ketbra{c}{c} U_G D(d) \ketbra{d}{d} U_G^\dagger \ketbra{a}{b} U_G D(e) \ketbra{e}{e} U_G^\dagger \ketbra{f}{f} dG \\
        &= \sum_{c, d, e, f}  e^{i (\lambda(c) - \lambda(a))x} e^{i (\lambda(b) - \lambda(f))y} \ketbra{c}{f} \nonumber \\
        &\quad \times \int D(d) D(e) dD \int \bra{c} U_G \ket{d} \bra{b} U_G \ket{e} \bra{d} U_G^\dagger \ket{a} \bra{e} U_G^\dagger \ket{f} dU_G \\
        &=  \sum_{c,d,f} e^{i (\lambda(c) - \lambda(a))x} e^{i (\lambda(b) - \lambda(f))y} \ketbra{c}{f} \nonumber \\ 
        &\quad \times \int \bra{c} U_G \ket{d} \bra{b} U_G \ket{d} \bra{a} \overline{U_G} \ket{d} \bra{f} \overline{U_G} \ket{d} dU_G \\
        &= \frac{1}{\dim^2 - 1} \sum_{c,d,f} e^{i (\lambda(c) - \lambda(a))x} e^{i (\lambda(b) - \lambda(f))y} \ketbra{c}{f} (\delta_{ca} \delta_{bf} + \delta_{cf}\delta_{ab})\parens{1 - \frac{1}{\dim}} \\
        &= \frac{1}{\dim + 1} \sum_{c,f} e^{i (\lambda(c) - \lambda(a))x} e^{i (\lambda(b) - \lambda(f))y} \ketbra{c}{f} (\delta_{ca} \delta_{bf} + \delta_{cf}\delta_{ab}) \\
        &= \frac{1}{\dim + 1} \parens{\ketbra{a}{b} + \delta_{ab} \sum_{c} e^{i(\lambda(c) - \lambda(a))(x-y)} \ketbra{c}{c} }.
    \end{align}
    Now re-indexing by $a \mapsto (i,j)$, $b \mapsto (k,l)$ and $c \mapsto (m,n)$ results in the expression given in the statement of the lemma.
\end{proof}

\begin{restatable}[Second Order Correction]{lemma}{secondOrderChannelHaar} \label{lem:big_one}
    Given a system Hamiltonian $H_{S}$, an environment Hamiltonian $H_{E}$, a simulation time $t$, and coupling coefficient $\alpha$, let $\Phi_G : \hilb_S \otimes \hilb_E \to \hilb_S \otimes \hilb_E$ denote the fixed interaction channel 
    \begin{equation}
        \Phi_G(\rho) = e^{+i (H + \alpha G)t} \rho e^{-i (H + \alpha G)t},
    \end{equation}
    where $H = H_S \otimes \identity + \identity \otimes H_E$. We compute the output of the averaged channel at $\alpha = 0$ for the basis $\ketbra{a}{b}$ of linear operators as:
 \begin{align}
     &\int \frac{\partial^2}{\partial \alpha^2} \Phi_G(\ketbra{i,j}{k,l})\bigg|_{\alpha = 0} dG \\
     &= -\frac{2  e^{i \Delta(i,j|k,l) t}}{\dim + 1} \bigg(\sum_{(a,b): \Delta(i,j|a,b) \neq 0} \frac{1 - i \Delta(i,j|a,b)t - e^{-i \Delta(i,j|a,b) t}}{\Delta(i,j|a,b)^2} \nonumber \\
     &~+ \sum_{(a,b): \Delta(k,l|a,b) \neq 0} \frac{1 + i \Delta(k,l|a,b) t - e^{i \Delta(k,l|a,b) t}}{\Delta(k,l|a,b)^2} + \frac{t^2}{2}(\eta(i,j) + \eta(k,l)) \bigg) \ketbra{i,j}{k,l} \nonumber \\
    &~ +\delta_{i,k} \delta_{j,l} \frac{2 e^{i \Delta(i,j|k,l)t}}{\dim+1} \parens{ \sum_{(a,b): \Delta(i,j|a,b) \neq 0 } \frac{2(1- \cos (\Delta(i,j|a,b)t))}{\Delta(i,j|a,b)^2} \ketbra{a,b}{a,b} + t^2 \sum_{(a,b) : \Delta(i,j|a,b) = 0} \ketbra{a,b}{a,b}}
 \end{align}
\end{restatable}
The proof of this is given in Appendix \ref{sec:haar_integral_appendix}. 

The goal for the remainder of this section is to simplify the expression for the channel output given by the above lemma. The first thing to note is that our channel does not appear to have large off-diagonal contributions to second order in $\alpha$. 
\begin{corollary}
    Given the inputs to Lemma \ref{lem:big_one} and a diagonal input state $\rho = \sum_{i,j} \rho_{i,j} \ketbra{i,j}{i,j}$, then we have $$\int \bra{k,l} \Phi_G(\rho)  \ket{m,n} ~dG~ \in \bigo{\alpha^3},$$ for $(k,l) \neq (m,n)$.
\end{corollary}
\begin{proof}
    We start from the Taylor's series of $\Phi_G$ with respect to $\alpha$:
    \begin{equation}
        \Phi_G(\rho) = \rho + \frac{\alpha^2}{2!} \int \frac{\partial^2}{\partial \alpha^2} \Phi_G(\rho) dG + \bigo{\alpha^3}. \label{eq:off_diagonal_taylors}
    \end{equation}
    First we note that $\bra{k,l}\rho \ket{m,n} =0$, as $\rho$ is diagonal and we assumed that $(k,l) \neq (m,n)$. Substituting Lemma \ref{lem:big_one} for the second order derivative and taking matrix elements yields:
    \begin{align}
        &\bra{k,l} \int \frac{\partial^2}{\partial \alpha^2} \Phi_G(\rho)\bigg|_{\alpha = 0} dG ~dG ~\ket{m,n}  \\
     &= -\frac{2 }{\dim + 1} \bigg(\sum_{(a,b): \Delta(i,j|a,b) \neq 0} \frac{1 - i \Delta(i,j|a,b)t - e^{-i \Delta(i,j|a,b) t}}{\Delta(i,j|a,b)^2} \nonumber \\
     &~+ \sum_{(a,b): \Delta(k,l|a,b) \neq 0} \frac{1 + i \Delta(k,l|a,b) t - e^{i \Delta(k,l|a,b) t}}{\Delta(k,l|a,b)^2} + \frac{t^2}{2}(\eta(i,j) + \eta(k,l)) \bigg) \bra{k,l} \rho \ket{m,n} \nonumber \\
    &~ + \frac{2}{\dim+1}\sum_{i,j} \rho_{i,j} \bigg(\sum_{(a,b): \Delta(i,j|a,b) \neq 0 } \frac{2(1- \cos (\Delta(i,j|a,b)t))}{\Delta(i,j|a,b)^2} \braket{k,l}{a,b} \braket{a,b}{m,n} \nonumber \\
    &~ + t^2 \sum_{(a,b) : \Delta(i,j|a,b) = 0} \braket{k,l}{a,b} \braket{a,b}{m,n} \bigg).
    \end{align}
    We see that the factors $\bra{k,l}\rho \ket{m,n}$ and $\braket{k,l}{i,j} \braket{i,j}{m,n}$ can never be non-zero due to the assumption $(k,l) \neq (m,n)$. The only remaining term in Eq. \ref{eq:off_diagonal_taylors} is of $\bigo{\alpha^3}$, yielding the stated result.
\end{proof}

Our last objective in this section is to yield a (relatively) concise statement for transition probabilities among diagonal matrix elements for density matrices. As we will be interested in computing trace distances 
\todo{Update the proof to include the expression for $T^{(2)}$}
\begin{theorem} \label{thm:second_order_transition_coeffs}
Given the inputs to Lemma \ref{lem:big_one}, let 
$$\tau(i,j | k,l) \coloneqq \frac{\alpha^2 }{2} \bra{k,l} \int \frac{\partial^2}{\partial \alpha^2} \Phi_G(\ketbra{i,j}{i,j})\bigg|_{\alpha = 0} dG \ket{k,l},$$ denote the second order transition coefficients in $\alpha$. These are expressed in terms of the eigenvalue differences $\Delta(i',j'|k',l')$ and other constants as 
$$\tau(i,j | k,l) = \begin{cases}
    - \frac{\alpha^2 t^2 }{\dim + 1} \parens{\sum_{(a,b) : \Delta(i,j | a,b) \neq 0} \frac{\sinc^2(\Delta(a,b|i,j) t}{2} + (\eta(i,j) - 1)} & (i,j) = (k,l) \\
    \frac{\alpha^2 t^2}{\dim + 1} & (i,j) \neq (k,l), \Delta(i,j | k,l) = 0 \\
    \frac{\alpha^2 t^2 }{\dim + 1} \sinc^2(\Delta t /2) & \Delta(i,j| k,l) \neq 0.
\end{cases}$$
. A summation shows that $\tau(i,j|i,j) = -\sum_{(k,l) \neq (i,j)} \tau(i,j|k,l)$, which as a byproduct shows that the mapping $\Phi$ is trace preserving to order $\bigo{\alpha^2}$.

We can write out the second order channel term for a thermal state density input with environment at $\beta_E$ as
\begin{equation}
    T^{(2)}(\rho_S(\beta), \alpha) = \sum_{i,j,k,l} \ketbra{k}{k} \frac{e^{-\beta \lambda_S(i)}}{\partfun_S(\beta)} \frac{e^{-\beta_E(j)}}{\partfun_E(\beta_E)} \tau(i,j | k,l) \label{eq:second_order_channel_with_tau}
\end{equation}
\end{theorem}
\begin{proof}
    We focus exclusively on diagonal inputs and outputs for density matrices, $\ketbra{a}{a}$ and $\ketbra{b}{b}$. We use Eq. \eqref{eq:second_order_output}, starting with transitions within the degenerate subspace of $a$. For the following we assume $a \neq b$ and $\Delta_{ab} = 0$: 
    \begin{align}
        &\prob{a \to b | a \neq b, \Delta_{ab} = 0} = \trace{\ketbra{b}{b} \int \Phi_G(\ketbra{a}{a}) dG} \\
        &= \braket{b}{a} \braket{a}{b} + \frac{\alpha^2}{2} \bra{b} \int \frac{\partial^2}{\partial \alpha^2} \Phi_G(\ketbra{a}{a})\bigg|_{\alpha = 0} dG \ket{b} \\
        &= -\frac{\sigma^2 \alpha^2}{\dim + 1} \bigg(\sum_{j: \Delta_{aj} \neq 0} \frac{1 - i \Delta_{aj}t - e^{-i \Delta_{aj} t}}{\Delta_{aj}^2} + \sum_{j: \Delta_{aj} \neq 0} \frac{1 + i \Delta_{aj} t - e^{i \Delta_{aj} t}}{\Delta_{aj}^2} + t^2 \eta(a) \bigg) \braket{b}{a} \braket{a}{b} \nonumber \\
        &~+ \frac{\alpha^2 \sigma^2}{\dim + 1} \parens{\sum_{i: \Delta_{ai} \neq 0} \frac{2(1- \cos(\Delta_{ai} t)}{\Delta_{ai}^2} \braket{b}{i}\braket{i}{b} + t^2 \sum_{i: \Delta_{ai} = 0 } \braket{b}{i} \braket{i}{b}} \label{eq:transition_intermediate} \\
        &= \frac{\sigma^2 \alpha^2 t^2}{\dim + 1}. 
    \end{align}
    We now proceed to the case that $b \neq a$ and $\Delta_{ab} \neq 0$, where we can start from Eq. \eqref{eq:transition_intermediate} and we get
    \begin{equation}
        \prob{a \to b | a \neq b, \Delta_{ab} \neq 0} = \frac{2 \sigma^2 \alpha^2 }{\dim + 1} \frac{1 - \cos (\Delta_{ab} t)}{\Delta_{ab}^2}.
    \end{equation}

    The remaining case to consider is when $b = a$. This involves simplifying the summations in Eq. \eqref{eq:second_order_output} and including the zeroth order $\bigo{\alpha^0}$ term which simply contributes a 1. 
    \begin{align}
        \prob{a \to a} &= 1 -\frac{\sigma^2 \alpha^2 }{\dim + 1} \bigg(\sum_{j: \Delta_{aj} \neq 0} \frac{1 - i \Delta_{aj}t - e^{-i \Delta_{aj} t}}{\Delta_{aj}^2} + \sum_{j: \Delta_{aj} \neq 0} \frac{1 + i \Delta_{aj} t - e^{i \Delta_{aj} t}}{\Delta_{aj}^2} + t^2 \eta(a) \bigg) \nonumber \\
    &~ +\frac{\sigma^2 \alpha^2 }{\dim+1} \parens{ \sum_{i: \Delta_{ai} \neq 0 } \frac{2(1- \cos (\Delta_{ai}t))}{\Delta_{ai}^2} \braket{a}{i} \braket{i}{a} + t^2  \sum_{i : \Delta_{ai} = 0} \braket{a}{i} \braket{i}{a}} \\
    &= 1 - \frac{\sigma^2 \alpha^2}{\dim + 1} \parens{\sum_{i: \Delta_{ai} \neq 0} \frac{2(1 - \cos(\Delta_{ai}t))}{\Delta_{ai}^2} + t^2 (\eta(a) - 1)}.
    \end{align}
    To simplify further we reduce the cosine terms to a sinc function as follows
    \begin{align}
        \frac{2(1 - \cos(\Delta t)}{\Delta^2} &= \frac{2(1 - \cos^2(\Delta t / 2) + \sin^2(\Delta t /2)}{\Delta^2} \\
        &= \frac{t^2}{2} \frac{\sin^2 (\Delta t / 2)}{ (\Delta t/ 2)^2} \\
        &= \frac{t^2}{2} \sinc^2(\Delta t /2).
    \end{align}
    This completes the transition probability computation. It is straightforward to see that $\sum_{b} \prob{a \to b} = 1$ and that $0 \leq \prob{a \to b} \leq 1$ given that $\alpha t \leq \sqrt{\dim + 1}$.
\end{proof}

\subsection{Remainder Bound}
We now aim to bound the spectral norm of the remainder term $R_{\Phi}$. This is rather tedious, as even to third order in $\alpha$ we have 24 terms to bound. For example, looking first at the $(A)$ term resulting from the second order derivative in Eq. \eqref{eq:second_order_deriv_intermediate_a} we have 4 multiplicative factors involving $\alpha$, leading to 4 terms from this single term of the second order derivative. As there are six terms in total, this yeilds 24 terms. We can profit from the fact though that the expressions for the second order derivatives do simplify and the final expression only has 3 terms, as two of the derivatives act similarly. We first will analyze the single term from Eq. \eqref{eq:second_order_deriv_intermediate_a} in detail. 


\begin{lemma} \label{lem:remainder_bound}
    Let $\Phi(\rho)$ be the thermalizing channel as defined in Def. \eqref{eq:phi_g_definition} and $R_{\Phi}$ the remainder term 
    \begin{equation}
        R_{\Phi}(\rho, \alpha) = \Phi(\rho, \alpha) - \rho - \frac{\alpha}{1!} \frac{\partial}{\partial \alpha} \Phi(\rho)\bigg|_{\alpha = 0} - \frac{\alpha^2}{2!} \frac{\partial^2}{\partial \alpha^2} \Phi(\rho) \bigg|_{\alpha = 0}.
    \end{equation}
    Then we have that
    \begin{equation}
        \alpha t \leq \parens{\frac{\epsilon_R}{2 \dim}}^{1/3} \implies \norm{R_{\Phi}} \leq \epsilon_R.
    \end{equation}
\end{lemma}
\begin{proof}
    First I need to write down what exactly we are trying to bound. What we are essentially trying to do is bound the matrix entries:
\begin{equation}
    \Phi(\rho) = \rho + \frac{\alpha}{1} \frac{\partial}{\partial \alpha} \Phi(\rho) \bigg|_{\alpha = 0} + \frac{\alpha^2}{2!} \frac{\partial^2}{\partial \alpha^2} \Phi(\rho) \bigg|_{\alpha = 0} + \frac{\alpha^3}{3!} \frac{\partial^3}{\partial \alpha^3} \Phi(\rho) \bigg|_{\alpha = \alpha^\star},
\end{equation}
where $\alpha^\star \in [0, \alpha]$. Our goal is then to bound the spectral norm of the remainder term for all such $\alpha^\star$.

\begin{align}
    &\norm{\frac{\partial}{\partial \alpha} (it)^2 \int_0^1 \parens{\int_0^1 e^{i s_1 s_2 (H+\alpha G)t} s_1 G e^{i s_1 (1-s_2) (H+\alpha G)t} ds_2} G e^{i(1-s_1) (H+\alpha G)t} ds_1 \rho e^{-i(H+\alpha G) t}} \\
    &=t^3 \norm{\int_0^1 \int_0^1 \int_0^1 e^{i s_1 s_2 s_3(H + \alpha G)t} G e^{i s_1 s_2 (1 - s_3)} ds_3 G e^{i s_1 (1-s_2)(H + \alpha G)t} ds_2 G e^{i(1-s_1)(H + \alpha G)t} ds_1 \rho e^{-i(H + \alpha G)t} }  + t^3 \norm{\ldots}\\
    &\leq t^3 \norm{G}^3 + t^3 \norm{\ldots} \\
    &\leq 4 t^3 \norm{G}^3,
\end{align}
where we used smoothness of the integrand to bring the norm inside the integral via the triangle inequality and submultiplicativity of the operator norm to simply break the norm of the product into the product of each of the norms. The operator norm of the unitary operators and the density matrix are each 1, and the resulting integrals yield only fractional values, which are upper bounded by 1. Now here we see the final simplification that can be made, each term $(B), (C)$, etc., will yield at most a cubic power of $\norm{G}$, so we can upper bound each term in the final sum as $t^3 \norm{G}^3$. In reality we would have terms such as $\norm{G} \norm{G^3}$ and all other polynomials, but we don't care.

Now as $G$ is a random matrix, we can only bound it's operator norm with a probabilistic guarantee. 

This gives
\begin{equation}
    \norm{\Phi(\rho) - \rho - \sum_{i,j,k,l} \ketbra{j}{j} \bra{i} \rho \ket{i} \frac{e^{-\beta_E \lambda_E(k)}}{\partfun_E(\beta_E)} \tau(i,k | j, l) } \leq 4 \alpha^3 t^3 \int \norm{G}^3 dG
\end{equation}

We now need to bound $\int \norm{G}^3 dG$. This is a fun probability question. First we get rid of the Haar integrals
\begin{align}
    \int \norm{G}^3 dG &= \int \int \norm{U_G D U_G^\dagger}^3  dU dD \\
    &= \int \norm{D}^3 dD \\
    &= \expect{\max_i |d_i|^3}
\end{align}
\todo{As it stands this is more like the skeleton of a proof than a final product. Need to clean this up.}

It is pretty straightforward to upper bound this integral by $\int \norm{D}^3 dD \leq 2 \dim$. This then gives
\begin{equation}
    \norm{R_{\Phi}} \leq 8 \alpha^3 t^3 \dim.
\end{equation}
So this says we need $\alpha t \leq \frac{\epsilon_{R}^{1/3}}{2 \dim^{1/3}}$ in order for $\norm{R_{\Phi}} \leq \epsilon_R$.
\end{proof}

\textcolor{red}{TODO: Change this to be the norm of the average instead of the average of the norm. The norm of the average should be reducible to the third moment of the distribution of eigenvalues. Even then though we might pick up dimensionful factors because we have to do a spectral decomposition of each random matrix $D$. This then introduces a sum over all 3 diagonals. We can use independence to get rid of two of them, but the last remaining one I'm not sure how to, as we have projectors interspersed throughout the factors. Using triangle inequality and submultiplicativity gives you a factor of dimension, in which case we are just as good with our dumb bound of the average of the norm cubed.}
\todo{Fix the definitions used here.}

\section{Fixed points and convergences}

In this first section we introduce the conditions required for more detailed analysis. The first is 
\begin{definition}[Well-separated Hamiltonian] \label{def:separated_hamiltonians}
    A Hamiltonian $H_S$ is $(\Delta_{\sinc}, \Delta_{\min}$)-separated if there are not only no degeneracies, but every eigenvalue difference is bounded from below and eigenvalue differences are sufficiently spaced.
    \begin{align}
        &\forall i \neq j : \Delta_{S}(i,j) \geq \Delta_{\min}, \\
        &\forall (i,j) \neq (k,l) : \abs{\Delta_S(i,j) - \Delta_S(k,l)} \geq \Delta_{\sinc} + \Delta_{\min}.
    \end{align}
    This rather strong condition on the spectrum is needed as we would like to make the guarantee that for a provided $\gamma$ there is at most 1 pair of indices $(i,j)$ such that $\sinc^2((\Delta_S(i,j) - \gamma)t/2) \geq 1 - \epsilon_{\sinc}$.
    \begin{equation}
        \abs{\set{(i,j): \abs{\Delta_S(i,j) - \gamma} \leq \Delta_{\sinc}}} \in \set{0, 1}.
    \end{equation}
    It will be useful when working from this to sample from these differences uniformly. Given a Hamiltonian with eigenvalues that are well separated there are $\frac{\dim_S (\dim_S - 1)}{2}$ positive differences. We create the following distribution which is a mixture of uniform distributions over the the eigenvalue differences, where each uniform distribution is of width $2 \Delta_{\sinc}$, so we can guarantee that any sampled value of $\gamma$ is close enough to an eigenvalue difference to satisfy the requirements of Corollary \ref{cor:gamma_difference_reqs}.

    \begin{equation}
        \prob{\gamma = x} = \begin{cases}
            \frac{2}{\dim_S (\dim_S - 1)}
            & x = \Delta_S(i,j) \text{ for some i,j} \\
            0 & \forall (i,j), x \neq \Delta_S(i,j).
        \end{cases}
    \end{equation}
    We will refer to this distribution as the \emph{perfect knowledge} distribution.
\end{definition}

The following lemma is purely a technical one to allow for easier computation of sums that appear from the second order correction of $\Phi$.
\begin{lemma} \label{lem:transition_idx_sub}
    Let $p, q$ denote probability distributions on the system and environment indices respectively and $\tau$ the transition amplitude from Theorem \ref{thm:second_order_transition_coeffs}. Then we have
    \begin{equation}
        \sum_{i,j,l} p(i) q(j) \tau(i,j | k,l) = \sum_{i \neq k} \sum_{j,l} (p(i) q(j) - p(k) q(l)) \tau(i,j |k,l)
    \end{equation}
\end{lemma}
\begin{proof}
    \begin{align}
        \sum_{i,j,l} p(i) q(j) \tau(i,j | k,l) &= \sum_{i \neq k} \sum_{j,l} \parens{p(i) q(j) \tau(i,j|k,l)} + \sum_{j,l} p(k) q(j) \tau(k,j | k,l) \label{eq:n_qubit_fixed_pt_intermediate_1}.
    \end{align}
    We expand the simpler sum on the right:
    \begin{align}
        \sum_{j,l}p(k) q(j) \tau(k,j| k,l) &= p(k) q(0) (\tau(k,0|k,0) + \tau(k,0|k,1)) + p(k) b(1) (\tau(k,1|k,0) + \tau(k,1|k,1)) \\
        &= - p(k) q(0) \sum_{c \neq k}\sum_{l} \tau(k,0 | c, l) - p(k) b(1) \sum_{c \neq k} \sum_{l}\tau(k,1|c,l) \\
        &= - \sum_{c \neq k} \sum_{j,l} p(k) q(j) \tau(k,j |c,l).
    \end{align}
    Plugging this into Eq. \ref{eq:n_qubit_fixed_pt_intermediate_1} allows us to simplify as follows
    \begin{align}
        \sum_{i,j,l} p(i) q(j) \tau(i,j|k,l) &= \sum_{i \neq k} \sum_{j,l} (p(i) q(j) \tau(i,j|k,l)) - \sum_{c \neq k} \sum_{j,l} p(k) q(j) \tau(k,j|c,l) \\
        &= \sum_{i \neq k} \sum_{j,l} (p(i) q(j) \tau(i,j|k,l)) - \sum_{i \neq k} \sum_{j,l} p(k) q(l) \tau(i,j|k,l) \\
        &= \sum_{i \neq k} \sum_{j,l} (p(i) q(j) - p(k) q(l)) \tau(i,j | k,l).
    \end{align}
\end{proof}

This lemma is also a purely technical one that allows us to substitute in simple values for the sums that appear in the transition terms. 
\begin{lemma} \label{lem:big_tau_sum_simplifier}
    Assume that $H_S$ is a $(\Delta_{\min}, \Delta_{\sinc}, \epsilon_{\sinc}$)-separated Hamiltonian satisfying Def. \ref{def:separated_hamiltonians}. Let $(\gamma_U,\gamma_L)$ denote the unique index pair such that $|\Delta_S(\gamma_U,\gamma_L) - \gamma| \leq \Delta_{\sinc}$ and for all $(i,j) \neq (\gamma_U, \gamma_L)$ we have $|\Delta_S(i,j) - \gamma| \geq \Delta_{\min}$. We let $\gamma_U$ denote the higher energy state, i.e $\lambda_S(\gamma_L) < \lambda_S(\gamma_U)$. Using $\delta \coloneqq \beta_E - \beta$, we show that the assumptions on $H$ and $\gamma$ along with the requirement $\delta \geq \beta_E \frac{\Delta_{\min}}{\Delta_S(u,l)} + \frac{1}{\Delta_S(u,l)} \ln \parens{\frac{1}{(2\dim + 1) \epsilon_{\sinc}}}$ imply the following bounds.
    Let $p(i) = \frac{e^{-\beta \lambda_S(i)}}{\partfun_S(\beta)}$ and $q(j) = \frac{e^{-\beta_E \lambda_E(j)}}{\partfun_E(\beta_E)}$. 
    \begin{align}
        \abs{\sum_{i, j, l} p(i) q(j) \tau(i,j| \gamma_L, l) - \frac{\alpha^2 t^2}{\dim + 1} p(\gamma_U) q(0)} &\leq 4 \alpha^2 t^2 \epsilon_{\sinc}  \\
        \abs{\sum_{i, j, l} p(i) q(j) \tau(i,j| \gamma_U, l) - \frac{\alpha^2 t^2}{\dim + 1} p(\gamma_L) q(1)} &\leq 4  \alpha^2 t^2 \epsilon_{\sinc} \\
        k \neq \gamma_L, \gamma_U \implies \abs{\sum_{i,j,l} p(i) q(j) \tau(i,j| k, l) } &\leq 4 \alpha^2 t^2 \epsilon_{\sinc}.
    \end{align}
\end{lemma}
\begin{proof}
    \todo{update this lemma to use $p,q$ instead of $a,b$. }
    For simplicity, we let $a(i') = \frac{e^{-\beta \lambda_S(i')}}{\partfun_S(\beta)}$ and $b(j') = \frac{e^{-\beta_E \lambda_E(j')}}{\partfun_E(\beta_E)}$. Using Lemma \ref{lem:transition_idx_sub} we write the summation as
    \begin{align}
        &\sum_{i,j,k} \sum_{i, j, k} \frac{e^{-\beta \lambda_S(i)} e^{-\beta_E \lambda_E(j)}}{\partfun_S(\beta) \partfun_E(\beta_E)} \tau(i,j|l,k) = \sum_{i \neq l} \sum_{j,k} (a(i) b(j) - a(l) b(k)) \tau(i,j|l,k) \\
        &= \sum_{j,k} (a(u) b(j) - a(l) b(k)) \tau(u,j | l,k) + \sum_{i \neq l,u} \sum_{j,k} (a(i) b(j) - a(l) b(k)) \tau(i,j| l,k).
    \end{align}
    Now we utilize the fact that $i \neq u,l$ implies that $\tau(i,j|l,k) \leq \frac{\alpha^2 t^2}{\dim + 1} \epsilon_{\sinc}$, and further that $\tau(u,j|l,k) \geq \frac{\alpha^2 t^2}{\dim + 1}\epsilon_{\sinc}$ if and only if $k = 1$ and $j = 0$. We also have that $-1 \leq a(i') b(j') - a(k') b(l') \leq 1$ for all $i', j', k', l'$ as $a,b$ are probabilities. These observations yield the upper bound
    \begin{align}
        &\sum_{i, j, k} \frac{e^{-\beta \lambda_S(i)} e^{-\beta_E \lambda_E(j)}}{\partfun_S(\beta) \partfun_E(\beta_E)} \tau(i,j|l,k) \nonumber \\
        &\leq (a(u) b(0) - a(l) b(1)) \tau(u,0|l,1) + 3 \frac{\alpha^2 t^2}{\dim + 1} \epsilon_{\sinc} + 4 (\dim_S - 2) \frac{\alpha^2 t^2}{\dim + 1} \epsilon_{\sinc} \\
        &\leq (a(u) b(0) - a(l) b(1)) \tau(u,0|l,1) + 2 \dim \epsilon_{\sinc} \frac{\alpha^2 t^2}{\dim + 1} \\
        &\leq (a(u) b(0) - a(l) b(1)) \tau(u,0|l,1) + 2 \epsilon_{\sinc} \alpha^2 t^2.
    \end{align}
    For the purposes of the upper bound, all we require is that $e^{-\beta_E \gamma + \beta \Delta_S(u,l)} \geq 0$, which implies $a(u) b(0) - a(l) b(1) \leq \frac{e^{-\beta \lambda_S(u)}}{\partfun_S(\beta) \partfun_E(\beta_E)}$. This, combined with the fact that $\tau(u,0|l,1) \leq \frac{\alpha^2 t^2}{\dim + 1}$, yields our final upper bound as
    \begin{equation}
        \sum_{i,j,k} \frac{e^{-\beta \lambda_S(i)}}{\partfun_S(\beta)} \frac{e^{-\beta_E \lambda_S(j)}}{\partfun_E(\beta_E)} \tau(i,j|l,k) \leq \frac{e^{-\beta \lambda_S(u)}}{\partfun_S(\beta) \partfun_E(\beta_E)} \frac{\alpha^2 t^2}{\dim + 1} + 2 \epsilon_{\sinc} \alpha^2 t^2. \label{eq:transition_prob_upper_bound_final}
    \end{equation}
    As constant factors are relatively unimportant, we will upper bound $2 \epsilon_{\sinc} \alpha^2 t^2 \leq 4 \epsilon_{\sinc} \alpha^2 t^2$ in the final lemma statement.

    We turn our attention now to the lower bound. This proceeds in a similar manner, of isolating the indices in the sum that are active in the transitions and those that aren't.
    \begin{align}
        &\sum_{i, j, k} \frac{e^{-\beta \lambda_S(i)} e^{-\beta_E \lambda_E(j)}}{\partfun_S(\beta) \partfun_E(\beta_E)} \tau(i,j|l,k) \nonumber \\
        &= \sum_{j,k} (a(u) b(j) - a(l) b(k)) \tau(u,j | l,k) + \sum_{i \neq l,u} \sum_{j,k} (a(i) b(j) - a(l) b(k)) \tau(i,j| l,k) \\
        &\geq (a(u) b(0) - a(l) b(1)) \tau(u,0 | l,1) -(\tau(u,0|l,0) + \tau(u,1|l,0) + \tau(u,1|l,1)) - \sum_{i \neq l,u} \sum_{j,k} \tau(i,j| l,k) \\
        &\geq (a(u) b(0) - a(l) b(1))\tau(u,0|l,1) - \frac{\alpha^2 t^2}{\dim + 1} \parens{ 3 \epsilon_{\sinc}  + 4 (\dim_S - 2) \epsilon_{\sinc} } \\
        &\geq (a(u) b(0) - a(l) b(1))\tau(u,0|l,1) - 2 \epsilon_{\sinc} \alpha^2 t^2. \label{eq:transition_prob_lower_bound}
    \end{align}
    We now return to the prefactor $a(u) b(0) - a(l) b(1)$ in a bit more detail. We write this as follows
    \begin{align}
        a(u) b(0) - a(l) b(1) &= \frac{e^{-\beta \lambda_S(u)}}{\partfun_S(\beta)} \frac{e^{-\beta_E \lambda_E(0)}}{\partfun_E(\beta_E)} - \frac{e^{-\beta \lambda_S(l)}}{\partfun_S(\beta)} \frac{e^{-\beta_E \lambda_E(1)}}{\partfun_E(\beta_E)} \\
        &= \frac{e^{-\beta \lambda_S(u)}}{\partfun_S(\beta) \partfun_E(\beta_E)} \parens{1 - e^{-\beta_E \gamma - \beta \lambda_S(l) + \beta \lambda_S(u)}}.
    \end{align}
    

    We need to understand the function $f(\beta) = 1 - e^{-\beta_E \gamma + \beta \Delta_S(u,l)}$ a bit better. We see that in the limit as $\gamma \to \Delta_S(u,l)$ and $\beta \to \beta_E$, $f(\beta) \to 0$. In the other regime we see as $\beta_E \to \infty$ then $f(\beta) \to 1$. This suggests that we need to look at the regime where $\delta \coloneqq \beta_E - \beta$ is sufficiently large. We first investigate when this function is positive
    \begin{align}
        1 - e^{-\beta_E \gamma + \beta \Delta_S(u,l)} \geq 0 &\iff \beta_E \gamma \geq \beta \Delta_S(u,l) \\
        &\iff 0 \leq \beta_E (\gamma - \Delta_S(u.l)) + (\beta_E - \beta) \Delta_S(u,l).
    \end{align}
    Now using $\delta = \beta_E - \beta$ and $|\Delta_S(u,l) - \gamma| \leq \Delta_{\sinc} \leq \Delta_{\min}$, which implies $\gamma - \Delta_S(u,l) \geq - \Delta_{\min}$, we get that 
    \begin{equation}
        -\beta_E \Delta_{\min} + \delta \Delta_S(u,l) \leq \beta_E (\gamma - \Delta_S(u,l)) + (\beta_E - \beta) \Delta_S(u,l).
    \end{equation}
    We then conclude that $0 \leq -\beta_E \Delta_{\min} + \delta \Delta_S(u,l)$, or $\delta \geq \beta_E \frac{\Delta_{\min}}{\Delta_S(u,l)}$, implies that $1 - e^{-\beta_E \gamma + \beta \Delta_S(u,l)} \geq 0$. Using this, we show that $\delta \geq \frac{\beta_E \Delta_{\min} + \ln(1/\epsilon_{exp})}{\Delta_S(u,l)}$ yields the following
    \begin{align}
        \delta \geq \frac{\beta_E \Delta_{\min} + \ln(1/\epsilon_{exp})}{\Delta_S(u,l)} &\implies \beta_E \gamma - \beta \Delta_S(u,l) \geq \ln 1/ \epsilon_{exp} \\
        &\implies e^{-\beta_E \gamma + \beta \Delta_S(u,l)} \leq \epsilon_{exp} \\
        &\implies 1 - e^{-\beta_E \gamma + \beta \Delta_S(u,l)} \geq 1 - \epsilon_{exp}.
    \end{align}

    We can profit off of the above upper bound, as well as the fact that $|\Delta_S(u,l) - \gamma| \leq \Delta_{\sinc}$ which implies $\tau(u,0|l,1) \geq \frac{\alpha^2 t^2}{\dim + 1} (1 - \epsilon_{\sinc})$, by simply plugging it into Eq. \eqref{eq:transition_prob_lower_bound} and simplifying
    \begin{equation}
        \sum_{i, j, k} \frac{e^{-\beta \lambda_S(i)} e^{-\beta_E \lambda_E(j)}}{\partfun_S(\beta) \partfun_E(\beta_E)} \tau(i,j|l,k) \geq \frac{e^{-\beta \lambda_S(u)}}{\partfun_S(\beta) \partfun_E(\beta_E)}\frac{\alpha^2 t^2}{\dim + 1} (1 - \epsilon_{exp})(1 - \epsilon_{\sinc}) - 2 \epsilon_{\sinc} \alpha^2 t^2.
    \end{equation}
    Using simple bounds allows us to rewrite this as
    \begin{equation}
        \sum_{i, j, k} \frac{e^{-\beta \lambda_S(i)} e^{-\beta_E \lambda_E(j)}}{\partfun_S(\beta) \partfun_E(\beta_E)} \tau(i,j|l,k) - \frac{e^{-\beta \lambda_S(u)}}{\partfun_S(\beta) \partfun_E(\beta_E)}\frac{\alpha^2 t^2}{\dim + 1} \geq - \alpha^2 t^2 \parens{\frac{\epsilon_{exp} + \epsilon_{\sinc}}{\dim + 1} + 2 \epsilon_{\sinc}}. \label{eq:transition_prob_lower_bound_final}
    \end{equation}
    We now use the fact that we only need to control $\epsilon_{exp}$ relatively loosely to simplify the bound. Setting $\epsilon_{exp} = (2 \dim + 1) \epsilon_{exp}$ for Eq. \eqref{eq:transition_prob_lower_bound_final}, along with the inequality from Eq. \eqref{eq:transition_prob_upper_bound_final}, yields the Lemma statement.

    The last inequality we need is the rather straightforward one where $k \neq \gamma_L, \gamma_U$
    \begin{align}
        \abs{\sum_{i,j,l} p(i) q(j) \tau(i,j|k,l)} &\leq \abs{\sum_{i\neq k} \sum_{j,l} (p(i)q(j) - p(k) q(l)) \tau(i,j|k,l)} \\
        &\leq \sum_{i \neq k} \sum_{j,l} \abs{p(i) q(j) - p(k) q(l)} \tau(i,j|k,l) \\
        &\leq \sum_{i \neq k} \sum_{j,l} \frac{\alpha^2 t^2}{\dim + 1} \epsilon_{\sinc} \\
        &\leq 4 \dim \frac{\alpha^2 t^2}{\dim + 1} \epsilon_{\sinc} \\
        &\leq 4 \alpha^2 t^2 \epsilon_{\sinc}
    \end{align}
\end{proof}

From the Detailed Balance arguments in Section \ref{sec:detailed_balance} we expect that in the infinite time limit, with perfect knowledge of the eigenvalue gaps, that the thermal state $\rho_S(\beta_E)$ is a fixed point of the channel $\Phi$. Here we give rigorous bounds showing that for all choices of the environment gap $\gamma$ we can make $\rho_S(\beta_E)$ as close to fixed as we want. As we know that $\gamma$ being far away from all eigenvalue differences yields a channel that is approximately the identity channel, aka it does nothing, we should aim to show that it is arbitrarily close to being a fixed point even when $\gamma$ is close to an eigenvalue gap $\Delta_S(i,j)$. 
\begin{theorem}[Approximate Fixed Points]
    Let $H_S$ be a well-separated Hamiltonian, per Def. \ref{def:separated_hamiltonians}. For all values of $\gamma \geq \Delta_{\min}$ we have that 
    \begin{align}
        2 \sqrt{2} \Delta_{\min} \dim \leq t& \text{ \emph{and} } \alpha \leq \min \set{\frac{\sqrt{\epsilon_{fix}}}{4 \sqrt{2} \Delta_{\min} \dim}, \parens{\frac{\epsilon_{fix}}{4 \dim^5}}^{1/3} \frac{1}{2 \sqrt{2} \Delta_{\min} }} \nonumber \\
        &\implies \norm{\rho_S(\beta_E) - \Phi(\rho_S(\beta_E))}_1 \leq \epsilon_{fix}. \label{eq:beta_e_fixed_point_reqs}
    \end{align}
\end{theorem}
\begin{proof}
    We start out with the expansion of the channel
    \begin{align}
        \norm{\rho_S(\beta_E) - \Phi(\rho_S(\beta_E))}_1 &= \norm{T^{(2)}(\rho_S(\beta_E), \alpha) + R_{\Phi}(\rho_S(\beta_E))}_1 \\
        &\leq \norm{T^{(2)}(\rho_S(\beta_E))}_1 + \norm{R_{\Phi}(\rho_S(\beta_E))}_1 \\
        &\leq \norm{T^{(2)}(\rho_S(\beta_E))}_1 + \dim \norm{R_{\Phi}} \\
        &\leq \norm{T^{(2)}(\rho_S(\beta_E))}_1 + \dim \norm{R_{\Phi}}.
    \end{align}
    Taking the straightforward approach of dividing our error budget equally among these two terms will be sufficient. Starting with the $\norm{R_{\Phi}}_1$ bound we have from Lemma \ref{lem:remainder_bound} that
    \begin{align}
        \alpha t \leq \parens{\frac{\epsilon_{fix}}{4 \dim^2}}^{1/3} \implies \dim \norm{R_{\Phi}} \leq \frac{\epsilon_{fix}}{2}. \label{tmp:first_alpha_t_req_fixed_pt}
    \end{align}

    We now bound the other half of our error budget, the term resulting from $\norm{T^{(2)}(\rho_S(\beta_E))}_1$. Plugging in the second order correction for the channel from Eq. \eqref{eq:second_order_channel_with_tau} gives us
    \begin{align}
        \norm{T^{(2)}(\rho_S(\beta_E))}_1 &= \sum_k \abs{\sum_{i,j,l} \frac{e^{-\beta_E \lambda_S(i)}}{\partfun_S(\beta_E)} \frac{e^{-\beta_E \lambda_S(j)}}{\partfun_E(\beta_E)} \tau(i,j|k,l)}.
    \end{align}
    For compactness, let $p(i) = \frac{e^{-\beta_E \lambda_S(i)}}{\partfun_S(\beta_E)}$ and $q(j) =\frac{e^{-\beta_E \lambda_S(j)}}{\partfun_E(\beta_E)}$. Using Lemma \ref{lem:big_tau_sum_simplifier} we can substitute this 
    \begin{align}
        &\sum_{k \neq \gamma_L, \gamma_U} \abs{\sum_{i,j,l} p(i) q(j) \tau(i,j|k,l) } + \abs{\sum_{i,j,l} p(i)q(j) \tau(i,j | \gamma_L, l)} + \abs{\sum_{i,j,l} p(i) q(j) \tau(i,j| \gamma_U, l)} \\
        &\leq (\dim_S - 2) 4 \alpha^2 t^2 \epsilon_{\sinc} + \abs{\sum_{i,j,l} p(i)q(j) \tau(i,j | \gamma_L, l)} + \abs{\sum_{i,j,l} p(i) q(j) \tau(i,j| \gamma_U, l)}. \label{int:fixed_pt_1}
    \end{align}
    Now we use the following upper bound
    \begin{align}
        \abs{\sum_{i,j,l} p(i) q(j) \tau(i,j|\gamma_L ,l)} &= \abs{\sum_{i,j,l} p(i) q(j) \tau(i,j|\gamma_L ,l) - \frac{\alpha^2 t^2}{\dim + 1} p(\gamma_U) q(0) + \frac{\alpha^2 t^2}{\dim + 1} p(\gamma_U) q(0)} \\
        &= \abs{\sum_{i,j,l} p(i) q(j) \tau(i,j|\gamma_L ,l) - \frac{\alpha^2 t^2}{\dim + 1} p(\gamma_U) q(0)} + \abs{\frac{\alpha^2 t^2}{\dim + 1} p(\gamma_U) q(0)} \\
        &\leq 4 \alpha^2 t^2 \epsilon_{\sinc} + \frac{\alpha^2 t^2}{\dim + 1} p(\gamma_U) q(0).
    \end{align}
    Plugging this into Eq. \eqref{int:fixed_pt_1} yields
    \begin{align}
        \norm{T^{(2)} (\rho_S(\beta_E))}_1 &\leq 2 \dim \alpha^2 t^2 \epsilon_{\sinc} + \frac{\alpha^2 t^2}{\dim + 1}(p(\gamma_U) q(0) + p(\gamma_L) q(1)) \\
        &\leq 2 \dim \alpha^2 t^2 \epsilon_{\sinc} + \frac{\alpha^2 t^2}{\dim + 1} \\
        &= \alpha^2 t^2 \parens{2 \dim \epsilon_{\sinc} + \frac{1}{\dim + 1}}.
    \end{align}
    Now here we use the requirement that $\epsilon_{\sinc} \leq \frac{1}{\dim + 1} \frac{1}{2 \dim}$ to yield the inequality
    \begin{equation}
        \norm{T^{(2)}(\rho_S(\beta_E))}_1 \leq 2 \alpha^2 t^2. \label{eq:upper_bound_on_T_2}
    \end{equation}
    We note that $\epsilon_{\sinc} \leq \frac{1}{2 \dim (\dim + 1)}$ is equivalent to the lower bound on $t$ of $2 \sqrt{2} \Delta_{\min} \dim \leq t$. Scaling of $t$ with dimension is expected in the worst case. 

    It is then straightforward to see that $\alpha^2 t^2 \leq \frac{\epsilon_{fix}}{4}$ implies $\norm{T^{(2)}(\rho_S(\beta_E))} \leq \epsilon_{fix} / 2$, which along with Eq. \eqref{tmp:first_alpha_t_req_fixed_pt} implies Eq. \eqref{eq:beta_e_fixed_point_reqs}. 
\end{proof}

We now have shown that the thermal state at the same temperature as the environment, $\rho_S(\beta_E)$, remains approximately fixed even when we choose a $\gamma$ that causes transitions in the system. This argument relies on taking a very small value of $\alpha$. However, as $\alpha$ gets too small and approaches zero we have that the channel acts as time evolution by non-interacting Hamiltonians $H_S$ and $H_E$, meaning that every state that is diagonal in the $H_S$ basis could also be an approximate fixed point! We show that this is not the case and that there is a temperature $\beta$ that causes $\rho_S(\beta)$ to not be an approximate fixed point. This relies on $\gamma$ being close to an eigenvalue difference $\Delta_S(i,j)$, as otherwise our channel approximates the identity on matrices diagonal in the $H_S$ basis. 
\begin{theorem}[Not Fixed Points] \label{thm:which_beta_not_fixed}
    Let $H_S$ be a well-separated Hamiltonian, $\gamma$ be $\Delta_{\sinc}$ close to $\Delta_S(\gamma_U, \gamma_L)$, and $\delta = \beta_E - \beta$. We show that
    \begin{equation}
        \delta \geq \delta_{LB} \implies \norm{\rho_S(\beta) - \Phi^{(2)}(\rho_S(\beta))}_1 \geq \epsilon_{fix}.
    \end{equation}
\end{theorem}
\begin{proof}
\textbf{THIS SHOULD NOT BE A PROOF IT IS INCOMPLETE}


     We use the expansion of $\Phi^{(2)}(X) = X + T^{(2)} (X)$, along with defining $p(i) = \frac{e^{-\beta \lambda_S(i)}}{\partfun_S(\beta)}$ and $q(j) = \frac{e^{-\beta_E \lambda_E(j)}}{\partfun_E(\beta_E)}$ to write
     \begin{align}
        &\norm{\rho_S(\beta) - \Phi^{(2)} (\rho_S(\beta))}_1 = \norm{T^{(2)}(\rho_S(\beta))}_1 = \sum_k \abs{\sum_{i,j,l} p(i) q(j) \tau(i,j | k,l)} \\
        &= \sum_{k \neq \gamma_U, \gamma_L} \abs{\sum_{i,j,l} p(i) q(j) \tau(i,j|k,l)} + \abs{\sum_{i,j,l} p(i) q(j) \tau(i,j | \gamma_U, l)} + \abs{\sum_{i,j,l} p(i) q(j) \tau(i,j | \gamma_L, l)} \\
        &\geq \abs{\sum_{i,j,l} p(i) q(j) \tau(i,j | \gamma_U, l)} + \abs{\sum_{i,j,l} p(i) q(j) \tau(i,j | \gamma_L, l)} \\
        &\geq \sum_{i,j,l} p(i) q(j) \tau(i,j | \gamma_U, l) + \sum_{i,j,l} p(i) q(j) \tau(i,j | \gamma_L, l)
     \end{align}
     
     It will prove simpler to lower bound one of the two terms and the other will follow almost the exact same manner.
     \begin{align}
        \sum_{i,j,l} p(i) q(j) \tau(i,j| \gamma_U,l) &= \sum_{i \neq \gamma_U} \sum_{j,l} (p(i) q(j) - p(\gamma_U) q(l)) \tau(i,j|\gamma_U, l) \\
        &= (p(\gamma_L) q(1) - p(\gamma_U) q(0)) \tau(\gamma_L, 1| \gamma_U, 0) + \sum_{(j,l) \neq (1, 0)} (p(\gamma_L) q(j) - p(\gamma_U) q(l)) \tau(\gamma_L, j| \gamma_U , l) + \sum_{i \neq \gamma_U, \gamma_L} \sum_{j,l} (p(i) q(j) - p(\gamma_U) q(l)) \tau(i,j | \gamma_U, l) \\
        &\geq (p(\gamma_L) q(1) - p(\gamma_U) q(0)) \tau(\gamma_L, 1| \gamma_U, 0) + \sum_{(j,l) \neq (1, 0)} (-1) \tau(\gamma_L, j | \gamma_U, l) + \sum_{i \neq \gamma_U, \gamma_L} \sum_{j,l} (-1) \tau(i,j | \gamma_U,l) \\
        &\geq (p(\gamma_L) q(1) - p(\gamma_U) q(0)) \tau(\gamma_L, 1| \gamma_U, 0) - 3 \frac{\alpha^2 t^2}{\dim + 1} \epsilon_{\sinc} - (\dim_S - 2) 4 \frac{\alpha^2 t^2}{\dim + 1}\epsilon_{\sinc} \\
        &\geq (p(\gamma_L) q(1) - p(\gamma_U) q(0)) \tau(\gamma_L, 1| \gamma_U, 0) - 2 \dim \frac{\alpha^2 t^2}{\dim + 1} \epsilon_{\sinc}.
     \end{align}
     Now we note that $\gamma$ is within $\Delta_{\sinc}$ of $\Delta_S(\gamma_U, \gamma_L)$ which allows us to say $1 - \epsilon_{\sinc} \leq \tau(\gamma_U, 0 | \gamma_L, 1) \frac{\dim + 1}{\alpha^2 t^2} \leq 1$ per Corollary \ref{cor:gamma_difference_reqs}. This simplifies the above bound to
     \begin{equation}
        (p(\gamma_L) q(1) - p(\gamma_U)q(0)) \frac{\alpha^2 t^2}{\dim + 1} - (2 \dim + 1) \epsilon_{\sinc} \frac{\alpha^2 t^2}{\dim + 1}.
     \end{equation}
     This is rather unfortunate, because it is the wrong term we should analyze first. To see this, as $\beta_E \to \infty$ we have $q(0) \to 1, q(1) \to 0$. This means the above quantity is negative, implying it only contributes to the norm lower bound because of the absolute value. We instead look for the opposite term.

     Writing out the same for $\gamma_L$ yields
     \begin{align}
        \sum_{i,j,l} p(i) q(j) \tau(i,j | \gamma_L, l) &\geq \parens{(p(\gamma_U) q(0) - p(\gamma_L) q(1))  - (2 \dim + 1) \epsilon_{\sinc}} \frac{\alpha^2 t^2}{\dim + 1}.
     \end{align}
     Now we see that this should be positive. To do so, we analyze the probability differences in more detail.
     \begin{align}
        p(\gamma_U) q(0) - p(\gamma_L) q(1) &= \frac{e^{-\beta \lambda_S(\gamma_U)}}{\partfun_S(\beta)} \frac{e^{-\beta_E \lambda_E(0)}}{\partfun_E(\beta_E)} - \frac{e^{-\beta \lambda_S(\gamma_L)}}{\partfun_S(\beta)} \frac{e^{-\beta_E \lambda_E(1)}}{\partfun_E(\beta_E)} \\
        &= \frac{e^{-\beta \lambda_S(\gamma_U)}}{\partfun_S(\beta) \partfun_E(\beta_E)} \parens{1 - e^{-\beta_E \gamma + \beta \Delta_S(\gamma_U, \gamma_L)}}.
     \end{align}
     For now, let $\Delta_S = \Delta_S(\gamma_U, \gamma_L)$ to save space. We show that $\delta \geq \frac{\beta_E \Delta_{\min} + \ln (1 / \epsilon_{exp})}{\Delta_S} \geq \beta_E \frac{\Delta_{\min}}{\Delta_S}$ implies $1 - e^{-\beta_E \gamma + \beta \Delta_S} \geq 1 - \epsilon_{exp}$. Note the assumption $\ln (1/ \epsilon_{exp}) \geq 0$. 
     \begin{align}
        \ln (1 / \epsilon_{exp}) &\leq \delta \Delta_S - \beta_E \Delta_{\min} \\
        \iff \ln (1 / \epsilon_{exp}) &\leq \beta_E (\Delta_S - \Delta_{\min}) - \beta \Delta_S \\
        &\leq \beta_E \gamma - \beta \Delta_S \\
        \iff \epsilon_{exp} &\geq e^{-\beta_E \gamma + \beta \Delta_S} \\
        \iff 1 - e^{-\beta \gamma + \beta \Delta_S}  &\geq 1 - \epsilon_{exp}.
     \end{align}

     Plugging this in, along with a simple substitution that $\epsilon_{exp} = \epsilon_{\sinc}$ yields 
     \begin{align}
        \sum_{i,j,l} p(i) q(j) \tau(i,j | \gamma_L, l) &\geq \parens{\frac{e^{-\beta \lambda_S(\gamma_U)}}{\partfun_S(\beta) \partfun_E(\beta_E)} (1 - \epsilon_{exp}) - (2 \dim + 1) \epsilon_{\sinc}} \frac{\alpha^2 t^2}{\dim + 1} \\
        &\geq \parens{\frac{e^{-\beta \lambda_S(\gamma_U)}}{\partfun_S(\beta) \partfun_E(\beta_E)} - 2 (\dim + 1) \epsilon_{\sinc}} \frac{\alpha^2 t^2}{\dim + 1}
     \end{align}
\end{proof}


% \begin{proof}
% \textbf{This is a bogus proof}
%     We need to show that there is no way to lower bound the distance of the actual channel, instead we have to lower bound the distance of $\rho_S(\beta_E)$ and the second order approximation to the channel. 
%     The following is what we want to show. \todo{this should be $\beta$ not $\beta_E$ ...}
%     \begin{align}
%         \norm{\rho_S(\beta_E) - \Phi^{(2)}(\rho_S(\beta))}_2^2 \geq \epsilon_{fix}.
%     \end{align}
%     We use Lemma (insert schatten deconstruction lemma here) to get
%     \begin{align}
%         \norm{\rho_S(\beta_E) - \Phi^{(2)}(\rho_S(\beta))}_2^2 &= \sum_k \abs{\frac{e^{-\beta_E \lambda_S(k)}}{\partfun_S(\beta_E)} - \frac{e^{-\beta \lambda_S(k)}}{\partfun_S(\beta)} - \sum_{i,j,l} \frac{e^{-\beta \lambda_S(i) - \beta_E \lambda_E(j)}}{\partfun_S(\beta) \partfun_E(\beta_E)} \tau(i,j|k ,l)}^2 \\
%         &\geq \abs{\frac{e^{-\beta_E \lambda_S(0)}}{\partfun_S(\beta_E)} - \frac{e^{-\beta \lambda_S(0)}}{\partfun_S(\beta)} - \sum_{i,j,l} \frac{e^{-\beta \lambda_S(i) - \beta_E \lambda_E(j)}}{\partfun_S(\beta) \partfun_E(\beta_E)}\tau(i,j|0,l)}^2 \\
%         &= \parens{\frac{e^{-\beta_E \lambda_S(0)}}{\partfun_S(\beta_E)} - \frac{e^{-\beta \lambda_S(0)}}{\partfun_S(\beta)}}^2 \nonumber \\
%         &\quad - 2 \parens{\frac{e^{-\beta_E \lambda_S(0)}}{\partfun_S(\beta_E)} - \frac{e^{-\beta \lambda_S(0)}}{\partfun_S(\beta)}} \sum_{i,j,l} \frac{e^{-\beta \lambda_S(i) - \beta_E \lambda_E(j)}}{\partfun_S(\beta) \partfun_E(\beta_E)} \tau(i,j|0,l) \nonumber \\
%         &\quad + \parens{\sum_{i,j,l} \frac{e^{-\beta \lambda_S(i) - \beta_E \lambda_E(j)}}{\partfun_S(\beta) \partfun_E(\beta_E)} \tau(i,j|0,l)}^2.
%     \end{align}
%     We now need to simplify
%     \begin{align}
%         \sum_{i,j,l} \frac{e^{-\beta \lambda_S(i) - \beta_E \lambda_E(j)}}{\partfun_S(\beta) \partfun_E(\beta_E)} \tau(i,j|0,l) &= \sum_{i \neq 0} (a(i) b(j) - a(k) b(l)) \tau(i,j|0,l) \\
%         &\geq \sum_{i \neq 0} \parens{(a(i) b(0) - a(0) b(1)) \tau(i, 0 | 0, 1) - 3 \epsilon_{\sinc}}
%     \end{align}
%     \todo{propagate missing factors of $\alpha^2 t^2$ on the $\epsilon_{\sinc}$}
%     Now we assume that there exists one and only one index $\kappa$ such that $\abs{\Delta_S(\kappa, 0) - \gamma} \leq \Delta_{\min}$. This ensures that if $i \neq \kappa$ then $\tau(i,j | 0, l) \leq \epsilon_{\sinc}$ and that $\epsilon_{\sinc} \leq \tau(\kappa, 0 | 0, 1) \leq 1$. This allows us to simplify the sum as
%     \begin{align}
%         \sum_{i \neq 0} \parens{(a(i) b(0) - a(0) b(1)) \tau(i, 0 | 0, 1) - 3 \epsilon_{\sinc}} &= \parens{(a(\kappa) b(0) - a(0) b(1)) \tau(\kappa, 0 | 0, 1) - 3 \epsilon_{\sinc}} \nonumber \\
%         &\quad + \sum_{i \neq 0, \kappa} \parens{(a(i) b(0) - a(0) b(1)) \tau(i, 0 | 0, 1) - 3 \epsilon_{\sinc}} \\
%         &\geq (a(\kappa) b(0) - a(0) b(1)) \tau(\kappa, 0 | 0, 1) - \epsilon_{\sinc}(3 + 4 (\dim - 2))
%     \end{align}

%     We now explore the prefactor in front of the transition factor,
%     \begin{align}
%         a(i)b(0) - a(0) b(1) &= \frac{1}{\partfun_S(\beta) \partfun_E(\beta_E)} \parens{e^{-\beta \lambda_S(i) - \beta_E \lambda_E(0)} - e^{-\beta \lambda_S(0) - \beta_E \lambda_E(1)}} \\
%         &= \frac{e^{-\beta \lambda_S(i)}}{\partfun_S(\beta) \partfun_E(\beta_E)} \parens{1 - e^{\beta \Delta_S(i,0) - \beta_E \gamma}}.
%     \end{align}
%     As we would like the theorem statement to hold in the limit as $\beta_E \to \infty$, we expect that this factor should be positive for finite $\beta$ and $\Delta_S(i,0)$. We utilize the assumptions $\beta_E - \beta > \delta$ and $|\Delta_S(i,0) - \gamma| \leq \Delta_{\min}$ to show that this factor is positive for certain values of $\delta$.
%     \begin{align}
%         1 - e^{\beta \Delta_S(i,0) - \beta_E \gamma} \geq 0 \iff \beta_E \gamma &\geq \beta \Delta_S(i,0) \\
%         &\iff 0 \leq \beta_E \gamma - \beta \Delta_S(i,0)  \\
%         &\iff 0 \leq \beta_E \parens{\gamma - \Delta_S(i,0)} + (\beta_E - \beta) \Delta_S(i,0).
%     \end{align}
%     Now we know that $|\Delta_S(i,0) - \gamma| \leq \Delta_{\min}$ implies that $\gamma - \Delta_S(i,0) \geq - \Delta_{\min}$. This shows that $\beta_E(\gamma - \Delta_S(i,0)) \geq - \beta_E \Delta_{\min}$. For the other term we know that $\beta_E - \beta > \delta$. Combining these shows that 
%     \begin{align}
%         &\beta_E(\gamma - \Delta_S(i,0)) + (\beta_E - \beta) \Delta_S(i,0) > - \beta_E \Delta_{\min} + \delta \Delta_S(i,0) \\
%         &\delta \geq \frac{\beta_E \Delta_{\min}}{\Delta_S(i,0)} \implies 1 - e^{\beta \Delta_S(i,0) - \beta_E \gamma} \geq 0
%     \end{align}

%     For now we will use $\Delta_S$ as shorthand for $\Delta_S(i,0)$. Here we want to show that if $1 - e^{-(\beta_E \gamma - \beta \Delta_S)}$ is greater than some value, say $\epsilon_{exp}$. To do this, all we have to show is that $x \geq \ln\parens{\frac{1}{\epsilon_{exp}}} \implies 1 - e^{-x} \geq 1 - \epsilon_{exp}$. As we know that $\beta_E \gamma - \beta \Delta_S > -\beta_E \Delta_{\min} + \delta \Delta_S$, this reduces to requiring that $\delta \geq \frac{\beta_E \Delta_{\min} + \ln{1/\epsilon_{exp}}}{\Delta_S(i,0)}$. So now we have that 
%     \begin{equation}
%         \delta \geq \frac{\beta_E \Delta_{\min} + \ln 1/\epsilon_{exp}}{\Delta_S(i,0)} \implies 1 - e^{-\beta_E \gamma + \beta \Delta_S(i,0)} \geq 1 - \epsilon_{exp}.
%     \end{equation}
%     This reduces our inequality to
%     \begin{equation}
%         \sum_{i,j,l} \frac{e^{-\beta \lambda_S(i) - \beta_E \lambda_E(j)}}{\partfun_S(\beta) \partfun_E(\beta_E)} \tau(i,j|0,l) \geq \frac{e^{-\beta \lambda_S(\kappa)}}{\partfun_S(\beta) \partfun_E(\beta_E)} (1 - \epsilon_{exp}) \tau(\kappa, 0 |0, 1) - \epsilon_{\sinc} (4 \dim - 5).
%     \end{equation}
%     Now we utilize our knowledge of $\Delta_S(\kappa, 0 )$ to create another inequality about $\tau$. If $\abs{\gamma - \Delta_S(\kappa, 0)} \leq \frac{2 \sqrt{\widetilde{\epsilon_{\sinc}}}}{t}$ then $\tau(\kappa, 0| 0, 1) \geq \frac{\alpha^2 t^2}{\dim + 1} (1 - \widetilde{\epsilon_{\sinc}})$. Our final lower bound for this summation is then
%     \begin{equation}
%         \sum_{i,j,l} \frac{e^{-\beta \lambda_S(i) - \beta_E \lambda_E(j)}}{\partfun_S(\beta) \partfun_E(\beta_E)} \tau(i,j|0,l) \geq \frac{e^{-\beta \lambda_S(\kappa)}}{\partfun_S(\beta) \partfun_E(\beta_E)} \frac{\alpha^2 t^2}{\dim + 1} (1 - \epsilon_{exp}) (1 - \widetilde{\epsilon}_{\sinc}) - \epsilon_{\sinc} (4 \dim - 5).
%     \end{equation}

%     Our next objective is to create an upper bound for this summation. We can again upper bound the trivial transitions with $\epsilon_{\sinc}$
%     \begin{align}
%         \sum_{i,j,l} \frac{e^{-\beta \lambda_S(i) - \beta_E \lambda_E(j)}}{\partfun_S(\beta) \partfun_E(\beta_E)} \tau(i,j|0,l) &= \sum_{j,l} (a(\kappa) b(j) - a(0) b(l)) \tau(\kappa,j|0,l) \nonumber \\
%         &\quad + \sum_{i \neq 0, \kappa} \sum_{j,l} (a(i) b(j) - a(0) b(l)) \tau(i,j|0,l) \\
%         &\leq (a(\kappa) b(0) -a(0) b(1)) \tau(\kappa, 0 | 0, 1) + \frac{\alpha^2 t^2}{\dim + 1} \epsilon_{\sinc}\parens{3 + 4(\dim - 2)}.
%     \end{align}
%     Now we repeat a similar process to before and upper bound $a(\kappa) b(0) - a(0) b(1) = \frac{e^{-\beta \lambda_S(\kappa)}}{\partfun_S(\beta) \partfun_E(\beta_E)}(1 - e^{-\beta_E \gamma + \beta \Delta_S(\kappa, 0)})$. This time we will only require the simple bound $1 - e^{-\beta_E \gamma + \beta \Delta_S(\kappa, 0)} \leq 1$, which is true as $e^x \geq 0$ for all finite $x$. We then use the fact that $\tau(\kappa, 0| 0,1) \leq \frac{\alpha^2 t^2}{\dim + 1}$ to produce our final upper bound
%     \begin{align}
%         \sum_{i,j,l} \frac{e^{-\beta \lambda_S(i) - \beta_E \lambda_E(j)}}{\partfun_S(\beta) \partfun_E(\beta_E)} \tau(i,j|0,l) &\leq \frac{\alpha^2 t^2}{\dim + 1}\parens{ \frac{e^{-\beta \lambda_S(\kappa)}}{\partfun_S(\beta) \partfun_E(\beta_E)} + 4 \dim \epsilon_{\sinc}}.
%     \end{align}
%     As we should think of $\epsilon_{\sinc}$ as approaching 0 as $t \to \infty$, we see that the sum should end up very close to $\frac{\alpha^2 t^2}{\dim + 1} \frac{e^{-\beta \lambda_S(\kappa)}}{\partfun_S(\beta) \partfun_E(\beta_E)}$. 
% \end{proof}

Want to find the ``single shot" capture radius. First should study the single-shot capture radius as $\beta_E \to \infty$, as this should simplify a lot of things. 
\begin{claim}
    We want to characterize the single-shot capture radius, or the $\delta = \beta_E - \beta$ such that 
    \begin{equation}
        \norm{\rho_S(\beta_E) - \mathbb{E}_{\gamma} \Phi_{\gamma}(\rho_S(\beta))}_1 \leq \frac{1}{2} \norm{\rho_S(\beta_E) - \rho_S(\beta)}_1 .
    \end{equation}
\end{claim}
\begin{proof}
See scratch work below.
\end{proof}
We first expand the channel using the Taylors series given by Eq. \eqref{idk}
\begin{align}
    \norm{\rho_S(\beta_E) - \mathbb{E}_{\gamma} \Phi_{\gamma}(\rho_S(\beta))}_1 &= \norm{\rho_S(\beta_E) - \rho_S(\beta) - \mathbb{E}_{\gamma} T_{\gamma}^{(2)}(\rho_S(\beta)) - \mathbb{E}_{\gamma} R_{\Phi}}_1 \\
    &\leq \norm{\rho_S(\beta_E) - \rho_S(\beta) - \mathbb{E}_{\gamma} T_{\gamma}^{(2)}(\rho_S(\beta))}_1 + \norm{\mathbb{E}_{\gamma} R_{\Phi}}_1.
\end{align}
As $T_{\gamma}^{(2)}$ is diagonal in the $H_S$ basis we can expand the one norm of the leftmost term as
\begin{align}
    \norm{\rho_S(\beta_E) - \rho_S(\beta) - \mathbb{E}_{\gamma} T_{\gamma}^{(2)}(\rho_S(\beta))}_1 &= \sum_k \abs{p_E(k) - p_{\beta}(k) - \mathbb{E}_{\gamma} \bra{k}  T_{\gamma}^{(2)}(\rho_S(\beta)) \ket{k}} .
\end{align}
We will investigate a generic term in the above expression by substituting in the Boltzmann remainder $R_0$ used in Lemma \ref{lem:thermal_state_diff_bound} and the characterization of the expected second order term from Lemma \ref{lem:expected_second_order_term}. Substituting these in gives
\begin{align}
    &\abs{p_E(k) - p_{\beta}(k) - \mathbb{E}_{\gamma} \bra{k} T_{\gamma}^{(2)}(\rho_S(\beta)) \ket{k}} \\
    &\leq \parens{\frac{4 \alpha}{\Delta_{\min}}}^2 + \abs{R_0(\beta, \beta_E, k) - \frac{\alpha^2 t^2}{\dim + 1} \frac{2}{\dim_S (\dim_S - 1)}\parens{\sum_{i \neq k} \frac{p_{\beta}(i)}{1 + e^{-\beta_E |\Delta_S(i,k)|}} - \frac{p_{\beta}(k)}{1 + e^{+ \beta_E |\Delta_S(i,k)|}}}}.
\end{align}
Zooming in even more, we need to substitute in one of the following forms of the remainder term
\begin{align}
    R_0(\beta, \beta_E, k) &= (\beta - \beta_E) \frac{e^{-\beta_{\star} \lambda_S(k)}}{\partfun_S(\beta_{\star})} \parens{\sum_j \frac{e^{-\beta_{\star} \lambda_S(j)}}{\partfun_S(\beta_{\star})} (\lambda_S(j) - \lambda_S(k))} \\
    &= \int_{\beta}^{\beta_E} (x - \beta) \frac{e^{-x \lambda_S(k)}}{\partfun_S(x)}\parens{\sum_j \frac{e^{-x \lambda_S(j)}}{\partfun_S(x)}( \lambda_S(j) - \lambda_S(k)},
\end{align}
where $\beta_{\star} \in (\beta, \beta_E)$ is an arbitrary point guaranteed by Taylor's Theorem. Note by restricting the sum to 
Want a positive $R_0$, so think about ground state for now. Call $\frac{\alpha^2 t^2}{\dim +1} \frac{2}{\dim_S(\dim_S - 1)}$ this some function $c$
\begin{align}
    c \sum_{i \neq 0} \frac{p_{\beta}(i)}{1 + e^{- \beta_E |\Delta_S(i,k)|}} - \frac{p_{\beta}(0)}{1 + e^{+ \beta_E |\Delta_S(i,k)|}} \geq \frac{1}{2} R_0(\beta, \beta_E, 0)
\end{align}
is our goal.  


\begin{claim}[Cooling Schedule]
    Suppose
\end{claim}
\begin{proof}
    Let $0 = \beta_0 \leq \beta_1 \leq \ldots \leq \beta_L = \beta$ denote a cooling schedule and $$\Phi_{i}(X) = \partrace{\hilb_E}{\int e^{-i (H + \alpha G)t} X \otimes \rho_E(\beta_i) e^{+i(H + \alpha G)t} dG},$$
    where we are simply changing the environment temperature at each step of the schedule with the end goal of preparing a state that approximates $\rho_S(\beta)$. We claim
    \begin{align}
        \norm{\rho_S(\beta) - \Phi_{L} \circ \Phi_{L - 1} \circ \ldots \circ \Phi_{1} (\identity / \dim)}_1 &= \norm{\rho_S(\beta_L) - \Phi_{L} (\rho_S(\beta_{L - 1})) + \Phi_L (\rho_S(\beta_{L-1})) - \Phi_L \circ \ldots \circ \Phi_1 (\rho_S(0))}_1 \\
        &\leq \norm{\rho_S(\beta_L) - \Phi_L (\rho_S(\beta_{L-1}))}_1 + \norm{\Phi_L \parens{\rho_S(\beta_{L-1}) - \Phi_{L-1} \circ \ldots \circ \Phi_1 (\rho_S(0))}}_1 \\
        &\leq \norm{\rho_S(\beta_L) - \Phi_L (\rho_S(\beta_{L-1}))}_1 + \norm{\rho_S(\beta_{L-1}) - \Phi_{L-1} \circ \ldots \circ \Phi_1 (\rho_S(0))}_1 \\
        & \ldots \nonumber \\
        &\leq \sum_{i = 1}^L \norm{\rho_S(\beta_i) - \Phi_i (\rho_S(\beta_{i - 1}))}_1
    \end{align}
\end{proof}

\bibliographystyle{unsrt}
\bibliography{bib}

\appendix




\section{Detailed Balance} \label{sec:detailed_balance}
Given that we are able to rotate the basis we work with into the System Hamiltonian's eigenbasis, and moreover we previously showed that off-diagonal coherences in this basis are negligible, we can treat this process as a stochastic one. Explicitly, if we give a diagonal state as input to the channel $\Phi$, to order $\bigo{\alpha^2}$ we get a diagonal state back. Since diagonal states can be thought of as probability distributions, the map $\Phi$ can roughly be thought of as a Markov process on the eigenstates of the Hamiltonian $H_S$. Since our goal is to show that this map produces a thermal state,
we would need to show that this probability distribution converges to the Boltzmann distribution $e^{-\beta_E \lambda_S(i)} / \partfun_S(\beta_E)$, for state $i$. In classical probability sampling literature this is done through a Detailed Balance calculation, which shows that the desired distribution is a fixed point. Ergodicity arguments of the desired Markov chain are then used to claim that the fixed point is unique and that therefore the process produces samples of the desired distribution.

Given that our map is only approximate, we cannot make rigorous claims regarding detailed balance at fixed $\alpha, t,$ or $\gamma$. However, in this section we show that in the appropriate limits that our mapping satisfies Detailed Balance exactly. This is enough to give solid evidence that for appropriate regimes our channel should converge approximately to this goal distribution.



\begin{align}
    &\prob{\text{System transition } i \to j | \text{ env at } \beta_E} \nonumber \\
    &\approx \frac{\alpha^2 t^2}{\dim + 1} \parens {\frac{e^{-\beta_E \gamma}}{1 + e^{-\beta_E \gamma}}  \sinc^2((\Delta(i,j) + \gamma)t) + \frac{1}{1 + e^{-\beta_E \gamma}} \sinc^2 ((\Delta(i,j) - \gamma)t) + 2 \sinc^2(\Delta(i,j) t)} \nonumber
\end{align}

\begin{align}
    &\prob{\text{System transition } i \to j | \text{ env at } \beta_E} \nonumber \\
    &= \bra{j} \Phi_{\gamma}(\ketbra{i}{i}) \ket{j} \\
    &= \braket{j}{i}\braket{i}{j} + \sum_{k,l} \tau(i, k | j, l) \frac{e^{-\beta_E \lambda_E(k)}}{\partfun_E(\beta_E)} + \bra{j} R_{\Phi}(\ketbra{i}{i})\ket{j}
\end{align}
Now as we are studying Detailed Balance, we assume that $i \neq j$. For our purposes, without loss of generality we let $\lambda_S(i) \geq \lambda_S(j)$, so by transitioning from $i \to j$ we are losing energy to the environment. We simplify the non-trivial term from above as
\begin{align}
    &\sum_{k,l} \tau(i,k| j,l) \frac{e^{-\beta_E \lambda_E(k)}}{\partfun_E(\beta_E)} = \frac{e^{-\beta_E \lambda_E(0)}}{\partfun_E(\beta_E)}(\tau(i,0|j,0) + \tau(i,0|j,1)) + \frac{e^{-\beta_E \lambda_E(1)}}{\partfun_E(\beta_E)} (\tau(i,1|j,0) + \tau(i,1|j,1) \\
    &= \frac{\alpha^2 t^2}{\dim + 1} \bigg(\frac{1}{1 + e^{-\beta_E \gamma}} (\sinc^2(\Delta_S(i,j)t/2) + \sinc^2((\Delta_S(i,j) - \gamma)t/2)) \nonumber \\
    &\quad \quad \quad \quad \quad \quad  +\frac{e^{-\beta_E \gamma}}{1 + e^{-\beta_E \gamma}} (\sinc^2((\Delta_S(i,j) + \gamma)t/2) + \sinc^2(\Delta_S(i,j) t/2)) \bigg) \\
    &\frac{\alpha^2 t^2}{\dim + 1} \bigg(\frac{1}{1 + e^{-\beta_E \gamma}} \sinc^2((\Delta_S(i,j) - \gamma)t/2) + \frac{e^{-\beta_E \gamma}}{1 + e^{-\beta_E \gamma}} \sinc^2((\Delta_S(i,j) + \gamma)t/2) + \sinc^2(\Delta_S(i,j)t/2)
\end{align}
As $\Delta_S(i,j) \geq 0 $ we expect that only the $\sinc^2((\Delta_S(i,j) - \gamma) t/2)$ term will contribute significantly to this sum. We can similarly write down the probability of the state to transition from $j \to i$ as
\begin{align}
    \prob{\text{System transition } j \to i | \text{ env at } \beta_E} = \sum_{k,l} \tau(j,k|i,l) \frac{e^{-\beta_E \lambda_E(k)}}{\partfun_E(\beta_E)} + \bra{i} R_{\Phi}(\ketbra{j}{j})\ket{i},
\end{align}
where we write the non-trivial term as
\begin{align}
    \sum_{k,l} \tau(j,k|i,l) \frac{e^{-\beta_E \lambda_E(k)}}{\partfun_E(\beta_E)} = \frac{\alpha^2 t^2}{\dim + 1} \bigg(&\frac{1}{1 + e^{-\beta_E \gamma}} (\sinc^2(\Delta_S(j,i)t/2) + \sinc^2((\Delta_S(j,i) - \gamma)t/2)) \nonumber \\
    &+\frac{e^{-\beta_E \gamma}}{1 + e^{-\beta_E \gamma}} (\sinc^2((\Delta_S(j,i) + \gamma)t/2) + \sinc^2(\Delta_S(j,i) t/2)) \bigg).
\end{align}
We simplify this by noting $\Delta_S(i,j) = - \Delta_S(j,i)$ and that $\sinc^2(x) = \sinc^2(-x)$ to get
\begin{align}
&\sum_{k,l} \tau(j,k|i,l) \frac{e^{-\beta_E \lambda_E(k)}}{\partfun_E(\beta_E)} \\
&= \frac{\alpha^2 t^2}{\dim + 1} \bigg(\frac{1}{1 + e^{-\beta_E \gamma}} \sinc^2((\Delta_S(i,j) + \gamma)t/2) + \frac{e^{-\beta_E \gamma}}{1 + e^{-\beta_E \gamma}} \sinc^2((\Delta_S(i,j) - \gamma)t/2) + \sinc^2(\Delta_S(i,j)t/2)
\end{align}

Now we would like to show some form of Detailed Balance, in appropriate limits, for the thermal state of the system. We will show that Detailed Balance holds in expectation, or with some non-zero probability. The thermal state give $\prob{\text{state in } i} = \frac{e^{-\beta_E \lambda_S(i)}}{\partfun_S(\beta_E)}$. We then need to expand differences
\begin{align}
    \chi(i,j) \coloneqq &\prob{\text{System transition } i \to j | \text{ env at } \beta_E} \prob{\text{state in } i} \nonumber \\
    &- \prob{\text{System transition } j \to i | \text{ env at } \beta_E} \prob{\text{state in } j}.
\end{align}
This expression is written in full glory as
\begin{align}
    \chi(i,j) &= \frac{e^{-\beta_E \lambda_S(i)}}{\partfun_S(\beta_E)} \bra{j} R_{\Phi}(\ketbra{i}{i})\ket{j} \nonumber \\
    &+ \frac{e^{-\beta_E \lambda_S(i)}}{\partfun_S(\beta_E)} \frac{\alpha^2 t^2}{\dim + 1} \frac{1}{1 + e^{-\beta_E \gamma}} \sinc^2((\Delta_S(i,j) - \gamma) t/2) \nonumber \\
    &+\frac{e^{-\beta_E \lambda_S(i)}}{\partfun_S(\beta_E)} \frac{\alpha^2 t^2}{\dim + 1} \frac{e^{-\beta_E \gamma}}{1 + e^{-\beta_E \gamma}} \sinc^2((\Delta_S(i,j) + \gamma) t/2) \nonumber \\
    &+ \frac{e^{-\beta_E \lambda_S(i)}}{\partfun_S(\beta_E)} \frac{\alpha^2 t^2}{\dim + 1} \sinc^2(\Delta_S(i,j)t/2) \nonumber \\
    &-\frac{e^{-\beta_E \lambda_S(j)}}{\partfun_S(\beta_E)} \bra{i} R_{\Phi}(\ketbra{j}{j})\ket{i} \nonumber \\
    &- \frac{e^{-\beta_E \lambda_S(j)}}{\partfun_S(\beta_E)} \frac{\alpha^2 t^2}{\dim + 1} \frac{1}{1 + e^{-\beta_E \gamma}} \sinc^2((\Delta_S(i,j) + \gamma) t/2) \nonumber \\
    &-\frac{e^{-\beta_E \lambda_S(j)}}{\partfun_S(\beta_E)} \frac{\alpha^2 t^2}{\dim + 1} \frac{e^{-\beta_E \gamma}}{1 + e^{-\beta_E \gamma}} \sinc^2((\Delta_S(i,j) - \gamma) t/2) \nonumber \\
    &- \frac{e^{-\beta_E \lambda_S(j)}}{\partfun_S(\beta_E)} \frac{\alpha^2 t^2}{\dim + 1} \sinc^2(\Delta_S(i,j)t/2).
\end{align}
In order to simplify this, we will group these expressions with an end goal in mind. As we would like this to hold for all $i \neq j$ but we have a fixed $\gamma$, we need to randomly choose a $\gamma$ and show that this holds in expectation. Further, we would like to bound the absolute value of these differences. So we want $\mathbb{E}_{\gamma}\abs{\chi(i,j)}$. We can then use the triangle inequality, the fact that $\abs{\bra{j}R_{\Phi}(\ketbra{i}{i})\ket{j}} \leq \norm{R_{\Phi}}$, $\abs{\frac{e^{-\beta_E \lambda_S(i)}}{\partfun_S(\beta_E)}} \leq 1$, 

along with the fact that transitions such as $\sinc^2(\Delta_S(i,j)t/2)$ are suppressed by $\epsilon_{\sinc}$ to get a simplified upper bound:
\begin{align}
    \mathbb{E}_{\gamma} \abs{\chi(i,j)} &\leq \mathbb{E}_{\gamma} 2 \norm{R_{\Phi}} \nonumber \\
    &+ \mathbb{E}_{\gamma} \frac{\alpha^2 t^2}{\dim + 1} 4 \epsilon_{\sinc}  \nonumber \\
    &+ \mathbb{E}_{\gamma} \frac{\alpha^2 t^2}{\dim + 1} \frac{1}{1 + e^{-\beta_E \gamma}} \frac{1}{\partfun_S(\beta_E)} \abs{\parens{e^{-\beta_E \lambda_S(i)} - e^{-\beta_E \gamma} e^{-\beta_E \lambda_S(j)}} \sinc^2((\Delta_S(i,j) - \gamma) t/2)} \\
    &\leq 2 \norm{R_{\Phi}} + 4 \epsilon_{\sinc} \frac{\alpha^2 t^2}{\dim + 1} \nonumber \\
    &+ \frac{\alpha^2 t^2}{\dim + 1} \mathbb{E}_{\gamma} \sinc^2((\Delta_S(i,j) - \gamma) t/2) \abs{1 - e^{\beta_E (\Delta_S(i,j) - \gamma)}}. \label{eq:detailed_balance_upper_bound}
\end{align}
This is where we have to introduce our distribution over $\gamma$. We note that if $\gamma$ is far away from $\Delta_S(i,j)$, then as $t \to \infty$ we have that this product is trivially 0. We want to show that Detailed Balance is satisfied even when $\gamma \to \Delta_S(i,j)$. To do so we upper bound $\sinc^2 \leq 1$ whenever $\abs{\Delta_S(i,j) - \gamma} \leq \Delta_{\min}$ and $\sinc^2 \leq \epsilon_{\sinc}$ whenever $\abs{\Delta_S(i,j) - \gamma} \geq \Delta_{\min}$. 

We now are at an impasse. Our goal for this argument is to show that if our channel does anything non-trivial, then in the appropriate limits ($t \to \infty, \alpha \to 0, \alpha t \to c_{small}$) it should satisfy detailed balance conditions, or at least get arbitrarily close to it. We see that when our channel has a $\gamma$ that is not close to any $\Delta_S(i,j)$ we do not induce any transitions among states (in the $t \to \infty$ limit). This then trivially satisfies Detailed balance, as $\prob{i \to j} = \prob{j \to i} = 0$ gives $0 = 0$ for Detailed Balance. We would like to show that all $i \neq j$ can get arbitrarily close to satisfying Detailed Balance by choosing a $\gamma$ randomly. To do so there are three obvious candidates for distributions of $\gamma$ that we could analyze theoretically:
\begin{enumerate}
    \item A maximum entropy prior, or choosing $\gamma$ uniformly from 0 to $\norm{H}$,
    \item A minimal entropy prior, or exact knowledge of $\Delta_S(i,j)$ for all $i,j$, where we choose indices or differences uniformly,
    \item Choose a difference $\Delta_S(i,j)$ uniformly and then add in noise, either in the form of a Gaussian or a simple uniform box centered about the gap $\Delta_S(i,j)$.
\end{enumerate}

The simplest for us to look at first is the maximum entropy prior, or the uniform distribution of $\gamma$ from 0 to $\norm{H}$. This then gives us two regimes, $|\Delta_S(i,j) - \gamma| \leq \Delta_{\min}$ and $|\Delta_S(i,j) - \gamma| > \Delta_{\min}$. The $\sinc^2$ term is upper bounded by 1 in the former and $\epsilon_{\sinc}$ in the latter. 
\begin{align}
    &\mathbb{E}_{\gamma} \sinc^2((\Delta_S(i,j) - \gamma)t/2) \abs{1 - e^{\beta_E (\Delta_S(i,j) - \gamma)}} \nonumber \\
    &= \frac{1}{\norm{H}} \int_0^{\norm{H}} \sinc^2((\Delta_S(i,j) - \gamma)t/2)\abs{1 - e^{\beta_E (\Delta_S(i,j) - \gamma)}} d\gamma \\
    &= \frac{1}{\norm{H}} \int_0^{\Delta_S(i,j) - \Delta_{\min}} \sinc^2((\Delta_S(i,j) - \gamma)t/2)\abs{1 - e^{\beta_E (\Delta_S(i,j) - \gamma)}} d\gamma \nonumber \\
    &\quad + \frac{1}{\norm{H}} \int_{\Delta_S(i,j) - \Delta_{\min}}^{\Delta_S(i,j) + \Delta_{\min}} \sinc^2((\Delta_S(i,j) - \gamma)t/2)\abs{1 - e^{\beta_E (\Delta_S(i,j) - \gamma)}} d\gamma \nonumber \\
    &\quad + \int_{\Delta_S(i,j) + \Delta_{\min}}^{\norm{H}} \sinc^2((\Delta_S(i,j) - \gamma)t/2)\abs{1 - e^{\beta_E (\Delta_S(i,j) - \gamma)}} d\gamma.
\end{align}
We will simplify these integrals separately. Starting with the first
\begin{align}
    &\frac{1}{\norm{H}} \int_0^{\Delta_S(i,j) - \Delta_{\min}} \sinc^2((\Delta_S(i,j) - \gamma)t/2)\abs{1 - e^{\beta_E (\Delta_S(i,j) - \gamma)}} d\gamma \nonumber \\
    &\leq \frac{\epsilon_{\sinc}}{\norm{H}} \int_0^{\Delta_S(i,j) - \Delta_{\min}} \left(e^{\beta_E(\Delta_S(i,j) - \gamma)} - 1\right) d\gamma \\
    &= \frac{\epsilon_{\sinc}}{\norm{H}}\parens{\Delta_{\min} - \Delta_S(i,j) + \frac{e^{\beta_E \Delta_S(i,j)}}{\beta_E}\left( 1 - e^{-\beta_E(\Delta_S(i,j) - \Delta_{\min})} \right)}.
\end{align}
We then compute the third, as it is more similar to the first as
\begin{align}
    &\frac{1}{\norm{H}} \int_{\Delta_S(i,j) + \Delta_{\min}}^{\norm{H}} \sinc^2((\Delta_S(i,j) - \gamma)t/2) \abs{1 - e^{\beta_E(\Delta_S(i,j) - \gamma)}} d\gamma \nonumber \\
    &\leq \frac{\epsilon_{\sinc}}{\norm{H}} \int_{\Delta_S(i,j) + \Delta_{\min}}^{\norm{H}} \left(1 - e^{\beta_E (\Delta_S(i,j) - \gamma)} \right) d\gamma \\
    &=\frac{\epsilon_{\sinc}}{\norm{H}}\parens{\norm{H} - (\Delta_S(i,j) + \Delta_{\min}) + \frac{1}{\beta_E} \left(e^{-\beta_E(\norm{H} - \Delta_S(i,j))} - e^{-\beta_E \Delta_{\min}} \right) }.
\end{align}
Adding the results of these two integrals yields
\begin{align}
    \frac{\epsilon_{\sinc}}{\norm{H}}\parens{\norm{H} - 2 \Delta_S(i,j) +  \frac{1}{\beta_E} e^{\beta_E \Delta_S(i,j)}(2 + e^{-\beta_E \norm{H}}) + \frac{e^{\beta_E \Delta_{\min}} - e^{-\beta_E \Delta_{\min}}}{\beta_E}}.
\end{align}
We make two observations. First that this is positive given that $\norm{H} \geq 2 \Delta_S(i,j)$ and tends towards infinity as $\beta_E \to \infty$. The second is that there is no time dependence on the factor within the parenthesis, meaning that for fixed $\beta_E$ we can make this quantity arbitrarily small by reducing $\epsilon_{\sinc} \propto 1/t^2$. 

The more reasonable distribution to analyze is the minimal entropy, or perfect knowledge distribution. In this distribution we pick an eigenvalue difference uniformly at random. We denote the set of eigenvalue gaps as $G_{\gamma}$. Then the expectation can be split into two: $S_{\gamma}$ being the set of gaps that are close to $\gamma$ and $T_{\gamma}$ as those that are far apart. Let $N_{diff}$ denote the number of differences. Specifically, let $S_{\gamma} = \set{\lambda_S(k) - \lambda_S(l) = \Delta_S(k,l) : \sinc^2((\Delta_S(i,j) - \Delta_S(k,l))t/2) \geq \epsilon_{\sinc}}$, and $T_{\gamma} = \set{\Delta_S(k,l) : \sinc^2((\Delta_S(i,j) - \Delta_S(k,l))t/2) < \epsilon_{\sinc}}$. Then the expected value over $\gamma$ becomes
\begin{align}
    &\mathbb{E}_{\gamma} \sinc^2((\Delta_S(i,j) - \gamma) t/2) \abs{1 - e^{\beta_E(\Delta_S(i,j) - \gamma)}} \nonumber \\
    &=\frac{1}{N_{diff}}\sum_{\gamma \in S_{\gamma}} \sinc^2((\Delta_S(i,j) - \gamma)t/2) \abs{1 - e^{\beta_E(\Delta_S(i,j) - \gamma)}} + \frac{1}{N_{diff}} \sum_{\gamma \in T_{\gamma}} \sinc^2((\Delta_S(i,j) - \gamma)t/2) \abs{1 - e^{\beta_E(\Delta_S(i,j) - \gamma)}} \\
    &\leq \frac{1}{N_{diff}} \sum_{\gamma \in S_{\gamma}} \abs{1 - e^{\beta_E(\Delta_S(i,j) -\gamma)}} + \frac{1}{N_{diff}} \epsilon_{\sinc} \sum_{\gamma \in T_{\gamma}}\abs{1 - e^{\beta_E(\Delta_S(i,j) - \gamma)}}.
\end{align}
We see that for the right hand sum the factor of $\epsilon_{\sinc}$, which vanishes as $t \to \infty$, causes the total sum to vanish as there are no explicit $t$ dependent terms. The set $T_{\gamma}$ does change with $t$, but it is upper bounded by a finite value and so is each possible summand. Now the real kicker is what happens to the leftmost summation. We investigate when an eigenvalue gap $\Delta_S(k,l)$ can be included in $S_{\gamma}$ as $t \to \infty$. 
\begin{align}
    \lim_{t \to \infty} \sinc^2((\Delta_S(i,j) - \Delta_S(k,l)) t/2) = \begin{cases}
        0 & \Delta_S(i,j) \neq \Delta_S(k,l) \\
        1 & \Delta_S(i,j) = \Delta_S(k,l).
    \end{cases}
\end{align}
Because of this $S_{\gamma} = \set{\Delta_S(i,j)}$ becomes a multiset consisting solely of $\Delta_S(i,j)$ with the number of degeneracies of the eigenvalue $S_{\gamma}$. This means that any term in the summation then becomes $\abs{1 - e^{\beta_E(\Delta_S(i,j) - \Delta_S(i,j)}} = 0$. Then given that $\epsilon_{\sinc} = 1/(\Delta_(\min)^2 t^2) \to 0$, we have that $\lim_{t \to \infty} \mathbb{E}_{\gamma} \sinc^2((\Delta_S(i,j) - \gamma)t/2) \abs{1 - e^{\beta_E(\Delta_S(i,j) - \gamma)}} = 0$. Looking at Eq. \eqref{eq:detailed_balance_upper_bound}, along with $\epsilon_{\sinc} = 1/(\Delta_{\min}^2 t^2)$ and $\alpha = \epsilon_{\alpha} / t$, we see that 
\begin{align}
    &\lim_{t \to \infty} \mathbb{E}_{\gamma} \abs{\chi(i,j)} \leq 2 \lim_{t \to \infty} \norm{R_{\Phi}} + 4 \lim_{t \to \infty} \frac{\epsilon_{\alpha}^2}{t^2 \Delta_{\min}^2 (\dim + 1)} \nonumber \\
    &+ \frac{\epsilon_{\alpha}^2}{\dim + 1} \lim_{t \to \infty} \sinc^2((\Delta_S(i,j) - \gamma)t/2) \abs{1 - e^{\beta_E(\Delta_S(i,j) - \gamma)}} \\
    &= 2 \lim_{t \to \infty} \norm{R_{\Phi}}.
\end{align}
As 



\begin{align}
    \frac{\prob{i \to j}}{\prob{j \to i}} &\approx \frac{\frac{1}{1 + e^{-\beta_E \gamma}}\sinc^2 ((\Delta(i,j) - \gamma)t)+ 3 \epsilon_{\sinc}}{\frac{e^{-\beta_E \gamma}}{1 + e^{-\beta_E \gamma}} \sinc^2((-\Delta(i,j) + \gamma)t) + 3 \epsilon_{\sinc}}  \nonumber \\
    \lim_{t \to \infty} &\implies e^{\beta_E \gamma} = e^{-\beta_E \Delta(i,j)} \nonumber \\
    &= \frac{\prob{\text{System in } j}}{ \prob{\text{System in } i}} \nonumber
\end{align}


\section{Thermal State Bounds}


\begin{lemma} \label{lem:thermal_state_diff_bound}
    Let $0 \leq \delta = \beta_E - \beta$ and $H_S$ be a Hamiltonian such that $\lambda_S(k) - \lambda_S(i) \geq \Delta_{\min}$ for all $k > i$. The trace distance between the thermal states $\rho_S(\beta)$ and $\rho_S(\beta_E)$ can be bounded from above and below as
    \begin{align}
        \frac{\Delta_{\min} \delta^2 \epsilon_{\rho}}{2 \dim_S} \leq \norm{\rho_S(\beta_E) - \rho_S(\beta)}_1 \leq 2 \dim_S \norm{H} \delta,
    \end{align}
    provided that $\beta_E$ satisfies the upper bound $\beta_E \leq \frac{1}{\Delta_{\min}} \parens{\ln (\dim_S - 1) + \ln \parens{1 + \frac{1}{\epsilon_{\rho}}}}$. 
\end{lemma}
\begin{proof}
    We first decompose the trace distance using the fact that thermal states are diagonal in the $H_S$ basis
    \begin{equation}
        \norm{\rho_S(\beta_E) - \rho_S(\beta)}_1 = \sum_k \abs{\frac{e^{-\beta_E \lambda_S(k)}}{\partfun_S(\beta_E)} - \frac{e^{-\beta \lambda_S(k)}}{\partfun_S(\beta)}}. \label{eq:thermal_diffs_1}
    \end{equation}
    This suggests a Taylor Series for the Boltzmann factors $\frac{e^{-\beta \lambda_S(k)}}{\partfun_S(\beta)}$ about $\beta = \beta_E$ would be useful. We will only need the zeroth order approximation,
    \begin{equation}
        \frac{e^{-\beta \lambda_S(k)}}{\partfun_S(\beta)} = \frac{e^{-\beta_E \lambda_S(k)}}{\partfun_S(\beta_E)} + R_0(\beta, \beta_E, k),
    \end{equation}
    and will use the integral form of $R_0$ for the lower bound and the Lagrange or mean-value form for the upper bound. 
    
    As the upper bound is easier we will compute it first. The remainder in this case is given by
    \begin{equation}
        R_0(\beta, \beta_E, k) = (\beta - \beta_E) \frac{d}{d\beta} \frac{e^{-\beta \lambda_S(k)}}{\partfun_S(\beta)} \bigg|_{\beta = \beta_{k}},
    \end{equation}
    where $\beta_{k} \in (\beta, \beta_E)$ is the arbitrary point guaranteed to exist by the Taylor's Theorem. To compute the derivative of the Boltzmann coefficient we first compute the derivative of the partition function
    \begin{align}
        \frac{d}{d\beta} \partfun_S(\beta) &= \frac{d}{d\beta} \sum_i e^{-\beta \lambda_S(i)} \\
        &= - \sum_i \lambda_S(i) e^{-\beta \lambda_S(i)} \\
        &= - \trace{H e^{-\beta H}}.
    \end{align}
    This gives the total derivative of the Boltzmann coefficient is computed as
    \begin{align}
        \frac{d}{d \beta}\frac{e^{-\beta \lambda_S(k)}}{\partfun_S(\beta)} &= - \lambda_S(k) \frac{e^{-\beta \lambda_S(k)}}{\partfun_S(\beta)} - \frac{e^{-\beta \lambda_S(k)}}{\partfun_S(\beta)^2} \frac{d}{d\beta}\partfun_S(\beta) \\
        &= \frac{e^{-\beta \lambda_S(k)}}{\partfun_S(\beta)} \parens{\trace{H \frac{e^{-\beta H}}{\partfun_S(\beta)}} - \lambda_S(k)}. \label{eq:boltzmann_derivative}
    \end{align}
    We can plug this in to Eq. \eqref{eq:thermal_diffs_1} to get
    \begin{align}
        \norm{\rho_S(\beta_E) - \rho_S(\beta)}_1 &= \sum_k \abs{\frac{e^{-\beta_E \lambda_S(k)}}{\partfun_S(\beta_E)} - \frac{e^{-\beta \lambda_S(k)}}{\partfun_S(\beta)}} \\
        &= \sum_k \abs{R_0(\beta, \beta_E, k)} \\
        &= \sum_k \abs{\beta - \beta_E} \frac{e^{-\beta_{k} \lambda_S(k)}}{\partfun_S(\beta_{k})} \abs{\trace{\rho_S(\beta_{k})H} - \lambda_S(k)} \\
        &\leq \delta \sum_k \parens{\abs{\sum_{j} \frac{e^{-\beta_k \lambda_S(j)}}{\partfun_S(\beta_k)} \lambda_S(j)} + |\lambda_S(k)|} \\
        &\leq \delta \sum_k \parens{\norm{H} \sum_j \frac{e^{-\beta_k \lambda_S(j)}}{\partfun_S(\beta_k)} + \norm{H}} \\
        &\leq 2 \delta \dim_S \norm{H}.
    \end{align}
    This gives the upper bound required in the lemma statement.

    We now move on to the lower bound. We use the same breakdown in the eigenbasis of $H_S$, but this time we drop all terms that are not the ground state 
    \begin{align}
        \norm{\rho_S(\beta_E) - \rho_S(\beta)}_1 &= \sum_k \abs{\frac{e^{-\beta_E \lambda_S(k)}}{\partfun_S(\beta_E)} - \frac{e^{-\beta \lambda_S(k)}}{\partfun_S(\beta)}} \\
        &\geq \abs{\frac{e^{-\beta_E \lambda_S(0)}}{\partfun_S(\beta_E)} - \frac{e^{-\beta \lambda_S(0)}}{\partfun_S(\beta)}} \\
        &= \abs{R_0(\beta, \beta_E, 0)} \\
        &\geq R_0(\beta, \beta_E, 0),
    \end{align}
    and we will stop writing the implicit argument of 0 at the end from now on. 
    
    We drop the higher order contributions for a few reasons. The first being that thermal states are typically used as close approximations to ground states, which approaches the exact ground state as $\beta_E \to \infty$, and we are imagining a scenario in which $\beta$ is somewhat close to $\beta_E$. In this situation, the dominant contribution to the trace distance comes from the difference in the ground state Boltzmann factors. The other rationale behind just studying the ground state contribution is the fact that $\beta_1 \leq \beta_2$ implies that $\frac{e^{-\beta_1 \lambda_S(0)}}{\partfun_S(\beta_1)} \leq \frac{e^{-\beta_2 \lambda_S(0)}}{\partfun_S(\beta_2)}$, or that the overlap that the thermal state has with the ground state, $\trace{\Pi_0 \rho_S(\beta)}$, is monotonically increasing with $\beta$. This is shown using a straightforward lower bound on the first derivative computed in Eq. \eqref{eq:boltzmann_derivative}:
    \begin{align}
        \frac{d}{d\beta} \frac{e^{-\beta \lambda_S(0)}}{\partfun_S(\beta)} &= \frac{e^{-\beta \lambda_S(0)}}{\partfun_S(\beta)} \parens{ \sum_k \left(\lambda_S(k) \frac{e^{-\beta \lambda_S(k)}}{\partfun_S(\beta)} \right) - \lambda_S(0) } \\
        &= \frac{e^{-\beta \lambda_S(0)}}{\partfun_S(\beta)} \parens{ \sum_k \left(\lambda_S(k) \frac{e^{-\beta \lambda_S(k)}}{\partfun_S(\beta)} \right) - \lambda_S(0) \sum_k \frac{e^{-\beta \lambda_S(k)}}{\partfun_S(k)} } \\
        &= \frac{e^{-\beta \lambda_S(0)}}{\partfun_S(\beta)} \sum_k (\lambda_S(k) - \lambda_S(0)) \frac{e^{-\beta \lambda_S(k)}}{\partfun_S(\beta)}  \\
        &\geq \frac{e^{-\beta \lambda_S(0)}}{\partfun_S(\beta)} \Delta_{\min} \sum_{k > 0} \frac{e^{-\beta \lambda_S(k)}}{\partfun_S(k)} \\
        &= \Delta_{\min} \frac{e^{-\beta \lambda_S(0)}}{\partfun_S(\beta)} \parens{1 - \frac{e^{-\beta \lambda_S(0)}}{\partfun_S(\beta)}}, \label{eq:boltzmann_derivative_lower_bound}
    \end{align}
    where we see that this is always positive for finite $\beta$ and 0 in the limit $\beta \to \infty$.  
    
    To compute this lower bound we will use the integral version of the remainder $R_0$, which is given as 
    \begin{align}
        R_0(\beta, \beta_E) = \int_{\beta}^{\beta_E} (x - \beta) \frac{d}{d\beta} \frac{e^{-\beta \lambda_S(0)}}{\partfun_S(\beta)}\bigg|_{\beta = x} dx ~.
    \end{align}
    Plugging in the derivative computed in Eq. \eqref{eq:boltzmann_derivative} gives
    \begin{align}
        R_0(\beta, \beta_E) = \int_{\beta}^{\beta_E} (x - \beta) \frac{e^{-x \lambda_S(0)}}{\partfun_S(x)}\parens{\trace{H \frac{e^{- x H}}{\partfun_S(x)}} - \lambda_S(0)} dx,
    \end{align}
    where we note that this is positive as $x > \beta$ and the derivative is always positive, as noted above. Plugging in the lower bound computed in Eq. \eqref{eq:boltzmann_derivative_lower_bound} yields
    \begin{align}
        R_0(\beta, \beta_E) &\geq \Delta_{\min} \int_{\beta}^{\beta_E} (x - \beta) \frac{e^{-x \lambda_S(0)}}{\partfun_S(x)} \parens{1 - \frac{e^{-x \lambda_S(0)}}{\partfun_S(x)}} dx \\
        &\geq \Delta_{\min} \frac{e^{-\beta \lambda_S(0)}}{\partfun_S(\beta)} \parens{1 - \frac{e^{-\beta_E \lambda_S(0)}}{\partfun_S(\beta_E)}} \int_{\beta}^{\beta_E} (x - \beta) dx \\
        &= \frac{\Delta_{\min} (\beta_E - \beta)^2}{2} \frac{e^{-\beta \lambda_S(0)}}{\partfun_S(\beta)} \parens{1 - \frac{e^{-\beta_E \lambda_S(0)}}{\partfun_S(\beta_E)}}. \label{eq:boltzmann_lower_bound_penultimate}
    \end{align}

    The last step is to lower bound the the Boltzmann factors as seen above. We can use the simplistic bound $\frac{e^{-\beta \lambda_S(0)}}{\partfun_S(\beta)} \geq \frac{1}{\dim_S}$ to take care of the one factor, and the other we need to show a bound of $\parens{1 - \frac{e^{-\beta_E \lambda_S(0)}}{\partfun_S(\beta_E}} \geq \epsilon_{\rho}$
    This will be done through a bound on the partition function, seen below
    \begin{align}
        \partfun_S(\beta_E) &=  \sum_k e^{-\beta_E \lambda_S(k)} = e^{-\beta_E \lambda_S(0)} \parens{1 + \sum_{k > 0} e^{-\beta_E (\lambda_S(k) - \lambda_S(0))}}.
    \end{align}
    Rearranging yields
    \begin{align}
        \frac{e^{-\beta_E \lambda_S(0)}}{\partfun_S(\beta_E)} &= \frac{1}{1 + \sum_{k > 0} e^{-\beta_E (\lambda_S(k) - \lambda_S(0))}}.
    \end{align}
    To upper bound this quantity we use fact that $\lambda_S(k) - \lambda_S(0) \leq 2 \norm{H}$ holds for all $k$, which yields
    \begin{align}
        -\beta_E(\lambda_S(k) - \lambda_S(0)) &\geq -\beta_E 2 \norm{H} \\
        \implies e^{-\beta_E (\lambda_S(k) - \lambda_S(0))} &\geq e^{-\beta_E 2 \norm{H}} \\
        \implies \sum_{k > 0} e^{-\beta_E (\lambda_S(k) - \lambda_S(0))} &\geq (\dim_S - 1) e^{-\beta_E 2 \norm{H}} \\
        \implies \frac{1}{1 + \sum_{k > 0} e^{-\beta_E (\lambda_S(k) - \lambda_S(0))} } &\leq \frac{1} {1 + (\dim_S - 1) e^{-\beta_E 2 \norm{H}} } .
    \end{align}
    Now this serves as an upper bound on $\frac{e^{-\beta_E \lambda_S(0)}}{\partfun_S(\beta_E)}$. We would therefore like that 
    \begin{align}
        \frac{e^{-\beta_E \lambda_S(0)}}{\partfun_S(\beta_E)} = \frac{1}{1 + \sum_{k > 0} e^{-\beta_E (\lambda_S(k) - \lambda_S(0))}} &\leq \frac{1}{1 + (\dim_S - 1) e^{-\beta_E 2 \norm{H}}} \\
        &\leq 1 - \epsilon_{\rho} \label{eq:epsilon_rho_inequality}
    \end{align}
    Through straightforward algebra, it can be seen that requiring 
    \begin{equation}
        \beta_E \leq \frac{1}{2 \norm{H}} \parens{\ln (\dim_S - 1) + \ln \left( \frac{1}{\epsilon_{\rho}} - 1\right) } \label{eq:beta_e_upper_bound_thermal_diffs}
    \end{equation}
    is sufficient to satisfy the above inequality in Eq. \eqref{eq:epsilon_rho_inequality}. Plugging Eq. \eqref{eq:epsilon_rho_inequality}, along with the inequality $\frac{e^{-\beta \lambda_S(0)}}{\partfun_S(\beta)} \geq \frac{1}{\dim_S}$, into Eq. \eqref{eq:boltzmann_lower_bound_penultimate} yields
    \begin{align}
        R_0(\beta, \beta_E) \geq \frac{\epsilon_{\rho} \Delta_{\min} \delta^2}{2 \dim_S},
    \end{align}
    which gives the lemma statement. 
    
\end{proof}



\section{Some Scratch}
\begin{lemma} \label{lem:expected_second_order_term}
    Let $T_{\gamma}^{(2)}(\rho_S(\beta))$ denote the second order term for a thermalizing channel, with a fixed value of $\gamma$, as defined in Eq. \eqref{def:taylor_series_terms}. We compute the expectation value of the $k$-th component of this matrix with respect to $\gamma$ as chosen from the perfect knowledge distribution of a well-separated Hamiltonian $H_S$, per Definition \ref{def:separated_hamiltonians}, as 
    \begin{align}
        \abs{\mathbb{E}_{\gamma} \bra{k} T_{\gamma}^{(2)}(\rho_S(\beta)) \ket{k} - \frac{\alpha^2 t^2}{\dim + 1} \frac{2}{\dim_S (\dim_S - 1)} \sum_{i \neq k} \parens{ p(i) \frac{1}{1 + e^{-\beta_E |\Delta_S(i,k)|}}  - p(k) \frac{e^{-\beta_E |\Delta_S(i,k)|}}{1 + e^{-\beta_E |\Delta_S(i,k)|}} }} \leq  \parens{\frac{4 \alpha}{\Delta_{\min}}}^2
    \end{align}
\end{lemma}
\begin{proof}
    We start by decomposing $T_{\gamma}^{(2)}$ for a \emph{fixed} value of $\gamma$, using the results of Lemma \ref{lem:transition_idx_sub}
    \begin{align}
        \bra{k} T_{\gamma}^{(2)}(\rho_S(\beta)) \ket{k} &= \sum_{i,j,l} p(i) q(j) \tau(i,j|k,l) \\
        &= \sum_{i \neq k} \sum_{j,l} (p(i) q(j) - p(k) q(l)) \tau(i,j| k,l) \\
        &= \frac{\alpha^2 t^2}{\dim + 1} \sum_{i \neq k} \sum_{j,l} \parens{ p(i) \frac{e^{-\beta_E \lambda_E(j)}}{\partfun_E(\beta_E)} - p(k) \frac{e^{-\beta_E \lambda_E(l)}}{\partfun_E(\beta_E)}} \sinc^2((\Delta_S(i,k) + \Delta_E(j,l)) t/2).
    \end{align}
    To tackle the sum over $i$, we note that $\Delta_S(i,k)$ is positive if $i > k$ and negative if $i < k$, due to the sorting of eigenvalues. We will first show the $i > k$ condition and the other case is the same argument.

    We expand the $j,l$ summation and note that 3 of the resulting terms are in the tail end of the sinc function and can therefore be upper bounded by $\epsilon_{\sinc}$, as $\Delta_S(i,k) \geq \Delta_{\min}$
    \begin{align}
        & \sum_{j,l} \parens{p(i) \frac{e^{-\beta_E \lambda_E(j)}}{\partfun_E(\beta_E)} - p(k) \frac{e^{-\beta_E \lambda_E(l)}}{\partfun_E(\beta_E)}} \sinc^2((\Delta_S(i,k) + \Delta_E(j,l)) t/2) \nonumber \\
        &= \parens{p(i) \frac{1}{\partfun_E(\beta_E)} - p(k) \frac{1}{\partfun_E(\beta_E)}} \sinc^2(\Delta_S(i,k)t/2) \nonumber \\
        &\quad + \parens{p(i) \frac{1}{\partfun_E(\beta_E)} - p(k) \frac{e^{-\beta_E \gamma}}{\partfun_E(\beta_E)}} \sinc^2((\Delta_S(i,k) - \gamma) t/2) \nonumber \\
        &\quad + \parens{p(i) \frac{e^{-\beta_E \gamma}}{\partfun_E(\beta_E)} - p(k) \frac{1}{\partfun_E(\beta_E)}} \sinc^2((\Delta_S(i,k) + \gamma) t/2) \nonumber \\
        &\quad + \parens{p(i) \frac{e^{-\beta_E \gamma}}{\partfun_E(\beta_E)} - p(k) \frac{e^{-\beta_E \gamma}}{\partfun_E(\beta_E)}} \sinc^2((\Delta_S(i,k)) t/2) \label{tmp:expected_second_order_coeff_1} \\
        &\leq 1 \cdot \sinc^2(\Delta_S(i,k) t/2) + \parens{p(i) \frac{1}{\partfun_E(\beta_E)} - p(k) \frac{e^{-\beta_E \gamma}}{\partfun_E(\beta_E)}} \sinc^2((\Delta_S(i,k) - \gamma) t/2) \nonumber \\
        &\quad + 1 \cdot \sinc^2((\Delta_S(i,k) + \gamma)t/2) + 1 \cdot \sinc^2(\Delta_S(i,k) t/2) \\
        &\leq 3 \epsilon_{\sinc} + \parens{p(i) \frac{1}{\partfun_E(\beta_E)} - p(k) \frac{e^{-\beta_E \gamma}}{\partfun_E(\beta_E)}} \sinc^2((\Delta_S(i,k) - \gamma) t/2).
    \end{align}
    Similar arguments show that Eq. \eqref{tmp:expected_second_order_coeff_1} is lower bounded by $-3 \epsilon_{\sinc} + \parens{p(i) \frac{1}{\partfun_E(\beta_E)} - p(k) \frac{e^{-\beta_E \gamma}}{\partfun_E(\beta_E)}} \sinc^2((\Delta_S(i,k) - \gamma) t/2)$. 

    Now we look at the case in which $i < k$. We have the same setup in Eq. \eqref{tmp:expected_second_order_coeff_1}, but $\Delta_S(i,k) \leq -\Delta_{\min}$. This gives virtually the same result as above, but the term that is not small is the term with $+ \gamma$ in the argument of the sinc function as opposed to the $-\gamma$ term. So for $i < k$ we have
    \begin{align}
        & \sum_{j,l} \parens{p(i) \frac{e^{-\beta_E \lambda_E(j)}}{\partfun_E(\beta_E)} - p(k) \frac{e^{-\beta_E \lambda_E(l)}}{\partfun_E(\beta_E)}} \sinc^2((\Delta_S(i,k) + \Delta_E(j,l)) t/2) \nonumber \\
        &\leq 3 \epsilon_{\sinc} + \parens{p(i) \frac{e^{-\beta_E \gamma}}{\partfun_E(\beta_E)} - p(k) \frac{1}{\partfun_E(\beta_E)} } \sinc^2((\Delta_S(i,k) + \gamma) t/2) \\
        &= 3 \epsilon_{\sinc} + \parens{p(i) \frac{e^{-\beta_E \gamma}}{\partfun_E(\beta_E)} - p(k) \frac{1}{\partfun_E(\beta_E)} } \sinc^2((|\Delta_S(i,k)| - \gamma) t/2).
    \end{align}
    Notice how the factor of $e^{-\beta_E \gamma}$ switched in the $i < k$ case from the $i > k$ one. Similar arguments give the same lower bound as before. 

    Now we introduce the expectation over $\gamma$, as expectations are linear we can do this term by term over the sum on $i \neq k$ in Eq. idk. As $H_S$ is well-separated, and $\gamma$ is distributed assuming perfect knowledge, then for each $i \neq k$ we will sample $\gamma = \Delta_S(i,k)$ exactly and every other term will be $\gamma \sim \Delta_S(a,b)$ for $(a,b) \neq (i,k)$. These other terms leave the resulting sinc function to be $\epsilon_{\sinc}$ small. We again will analyze this expectation first for terms in which $i > k$ and then show the $i < k$ case.

    \begin{align}
        & \mathbb{E}_{\gamma}\sum_{j,l} \parens{p(i) \frac{e^{-\beta_E \lambda_E(j)}}{\partfun_E(\beta_E)} - p(k) \frac{e^{-\beta_E \lambda_E(l)}}{\partfun_E(\beta_E)}} \sinc^2((\Delta_S(i,k) + \Delta_E(j,l)) t/2) \nonumber \\
        &\leq \mathbb{E}_{\gamma} 3 \epsilon_{\sinc} + \mathbb{E}_{\gamma} \parens{p(i) \frac{1}{\partfun_E(\beta_E)} - p(k) \frac{e^{-\beta_E \gamma}}{\partfun_E(\beta_E)}} \sinc^2((\Delta_S(i,k) - \gamma) t/2) \\
        &= 3 \epsilon_{\sinc} + \sum_{(a,b) : a > b} \frac{2}{\dim_S (\dim_S - 1)} \parens{p(i) \frac{1}{1 + e^{-\beta_E \Delta_S(a,b)}} - p(k) \frac{e^{-\beta_E \Delta_S(a,b)}}{1 + e^{-\beta_E \Delta_S(a,b)}}} \sinc^2((\Delta_S(i,k) - \Delta_S(a,b)) t/2) \\
        &= 3 \epsilon_{\sinc} \nonumber \\
        &+ \sum_{(a,b): a > b, (a,b) \neq (i,k)} \frac{2}{\dim_S (\dim_S - 1)} \parens{p(i) \frac{1}{1 + e^{-\beta_E \Delta_S(a,b)}} - p(k) \frac{e^{-\beta_E \Delta_S(a,b)}}{1 + e^{-\beta_E \Delta_S(a,b)}}} \sinc^2((\Delta_S(i,k) - \Delta_S(a,b)) t/2) \nonumber \\
        &+ \frac{2}{\dim_S (\dim_S - 1)} \parens{p(i) \frac{1}{1 + e^{-\beta_E \Delta_S(i,k)}} - p(k) \frac{e^{-\beta_E \Delta_S(i,k)}}{1 + e^{-\beta_E \Delta_S(i,k)}}} \sinc^2((\Delta_S(i,k) - \Delta_S(i,k)) t/2) \\
        &\leq 3 \epsilon_{\sinc} + \epsilon_{\sinc} \frac{2}{\dim_S (\dim_S - 1)} \sum_{(a,b): a > b, (a,b) \neq (i,k)} 1 \nonumber \\
        &\quad + \frac{2}{\dim_S (\dim_S - 1)} \parens{p(i) \frac{1}{1 + e^{-\beta_E \Delta_S(i,k)}} - p(k) \frac{e^{-\beta_E \Delta_S(i,k)}}{1 + e^{-\beta_E \Delta_S(i,k)}}} \\
        &\leq 4 \epsilon_{\sinc} + \frac{2}{\dim_S (\dim_S - 1)} \parens{p(i) \frac{1}{1 + e^{-\beta_E \Delta_S(i,k)}} - p(k) \frac{e^{-\beta_E \Delta_S(i,k)}}{1 + e^{-\beta_E \Delta_S(i,k)}}}.
    \end{align}
    Similar lower bounds simply swap the sign of $4 \epsilon_{\sinc}$ to $- 4 \epsilon_{\sinc}$. As with the above, in the situation with $i < k$ we simply move the $e^{-\beta_E \gamma}$ factor, as the sinc function evaluates to 1 in term in the expectation that actually contributes
    \begin{align}
        & \mathbb{E}_{\gamma}\sum_{j,l} \parens{p(i) \frac{e^{-\beta_E \lambda_E(j)}}{\partfun_E(\beta_E)} - p(k) \frac{e^{-\beta_E \lambda_E(l)}}{\partfun_E(\beta_E)}} \sinc^2((\Delta_S(i,k) + \Delta_E(j,l)) t/2) \nonumber \\
        &\leq 4 \epsilon_{\sinc} + \frac{2}{\dim_S (\dim_S - 1)} \parens{p(i) \frac{e^{-\beta_E \Delta_S(i,k)}}{1 + e^{-\beta_E \Delta_S(i,k)}} - p(k) \frac{1}{1 + e^{-\beta_E \Delta_S(i,k)}}}.
    \end{align}
    This then leads to the bounds for the expectation value of $T_{\gamma}^{(2)}$ as 
    \begin{align}
        \mathbb{E}_{\gamma} \bra{k} T_{\gamma}^{(2)}(\rho_S(\beta)) \ket{k} &\leq \frac{\alpha^2 t^2}{\dim + 1} \sum_{i \neq k} 4 \epsilon_{\sinc} \nonumber \\
        &\quad + \frac{\alpha^2 t^2}{\dim + 1} \sum_{i < k} \frac{2}{\dim_S (\dim_S - 1)} \parens{p(i) \frac{e^{-\beta_E \Delta_S(i,k)}}{1 + e^{-\beta_E \Delta_S(i,k)}} - p(k) \frac{1}{1 + e^{-\beta_E \Delta_S(i,k)}}} \nonumber \\
        &\quad + \frac{\alpha^2 t^2}{\dim + 1} \sum_{i > k} \frac{2}{\dim_S (\dim_S - 1)} \parens{p(i) \frac{1}{1 + e^{-\beta_E \Delta_S(i,k)}} - p(k) \frac{e^{-\beta_E \Delta_S(i,k)}}{1 + e^{-\beta_E \Delta_S(i,k)}} } \\
        &\leq 4 \alpha^2 t^2 \epsilon_{\sinc} + \frac{\alpha^2 t^2}{\dim + 1} \frac{2}{\dim_S (\dim_S - 1)} \sum_{i \neq k} \parens{ p(i) \frac{1}{1 + e^{-\beta_E |\Delta_S(i,k)|}}  - p(k) \frac{e^{-\beta_E |\Delta_S(i,k)|}}{1 + e^{-\beta_E |\Delta_S(i,k)|}} }.
    \end{align}
    The exact same lower bound but with $4 \alpha^2 t^2 \epsilon_{\sinc}$ goes to $-4\alpha^2 t^2 \epsilon_{\sinc}$ holds. This, along with $\epsilon_{\sinc} = \frac{4}{t^2 \Delta_{\min}^2}$, gives the final result
    \begin{align}
        \abs{\mathbb{E}_{\gamma} \bra{k} T_{\gamma}^{(2)}(\rho_S(\beta)) \ket{k} - \frac{\alpha^2 t^2}{\dim + 1} \frac{2}{\dim_S (\dim_S - 1)} \sum_{i \neq k} \parens{ p(i) \frac{1}{1 + e^{-\beta_E |\Delta_S(i,k)|}}  - p(k) \frac{e^{-\beta_E |\Delta_S(i,k)|}}{1 + e^{-\beta_E |\Delta_S(i,k)|}} }} \leq  \parens{\frac{4 \alpha}{\Delta_{\min}}}^2
    \end{align}
\end{proof}

\begin{theorem}[Single-Shot capture]
    Assume that $H_S$ is a well separated Hamiltonian according to Def. \ref{def:separated_hamiltonians}. Our goal is to show that 
    Need to find $\beta_{UB}$ such that 
    \begin{equation}
        \beta_{LB} \leq \beta \leq \beta_{UB} \implies \norm{\rho_S(\beta_E) - \Phi(\rho_S(\beta))}_1 \leq \epsilon_{fix}.
    \end{equation}
    Note we should also show this by averaging over the perfect knowledge eigenvalue difference distribution.
\end{theorem}
\begin{proof}
    We first expand the norm as usual, assuming that $\gamma$ is $\Delta_{\sinc}$ close to the eigenvalue difference $\Delta_S(\gamma_U, \gamma_L)$,
    \begin{align}
        \norm{\rho_S(\beta_E) - \Phi_\gamma(\rho_S(\beta))}_1 &\leq \norm{\rho_S(\beta_E) - \rho_S(\beta) - T^{(2)}_{\gamma}(\rho_S(\beta))}_1 + \norm{R_[\Phi]}_1.
    \end{align}
    Ignoring the channel approximation error term for now, we then expand the one norm to a sum as the matrix is diagonal, letting $p_E(i) = \frac{e^{-\beta_E \lambda_S(i)}}{\partfun_S(\beta_E)}$, $p_{\beta}(i) = \frac{e^{-\beta \lambda_S(i)}}{\partfun_S(\beta)}$, and $q(j) = \frac{e^{-\beta_E \lambda_E(j)}}{\partfun_E(\beta_E)}$,
    \begin{align}
        &\norm{\rho_S(\beta_E) - \rho_S(\beta) - T^{(2)}_{\gamma}(\rho_S(\beta))}_1 \nonumber \\
        &= \sum_k \abs{p_E(k) - p_{\beta}(k) - \sum_{i,j,l} p_{\beta}(i) q(j) \tau(i,j | k, l) } \\
        &= \sum_{k \neq \gamma_U, \gamma_L} \abs{p_E(k) - p_{\beta}(k) - \sum_{i,j,l} p_{\beta}(i) q(j) \tau(i,j | k, l) } \nonumber \\
        &\quad + \abs{p_E(\gamma_U) - p_{\beta}(\gamma_U) - \sum_{i,j,l} p_{\beta}(i) q(j) \tau(i,j| \gamma_U, l) } \nonumber \\
        &\quad +\abs{p_E(\gamma_U) - p_{\beta}(\gamma_U) - \sum_{i,j,l} p_{\beta}(i) q(j) \tau(i,j| \gamma_U, l) }.
    \end{align}
    The first term is where we take the hit by not averaging over $\gamma$. For now, I will expand the other terms about $\beta \to \beta_E$.

    \begin{align}
        &\abs{p_E(\gamma_U) - p_{\beta}(\gamma_U) - \sum_{i,j,l} p_{\beta}(i) q(j) \tau(i,j| \gamma_U, l) } \nonumber \\
        &= \abs{p_E(\gamma_U) - p_{\beta}(\gamma_U) - \frac{\alpha^2 t^2}{\dim + 1} p_{\beta}(\gamma_L) q(1) + \frac{\alpha^2 t^2}{\dim + 1} p_{\beta}(\gamma_L) q(1) - \sum_{i,j,l} p_{\beta}(i) q(j) \tau(i,j| \gamma_U, l) } \\
        &\leq \abs{p_E(\gamma_U) - p_{\beta}(\gamma_U) - \frac{\alpha^2 t^2}{\dim + 1} p_{\beta}(\gamma_L) q(1)} + \abs{\frac{\alpha^2 t^2}{\dim + 1} p_{\beta}(\gamma_L) q(1) - \sum_{i,j,l} p_{\beta}(i) q(j) \tau(i,j| \gamma_U, l) } \\
        &\leq \abs{p_E(\gamma_U) - p_{\beta}(\gamma_U) - \frac{\alpha^2 t^2}{\dim + 1} p_{\beta}(\gamma_L) q(1)} + 16 \frac{\alpha^2}{\Delta_{\min}^2}.
    \end{align}
    We have now arrived at the expression we need to expand, 
    \begin{align}
        g(\beta) &= p_{\beta}(\gamma_U) + p_{\beta}(\gamma_L) q(1) \frac{\alpha^2 t^2}{\dim + 1} \\
        &= \frac{e^{-\beta \lambda_S(\gamma_U)}}{\partfun_S(\beta)} + \frac{e^{-\beta \lambda_S(\gamma_L)}}{\partfun_S(\beta)} q(1) \frac{\alpha^2 t^2}{\dim  +1}.
    \end{align}
     We compute the zeroth order term as
    \begin{align}
        g(\beta_E) = \frac{e^{-\beta_E \lambda_S(\gamma_U)}}{\partfun_S(\beta_E)} + \frac{e^{-\beta_E \lambda_S(\gamma_L)}}{\partfun_S(\beta_E)} q(1) \frac{\alpha^2 t^2}{\dim  +1}.
    \end{align}
    In order to compute the derivative we will first compute $\frac{d}{d\beta} \frac{e^{-\beta x}}{\partfun_S(\beta)}$, using the notation $\anglebrackets{H}_{\beta} = \trace{H \rho_S(\beta)}$. First we compute the derivative of the partition function
    \begin{align}
        \frac{d}{d\beta} \partfun_S(\beta) &= \frac{d}{d\beta} \sum_i e^{-\beta \lambda_S(i)} \\
        &= - \sum_i \lambda_S(i) e^{-\beta \lambda_S(i)} \\
        &= - \trace{H e^{-\beta H}}.
    \end{align}
    Then computing the total Boltzmann factor derivative
    \begin{align}
        \frac{d}{d \beta} \frac{e^{-\beta x}}{\partfun_S(\beta)} &= - x \frac{e^{-\beta x}}{\partfun_S(\beta)} - \frac{e^{-\beta x}}{\partfun_S(\beta)^2} \frac{d}{d\beta}\partfun_S(\beta) \\
        &= \frac{e^{-\beta x}}{\partfun_S(\beta)} \parens{\anglebrackets{H} - x}.
    \end{align}
    Now we can compute the total first order derivative as
    \begin{align}
        \frac{d}{d\beta} g(\beta) &= \frac{d}{d\beta} \frac{e^{-\beta \lambda_S(\gamma_U)}}{\partfun_S(\beta)} + \frac{d}{d\beta} \frac{e^{-\beta \lambda_S(\gamma_L)}}{\partfun_S(\beta)} q(1) \frac{\alpha^2 t^2}{\dim + 1} \\
        &= \frac{e^{-\beta \lambda_S(\gamma_U)}}{\partfun_S(\beta)}\parens{\anglebrackets{H}_{\beta} - \lambda_S(\gamma_U)} + \frac{e^{-\beta \lambda_S(\gamma_L)}}{\partfun_S(\beta)}\parens{\anglebrackets{H}_{\beta} - \lambda_S(\gamma_L)} q(1) \frac{\alpha^2 t^2}{\dim + 1}
    \end{align}

\end{proof}

\section{Haar Integral Proofs} \label{sec:haar_integral_appendix}
\secondOrderChannelHaar*
\begin{proof}
To start we would like to note that we will use a single index notation to refer to the joint system-environment eigenbasis during this proof to help shorten the already lengthy expressions. We will convert back to a double index notation to match the statement of the theorem. We start from the expression for the first derivative of the channel $\frac{\partial}{\partial \alpha} \Phi_G(\rho_S)$ given by Eq. \eqref{eq:first_order_alpha_derivative}. To take the second derivative there are six factors involving $\alpha$, so we will end up with six terms. We repeat Eq. \eqref{eq:first_order_alpha_derivative} below, add a derivative, and label each factor containing an $\alpha$ for easier computation
\begin{align}
    \frac{\partial^2}{\partial \alpha^2} \Phi_G(\rho_S) =& \frac{\partial}{\partial \alpha} \parens{\int_{0}^{1} \underset{\substack{\downarrow \\ (A)}}{e^{i s (H+\alpha G)t}} (i t G) \underset{\substack{\downarrow \\ (B)}}{e^{i (1-s) (H+\alpha G)t}} ds ~ \rho \underset{\substack{\downarrow \\ (C)}}{e^{-i(H+\alpha G)t}} } \nonumber \\
    &~ ~+\frac{\partial}{\partial \alpha} \parens{ \underset{\substack{\downarrow \\ (D)} }{e^{i(H+\alpha G)t}} \rho \int_{0}^1 \underset{\substack{\downarrow \\ (E)} }{e^{-i s (H+\alpha G) t} } (- i t G) \underset{\substack{\downarrow \\ (F)}}{e^{-i (1-s) (H+\alpha G)t}} ds }.
\end{align}
Our goal is to get each of these terms in a form in which we can use either Lemma \ref{lem:two_heisenberg_interactions} or \ref{lem:sandwiched_interaction}. 
\begin{align}
    (A) &=i t\int_0^1 \parens{\frac{\partial}{\partial \alpha} e^{i s_1 (H+ \alpha G)t}} G e^{i(1-s_1)(H+\alpha G)t} ds_1 \rho e^{-i (H+\alpha G)t} \bigg|_{\alpha=0} \\
    &= (it)^2 \int_0^1 \parens{\int_0^1 e^{i s_1 s_2 (H+\alpha G)t} s_1 G e^{i s_1 (1-s_2) (H+\alpha G)t} ds_2} G e^{i(1-s_1) (H+\alpha G)t} ds_1 \rho e^{-i(H+\alpha G) t} \bigg|_{\alpha=0} \label{eq:second_order_deriv_intermediate_a}\\
    &= -t^2 \int_0^1 \int_0^1 e^{i s_1 s_2 H t} G e^{-i s_1 s_2 H t} e^{i s_1 H t} G e^{-i s_1 H t} s_1 ds_1 ds_2 e^{i H t} \rho e^{-i H t} \\
    &= -t^2 \int_0^1 \int_0^1 G(s_1 s_2 t) G(s_1 t) s_1 ds_1 ds_2 \rho(t). \label{eq:second_deriv_alpha_first_term}
\end{align}

\begin{align}
    (B) &= it \int_0^1 e^{i s_1 (H + \alpha G)t} G \frac{\partial}{\partial \alpha}\parens{e^{i(1-s_1)(H + \alpha G)t}} ds_1 \rho e^{-i(H + \alpha G) t} \bigg|_{\alpha = 0} \\
    &= (it)^2 \int_0^1 e^{i s_1 (H + \alpha G)t} G \parens{\int_0^1 e^{i(1-s_1)s_2 (H + \alpha G)t} (1-s_1) G e^{i(1 - s_1)(1 - s_2)(H + \alpha G)t} ds_2} ds_1 ~ \rho e^{-i ( H + \alpha G)t} \bigg|_{\alpha = 0} \\
    &= -t^2 \int_0^1 \int_0^1 e^{i s_1 H t} G e^{i(1-s_1)s_2 H t} G e^{i(1-s_1)(1-s_2) H t} (1-s_1) ds_1 ds_2 ~ \rho e^{-i H t}\\ 
    &= -t^2 \int_0^1 \int_0^1 e^{i s_1 H t} G e^{-i s_1 H t} e^{i(s_1 + s_2 - s_1 s_2) H t} G e^{-i (s_1 + s_2 - s_1 s_2) H t} (1-s_1) ds_1 ds_2 ~ \rho(t) \\
    &= -t^2 \int_0^1 \int_0^1 G(s_1 t) G((s_1 + s_2 - s_1 s_2)t) (1-s_1) ds_1 ds_2 ~ \rho(t)
\end{align}

\begin{align}
    (C) &= it \int_0^1 e^{i s (H + \alpha G)t} G e^{i(1-s) (H + \alpha G) t} ds ~\rho ~ \frac{\partial}{\partial \alpha} \parens{ e^{-i (H + \alpha G) t} } \bigg|_{\alpha = 0} \\
    &= (i t) (-it) \int_0^1 e^{i s (H + \alpha G)t} G e^{i (1 - s) (H + \alpha G)t} ds ~ \rho ~ \parens{ \int_0^1 e^{-i s (H + \alpha G)t} G e^{-i (1- s) ( H + \alpha G)t } ds}\bigg|_{\alpha = 0} \\
    &= + t^2 \parens{\int_0^1 e^{i s H t} G e^{-i s H t} ds} e^{i H t} \rho e^{-i H t} \parens{\int_0^1 e^{i (1-s) H t} G e^{-i (1-s) H t} ds} \\
    &= + t^2 \int_0^1 G(st) ds ~ \rho(t) \int_0^1 G((1-s)t) ds
\end{align}

\begin{align}
    (D) &= (-it) \frac{\partial}{\partial \alpha} \parens{e^{i(H + \alpha G)t}} \rho \int_0^1 e^{-i s (H + \alpha G)t} G e^{-i (1-s)(H + \alpha G)t} ds \bigg|_{\alpha = 0} \\
    &= t^2 \parens{\int_0^1 e^{i s (H+ \alpha G)t} G e^{i (1-s) (H + \alpha G)t}ds} \rho \int_0^1 e^{-i s (H + \alpha G)t} G e^{-i (1-s)(H + \alpha G)t} ds \bigg|_{\alpha = 0} \\
    &=  t^2 \int_0^1 e^{i s H t} G e^{-i s H t} ds ~\rho(t) \int_0^1 e^{i (1-s) H t} G e^{-i (1-s) H t} ds \\
    &= t^2 \int_0^1 G(st) ds ~ \rho(t) ~ \int_0^1 G((1-s)t) ds
\end{align}

\begin{align}
    (E) &= (-it) e^{i (H+ \alpha G) t} ~ \rho ~\int_0^1 \frac{\partial}{\partial \alpha} \parens{e^{-i s_1 (H + \alpha G)t}} G e^{-i (1-s_1)(H + \alpha G)t} ds_1 \bigg|_{\alpha = 0} \\
    &= - t^2 e^{i(H + \alpha G)t} ~ \rho ~\int_0^1 \parens{\int_0^1 e^{-i s_1 s_2 (H + \alpha G) t} (s_1 G) e^{-i s_1 (1-s_2) (H + \alpha G)t} ds_2} G e^{-i(1-s_1)(H + \alpha G)t} ds_1 \bigg|_{\alpha = 0} \\
    &= -t^2 e^{i H t} \rho e^{-i H t} \int_0^1 \int_0^1 e^{i (1 - s_1 s_2) H t} G e^{-i (s_1 - s_1 s_2)H t} G e^{-i (1-s_1)H t} s_1 ds_1 ds_2 \\
    &= -t^2 \rho(t) \int_0^1 \int_0^1 G((1- s_1 s_2) t) G((1-s_1)t) s_1 ds_1 ds_2
\end{align}

\begin{align}
    (F) &= (-it) e^{i(H + \alpha G) t} \rho \int_0^1 e^{-i s_1 ( H + \alpha G) t} G \frac{\partial}{\partial \alpha} \parens{ e^{-i (1-s_1) ( H +\alpha G)t}} ds_1 \bigg|_{\alpha = 0} \\
    &= (-it)^2 e^{i (H + \alpha G)t} \rho \int_0^1 e^{-i s_1 (H + \alpha G)t} G \parens{\int_0^1 e^{-i(1-s_1) s_2 (H + \alpha G)t} (1-s_1) G e^{-i(1-s_1) (1-s_2) (H + \alpha G) t} ds_2} ds_1 \bigg|_{\alpha = 0} \\
    &= -t^2 e^{-i H t} \rho e^{-i H t} \int_0^1 \int_0^1 e^{i (1- s_1) H t} G e^{-i (1-s_1) H t} e^{i (1-s_1)(1-s_2) H t} G e^{-i(1-s_1)(1-s_2) H t} (1-s_1) ds_1 ds_2 \\
    &= -t^2 \rho(t) \int_0^1 \int_0^1 G((1-s_1)t) G((1-s_1)(1 - s_2) t) (1-s_1)ds_1 ds_2
\end{align}

Now our goal is to compute the effects of averaging over the interaction $G$ on the above terms, starting with $(A)$. As this involves a lot of index manipulations, similarly to the proofs of Lemmas \ref{lem:two_heisenberg_interactions} and \ref{lem:sandwiched_interaction} we will use a single index for the total system-environment Hilbert space and switch back to a double index to state the results. We will make heavy use of Lemma \ref{lem:two_heisenberg_interactions}.
\begin{align}
    \int (A) dG &= -t^2 \int_0^1 \int_0^1 \int G(s_1 s_2 t) G(s_1 t) dG s_1 ds_1 ds_2 \rho(t) \\
    &= \frac{-t^2 }{\dim + 1} \int_0^1 \int_0^1 \parens{\sum_{i,j} e^{i (\lambda(i) - \lambda(j)) (s_1 s_2 t - s_1 t)} \ketbra{i}{i} + \identity} s_1 ds_1 ds_2 \rho(t) \\
    &= \frac{- t^2 }{\dim + 1} \parens{\sum_{i} \sum_{j : \lambda(i) \neq \lambda(j)} \int_0^1 \int_0^1 e^{i(\lambda(i) - \lambda(j))t (s_1 s_2 - s_1)} s_1 ds_1 ds_2 \ketbra{i}{i} + \sum_{i} \sum_{j : \lambda(i) = \lambda(j)}\frac{1}{2} \ketbra{i}{i} + \frac{1}{2} \identity} \rho(t) \\
    &= \frac{- t^2 }{\dim + 1} \parens{\sum_i \sum_{j : \lambda(i) \neq \lambda(j)} \frac{1 - i (\lambda(i) - \lambda(j))t - e^{-i (\lambda(i) - \lambda(j))t}}{t^2 (\lambda(i) - \lambda(j))^2} \ketbra{i}{i} + \frac{1}{2} \sum_{i} (\eta(i) + 1) \ketbra{i}{i} } \rho(t) \\
    &= \frac{- 1}{\dim + 1}\parens{\sum_{i} \sum_{j: \Delta_{ij} \neq 0} \frac{1 - i \Delta_{ij}t - e^{-i \Delta_{ij} t}}{\Delta_{ij}^2} \ketbra{i}{i} + \frac{t^2}{2} \sum_{i} (\eta(i) + 1)\ketbra{i}{i} } \rho(t)
\end{align}

We can similarly compute the averaged $(B)$ term:
\begin{align}
    \int (B) dG &= -t^2 \int_0^1 \int_0^1 \int G(s_1 t) G((s_1 + s_2 - s_1 s_2) t) dG (1-s_1) ds_1 ds_2 ~ \rho(t) \\
    &= \frac{- t^2 }{\dim + 1} \int_0^1 \int_0^1 \parens{\sum_{i,j} e^{i (\lambda(i) - \lambda(j))(s_1 s_2 - s_2) t} \ketbra{i}{i} + \identity} (1 -s_1) ds_1 ds_2 \rho \\
    &= \frac{- t^2 }{\dim + 1} \parens{\sum_{i} \sum_{j : \lambda(i) \neq \lambda(j)} \int_0^1 \int_0^1 e^{i(\lambda(i) - \lambda(j))t (s_1 s_2 - s_2)} (1 - s_1) ds_1 ds_2 \ketbra{i}{i} + \sum_{i} \sum_{j : \lambda(i) = \lambda(j)}\frac{1}{2} \ketbra{i}{i} + \frac{1}{2} \identity} \rho(t) \\
    &= \frac{- t^2 }{\dim + 1} \parens{\sum_i \sum_{j : \lambda(i) \neq \lambda(j)} \frac{1 - i (\lambda(i) - \lambda(j))t - e^{-i (\lambda(i) - \lambda(j))t}}{t^2 (\lambda(i) - \lambda(j))^2} \ketbra{i}{i} + \frac{1}{2} \sum_{i} (\eta(i) + 1) \ketbra{i}{i} } \rho(t) \\
    &= \frac{-1}{\dim + 1}\parens{\sum_{i} \sum_{j: \Delta_{ij} \neq 0} \frac{1 - i \Delta_{ij}t - e^{-i \Delta_{ij} t}}{\Delta_{ij}^2} \ketbra{i}{i} + \frac{t^2}{2} \sum_{i} (\eta(i) + 1)\ketbra{i}{i} } \rho(t),
\end{align}
which we note is identical to $\int (A) dG$. As terms $(C)$ and $(D)$ involve a different method of computation we skip them for now and compute $(E)$ and $(F)$. 
\begin{align}
    \int (E) dG &= -t^2 \rho(t) \int_0^1 \int_0^1 \int G((1- s_1 s_2) t) G((1-s_1)t) dG s_1 ds_1 ds_2 \\
    &= \frac{- t^2}{\dim + 1} \rho(t) \int_0^1 \int_0^1 \parens{\sum_{i,j} e^{i(\lambda(i) - \lambda(j)) t (s_1 - s_1 s_2)} \ketbra{i}{i} + \identity } s_1 ds_1 ds_2 \\
    &= \frac{- t^2}{\dim + 1} \rho(t) \parens{\sum_i \sum_{j : \lambda(i) \neq \lambda(j)} \frac{1 + i (\lambda(i) - \lambda(j))t - e^{i(\lambda(i) - \lambda(j))t}}{t^2 (\lambda(i) - \lambda(j))^2}\ketbra{i}{i} + \frac{1}{2} \sum_{i} (\eta(i) + 1 )\ketbra{i}{i}} \\
    &= \frac{- 1}{\dim + 1} \rho(t) \parens{\sum_i \sum_{j: (\Delta_{ij} \neq 0)} \frac{1 + i \Delta_{ij}t - e^{i\Delta_{ij}t}}{\Delta_{ij}^2} \ketbra{i}{i} + \frac{t^2}{2}\sum_i (\eta(i) + 1) \ketbra{i}{i}}.
\end{align}
Computing $(F)$ yields
\begin{align}
    \int (F) dG &= -t^2 \rho(t) \int_0^1 \int_0^1 \int G((1-s_1)t) G((1-s_1)(1 - s_2) t) dG (1-s_1)ds_1 ds_2 \\
    &= \frac{- t^2 \sigma ^2}{\dim + 1} \rho(t) \int_0^1 \int_0^1 \parens{\sum_{i,j} e^{i(\lambda(i) - \lambda(j))t (s_2 - s_1 s_2)}\ketbra{i}{i} + \identity} (1-s_1) ds_1 ds_2 \\
    &= \frac{- t^2 }{\dim + 1} \rho(t) \parens{\sum_{i} \sum_{j : \lambda(i) \neq \lambda(j)} \frac{1 + i (\lambda(i) - \lambda(j))t - e^{i (\lambda(i) - \lambda(j))t}}{t^2 (\lambda(i) - \lambda(j))^2} \ketbra{i}{i} +\frac{1}{2} \sum_{i} (\eta(i) + 1) \ketbra{i}{i}} \\
    &= \frac{- 1}{\dim + 1} \rho(t) \parens{\sum_i \sum_{j: (\Delta_{ij} \neq 0)} \frac{1 + i \Delta_{ij}t - e^{i\Delta_{ij}t}}{\Delta_{ij}^2} \ketbra{i}{i} + \frac{t^2}{2}\sum_i (\eta(i) + 1) \ketbra{i}{i}}
\end{align}
 which is identical to $\int (E) dG$.

 The last two terms $(C) = (D)$ are computed as follows:
 \begin{align}
     \int (C) dG &= t^2 \int_0^1 \int_0^1 \int G(s_1 t) \rho(t) G((1-s_2)t) ~dG ~ ds_1 ds_2 \\
     &= t^2 \sum_{i,j} \rho_{ij} e^{i(\lambda(i) - \lambda(j))t} \int_0^1 \int_0^1 \int G(s_1 t) \ketbra{i}{j} G((1-s_2)t) ~ dG ~ ds_1 ds_2 \\
     &= \frac{ t^2}{\dim + 1} \sum_{i,j} \rho_{ij} e^{i(\lambda(i) - \lambda(j))t} \parens{ \ketbra{i}{j} + \delta_{ij} \sum_{a} \int_0^1 \int_0^1 e^{i(\lambda(a) - \lambda(i))(s_1 + s_2 - 1)t} ds_1 ds_2 \ketbra{a}{a}} \\
     &= \frac{ t^2}{\dim + 1} \sum_{i,j} \rho_{ij} e^{i \Delta_{ij} t} \parens{\ketbra{i}{j} + \delta_{ij} \sum_{a : \Delta_{ai} \neq 0} \frac{2( 1- \cos (\Delta_{ai} t))}{\Delta_{ai}^2 t^2} \ketbra{a}{a} + \delta_{ij} \sum_{a : \Delta_{ai} = 0} \ketbra{a}{a}}
 \end{align}

 We can now combine each of these terms to offer the full picture of the output of the channel to second order. We make two modifications to the results from each sum: first, we will switch to double index notation to make for easier use in other areas, and secondly we let $\rho = \ketbra{i,j}{k,l}$. We note that the first term in the following equation is provided by $(A) + (B)$, the second through $(E) + (F)$, and the last two through $(C) + (D)$. 
 \begin{align}
     &\int \frac{\partial^2}{\partial \alpha^2} \Phi_G(\ketbra{i,j}{k,l})\bigg|_{\alpha = 0} dG \\
     &= -\frac{2  e^{i \Delta(i,j|k,l) t}}{\dim + 1} \bigg(\sum_{(a,b): \Delta(i,j|a,b) \neq 0} \frac{1 - i \Delta(i,j|a,b)t - e^{-i \Delta(i,j|a,b) t}}{\Delta(i,j|a,b)^2} \nonumber \\
     &~+ \sum_{(a,b): \Delta(k,l|a,b) \neq 0} \frac{1 + i \Delta(k,l|a,b) t - e^{i \Delta(k,l|a,b) t}}{\Delta(k,l|a,b)^2} + \frac{t^2}{2}(\eta(i,j) + \eta(k,l)) \bigg) \ketbra{i,j}{k,l} \nonumber \\
    &~ +\delta_{i,k} \delta_{j,l} \frac{2 e^{i \Delta(i,j|k,l)t}}{\dim+1} \parens{ \sum_{(a,b): \Delta(i,j|a,b) \neq 0 } \frac{2(1- \cos (\Delta(i,j|a,b)t))}{\Delta(i,j|a,b)^2} \ketbra{a,b}{a,b} + t^2 \sum_{(a,b) : \Delta(i,j|a,b) = 0} \ketbra{a,b}{a,b}} \label{eq:second_order_output}
 \end{align}
\end{proof}

\end{document}