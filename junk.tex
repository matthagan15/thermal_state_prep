\begin{proof}
    \textbf{leftover work that still might be useful.}
        First we want to define the (multi)set
        \begin{equation}
            S = \set{\Delta_S(i,j) : 0 \leq i < j \leq \dim_S}.
        \end{equation}
        We assume that if $(i,j) \neq (i', j')$ then $\Delta_S(i,j) \neq \Delta_S(i', j')$. This is a significant restriction, but it greatly simplifies the following proofs. 
        
        Our next goal is to study the distance reduction achieved by an application of our repeated interactions channel. The most convenient distance to study is the square of the Sch\"{a}tten 2-norm, $\norm{\cdot}_2^2$. Although this does not have a nice operational interpretation, as the trace distance does in regards to state distinguishability, it can be used to give a loose upper bound on the trace distance. We break down our distance into two terms based on the second order decomposition of the channel
    \begin{align}
        \norm{\rho_S(\beta_E) - \Phi_{\gamma}(\rho_S)}_2^2 &= \norm{\rho_S(\beta_E) - \rho_S - \sum_{i, j, k, l} \bra{i} \rho_S \ket{i} \frac{e^{-\beta_E \lambda_E(j)}}{\partfun_E(\beta_E)}   \tau(i, j| k,l) \ketbra{k}{k} + R(\rho_S)}_2^2.
    \end{align}
    For simplicity, denote the operator $\rho_S(\beta_E) - \rho_S \eqqcolon A$ and $\sum_{i, j, k, l} \bra{i} \rho_S \ket{i} \frac{e^{-\beta_E \lambda_E(j)}}{\partfun_E(\beta_E)} \tau(i, j| k,l) \ketbra{k}{k} \eqqcolon B$. This allows us to isolate the remainder part as follows
    \begin{align}
        \norm{\rho_S(\beta_E) - \Phi_\gamma(\rho_S)}_2^2 &= \norm{A - B + R(\rho_S)}_2^2 \\
        &\leq \parens{\norm{A - B}_2 + \norm{R(\rho_S)}}^2 \\
        &\leq \norm{A - B}_2^2 + 2 \norm{A - B}_2 \norm{R(\rho_S(\beta))}_2 + \norm{R(\rho_S)}_2^2
    \end{align}
    We see that since we have a term linear in $\norm{R}_2$ we should only analyze $\norm{A - B}_2^2$ to order $\bigo{\alpha^2}$. We will now simplify this to a linear expression in $\norm{R}_2$ by using simple upper bounds on norms of $A$ and $B$. \textcolor{red}{NOTE: Need to bound the norm of the Taylor remainder of the channel for all states by some epsilon.} 
    \begin{align}
        \norm{A - B}_2 &\leq \norm{A - B}_1 \\
        &\leq \norm{A}_1 + \norm{B}_1 \\
        &= \sum_{i} \abs{\bra{i}\rho_S(\beta_E) \ket{i} - \bra{i} \rho_S \ket{i}} + \sum_k \abs{\sum_{i,j,l} \bra{i} \rho_S \ket{i} \frac{e^{-\beta_E \lambda_E(j)}}{\partfun_E(\beta_E)} \tau(i,j | k,l)} \\
        &\leq \sum_i 2 + \sum_k \sum_{i,j,l} 1 \\
        &\leq \dim^2 + 2 \dim \\
        &\leq 2 \dim^2.
    \end{align}
    This then gives our remainder term as
    \begin{equation}
        \norm{\rho_S(\beta_E) - \Phi_{\gamma}(\rho_S)}_2^2 \leq \norm{A - B}_2^2 + (4 \dim^2 + 1) \norm{R_{\Phi}}_2.
    \end{equation}
    Now we can bound $\norm{R_{\Phi}}_2 \leq \sqrt{\dim} \norm{R_{\Phi}} \leq \epsilon_{R} \sqrt{\dim}$. This gives the final inequality as
    \begin{equation}
        \norm{\rho_S(\beta_E) - \Phi_{\gamma}(\rho_S)}_2^2 \leq \norm{A - B}_2^2 + 5 \dim^{5/2} \epsilon_{R}
    \end{equation}
    
    Note we can improve this bound with improvements to the $\norm{B}_2$ bound. 
    \end{proof}
    
    
    Shifting our attention to $\norm{A-B}_2^2$, we decompose this into trace calculations straightforwardly
    \begin{align}
        \norm{A-B}_2^2 &= \trace{(A-B)^\dagger (A-B)} \\
        &= \trace{A^2} - 2 \trace{AB} + \trace{B^2}
    \end{align}
    In order to bound each of these traces it will be helpful to organize the indices of our system into two sets, one of indices which are ``active" in the transition and those that are ``dead". The active set is denoted as $S_{\gamma}$ and is defined as
    \begin{equation}
        S_{\gamma} \coloneqq \set{(i,j) : \lambda_S(i) \leq \lambda_S(j), |\Delta_S(i,j) - \gamma| \leq \Delta_{\min}},
    \end{equation}
    whereas the ``dead" set is denoted and defined as
    \begin{equation}
    T_{\gamma} \coloneqq \set{i : \forall j, |\Delta_S(i,j) - \gamma| \geq \Delta_{\min}}. 
    \end{equation}
    
    
    \noindent \textbf{Bounding: }$\trace{B^2}$
    
    
    We first bound $\norm{B}_2^2 = \trace{B^2}$. We split this trace into sums over $S_{\gamma}$ and $T_{\gamma}$. 
    \begin{align}
        k \in T_{\gamma} \implies B(k) &= \sum_{i \neq k} \sum_{j,l} (a(i) b(j) - a(k) b(l)) \tau(i,j|k,l) \\
        &\leq \sum_{i \neq k} \sum_{j,l} \tau(i,j|k,l) \\
        &\leq 4 (\dim_S - 1) \frac{\alpha^2 t^2}{\dim + 1}\epsilon_{\sinc} \\
        &\leq 2 \alpha^2 t^2 \epsilon_{\sinc}.
    \end{align}
    
    
    To perform the sum over the remaining indices we introduce a function $f_U$ that maps a given index to a subset of indices that satisfy
    \begin{align}
        f_U(k | \gamma) \coloneqq \set{k' : \abs{\Delta_S(k, k') - \gamma} \leq \Delta_{\min} \text{ and } \Delta_S(k,k') < 0} \\
        f_L(k | \gamma) \coloneqq \set{k' : \abs{\Delta_S(k, k') - \gamma} \leq \Delta_{\min} \text{ and } \Delta_S(k, k') > 0}.
    \end{align}
    We then simplify a term for $\trace{B^2}$ as
    \begin{align}
        B(k) &= \sum_{i \neq k} \sum_{j,l} (a(i) b(j) - a(k) b(l)) \tau(i,j|k,l) \\
        &\leq \sum_{i \in T_{\gamma}(k)} \sum_{j,l} \frac{\alpha^2 t^2}{\dim + 1}\epsilon_{\sinc} + \sum_{k' \in f_U(k)} \parens{\frac{\alpha^2 t^2}{\dim + 1} 3 \epsilon_{\sinc} + \tau(k', 0 | k, 1)} + \sum_{k'' \in f_L(k)} \parens{\tau(k'', 1 | k, 0) + 3\frac{\alpha^2 t^2}{\dim + 1} \epsilon_{\sinc}} \\
        &\leq \epsilon_{\sinc}\frac{\alpha^2 t^2}{\dim + 1} \parens{4 |T_{\gamma}(k)| + 3 | f_U(k)| + 3 |f_L(k)|} + \sum_{k' \in f_U(k)} (a(k')b(0) - a(k) b(1))\tau(k', 0| k, 1) \nonumber \\
        &\quad + \sum_{k'' \in f_L(k)} (a(k'') b(1) - a(k) b(0))\tau(k'', 1| k, 0) \\
        &\leq \epsilon_{\sinc} \dim \frac{\alpha^2 t^2}{\dim + 1} + \sum_{k' \in f_U(k)} (a(k')b(0) - a(k) b(1))\tau(k', 0| k, 1) + \sum_{k'' \in f_L(k)} (a(k'') b(1) - a(k) b(0))\tau(k'', 1| k, 0) \\
        &= \epsilon_{\sinc}\alpha^2 t^2 + b(0) \parens{\sum_{k' \in f_U(k)} a(k') \tau(k', 0 | k, 1) - \sum_{k'' \in f_L(k)} a(k'') \tau(k'', 1| k, 0)} \nonumber \\
        &\quad + b(1) \parens{-\sum_{k' \in f_U(k)} a(k') \tau(k', 0 | k, 1) + \sum_{k'' \in f_L(k)} a(k'') \tau(k'', 1| k, 0)} \\
        &\eqqcolon \widetilde{B}.
    \end{align}
    We let $\widetilde{B}$ denote the upper bound on $B(k)$. 
    
    this gives the final bound for $\trace{B^2}$ as
    \begin{equation}
        \trace{B^2} \leq 4 |T_{\gamma}| (\alpha^2 t^2 \epsilon_{\sinc})^2 + |S_{\gamma}| \widetilde{B}^2
    \end{equation}
    
    
    
    \newpage
    \noindent\rule{\textwidth}{1pt}
    \noindent\rule{\textwidth}{1pt}
    
    \begin{claim}
        Given inputs $H_S$, $\rho_S$, $\beta_E$, and $\gamma$, we compute an upper bound on the distance of the thermalizing channel as
        \begin{align}
            \norm{\rho_S(\beta_E) - \Phi(\rho_S(\beta))}_2^2 \leq \norm{\rho_S(\beta_E) - \rho_S}_2^2 - \epsilon
        \end{align}
        
    \end{claim}
    
    
    Starting from the beginning:
    \begin{align}
        \norm{\rho_S(\beta_E) - \Phi(\rho_S(\beta))}_2^2 &= \norm{\rho_S(\beta_E) - \rho_S(\beta) - \sum_{i,j,k,l} \frac{e^{-\beta_E \lambda_E(j) - \beta \lambda_S(i)}}{\partfun_E(\beta_E) \partfun_S(\beta)} \tau(i,j|k,l) \ketbra{k}{k} }_2^2 + \bigo{\alpha^6} \\
        &\approx \norm{A - B}_2^2 \\
        &= \trace{(A - B)^\dagger (A - B)} \\
        &= \trace{A^2} - 2 \trace{A B} + \trace{B^2},
    \end{align}
    where $A \coloneqq \rho_S(\beta_E) - \rho_S(\beta)$ and $B \coloneqq \sum_{i,j,k,l} \frac{e^{-\beta_E \lambda_E(j) - \beta \lambda_S(i)}}{\partfun_E(\beta_E) \partfun_S(\beta)} \tau(i,j|k,l) \ketbra{k}{k} $ are Hermitian and diagonal in the system's Hamiltonian basis. As $\trace{A^2}$ represents the distance we are trying to reduce, the goal is to show that $2 \trace{AB} \geq \trace{B^2} + \epsilon$. 
    
    We use this sorting to create the (ordered) set of indices as before,
    
    This means that each pair of indices $(i,j)$, such that their eigenvalue difference is close to $\gamma$, is stored only once in the set. This allows us to define the function $S_\gamma(i) = j$, as we have assumed non-degeneracy of the system eigenvalues. In a similar spirit, we say $i \in S_{\gamma}$ if it is the first index of a pair $(i,j) \in S_{\gamma}$. We can further define the set of ``dead" indices, those indices that do not have another eigenvalue that such that their difference is close to $\gamma$. This gives:
    
    Now our goal is to compute $\trace{A B} = \sum_{k} A(k) B(k)$. We first compute $A(k) B(k)$ for $k \in T_{\gamma}$ and then for $k \in S_{\gamma}$. 
    
    
    
    Now we move on to bounding the paired indices. Let $(k_1, k_2) \in S_{\gamma}$ be a single pair. We look at the sum of $A(k_1) B(k_1) + A(k_2) B(k_2)$. 
    \begin{align}
        B(k_1) &\leq  \parens{\sum_{i \neq k_1, k_2} \sum_{j,l} (a(i) b(j) - a(k_1) b(l)) \tau(i,j|k_1, l) + 3 \frac{\alpha^2 t^2}{\dim + 1} \epsilon_{\sinc} + (a(k_2) b(0) - a(k_1) b(1)) \tau(k_2, 0 | k_1, 1)} \\
        &\leq \parens{\sum_{i \neq k_1, k_2} \sum_{j,l} \frac{\alpha^2 t^2}{\dim + 1}\epsilon_{\sinc} + 3 \frac{\alpha^2 t^2}{\dim + 1}\epsilon_{\sinc} + (a(k_2) b(0) - a(k_1) b(1)) \tau(k_2, 0 | k_1, 1)} \\
        &= (4 \dim_S - 5) \frac{\alpha^2 t^2}{\dim + 1} \epsilon_{\sinc} + (a(k_2) b(0) - a(k_1) b(1)) \tau(k_2, 0 | k_1, 1) \\
        &\leq 4 \dim_S \frac{\alpha^2 t^2}{\dim + 1} \epsilon_{\sinc} + \tau(k_2, 0| k_1, 1)
    \end{align}
    
    
    
    and similarly we compute
    \begin{align}
        B(k_2) &\leq  \parens{\sum_{i \neq k_1, k_2} \sum_{j,l} (a(i) b(j) - a(k_1) b(l)) \tau(i,j|k_2, l) + 3 \frac{\alpha^2 t^2}{\dim + 1}\epsilon_{\sinc} + (a(k_1) b(1) - a(k_2) b(0)) \tau(k_1, 1 | k_2, 0)} \\
        &\leq \parens{\sum_{i \neq k_1, k_2} \sum_{j,l} \frac{\alpha^2 t^2}{\dim + 1} \epsilon_{\sinc} + 3 \frac{\alpha^2 t^2}{\dim + 1} \epsilon_{\sinc} + (a(k_1) b(1) - a(k_2) b(0)) \tau(k_1, 1 | k_2, 0)} \\
        &= (4 \dim_S - 5) \frac{\alpha^2 t^2}{\dim + 1} \epsilon_{\sinc} + (a(k_1) b(1) - a(k_2) b(0)) \tau(k_1, 1 | k_2, 0) \\
        &\leq 4 \dim_S \frac{\alpha^2 t^2}{\dim + 1} \epsilon_{\sinc} + \tau(k_1, 1| k_2, 0).
    \end{align}
    Given that $\tau$ is symmetric about input and output, we have $B(k_1) - (4 \dim_S - 5) \frac{\alpha^2 t^2}{\dim + 1} \epsilon_{\sinc} = - (B(k_2) - (4\dim_S - 5) \frac{\alpha^2 t^2}{\dim + 1} \epsilon_{\sinc})$. This allows us to compute the sum
    \begin{align}
        A(k_1) B(k_1) + A(k_2) B(k_2) &\leq (4 \dim_S - 5) \epsilon_{\sinc} \parens{A(k_1) + A(k_2)} + (a(k_2) b(0) - a(k_1) b(1)) \tau(k_1, 1| k_2, 0) (A(k_1) - A(k_2)) \\
        &\leq 2 (4 \dim_S - 5) \epsilon_{\sinc} + (a(k_2) b(0) - a(k_1) b(1)) \tau(k_1, 1| k_2, 0) (A(k_1) - A(k_2)) \\
        &= 2 (4 \dim_S - 5) \epsilon_{\sinc} + \frac{e^{-\beta \lambda_S(k_2) - \beta_E \lambda_E(0)} - e^{-\beta \lambda_S(k_1) - \beta_E \lambda_E(1)}}{\partfun_S(\beta) \partfun_E(\beta_E)} \tau(k_1, 1| k_2, 0) (A(k_1) - A(k_2)).
    \end{align}
    As we want this to be negative, we require
    \begin{align}
        e^{-\beta \lambda_S(k_2) - \beta_E \lambda_E(0)} - e^{-\beta \lambda_S(k_1) - \beta_E \lambda_E(1)} &\leq 0 \\
        \beta \lambda_S(k_2) &\geq \beta \lambda_S(k_1) + \beta_E\gamma \\
        \beta \Delta_S(k_2, k_1) &\geq \beta_E \gamma.
    \end{align}
    
    We will need an upper bound on $\trace{B^2}$
    \begin{align}
        \trace{B^2} &= \sum_k B(k)^2 \\
        &= \sum_{k \in T_{\gamma}} B(k)^2 + \sum_{(k_1, k_2) \in S_{\gamma}} B(k_1)^2 + B(k_2)^2 \\
        &\leq 16 |T_{\gamma}| (\dim_S - 1)^2 \frac{\alpha^4 t^4}{(\dim + 1)^2}\epsilon_{\sinc}^2 + \sum_{(k_1, k_2) \in S_{\gamma}} B(k_1)^2 + B(k_2)^2
    \end{align}
    In order to compute the above we need an upper bound on $B(k_1)^2$ and $B(k_2)^2$. For ease of notation, we define the following variables:
    \begin{align}
        S_1 &\coloneqq \sum_{i \neq k_1, k_2} \sum_{j,l} (a(i) b(j) - a(k_1) b(l)) \tau(i, j|k_1, l) \\
        S_2 &\coloneqq \sum_{(j,l) \neq (0, 1)} (a(k_2) b(j) - a(k_1) b(l)) \tau(k_2, j| k_1, l) \\
        r &\coloneqq a(k_2) b(0) - a(k_1) b(1) 
    \end{align}
    We can bound the absolute value of these as:
    \begin{align}
        \abs{S_1} &\leq \sum_{i \neq k_1, k_2} \sum_{j,l} |a(i) b(j) - a(k_1) b(l)| |\tau(i, j|k_1, l)| \\
        &\leq \frac{\alpha^2 t^2}{\dim + 1} \epsilon_{\sinc} \sum_{i \neq k_1, k_2} \sum_{j,l} 1\\
        &\leq 4 (\dim_S - 2) \frac{\alpha^2 t^2}{\dim + 1} \epsilon_{\sinc} \\
        \abs{S_2} &\leq \sum_{(j,l) \neq (0,1)} |a(k_1) b(j) - a(k_2) b(j)| |\tau(k_2, j | k_1, l)| \\
        &\leq \frac{\alpha^2 t^2}{\dim + 1} \epsilon_{\sinc} \sum_{(j,l) \neq (0,1)} 1 \\
        &\leq 3 \frac{\alpha^2 t^2}{\dim + 1} \epsilon_{\sinc}
    \end{align}
    This allows us to compute bounds on $B(k_1)^2$ and $B(k_2)^2$ as:
    \begin{align}
        B(k_1)^2 &= \bigg( S_1 + S_2 + r \tau(k_2, 0| k_1, 1) \bigg)^2 \\
        &\leq (|S_1| + |S_2|)^2 + 2(S_1 + S_2) r \tau(k_2, 0| k_1, 1) + r^2 \tau(k_2, 0| k_1, 1)^2 \\
        B(k_2)^2 &= \bigg( S_1 + S_2 - r \tau(k_2, 0 | k_1, 1) \bigg)^2 \\
        &\leq (|S_1| + |S_2|)^2 - 2 (S_1 + S_2) r \tau(k_2, 0 | k_1, 1) + r^2 \tau(k_2, 0 | k_1, 1)^2 \\
        B(k_1)^2 + B(k_2)^2 &\leq 2 (|S_1| + |S_2|)^2 + 2 r^2 \tau(k_2, 0 | k_1, 1)^2 \\
        &\leq 32 \dim_S^2 \frac{\alpha^4 t^4}{(\dim + 1)^2} \epsilon_{\sinc}^2 + 2 r^2 \tau(k_2, 0 | k_1, 1)^2.
    \end{align}
    We can now bound $\trace{B^2}$ as
    \begin{align}
        \trace{B^2} &= \sum_{k \in T_{\gamma}} B(k)^2 + \sum_{(k_1, k_2) \in S_{\gamma}} B(k_1)^2 + B(k_2)^2 \\
        &\leq 16 |T_{\gamma}| (\dim_S - 1)^2 \frac{\alpha^4 t^4}{(\dim + 1)^2} \epsilon_{\sinc}^2 + 16 \dim_S^2 (2 |S_{\gamma}|) \frac{\alpha^4 t^4}{(\dim + 1)^2} \epsilon_{\sinc}^2 + 2 \sum_{(k_1, k_2) \in S_{\gamma}} r^2 \tau(k_1, 1 | k_2, 0)^2 \\
        &\leq 16 \dim_S^3 \frac{\alpha^4 t^4}{(\dim + 1)^2} \epsilon_{\sinc}^2 + 2 \sum_{(k_1, k_2) \in S_{\gamma}} r^2 \tau(k_1, 1 | k_2, 0)^2
    \end{align}
    
    Now our goal is to upper bound $-2\trace{AB}$. We make use of the fact that $|A(k)| \leq 1$ for all $k$ and that $k \in T_{\gamma}$ implies that $|B(k)| \leq 4 (\dim_S - 1) \epsilon_{\sinc}$.
    \begin{align}
        -2 \trace{AB} &= -2 \sum_{k \in T_{\gamma}} A(k) B(k) - 2 \sum_{(k_1, k_2) \in S_{\gamma}} A(k_1) B(k_1) + A(k_2) B(k_2) \\
        &\leq + 2 \sum_{k \in T_{\gamma}} |A(k)| |B(k)| - 2 \sum_{(k_1, k_2) \in S_{\gamma}} A(k_1) B(k_1) + A(k_2) B(k_2) \\
        &\leq 8 \dim_S |T_{\gamma}| \frac{\alpha^2 t^2}{\dim + 1} \epsilon_{\sinc} - 2 \sum_{(k_1, k_2) \in S_{\gamma}} A(k_1) B(k_1) + A(k_2) B(k_2).
    \end{align}
    We now investigate the right term in the above. Using the notation from before
    \begin{align}
        A(k_1) B(k_1) &= A(k_1) (S_1(k_1) + S_2(k_1) + r \tau(k_1, 1 | k_2, 0)) \\
        A(k_2) B(k_2) &= A(k_2) (S_1(k_2) + S_2(k_2) - r \tau(k_1, 1 | k_2, 0)) \\
        -2A(k_1) B(k_1) -2 A(k_2) B(k_2) & = -2 A(k_1)(S_1(k_1) + S_2(k_1))  - 2A(k_2)( S_1(k_2) + S_2(k_2)) + 2 r \tau(k_1, 1 | k_2, 0)( A(k_2) - A(k_1)) \\
        &\leq 2 |A(k_1)|(|S_1(k_1)| + |S_2(k_1)|) + 2 |A(k_2)|( |S_1(k_2)| + |S_2(k_2)|) \nonumber \\
        &+ 2 r \tau(k_1, 1 | k_2, 0)(A(k_2) - A(k_1)) \\
        &\leq 16 \dim_S \frac{\alpha^2 t^2}{\dim + 1} \epsilon_{\sinc} + 2 r \tau(k_1, 1 | k_2, 0)( A(k_2) - A(k_1)).
    \end{align}
    This allows us to plug in to our sum:
    \begin{align}
        -2 \trace{AB} &\leq 8 \dim_S(|T_{\gamma}| + 2 |S_{\gamma}|) \frac{\alpha^2 t^2}{\dim + 1} \epsilon_{\sinc} + 2 \sum_{(k_1, k_2) \in S_{\gamma}} r \tau(k_1, 1 | k_2, 0) ( A(k_2) - A(k_1)) \\
        &= 8 \dim_S^2 \frac{\alpha^2 t^2}{\dim + 1} \epsilon_{\sinc} + 2 \sum_{(k_1, k_2) \in S_{\gamma}} r \tau(k_1, 1 | k_2, 0) ( A(k_2) - A(k_1)).
    \end{align}
    Combining with the bound for $\trace{B^2}$ we get the following
    \begin{align}
        \trace{B^2} - 2 \trace{AB} &\leq 16 \dim_S^3 \frac{\alpha^4 t^4}{(\dim + 1)^2} \epsilon_{\sinc}^2 + 8 \dim_S^2 \frac{\alpha^2 t^2}{\dim + 1} \epsilon_{\sinc} + 2 \sum_{(k_1, k_2) \in S_{\gamma}} \parens{r^2 \tau(k_1, 1 |k_2, 0)^2 + r \tau(k_1, 1| k_2, 0) (A(k_2) - A(k_1))}.
    \end{align}
    To prove the required bound, we will require the following two inequalities:
    \begin{align}
        16 \dim_S^3 \frac{\alpha^4 t^4}{(\dim + 1)^2}\epsilon_{\sinc}^2 + 8 \dim_S^2 \frac{\alpha^2 t^2}{\dim + 1} \epsilon_{\sinc} &\leq \frac{\epsilon}{2} \\
        2 \sum_{(k_1, k_2) \in S_{\gamma}} r \tau(k_1, 1 | k_2, 0) (r \tau(k_1, 1 | k_2, 0)  + A(k_2) - A(k_1)) &\leq - \frac{3\epsilon}{2}.
    \end{align}
    
    Everything below this line has not been updated with the fix to $\tau(i,j|k,l) \leq \frac{\alpha^2 t^2}{\dim + 1} \epsilon_{\sinc}$. 
    
    \noindent\rule{\textwidth}{1pt}
    
    The first inequality is rather straightforward. We use the intuition that the left term (ignoring constant factors) is nearly the square of the right term, and since we want to bound their sum with something small we can then deduce that the right term should be dominant. This yields the intuition that
    \begin{align}
        16 \dim_S^3 \frac{\alpha^4 t^4}{(\dim + 1)^2} \epsilon_{\sinc}^2 &\leq 8 \dim_S^2 \frac{\alpha^2 t^2}{\dim + 1} \epsilon_{\sinc} \\
        \epsilon_{\sinc} &\leq \frac{\dim + 1}{2 \dim_S} \frac{1}{(\alpha^2 t^2)} \\
        &= \parens{1 + \frac{1}{\dim}} \frac{1}{\alpha^2 t^2}.
    \end{align}
    We then note that because the bound $\epsilon_{\sinc} \geq 1/(\Delta_{\min}^2 t^2)$ has to be satisfied, this yields a ``sandwich'' of acceptable ranges for $\epsilon_{\sinc}$ when combined with the upper bound above. To make this sandwich reasonable we require
    \begin{align}
        \frac{1}{\Delta_{\min}^2 t^2} &\leq \epsilon_{\sinc} \leq \parens{1 + \frac{1}{\dim}} \frac{1}{\alpha^2 t^2} \\
        \alpha^2 &\leq \Delta_{\min}^2 \parens{1 + \frac{1}{\dim}}.
    \end{align}
    Curiously, there are no requirements on the value of $t$. To complete this argument, we now simplify the lower bound on $\epsilon$ by noting,
    \begin{align}
        16 \dim_S^3 \frac{\alpha^4 t^4}{(\dim + 1)^2} \epsilon_{\sinc}^2 + 8 \dim_S^2 \frac{\alpha^2 t^2}{\dim + 1}\epsilon_{\sinc} &\leq 16 \dim_S^2 \frac{\alpha^2 t^2}{\dim + 1}\epsilon_{\sinc},
    \end{align}
    which yields $\epsilon \geq 32 \dim_S^2 \epsilon_{\sinc} \frac{\alpha^2 t^2}{\dim + 1}$.
    
    The second inequality is a lot trickier. We will work with this as a quadratic expression in $\tau$ and find values of $\tau$ such that it is satisfied. We plug in for $\tau$ and rewrite the inequality as 
    \begin{align}
        &\frac{(\alpha t)^4}{(\dim + 1)^2} \parens{2  \sum_{(k_1, k_2) \in S_{\gamma}} r^2 \sinc((\Delta_S(k_1, k_2) - \gamma) t/2)^4} \nonumber \\
        &+ \frac{(\alpha t)^2}{\dim + 1} \parens{2  \sum_{(k_1, k_2) \in S_{\gamma}}r \sinc((\Delta_S(k_1, k_2) - \gamma) t/2)^2 (A(k_2) - A(k_1))} \nonumber \\
        &+ \parens{\frac{\alpha t}{\dim + 1}}^0 \frac{3 \epsilon}{2} \leq 0
    \end{align}
    We first simplify the coefficient for $\alpha^4$ as 
    \begin{align}
        2 \sum_{(k_1, k_2) \in S_{\gamma}} r^2 \sinc((\Delta_S(k_1, k_2) - \gamma) t/2)^4 &\leq 2 |S_{\gamma}|.
    \end{align}
    Since everything is positive, if this simplified inequality holds then so does the original. 
    
    For simplicity we denote the summation in the coefficient for $\alpha^2$ as
    $$d \coloneqq 2 \sum_{(k_1, k_2) \in S_{\gamma}} r \sinc((\Delta_S(k_1, k_2) - \gamma) t/2)^2 (A(k_2) - A(k_1)).$$
    Letting $x$ play the role of $(\alpha t)^2 / (\dim + 1)$ gives us a simple quadratic expression:
    \begin{equation}
        2 x^2 |S_{\gamma}| + x d + \frac{3 \epsilon}{2} \leq 0.
    \end{equation}
    As $(\alpha t)^2 / (\dim + 1) \geq 0$ we have $x$ must be positive as well. This requires for not only $d$ to be negative, but that the roots of the quadratic must both be real and at least one of them must be positive. Denote the roots $x_{\pm}$, which can be seen as:
    \begin{align}
        x_{\pm} &= \frac{-d}{4 |S_{\gamma}|} \pm \frac{1}{4 |S_{\gamma}|} \sqrt{d^2 - 12 |S_{\gamma}| \epsilon} \\
        &= \frac{-d}{4 |S_{\gamma}|} \parens{1 \mp \sqrt{1 - \frac{12 |S_{\gamma}| \epsilon}{d^2}}}.
    \end{align}
    We want to guarantee that these roots are real, so we require 
    \begin{equation}
        d^2 \geq 12 |S_{\gamma}| \epsilon. \label{eq:d_squared_epsilon_bound}
    \end{equation}
    Note that inside the radical, $12  |S_{\gamma}| \epsilon / d^2$ is always positive, implying that the radical is always going to be less than 1. This leads to both roots being positive so long as $d \leq 0$ and $d^2$ is lower bounded as mentioned.
    
    If these two conditions are met, then we can set $\alpha^2$ to be the average of the two roots and the distance reduction claim is satisfied. Ideally we would like the lower root to be as close to 0 as possible, so that way we can simply reduce $\alpha$ to get our desired distance reduction. Regardless, our last objective still is to produce bounds on $d$ that satisfy the given inequalities, we therefore turn our attention to the sum in question. To simplify $d$ we first investigate the sign of $r$, assuming it to be positive we see what conditions result
    \begin{align}
        r = a(k_2) b(0) - a(k_1) b(1) &\geq 0 \\
        \bra{k_2} \rho_S \ket{k_2} \frac{e^{-\beta_E \lambda_E(0)}}{\partfun_E(\beta_E)} - \bra{k_1} \rho_S \ket{k_1} \frac{e^{-\beta_E \lambda_E(1)}}{\partfun_E(\beta_E)} &\geq 0 \\
        \frac{\bra{k_2} \rho_S \ket{k_2}}{\bra{k_1} \rho_S \ket{k_1}} &\geq e^{-\beta_E \gamma}. \label{eq:bound_on_rho_s_for_r}
    \end{align}
    If we were using thermal states, $\rho_S = e^{-\beta H_S} / \partfun_S(\beta)$, the left hand side of the above would be $e^{-\beta \Delta_S(k_2, k_1)}$. Simplifying would result in $\beta \leq \frac{\gamma}{\Delta_S(k_2, k_1)} \beta_E$. Since we expect the distance moved by the thermalizing channel to be greatest as $\beta_E \to \infty$ and $\beta \to 0$, requiring this inequality to be true, and therefore $r \geq 0$, seems to be a reasonable condition to impose.
    
    With the sign of $r$ sorted out, we now have to bound the summation for $d$. Since we require $d \leq 0$ in order to get positive values for $(\alpha t)^2 $, we require
    \begin{align}
        \sum_{(k_1, k_2) \in S_{\gamma}} \frac{\sinc((\Delta_S(k_2, k_1) - \gamma)t/2)^2}{\dim + 1} (A(k_2) - A(k_1)) &\leq 0.
    \end{align}
    To prove this we bound the following, note we surpress the arguments to $\sinc$ to save space, and we make use of the bound required in Eq. \eqref{eq:bound_on_rho_s_for_r}
    \begin{align}
        \sum_{(k_1, k_2) \in S_{\gamma}} \frac{\sinc^2}{\dim + 1} A(k_2) &= \sum_{(k_1, k_2) \in S_{\gamma}} \frac{\sinc^2}{\dim + 1} \parens{\frac{e^{-\beta_E \lambda_S(k_2)}}{\partfun_S(\beta_E)}  - \bra{k_2} \rho_S \ket{k_2}} \\
        &\leq \sum_{(k_1, k_2) \in S_{\gamma}} \frac{\sinc^2}{\dim + 1} \parens{\frac{e^{-\beta_E \lambda_S(k_2)}}{\partfun_S(\beta_E)}  - e^{-\beta_E \gamma} \bra{k_1} \rho_S \ket{k_1}} \\
        &= \sum_{(k_1, k_2) \in S_{\gamma}} \frac{\sinc^2}{\dim + 1} \parens{\frac{e^{-\beta_E (\lambda_S(k_1) - \lambda_S(k_1) + \lambda_S(k_2) )}}{\partfun_S(\beta_E)}  - e^{-\beta_E \gamma} \bra{k_1} \rho_S \ket{k_1}} \\
        &= \sum_{(k_1, k_2) \in S_{\gamma}} \frac{\sinc^2}{\dim + 1} \parens{\frac{e^{-\beta_E \lambda_S(k_1)}}{\partfun_S(\beta_E)} e^{-\beta_E \Delta_S(k_2, k_1)}  - e^{-\beta_E \gamma} \bra{k_1} \rho_S \ket{k_1}}.
    \end{align}
    We note right away that as $\beta_E \to 0$ this yields $\sum_{(k_1, k_2) \in S_{\gamma}} \frac{\sinc^2}{\dim + 1} (A(k_2) - A(k_1)) \leq 0$. 
    
    Now subtracting the sum with $A(k_1)$ and simplifying yields
    \begin{align}
        \sum_{(k_1, k_2) \in S_{\gamma}} \frac{\sinc^2}{\dim + 1} (A(k_2) - A(k_1)) &\leq \sum_{(k_1, k_2) \in S_{\gamma}} \frac{\sinc^2}{\dim + 1} \parens{\frac{e^{-\beta_E \lambda_S(k_1)}}{\partfun_S(\beta_E)} (e^{-\beta_E \Delta_S(k_2, k_1)} - 1) - \bra{k_1} \rho_S \ket{k_1} (e^{-\beta_E \gamma} - 1) } .
    \end{align}
    Since we have to upper bound this summation by 0, a good first step would be to understand when a given term is positive or negative. Taking a generic term, disregarding the prefactor of $\frac{\sinc^2}{\dim + 1}$ and simplifying leads to
    \begin{align}
        \bra{k_1} \rho_S \ket{k_1}& \parens{1 - e^{-\beta_E \gamma}} -\frac{e^{-\beta_E \lambda_S(k_1)}}{\partfun_S(\beta_E)}\parens{1 - e^{-\beta_E \Delta_S}} \leq 0 \\
        \bra{k_1} \rho_S \ket{k_1}&  \leq \frac{e^{-\beta_E \lambda_S(k_1)}}{ \partfun_S(\beta_E)}\frac{1 - e^{-\beta_E \Delta_S}}{1 - e^{-\beta_E \gamma}}.
    \end{align}
    In order to proceed with the analysis we must impose some kind of structure onto $\rho_S$. We will investigate a few limits, one in which $\rho_S$ is a thermal state with $\beta$ close enough to $\beta_E$, one in which $\beta_E \to \infty$, and another in which we bound the operator distance of $\rho_S$ from $\rho_S(\beta_E)$. 
    
    The condition in which we expect the most rapid thermalization is one in which the environment is in it's ground state (temperature of 0 or $\beta_E \to \infty$) and the system is in the maximally mixed state (temperature to infinity or $\beta \to 0$). In this situation, the factors $e^{-\beta_E \Delta_S} \to 0$ and $e^{-\beta_E \gamma} \to 0$. Further, the Boltzmann factors approach 0 if the eigenvalue is not a minimal eigenvalue or 1 if it is a minimal eigenvalue (ground state energy). In this situation, if $(0, i) \notin S_{\gamma}$, for all $i$, then we have that the Boltzmann factors $e^{-\beta_E \lambda_S(k_1)} = 0$. In this case we can also directly evaluate the summation $A(k_2) - A(k_1)$, as this is 
    \begin{equation}
        A(k_2) - A(k_1) = \frac{e^{-\beta_E \lambda_S(k_2)}}{\partfun_S(\beta_E)} - \frac{1}{\dim_S} - \frac{e^{-\beta_E \lambda_S(k_1)}}{\partfun_S(\beta_E)} + \frac{1}{\dim_S} = 0.
    \end{equation}
    From Eq. \eqref{eq:d_squared_epsilon_bound} we see that this implies that $\epsilon = 0$ and we do not have any distance reduction to the ground state possible. This makes intuitive sense, as any probability mass that gets shuffled from high energy states to lower energy states does not get us any closer to the ground state. 
    However, whenever the 0 temperature environment and infinite temperature system are coupled when $\gamma$ is close enough to a transition between a system ground state and an excited state we can get distance reduction. We will analyze this situation now. In the case in which there exists a pair $(0, i) \in S_{\gamma}$, meaning $e^{-\beta_E \lambda_S(k)} / \partfun_S(\beta_E) = \delta_{k,0}$ and $|\Delta_S(i, 0) - \gamma| \leq \Delta_{\min}$. In this case, if $(k_1, k_2) \in S_{\gamma}$ and $k_1 \neq 0$, then $A(k_2) - A(k_1) = 0$. However, for $(0, k)$ $A(k) - A(0) = 0 - 1/\dim_S - 1 + 1/\dim_S = -1$. Therefore, the total sum is then 
    \begin{align}
        \sum_{(k_1, k_2) \in S_{\gamma}} \frac{\sinc^2}{\dim + 1} (A(k_2) - A(k_1)) &= - \sum_{(0, k) \in S_{\gamma}} \frac{\sinc^2}{\dim + 1} \\
        &\leq \frac{- \epsilon_{\sinc}}{\dim + 1}.
    \end{align}
    In addition, we can provide the simplistic bound 
    \begin{equation}
        \sum_{(k_1, k_2) \in S_{\gamma}} \frac{\sinc^2}{\dim + 1}(A(k_2) - A(k_1)) \geq \frac{-1}{\dim + 1}.
    \end{equation}
    This allows us to argue that $\frac{\epsilon_{\sinc}^2}{(\dim + 1)^2} \leq d^2 \leq \frac{1}{(\dim + 1)^2}$. Propagating this through to $\epsilon$ via Eq. \eqref{eq:d_squared_epsilon_bound} yields 
    \begin{align}
        d^2 &\geq \frac{12 t^4 |S_{\gamma}| \epsilon}{(\dim + 1)^2} \\
        \epsilon &\leq \frac{1}{12 t^4 |S_{\gamma}|}.
    \end{align}
    Now we look at the lower bound for $\epsilon$, which is given by 
    \begin{align}
        \epsilon \geq 32 \epsilon_{\sinc} \alpha^2 t^2 \frac{\dim_S^2}{\dim + 1}
    \end{align}
    
    
    Without adding any structure to $\rho_S$ this is essentially a requirement we must impose on the state. We will go on to investigate for what ranges of $\beta$ this holds in the case that $\rho_S$ is a thermal state. We see as $\beta_E \to \infty$ that this inequality is trivially satisfied for $k_1$ being the ground state, the RHS approaches 1.
    
    As we can see that this bound is pretty much dependent on the structure of the state, we now move on to bounding the value of $\epsilon$ that can be achieved. This comes from the bound in Eq. \eqref{eq:d_squared_epsilon_bound}, where we note that since $d$ is negative (from $r$ being positive) this amounts to an upper bound on $d$ (or a lower bound on $\epsilon$).
    



\section{Single Qubit System \& Environment}
We now will analyze the effects of $\Phi$ on thermalizing a system state with only a single qubit environment. The single qubit case makes the effect of tracing out the environment tractable. For absolute simplicity, we first study the case of two qubits. After this situation is analyzed in detail we extend the results to arbitrary system Hamiltonians. For the remainder of this section we assume a a system Hamiltonian $H_S = \begin{bmatrix}
    0 & 0 \\ 0 & \Delta_S
\end{bmatrix}$ and an environment Hamiltonian $H_E = \begin{bmatrix}
    0 & 0 \\ 0 & \gamma
\end{bmatrix}$. Further we will use an environment prepared in the state $\rho_E(\beta) = \frac{e^{-\beta H_E}}{\partfun_E(\beta)}$. Another restriction we will make is that we assume that $t \geq \frac{2}{\Delta_{\min} \sqrt{\epsilon_{\sinc}}}$, where $\lambda(i,j) = \lambda(k,l)$ or $\lambda(i,j) - \lambda(k,l) \geq \Delta_{\min}$, per Lemma \ref{lem:sinc_poly_approx}. 

Given the two qubit setup as mentioned, we want to compute the trace distance 
\begin{equation}
    \norm{\rho_S(\beta) - \Phi(\rho_S(\beta))}_1.
\end{equation}
We first start with the trace distance and reduce it to a computation of transition coefficients. 
\begin{align}
    \norm{\rho_S(\beta) - \Phi(\rho_S(\beta))}_1  &= \norm{\frac{\alpha^2}{2} \partrace{\hilb_E}{\int \frac{\partial^2}{\partial \alpha^2} \Phi_G(\rho_S(\beta) \otimes \rho_E(\beta)) \bigg|_{\alpha = 0} ~dG}}_1 + \bigo{\alpha^3} \\
    &\approx \norm{\sum_{i,j} e^{-\beta \lambda(i,j)} \partfun(\beta)^{-1}\partrace{\hilb_E}{ \frac{\alpha^2}{2} \int \frac{\partial^2}{\partial \alpha^2} \Phi_G(\ketbra{i,j}{i,j})\bigg|_{\alpha=0} dG} }_1 \\
    &= \norm{\sum_{i,j} e^{-\beta \lambda(i,j)} \partfun(\beta)^{-1} \partrace{\hilb_E}{\sum_{k,l} \tau(i,j | k,l) \ketbra{k,l}{k,l}}}_1 \\
    &= \norm{\sum_{i,j} e^{-\beta \lambda(i,j)} \partfun(\beta)^{-1} \sum_{k,l} \tau(i,j |k,l) \ketbra{k}{k}}_1 \\
    &= \sum_k \abs{\sum_{i,j} e^{-\beta \lambda(i,j)} \partfun(\beta)^{-1} \sum_{l} \tau(i,j |k,l)} \\
    &= \frac{1}{|\partfun(\beta)|} \sum_k \abs{\sum_{i,j} e^{-\beta \lambda(i,j)} \sum_{l} \tau(i,j |k,l)} \label{eq:fixed_point_as_transition_coeffs}
\end{align}
\begin{align}
&\norm{\rho_S(\beta_E) - \Phi(\rho_S(\beta))}_p^p \\
&= \norm{\rho_S(\beta_E) - \rho_S(\beta) - \frac{\alpha^2}{2} \partrace{\hilb_E}{\int \frac{\partial^2}{\partial \alpha^2} \Phi_G(\rho_S(\beta) \otimes \rho_E(\beta_E)) \bigg|_{\alpha=0} dG}}_p^p + \bigo{\alpha^3} \\
&\approx \norm{\rho_S(\beta_E) - \rho_S(\beta) - \sum_{i,j} \frac{e^{-\beta \lambda_S(i) -\beta_E \lambda_E(j)}}{\partfun_S(\beta) \partfun_E(\beta_E)} \frac{\alpha^2}{2}\partrace{\hilb_E}{\int \frac{\partial^2}{\partial \alpha^2} \Phi_G(\ketbra{i,j}{i,j}) \bigg|_{\alpha=0} dG} }_p^p \\
&= \norm{\rho_S(\beta_E) - \rho_S(\beta) - \sum_{i,j} \frac{e^{-\beta \lambda_S(i) -\beta_E \lambda_E(j)}}{\partfun_S(\beta) \partfun_E(\beta_E)} \partrace{\hilb_E}{\sum_{k,l} \tau(i,j|k,l) \ketbra{k,l}{k,l}} }_p^p \\
&= \norm{\rho_S(\beta_E) - \rho_S(\beta) - \sum_{i,j} \sum_{k,l} \frac{e^{-\beta \lambda_S(i) -\beta_E \lambda_E(j)}}{\partfun_S(\beta) \partfun_E(\beta_E)} \tau(i,j|k,l) \ketbra{k}{k} }_p^p \\
&= \sum_k \abs{\frac{e^{-\beta_E \lambda_S(k)}}{\partfun_S(\beta_E)} - \frac{e^{-\beta \lambda_S(k)}}{\partfun_S(\beta)} - \sum_{i,j,l} \frac{e^{-\beta \lambda_S(i) - \beta_E \lambda_E(j)}}{\partfun_S(\beta) \partfun_E(\beta_E)} \tau(i,j|k,l)}^p. \label{eq:trace_dist_p_norm}
\end{align}
We note that the last step is due to the fact that $\rho_S(x)$ is diagonal in the eigenbasis that we are working over. We drop the $\bigo{\alpha^3}$ term, we could possibly upper bound it by some additional error contribution $\epsilon_{\alpha}$ or by stating that since it is less than the $\alpha^2$ contribution we could simply add a factor of 2 in front of the norm. These details should be cleaned up. Our goal is to now compute these sums for fixed $k$ in 0,1.

\subsection{Fixed Points}

In the following we set $p=1$ and $\beta = \beta_E$.
    We make heavy use of Lemma \ref{thm:second_order_transition_coeffs}. 
    We now compute the term for $k=0, i=1$
    \begin{align}
        \sum_{j, l} e^{-\beta \lambda(1,j)} \tau(1,j| 0, l) &=  e^{-\beta \Delta_S} \parens{\tau(1, 0| 0, 0) + \tau(1, 0 | 0, 1)} +  e^{-\beta (\Delta_S + \gamma)} \parens{\tau(1, 1 | 0, 0) + \tau(1, 1| 0, 1)} \label{eq:single_qubit_fixed_pt_1} 
    \end{align}
    Now we compute the term for $k=0, i= 0$, which involves using the self-transition substitution for $\tau(a,b|a,b)$.
    \begin{align}
        \sum_{j,l} e^{-\beta \lambda(0,j)} \tau(0,j|0,l) &= e^{-\beta \cdot 0} \parens{\tau(0,0| 0,0) + \tau(0,0 | 0,1)} + e^{-\beta \gamma} \parens{\tau(0,1 | 0,0) + \tau(0,1|0,1)} \\
        &= -(\tau(0,0| 0,1) + \tau(0,0| 1,0) + \tau(0,0|1,1)) + \tau(0,0|0,1) \nonumber \\
        &~ + e^{-\beta \gamma} \parens{\tau(0,1|0,0) - (\tau(0,1|0,0) + \tau(0,1| 1,0) + \tau(0,1|1,1))} \\
        &= - \tau(0,0|1,0)  - \tau(0,0|1,1) - e^{-\beta \gamma} \tau(0,1|1,0) - e^{-\beta \gamma} \tau(0,1|1,1) \label{eq:single_qubit_fixed_pt_2}
    \end{align}
    Using the fact that $\tau(a,b|c,d) = \tau(c,d|a,b)$ we can add Eqs. \ref{eq:single_qubit_fixed_pt_1} and \ref{eq:single_qubit_fixed_pt_2} to get:
    \begin{align}
        \sum_{i,j} e^{-\beta \lambda(i,j)} \sum_l \tau(i,j|0,l) &= \tau(0,0|1,0)(e^{-\beta \Delta_S} -1) + \tau(0,1|1,0)(e^{-\beta \Delta_S} - e^{-\beta \gamma}) \nonumber \\
        &\quad + \tau(0,0,| 1,1) (e^{-\beta (\Delta_S + \gamma)} - 1) + \tau(0,1|1,1) (e^{-\beta (\Delta_S + \gamma)} - e^{-\beta \gamma})
    \end{align}
    We can now use the fact that $\tau(a,b|c,d)$ is positive for $(a,b) \neq (c,d)$ and that $\beta > 0, \Delta_S > 0$ and $\gamma > 0$ to rewrite the above as
    \begin{align}
        \sum_{i,j} e^{-\beta \lambda(i,j)} \sum_l \tau(i,j|0,l) &= \tau(0,1|1,0)(e^{-\beta \Delta_S} - e^{-\beta \gamma}) - \tau(0,0|1,0)(1 - e^{-\beta \Delta_S})  \nonumber \\
        &\quad - \tau(0,0,| 1,1) (1 - e^{-\beta (\Delta_S + \gamma)}) - \tau(0,1|1,1) e^{-\beta \gamma} (1 - e^{-\beta \Delta_S})
    \end{align}
    Now using the intution that non-degenerate transitions are going to be significantly suppressed, we use the triangle inequality. 
    \begin{align}
        &\bigg| \tau(0,1|1,0)(e^{-\beta \Delta_S} - e^{-\beta \gamma}) - \tau(0,0|1,0)(1 - e^{-\beta \Delta_S})  \nonumber \\
        &\quad - \tau(0,0,| 1,1) (1 - e^{-\beta (\Delta_S + \gamma)}) - \tau(0,1|1,1) e^{-\beta \gamma} (1 - e^{-\beta \Delta_S})\bigg| \\
        &\leq \abs{\tau(0,1|1,0) (e^{-\beta \Delta_S} - e^{-\beta \gamma})} + \abs{\tau(0,0|1,0)} + |\tau(0,0|1,1)| + |\tau(0,1|1,1)| \\
        &\leq \abs{\tau(0,1|1,0) (e^{-\beta \Delta_S} - e^{-\beta \gamma})} + 3 \frac{\epsilon_{\sinc} \alpha^2 t^2}{2(\dim + 1)} \\
        &= \frac{\alpha^2 t^2}{2(\dim + 1)} \parens{\sinc^2((\Delta_S - \gamma)t/2) |e^{-\beta \Delta_S} - e^{-\beta \gamma}| + 3 \epsilon_{\sinc}}\\
        &= \frac{\alpha^2 t^2}{2 (\dim + 1)} \parens{\sinc^2((\Delta_S - \gamma)t/2) e^{-\beta \Delta_S} |1 - e^{\beta (\Delta_S - \gamma)}| + 3 \epsilon_{\sinc}} \\
        &=: \frac{\alpha^2 t^2}{2(\dim + 1)} (e^{-\beta \Delta_S} g(\Delta_S - \gamma) + 3 \epsilon_{\sinc})
    \end{align}

    Now when computing the summation $|\sum_{i,j} e^{-\beta \lambda(i,j)} \sum_l \tau(i,j |1,l)|$, for $k=1$, the same result as $k=0$ is found. This allows us to say that the trace distance is upper bounded by
    \begin{equation}
        \norm{\rho_S(\beta) - \Phi(\rho_S(\beta))}_1 \leq \frac{\alpha^2 t^2}{(\dim + 1)\partfun} (e^{-\beta \Delta_S} g(\Delta_S - \gamma) + 3 \epsilon_{\sinc}).
    \end{equation}
    We note that the following upper bound is sufficient to show that the trace distance is bounded:
    \begin{align}
        g(\Delta_S - \gamma) &= \sinc^2((\Delta_S - \gamma)t/2) \abs{1 - e^{\beta(\Delta_S - \gamma)}} \\
        &\leq \abs{e^{\beta(\Delta_S - \gamma)} - 1} \\
        &\leq e^{\beta \Delta_S} - 1,
    \end{align}
    where we assumed $\gamma \geq 0$ in the last step. This yields the following upper bound on the trace distance
    \begin{equation}
        \norm{\rho_S(\beta) - \Phi(\rho_S(\beta))}_1 \leq \frac{\alpha^2 t^2}{\partfun (\dim+1)} (1 - e^{-\beta \Delta_S} + 3 \epsilon_{\sinc}).
    \end{equation}
    We can even drop the $e^{-\beta \Delta_S}$ term and have 
    $$\alpha^2 \leq \epsilon_{\beta} \frac{ \partfun (\dim + 1)}{t^2 (1 - e^{-\beta \Delta_S} + 3 \epsilon_{\sinc})}$$
    to imply $\norm{\rho_S(\beta) - \Phi(\rho_S(\beta))}_1 \leq \epsilon_{\beta}$.

    This is not the complete picture, however. We also have to satisfy the inequality $\alpha^2 t^2 / (\dim + 1) \leq 1$ in order for $\tau$ to represent valid transition probabilities. 


    Another weird thing is that as $\gamma \to \Delta_S$, aka we have a degenerate transition possible, it seems that the contribution of this transition to the trace distance vanishes? If you look at the temperature contributions we have $\tau(0,1|1,0)|e^{-\beta \Delta_S} - e^{-\beta \gamma}|$, which goes to zero as $\gamma \to \Delta_S$. This is seemingly contradictory, as we are getting probability mass shuffling between the energy levels. 
    
    Our goal is to produce an upper bound on $g(\Delta_S - \gamma)$. To do so there are two different approaches we explore. The first is to sample $\gamma$ from a probability distribution and look at the expected value of the trace distance. The easiest distribution to start with is a uniform distribution from $\Delta_S - \frac{1}{2} \epsilon \Delta_{\text{min}}$ to $\Delta_S + \frac{1}{2} \epsilon \Delta_{\text{min}}$. Within this range we can approximate $\sinc^2 ((\Delta_S - \gamma)t/2)$ as $1$ with only an error of at most $\epsilon$. 
    \begin{align}
        \int_{\Delta_S - \epsilon \Delta_{\text{min}}/2}^{\Delta_S + \epsilon \Delta_{\text{min}}/2} g(\Delta_S - \gamma) \prob{\gamma} d\gamma &= \int_{\epsilon \Delta_{\text{min}}/2}^{-\epsilon \Delta_{\text{min}}/2} g(u) \prob{u(\gamma)} (-du) \\
        &=\frac{1}{\epsilon \Delta_{\text{min}}} \int_{-\epsilon \Delta_{\text{min}}/2}^{\epsilon \Delta_{\text{min}}/2} \sinc^2(u t /2) |1 - e^{\beta u}| du \\
        &\leq  \parens{\frac{1}{\epsilon \Delta_{\text{min}}} \int_{-\epsilon \Delta_{\text{min}}/2}^{\epsilon \Delta_{\text{min}}/2} |1 - \epsilon| |1 - e^{\beta u}| du + 3 \epsilon_{\sinc}} \\
        &= \parens{\frac{1}{ \Delta_{\text{min}}} \parens{\frac{1}{\epsilon } - 1}\frac{2(\cosh (\beta \epsilon \Delta_{\text{min}}/2) - 1)  }{\beta} + 3 \epsilon_{\sinc}}
    \end{align}
    We see that in the limit as $\epsilon \to 0$, $\cosh(\beta \epsilon \Delta_{\min} /2) - 1 \in \bigo{\epsilon^2}$, so the total contribution is of order $\bigo{\epsilon}$. Similar logic holds for the limit as $\Delta_{\min} \to 0$. As $\beta \to \infty$, the $\cosh$ term simply approaches an exponential $e^{\beta \epsilon \Delta_{\min} /2} / \beta$. When looking at this as a contribution to the trace distance, we see that $e^{-\beta \Delta_S} g(\Delta_S - \gamma) \to e^{-\beta (\Delta_S - \epsilon \Delta_{\min}/2)} / \beta$, and since $\Delta_{\min} \leq \Delta_S$ by definition we have a vanishing trace distance.

    One thing to note is that if $\gamma$ is too large then we fall into the ``near-zero" approximation regime due to the large value of $t$. We had set $t$ such that for $\Delta \geq \Delta_{\min}$, then $\sinc^2(\Delta t/2) \leq \epsilon_{\sinc}$. This means that we require $\Delta_S - \gamma \leq \Delta_{\min}$ to fall within the window for any reasonable contribution from the $\sinc$. This gives us a window of $\pm \Delta_{\min}$ around $\Delta_S$ that we should have Now we let $\epsilon$ be a constant factor, say $\epsilon = 1/2$. This seems sufficient for now. Plugging in for the upper bound on the trace distance yields:
    \begin{equation}
        \norm{\rho_S(\beta) - \Phi(\rho_S(\beta))}_1 \leq \frac{\alpha^2 t^2}{\partfun (\dim + 1)} \parens{ \frac{2 e^{-\beta \Delta_S}}{\beta \Delta_{\min}} (\cosh(\beta \Delta_{\min} / 4) - 1) + 3 \epsilon_{\sinc} }
    \end{equation}

    What if instead of integrating really close to $\Delta_S$, we actually integrate the function over some much larger interval? Would this work? 
    \begin{align}
        \int_{\Delta_{\min}}^{\Delta_{\max}} g(\Delta_S - \gamma) \prob{\gamma} d\gamma &= \int_{\Delta_{\min}}^{\Delta_{\max}} \sinc^2((\Delta_S - \gamma)t/2)  \frac{\abs{1 - e^{\beta (\Delta_S - \gamma)}}}{\Delta_{\max} - \Delta_{\min}} d\gamma \\
        &= \int_{\Delta_{\min}}^{\Delta_S} \sinc^2((\Delta_S - \gamma)t/2) \frac{e^{\beta (\Delta_S - \gamma)} - 1}{\Delta_{\max} - \Delta_{\min} } d\gamma + \int_{\Delta_S}^{\Delta_{\max}} \sinc^2((\Delta_S - \gamma)t/2) \frac{1 - e^{- \beta (\gamma - \Delta_S)}}{\Delta_{\max} - \Delta_{\min} } d\gamma
    \end{align}
 
\begin{align}
    \norm{A}_1 &= \trace{\sqrt{A A^\dagger}} \\
    &= \trace{\sqrt{\sum_i |a_i|^2 \ketbra{i}{i}}} \\
    &= \trace{\sum_i |a_i| \ketbra{i}{i}} \\
    &= \sum_i |a_i|.
\end{align}

\subsection{Distance Reduction}
We now would like to show that some distance metric between an input state $\rho_S(\beta)$ and the desired output state $\rho_S(\beta_E)$ is decreasing upon application of $\Phi$. The most straightforward metric to use is the $\norm{\cdot}_2^2$ distance. Again, we restrict ourselves to the single qubit system as defined at the beginning of the section.

\begin{align}
    \norm{\rho_s(\beta_E) - \Phi(\rho_S(\beta))}_2^2 &= \norm{\rho_S(\beta_E) - \rho_S(\beta) - \sum_{i,j,k,l} \frac{e^{-\beta_E \lambda_E(j)}}{\partfun_E(\beta_E)} \frac{e^{-\beta \lambda_S(i)}}{\partfun_S(\beta)} \tau(i,j|k,l) \ketbra{k}{k} + R_3(\Phi)} _2^2
\end{align}
To simplify notation, we note that $\norm{R_3(\Phi)}_2^2 \leq \norm{R_3(\Phi)}_1^2 \in \bigo{\alpha^6}$ and introduce $A = \rho_S(\beta_E) - \rho_S(\beta)$ and $B = \sum_{i,j,k,l} \frac{e^{-\beta_E \lambda_E(j)}}{\partfun_E(\beta_E)} \frac{e^{-\beta \lambda_S(i)}}{\partfun_S(\beta)} \tau(i,j|k,l) \ketbra{k}{k}$. By dropping terms of order $\bigo{\alpha^6}$ for now we can compute
\begin{align}
    \norm{\rho_s(\beta_E) - \Phi(\rho_S(\beta))}_2^2 &\approx \norm{A - B}_2^2 \\
    &= \trace{(A-B)^\dagger (A - B)} \\
    &= \trace{A^2} - 2 \trace{A B} + \trace{B^2} \\
    &= \norm{A}_2^2 + \norm{B}_2^2 - 2 \trace{A B}.
\end{align}
Now in order to show that $\norm{\rho_s(\beta_E) - \Phi(\rho_S(\beta))}_2^2 \leq \norm{\rho_S(\beta_E) - \rho_S(\beta)}_2^2$, we need to show that $2 \trace{AB} \geq \trace{B^2}$.

First we note that since $A$ and $B$ are diagonal, we have $\trace{AB} = \sum_{k} A(k) B(k) = A(0)B(0) + A(1) B(1)$. We first compute the $k=0$ term and then $k=1$. For brevity, allow the following variables:
\begin{equation}
    a(i) = \frac{e^{-\beta \lambda_S(i)}}{\partfun_S(\beta)}, \quad b(j) = \frac{e^{-\beta_E \lambda_E(j)}}{\partfun_E(\beta_E)}.
\end{equation}

We first compute $B$, starting with $k=0$. We note that $\tau(i,j| k,l) = \tau(k,l |i,j)$ and that $\tau(i,j|i,j) = -\sum_{(a,b) \neq (i,j)} \tau(i,j| a,b)$.
\begin{align}
    B(0) &= \sum_{i,j,l} a(i) b(j) \tau(i,j|0,l) \\
    &= a(0) b(0) (\tau(0,0|0,0) + \tau(0,0|0,1)) + a(0)b(1) (\tau(0,1| 0,0) + \tau(0,1| 0,1)) \nonumber \\
    &\quad + a(1) b(0) (\tau(1,0|0,0) + \tau(1,0|0,1)) + a(1) b(1) (\tau(1,1|0,0) + \tau(1,1|0,1)) \\
    &= a(0) b(0) (-\tau(0,0|1,0) - \tau(0,0|1,1)) + a(0)b(1) (- \tau(0,1|1,0) -\tau(0,1|1,1)) \nonumber \\
    &\quad + a(1) b(0) (\tau(1,0|0,0) + \tau(1,0|0,1)) + a(1) b(1) (\tau(1,1|0,0) + \tau(1,1|0,1)) \\
    &= \tau(0,0|1,0) (-b(0) a(0) + b(0) a(1)) + \tau(0,0|1,1)(-b(0) a(0) + b(1) a(1)) \nonumber \\
    &\quad + \tau(0,1|1,0) (-b(1) a(0) + b(0) a(1)) + \tau(0,1|1,1)(-b(1) a(0) + b(1) a(1))
\end{align}
Next we compute $B(1)$ as
\begin{align}
    B(1) &= \sum_{i,j,l} \tau(i,j|1,l) \\
    &= a(0) b(0) (\tau(0,0|1,0) + \tau(0,0|1,1)) + a(0)b(1) (\tau(0,1| 1,0) + \tau(0,1| 1,1)) \nonumber \\
    &\quad + a(1) b(0) (\tau(1,0|1,0) + \tau(1,0|1,1)) + a(1) b(1) (\tau(1,1|1,0) + \tau(1,1|1,1)) \\
    &= a(0) b(0) (\tau(0,0|1,0) + \tau(0,0|1,1)) + a(0)b(1) (\tau(0,1| 1,0) + \tau(0,1| 1,1)) \nonumber \\
    &\quad + a(1) b(0) (- \tau(1,0|0,1) - \tau(1,0|0,0)) + a(1) b(1) (-\tau(1,1|0,0) - \tau(1,1|0,1)) \\
    &= \tau(0,0|1,0) (b(0) a(0) - b(0) a(1)) + \tau(0,0|1,1)(b(0) a(0) - b(1) a(1)) \nonumber \\
    &\quad + \tau(0,1|1,0) (b(1) a(0) - b(0) a(1)) + \tau(0,1|1,1)(b(1) a(0) - b(1) a(1)) \\
    &= - B(0).
\end{align}
This then leads to the equality $\trace{AB} = A(0) B(0) + A(1)B(1) = B(0) (A(0) - A(1))$. We investigate when each of these can be negative. 

\begin{align}
    A(0) - A(1) &= \frac{e^{-\beta_E \lambda_S(0)}}{\partfun_S(\beta_E)} - \frac{e^{-\beta \lambda_S(0)}}{\partfun_S(\beta)} - \frac{e^{-\beta_E \lambda_S(1)}}{\partfun_S(\beta_E)} + \frac{e^{-\beta \lambda_S(1)}}{\partfun_S(\beta)} \\
    &= \frac{e^{-\beta_E \lambda_S(0)} - e^{-\beta_E \lambda_S(1)}}{e^{-\beta_E \lambda_S(0)} + e^{-\beta_E \lambda_S(1)}} - \frac{e^{-\beta \lambda_S(0)} - e^{-\beta \lambda_S(1)}}{e^{-\beta \lambda_S(0)} + e^{-\beta \lambda_S(1)}} \\
    &= \frac{e^{\beta_E \Delta_S} - 1}{e^{\beta_E \Delta_S} + 1} - \frac{e^{\beta \Delta_S} - 1}{e^{\beta \Delta_S} + 1} \\
    &= \tanh(\beta_E \Delta_S / 2) - \tanh(\beta \Delta_S / 2).
\end{align}
Since $\tanh$ is a monotonic function, $\beta_E \geq \beta$ implies $A(0) - A(1) \geq 0$. For ease of notation, we define $T(\delta) \coloneqq \tanh(\beta_E \Delta_S /2) - \tanh((\beta_E -\delta)\Delta_S/2)$, where $\beta = \beta_E - \delta$.

 In order to determine what kind of bound should be used, we first simplify the overall expression:
\begin{align}
    \trace{A^2} - 2 \trace{AB} + \trace{B^2} &= \trace{A^2} - 2 (A(0) B(0) + A(1)B(1)) + B(0)^2 + B(1)^2 \\
    &= \trace{A^2} - 2 B(0)(A(0) - A(1)) + 2 B(0)^2 \\
    &= \trace{A^2} + 2 B(0)(B(0) - A(0) + A(1)) \label{eq:dist_decrease_a_and_b}
\end{align}
As our goal is to show $\norm{A - B}_2^2 \leq \norm{A}_2^2 - \epsilon$, we want to show $2 B(0) (B(0) -A(0) + A(1)) \leq - \epsilon$.

Now we work on bounding $B(0)$. We note that most of the non-energy preserving transitions (i.e all other than $\tau(0,1|1,0)$) can be bounded by a trivial bound of 0. Working term by term, along with the facts: $a(i) \geq 0$, $b(i) \geq 0$, and $i \leq j \implies a(i) \geq a(j) \& b(i) \geq b(j)$, $\tau(i,j|k,l) \geq 0$ if $(i,j) \neq (k,l)$ we get the following
\begin{align}
    \tau(0,0|1,0) b(0)(a(1) - a(0)) &\leq 0 \\
    \tau(0,0|1,1) (-b(0) a(0) + b(1) a(1)) &\leq \tau(0,0|1,1) b(0) (-a(0) + a(1)) \leq 0 \\
    \tau(0,1|1,1) b(1) (-a(0) + a(1)) &\leq 0.
\end{align}
This produces the bound $B(0) \leq \tau(0,1|1,0) (a(1)b(0) - a(0)b(1))$. Our goal is to further reduce this upper bound:
\begin{align}
    B(0) &\leq \tau(0,1|1,) \frac{1}{\partfun_E(\beta_E) \partfun_S(\beta)}\parens{e^{-\beta \lambda_S(1)} e^{-\beta_E \lambda_E(0)} - e^{-\beta \lambda_S(0)} e^{-\beta_E \lambda_E(1)}} \\
    &= \tau(0,1|1,0) \frac{e^{-\beta \lambda_S(1)}}{\partfun_E(\beta_E) \partfun_S(\beta)} (1 - e^{\beta \Delta_S - \beta_E \gamma })
\end{align}
Now we see that if $\beta \Delta_S \geq \beta_E \gamma$ then our bound becomes negative. If we let $\beta = \beta_E - \delta$ we have $\beta \Delta_S - \beta_E \gamma = \beta_E o$.
However, we could simply use the dumb bound:
\begin{align}
    B(0) &\leq \tau(0,1|1,0) (a(1) b(0) - a(0) b(1)) \\
    &\leq \tau(0,1|1,0) a(1) b(0) \\
    &\leq \frac{\alpha^2 t^2}{\dim + 1} \sinc^2((\Delta_S -\gamma)t/2)
\end{align}
We will use this for now.


The next tast is to construct a lower bound for $B(0)$. For this we repeat a similar term-by-term process as above. For a lower bound, however, we can take advantage that the non-degenerate transitions are upper bounded by $\epsilon_{\sinc}$. We also have $a(i),b(j) \leq 1$ as they are Boltzmann factors.
\begin{align}
    \tau(0,0|1,0) b(0) (a(1) - a(0)) \geq - \tau(0,0|1,0) a(0) b(0) \geq - \epsilon_{\sinc} \\
    \tau(0,0|1,1)(-b(0) a(0) + b(1)a(1)) \geq - \tau(0,0|1,1) a(0) b(0) \geq -\epsilon_{\sinc} \\
    \tau(0,1|1,1) b(1)(-a(0) + a(1) \geq - \tau(0,1|1,1) a(0)b(1) \geq - \epsilon_{\sinc}
\end{align}
The one remaining term is
\begin{align}
    \tau(0,1|1,0) (a(1)b(0) - a(0)b(1)) &= \frac{\alpha^2 t^2 \sinc^2((\Delta_S - \gamma) t/2)}{\dim + 1} \frac{1}{\partfun_E(\beta_E) \partfun_S(\beta)}(e^{-\beta \lambda_S(1) - \beta_E \lambda_E(0)} - e^{-\beta \lambda_S(0) - \beta_E \lambda_E(1)}) \\
    &= \frac{\alpha^2 t^2 \sinc^2((\Delta_S - \gamma) t/2)}{\dim + 1} \frac{e^{-\beta \lambda_S(1)}}{\partfun_E(\beta_E) \partfun_S(\beta)} (1 - e^{\beta \Delta_S - \beta_E \gamma})
\end{align}
Now we let $\beta = \beta_E - \delta$ and $\gamma = \Delta_S + \widetilde{\gamma}$. We get
\begin{align}
    1 - e^{(\beta_E - \delta)\Delta_S - \beta_E (\Delta_S + \widetilde{\gamma})} &= 1 - e^{-\delta \Delta_S - \beta_E \Delta_S - \beta_E \widetilde{\gamma}} \\
    &= 1 - e^{-\beta_E \Delta_S (1 - \delta / \beta_E + \widetilde{\gamma} / \Delta_S)}
\end{align}
We see that this is positive so long as $\widetilde{\gamma} \geq -\Delta_S(1 - \frac{\delta}{\beta_E})$. This is a fairly significant range if $\Delta_S$ is much larger than $\Delta_{\min}$. 

However, we can get a much simpler bound if we simply ignore these prefactors. We could bound it by:
\begin{align}
    \tau(0,1|1,0)(a(1) b(0) - a(0) b(1)) &\geq - \tau(0,1|1,0) a(0) b(1) \\
    &\geq - \frac{\alpha^2 t^2}{\dim + 1} \sinc^2((\Delta_S - \gamma)t/2)
\end{align}
Which yields $B(0) \geq -\parens{3 \epsilon_{\sinc} + \frac{\alpha^2 t^2}{\dim + 1} \sinc^2((\Delta_S - \gamma)t/2)}$.

Now that we have upper and lower bounds, we return to the original task of bounding the distance decrease. Our first goal is to upper bound $2 B(0)^2$. The issue we have is that $B(0)$ could be negative, in which case $\text{LB} \leq B(0) ~\&~ B(0) \leq \text{UB} \implies B(0)^2 \leq \min \set{\text{LB}^2, \text{UB}^2}$. So given the bounds we have computed and noting that $\epsilon_{\sinc} \geq 0$, we have $B(0)^2 \leq \frac{\alpha^2 t^2}{\dim  + 1} \sinc^2((\Delta_S - \gamma)t/2)$. 
\begin{align}
    &2 B(0)^2 - 2 B(0) (A(0) - A(1)) \\
    &\leq \parens{\frac{2 \alpha^2 t^2}{\dim + 1} \sinc^2 ((\Delta_S -\gamma)t/2)}^2 - 2\parens{3 \epsilon_{\sinc} + \frac{\alpha^2 t^2}{\dim + 1} \sinc^2 ((\Delta_S -\gamma)t/2)} T(\delta) \\
    &= - \epsilon
\end{align}
As we expect the $\alpha^4$ term to be dominated by the $\alpha^2$ term, we have that 
$$\epsilon \leq 2 (3 \epsilon_{\sinc} + \frac{\alpha^2 t^2}{\dim + 1} \sinc^2((\Delta_S - \gamma)t/2) T(\delta) - \parens{\frac{2 \alpha^2 t^2}{\dim + 1} \sinc^2 ((\Delta_S -\gamma)t/2)}^2.$$
Now we see the importance of knowing $\Delta_S$ to high precision, if $\Delta_S - \gamma \geq \Delta_{\min}$, then we see that $\sinc^2((\Delta_S - \gamma)t/2) \leq \epsilon_{\sinc}$ and we have that $\epsilon \in \bigo{\epsilon_{\sinc}}$. Since $\epsilon_{\sinc}$ is a user-defined paramater that we would like to take as small as possible, this would severely limit the distance that we converge to. Further, note that as $\delta \to 0$ $T(\delta) \to 0$, and we have that $\epsilon$ becomes negative, implying that our bounds break down at some point. 

We can then say that $\norm{\rho_S(\beta_E) - \Phi(\rho_S(\beta))}_2^2 \leq \norm{\rho_S(\beta_E) - \rho_S(\beta)}_2^2 - \epsilon$. By setting $\epsilon = \epsilon' \norm{\rho_S(\beta_E) - \rho_S(\beta)}_2^2$, we have
\begin{equation}
    \norm{\rho_S(\beta_E) - \Phi(\rho_S(\beta))}_2 \leq \norm{\rho_S(\beta_E) - \rho_S(\beta)}_2 \sqrt{1 - \epsilon'}
\end{equation}
It follows from standard Sch\"atten norm inequalities that $\norm{X}_2 \leq \sqrt{\dim} \norm{X}_{\infty} \leq \sqrt{\dim} \norm{X}_1$, which we can use to say:
\begin{align}
    \norm{\rho_S(\beta_E) - \Phi(\rho_S(\beta))}_1 &\leq  \sqrt{\dim} \norm{\rho_S(\beta_E) - \Phi(\rho_S(\beta))}_2 \\
    &\leq \parens{\dim \sqrt{1 - \epsilon'}} \norm{\rho_S(\beta_E) - \rho_S(\beta)}_1
\end{align}

\begin{proof}
\textbf{leftover work that still might be useful.}
    First we want to define the (multi)set
    \begin{equation}
        S = \set{\Delta_S(i,j) : 0 \leq i < j \leq \dim_S}.
    \end{equation}
    We assume that if $(i,j) \neq (i', j')$ then $\Delta_S(i,j) \neq \Delta_S(i', j')$. This is a significant restriction, but it greatly simplifies the following proofs. 
    
    Our next goal is to study the distance reduction achieved by an application of our repeated interactions channel. The most convenient distance to study is the square of the Sch\"{a}tten 2-norm, $\norm{\cdot}_2^2$. Although this does not have a nice operational interpretation, as the trace distance does in regards to state distinguishability, it can be used to give a loose upper bound on the trace distance. We break down our distance into two terms based on the second order decomposition of the channel
\begin{align}
    \norm{\rho_S(\beta_E) - \Phi_{\gamma}(\rho_S)}_2^2 &= \norm{\rho_S(\beta_E) - \rho_S - \sum_{i, j, k, l} \bra{i} \rho_S \ket{i} \frac{e^{-\beta_E \lambda_E(j)}}{\partfun_E(\beta_E)}   \tau(i, j| k,l) \ketbra{k}{k} + R(\rho_S)}_2^2.
\end{align}
For simplicity, denote the operator $\rho_S(\beta_E) - \rho_S \eqqcolon A$ and $\sum_{i, j, k, l} \bra{i} \rho_S \ket{i} \frac{e^{-\beta_E \lambda_E(j)}}{\partfun_E(\beta_E)} \tau(i, j| k,l) \ketbra{k}{k} \eqqcolon B$. This allows us to isolate the remainder part as follows
\begin{align}
    \norm{\rho_S(\beta_E) - \Phi_\gamma(\rho_S)}_2^2 &= \norm{A - B + R(\rho_S)}_2^2 \\
    &\leq \parens{\norm{A - B}_2 + \norm{R(\rho_S)}}^2 \\
    &\leq \norm{A - B}_2^2 + 2 \norm{A - B}_2 \norm{R(\rho_S(\beta))}_2 + \norm{R(\rho_S)}_2^2
\end{align}
We see that since we have a term linear in $\norm{R}_2$ we should only analyze $\norm{A - B}_2^2$ to order $\bigo{\alpha^2}$. We will now simplify this to a linear expression in $\norm{R}_2$ by using simple upper bounds on norms of $A$ and $B$. \textcolor{red}{NOTE: Need to bound the norm of the Taylor remainder of the channel for all states by some epsilon.} 
\begin{align}
    \norm{A - B}_2 &\leq \norm{A - B}_1 \\
    &\leq \norm{A}_1 + \norm{B}_1 \\
    &= \sum_{i} \abs{\bra{i}\rho_S(\beta_E) \ket{i} - \bra{i} \rho_S \ket{i}} + \sum_k \abs{\sum_{i,j,l} \bra{i} \rho_S \ket{i} \frac{e^{-\beta_E \lambda_E(j)}}{\partfun_E(\beta_E)} \tau(i,j | k,l)} \\
    &\leq \sum_i 2 + \sum_k \sum_{i,j,l} 1 \\
    &\leq \dim^2 + 2 \dim \\
    &\leq 2 \dim^2.
\end{align}
This then gives our remainder term as
\begin{equation}
    \norm{\rho_S(\beta_E) - \Phi_{\gamma}(\rho_S)}_2^2 \leq \norm{A - B}_2^2 + (4 \dim^2 + 1) \norm{R_{\Phi}}_2.
\end{equation}
Now we can bound $\norm{R_{\Phi}}_2 \leq \sqrt{\dim} \norm{R_{\Phi}} \leq \epsilon_{R} \sqrt{\dim}$. This gives the final inequality as
\begin{equation}
    \norm{\rho_S(\beta_E) - \Phi_{\gamma}(\rho_S)}_2^2 \leq \norm{A - B}_2^2 + 5 \dim^{5/2} \epsilon_{R}
\end{equation}

Note we can improve this bound with improvements to the $\norm{B}_2$ bound. 
\end{proof}


Shifting our attention to $\norm{A-B}_2^2$, we decompose this into trace calculations straightforwardly
\begin{align}
    \norm{A-B}_2^2 &= \trace{(A-B)^\dagger (A-B)} \\
    &= \trace{A^2} - 2 \trace{AB} + \trace{B^2}
\end{align}
In order to bound each of these traces it will be helpful to organize the indices of our system into two sets, one of indices which are ``active" in the transition and those that are ``dead". The active set is denoted as $S_{\gamma}$ and is defined as
\begin{equation}
    S_{\gamma} \coloneqq \set{(i,j) : \lambda_S(i) \leq \lambda_S(j), |\Delta_S(i,j) - \gamma| \leq \Delta_{\min}},
\end{equation}
whereas the ``dead" set is denoted and defined as
\begin{equation}
T_{\gamma} \coloneqq \set{i : \forall j, |\Delta_S(i,j) - \gamma| \geq \Delta_{\min}}. 
\end{equation}


\noindent \textbf{Bounding: }$\trace{B^2}$


We first bound $\norm{B}_2^2 = \trace{B^2}$. We split this trace into sums over $S_{\gamma}$ and $T_{\gamma}$. 
\begin{align}
    k \in T_{\gamma} \implies B(k) &= \sum_{i \neq k} \sum_{j,l} (a(i) b(j) - a(k) b(l)) \tau(i,j|k,l) \\
    &\leq \sum_{i \neq k} \sum_{j,l} \tau(i,j|k,l) \\
    &\leq 4 (\dim_S - 1) \frac{\alpha^2 t^2}{\dim + 1}\epsilon_{\sinc} \\
    &\leq 2 \alpha^2 t^2 \epsilon_{\sinc}.
\end{align}


To perform the sum over the remaining indices we introduce a function $f_U$ that maps a given index to a subset of indices that satisfy
\begin{align}
    f_U(k | \gamma) \coloneqq \set{k' : \abs{\Delta_S(k, k') - \gamma} \leq \Delta_{\min} \text{ and } \Delta_S(k,k') < 0} \\
    f_L(k | \gamma) \coloneqq \set{k' : \abs{\Delta_S(k, k') - \gamma} \leq \Delta_{\min} \text{ and } \Delta_S(k, k') > 0}.
\end{align}
We then simplify a term for $\trace{B^2}$ as
\begin{align}
    B(k) &= \sum_{i \neq k} \sum_{j,l} (a(i) b(j) - a(k) b(l)) \tau(i,j|k,l) \\
    &\leq \sum_{i \in T_{\gamma}(k)} \sum_{j,l} \frac{\alpha^2 t^2}{\dim + 1}\epsilon_{\sinc} + \sum_{k' \in f_U(k)} \parens{\frac{\alpha^2 t^2}{\dim + 1} 3 \epsilon_{\sinc} + \tau(k', 0 | k, 1)} + \sum_{k'' \in f_L(k)} \parens{\tau(k'', 1 | k, 0) + 3\frac{\alpha^2 t^2}{\dim + 1} \epsilon_{\sinc}} \\
    &\leq \epsilon_{\sinc}\frac{\alpha^2 t^2}{\dim + 1} \parens{4 |T_{\gamma}(k)| + 3 | f_U(k)| + 3 |f_L(k)|} + \sum_{k' \in f_U(k)} (a(k')b(0) - a(k) b(1))\tau(k', 0| k, 1) \nonumber \\
    &\quad + \sum_{k'' \in f_L(k)} (a(k'') b(1) - a(k) b(0))\tau(k'', 1| k, 0) \\
    &\leq \epsilon_{\sinc} \dim \frac{\alpha^2 t^2}{\dim + 1} + \sum_{k' \in f_U(k)} (a(k')b(0) - a(k) b(1))\tau(k', 0| k, 1) + \sum_{k'' \in f_L(k)} (a(k'') b(1) - a(k) b(0))\tau(k'', 1| k, 0) \\
    &= \epsilon_{\sinc}\alpha^2 t^2 + b(0) \parens{\sum_{k' \in f_U(k)} a(k') \tau(k', 0 | k, 1) - \sum_{k'' \in f_L(k)} a(k'') \tau(k'', 1| k, 0)} \nonumber \\
    &\quad + b(1) \parens{-\sum_{k' \in f_U(k)} a(k') \tau(k', 0 | k, 1) + \sum_{k'' \in f_L(k)} a(k'') \tau(k'', 1| k, 0)} \\
    &\eqqcolon \widetilde{B}.
\end{align}
We let $\widetilde{B}$ denote the upper bound on $B(k)$. 

this gives the final bound for $\trace{B^2}$ as
\begin{equation}
    \trace{B^2} \leq 4 |T_{\gamma}| (\alpha^2 t^2 \epsilon_{\sinc})^2 + |S_{\gamma}| \widetilde{B}^2
\end{equation}



\newpage
\noindent\rule{\textwidth}{1pt}
\noindent\rule{\textwidth}{1pt}

\begin{claim}
    Given inputs $H_S$, $\rho_S$, $\beta_E$, and $\gamma$, we compute an upper bound on the distance of the thermalizing channel as
    \begin{align}
        \norm{\rho_S(\beta_E) - \Phi(\rho_S(\beta))}_2^2 \leq \norm{\rho_S(\beta_E) - \rho_S}_2^2 - \epsilon
    \end{align}
    
\end{claim}


Starting from the beginning:
\begin{align}
    \norm{\rho_S(\beta_E) - \Phi(\rho_S(\beta))}_2^2 &= \norm{\rho_S(\beta_E) - \rho_S(\beta) - \sum_{i,j,k,l} \frac{e^{-\beta_E \lambda_E(j) - \beta \lambda_S(i)}}{\partfun_E(\beta_E) \partfun_S(\beta)} \tau(i,j|k,l) \ketbra{k}{k} }_2^2 + \bigo{\alpha^6} \\
    &\approx \norm{A - B}_2^2 \\
    &= \trace{(A - B)^\dagger (A - B)} \\
    &= \trace{A^2} - 2 \trace{A B} + \trace{B^2},
\end{align}
where $A \coloneqq \rho_S(\beta_E) - \rho_S(\beta)$ and $B \coloneqq \sum_{i,j,k,l} \frac{e^{-\beta_E \lambda_E(j) - \beta \lambda_S(i)}}{\partfun_E(\beta_E) \partfun_S(\beta)} \tau(i,j|k,l) \ketbra{k}{k} $ are Hermitian and diagonal in the system's Hamiltonian basis. As $\trace{A^2}$ represents the distance we are trying to reduce, the goal is to show that $2 \trace{AB} \geq \trace{B^2} + \epsilon$. 

We use this sorting to create the (ordered) set of indices as before,

This means that each pair of indices $(i,j)$, such that their eigenvalue difference is close to $\gamma$, is stored only once in the set. This allows us to define the function $S_\gamma(i) = j$, as we have assumed non-degeneracy of the system eigenvalues. In a similar spirit, we say $i \in S_{\gamma}$ if it is the first index of a pair $(i,j) \in S_{\gamma}$. We can further define the set of ``dead" indices, those indices that do not have another eigenvalue that such that their difference is close to $\gamma$. This gives:

Now our goal is to compute $\trace{A B} = \sum_{k} A(k) B(k)$. We first compute $A(k) B(k)$ for $k \in T_{\gamma}$ and then for $k \in S_{\gamma}$. 



Now we move on to bounding the paired indices. Let $(k_1, k_2) \in S_{\gamma}$ be a single pair. We look at the sum of $A(k_1) B(k_1) + A(k_2) B(k_2)$. 
\begin{align}
    B(k_1) &\leq  \parens{\sum_{i \neq k_1, k_2} \sum_{j,l} (a(i) b(j) - a(k_1) b(l)) \tau(i,j|k_1, l) + 3 \frac{\alpha^2 t^2}{\dim + 1} \epsilon_{\sinc} + (a(k_2) b(0) - a(k_1) b(1)) \tau(k_2, 0 | k_1, 1)} \\
    &\leq \parens{\sum_{i \neq k_1, k_2} \sum_{j,l} \frac{\alpha^2 t^2}{\dim + 1}\epsilon_{\sinc} + 3 \frac{\alpha^2 t^2}{\dim + 1}\epsilon_{\sinc} + (a(k_2) b(0) - a(k_1) b(1)) \tau(k_2, 0 | k_1, 1)} \\
    &= (4 \dim_S - 5) \frac{\alpha^2 t^2}{\dim + 1} \epsilon_{\sinc} + (a(k_2) b(0) - a(k_1) b(1)) \tau(k_2, 0 | k_1, 1) \\
    &\leq 4 \dim_S \frac{\alpha^2 t^2}{\dim + 1} \epsilon_{\sinc} + \tau(k_2, 0| k_1, 1)
\end{align}



and similarly we compute
\begin{align}
    B(k_2) &\leq  \parens{\sum_{i \neq k_1, k_2} \sum_{j,l} (a(i) b(j) - a(k_1) b(l)) \tau(i,j|k_2, l) + 3 \frac{\alpha^2 t^2}{\dim + 1}\epsilon_{\sinc} + (a(k_1) b(1) - a(k_2) b(0)) \tau(k_1, 1 | k_2, 0)} \\
    &\leq \parens{\sum_{i \neq k_1, k_2} \sum_{j,l} \frac{\alpha^2 t^2}{\dim + 1} \epsilon_{\sinc} + 3 \frac{\alpha^2 t^2}{\dim + 1} \epsilon_{\sinc} + (a(k_1) b(1) - a(k_2) b(0)) \tau(k_1, 1 | k_2, 0)} \\
    &= (4 \dim_S - 5) \frac{\alpha^2 t^2}{\dim + 1} \epsilon_{\sinc} + (a(k_1) b(1) - a(k_2) b(0)) \tau(k_1, 1 | k_2, 0) \\
    &\leq 4 \dim_S \frac{\alpha^2 t^2}{\dim + 1} \epsilon_{\sinc} + \tau(k_1, 1| k_2, 0).
\end{align}
Given that $\tau$ is symmetric about input and output, we have $B(k_1) - (4 \dim_S - 5) \frac{\alpha^2 t^2}{\dim + 1} \epsilon_{\sinc} = - (B(k_2) - (4\dim_S - 5) \frac{\alpha^2 t^2}{\dim + 1} \epsilon_{\sinc})$. This allows us to compute the sum
\begin{align}
    A(k_1) B(k_1) + A(k_2) B(k_2) &\leq (4 \dim_S - 5) \epsilon_{\sinc} \parens{A(k_1) + A(k_2)} + (a(k_2) b(0) - a(k_1) b(1)) \tau(k_1, 1| k_2, 0) (A(k_1) - A(k_2)) \\
    &\leq 2 (4 \dim_S - 5) \epsilon_{\sinc} + (a(k_2) b(0) - a(k_1) b(1)) \tau(k_1, 1| k_2, 0) (A(k_1) - A(k_2)) \\
    &= 2 (4 \dim_S - 5) \epsilon_{\sinc} + \frac{e^{-\beta \lambda_S(k_2) - \beta_E \lambda_E(0)} - e^{-\beta \lambda_S(k_1) - \beta_E \lambda_E(1)}}{\partfun_S(\beta) \partfun_E(\beta_E)} \tau(k_1, 1| k_2, 0) (A(k_1) - A(k_2)).
\end{align}
As we want this to be negative, we require
\begin{align}
    e^{-\beta \lambda_S(k_2) - \beta_E \lambda_E(0)} - e^{-\beta \lambda_S(k_1) - \beta_E \lambda_E(1)} &\leq 0 \\
    \beta \lambda_S(k_2) &\geq \beta \lambda_S(k_1) + \beta_E\gamma \\
    \beta \Delta_S(k_2, k_1) &\geq \beta_E \gamma.
\end{align}

We will need an upper bound on $\trace{B^2}$
\begin{align}
    \trace{B^2} &= \sum_k B(k)^2 \\
    &= \sum_{k \in T_{\gamma}} B(k)^2 + \sum_{(k_1, k_2) \in S_{\gamma}} B(k_1)^2 + B(k_2)^2 \\
    &\leq 16 |T_{\gamma}| (\dim_S - 1)^2 \frac{\alpha^4 t^4}{(\dim + 1)^2}\epsilon_{\sinc}^2 + \sum_{(k_1, k_2) \in S_{\gamma}} B(k_1)^2 + B(k_2)^2
\end{align}
In order to compute the above we need an upper bound on $B(k_1)^2$ and $B(k_2)^2$. For ease of notation, we define the following variables:
\begin{align}
    S_1 &\coloneqq \sum_{i \neq k_1, k_2} \sum_{j,l} (a(i) b(j) - a(k_1) b(l)) \tau(i, j|k_1, l) \\
    S_2 &\coloneqq \sum_{(j,l) \neq (0, 1)} (a(k_2) b(j) - a(k_1) b(l)) \tau(k_2, j| k_1, l) \\
    r &\coloneqq a(k_2) b(0) - a(k_1) b(1) 
\end{align}
We can bound the absolute value of these as:
\begin{align}
    \abs{S_1} &\leq \sum_{i \neq k_1, k_2} \sum_{j,l} |a(i) b(j) - a(k_1) b(l)| |\tau(i, j|k_1, l)| \\
    &\leq \frac{\alpha^2 t^2}{\dim + 1} \epsilon_{\sinc} \sum_{i \neq k_1, k_2} \sum_{j,l} 1\\
    &\leq 4 (\dim_S - 2) \frac{\alpha^2 t^2}{\dim + 1} \epsilon_{\sinc} \\
    \abs{S_2} &\leq \sum_{(j,l) \neq (0,1)} |a(k_1) b(j) - a(k_2) b(j)| |\tau(k_2, j | k_1, l)| \\
    &\leq \frac{\alpha^2 t^2}{\dim + 1} \epsilon_{\sinc} \sum_{(j,l) \neq (0,1)} 1 \\
    &\leq 3 \frac{\alpha^2 t^2}{\dim + 1} \epsilon_{\sinc}
\end{align}
This allows us to compute bounds on $B(k_1)^2$ and $B(k_2)^2$ as:
\begin{align}
    B(k_1)^2 &= \bigg( S_1 + S_2 + r \tau(k_2, 0| k_1, 1) \bigg)^2 \\
    &\leq (|S_1| + |S_2|)^2 + 2(S_1 + S_2) r \tau(k_2, 0| k_1, 1) + r^2 \tau(k_2, 0| k_1, 1)^2 \\
    B(k_2)^2 &= \bigg( S_1 + S_2 - r \tau(k_2, 0 | k_1, 1) \bigg)^2 \\
    &\leq (|S_1| + |S_2|)^2 - 2 (S_1 + S_2) r \tau(k_2, 0 | k_1, 1) + r^2 \tau(k_2, 0 | k_1, 1)^2 \\
    B(k_1)^2 + B(k_2)^2 &\leq 2 (|S_1| + |S_2|)^2 + 2 r^2 \tau(k_2, 0 | k_1, 1)^2 \\
    &\leq 32 \dim_S^2 \frac{\alpha^4 t^4}{(\dim + 1)^2} \epsilon_{\sinc}^2 + 2 r^2 \tau(k_2, 0 | k_1, 1)^2.
\end{align}
We can now bound $\trace{B^2}$ as
\begin{align}
    \trace{B^2} &= \sum_{k \in T_{\gamma}} B(k)^2 + \sum_{(k_1, k_2) \in S_{\gamma}} B(k_1)^2 + B(k_2)^2 \\
    &\leq 16 |T_{\gamma}| (\dim_S - 1)^2 \frac{\alpha^4 t^4}{(\dim + 1)^2} \epsilon_{\sinc}^2 + 16 \dim_S^2 (2 |S_{\gamma}|) \frac{\alpha^4 t^4}{(\dim + 1)^2} \epsilon_{\sinc}^2 + 2 \sum_{(k_1, k_2) \in S_{\gamma}} r^2 \tau(k_1, 1 | k_2, 0)^2 \\
    &\leq 16 \dim_S^3 \frac{\alpha^4 t^4}{(\dim + 1)^2} \epsilon_{\sinc}^2 + 2 \sum_{(k_1, k_2) \in S_{\gamma}} r^2 \tau(k_1, 1 | k_2, 0)^2
\end{align}

Now our goal is to upper bound $-2\trace{AB}$. We make use of the fact that $|A(k)| \leq 1$ for all $k$ and that $k \in T_{\gamma}$ implies that $|B(k)| \leq 4 (\dim_S - 1) \epsilon_{\sinc}$.
\begin{align}
    -2 \trace{AB} &= -2 \sum_{k \in T_{\gamma}} A(k) B(k) - 2 \sum_{(k_1, k_2) \in S_{\gamma}} A(k_1) B(k_1) + A(k_2) B(k_2) \\
    &\leq + 2 \sum_{k \in T_{\gamma}} |A(k)| |B(k)| - 2 \sum_{(k_1, k_2) \in S_{\gamma}} A(k_1) B(k_1) + A(k_2) B(k_2) \\
    &\leq 8 \dim_S |T_{\gamma}| \frac{\alpha^2 t^2}{\dim + 1} \epsilon_{\sinc} - 2 \sum_{(k_1, k_2) \in S_{\gamma}} A(k_1) B(k_1) + A(k_2) B(k_2).
\end{align}
We now investigate the right term in the above. Using the notation from before
\begin{align}
    A(k_1) B(k_1) &= A(k_1) (S_1(k_1) + S_2(k_1) + r \tau(k_1, 1 | k_2, 0)) \\
    A(k_2) B(k_2) &= A(k_2) (S_1(k_2) + S_2(k_2) - r \tau(k_1, 1 | k_2, 0)) \\
    -2A(k_1) B(k_1) -2 A(k_2) B(k_2) & = -2 A(k_1)(S_1(k_1) + S_2(k_1))  - 2A(k_2)( S_1(k_2) + S_2(k_2)) + 2 r \tau(k_1, 1 | k_2, 0)( A(k_2) - A(k_1)) \\
    &\leq 2 |A(k_1)|(|S_1(k_1)| + |S_2(k_1)|) + 2 |A(k_2)|( |S_1(k_2)| + |S_2(k_2)|) \nonumber \\
    &+ 2 r \tau(k_1, 1 | k_2, 0)(A(k_2) - A(k_1)) \\
    &\leq 16 \dim_S \frac{\alpha^2 t^2}{\dim + 1} \epsilon_{\sinc} + 2 r \tau(k_1, 1 | k_2, 0)( A(k_2) - A(k_1)).
\end{align}
This allows us to plug in to our sum:
\begin{align}
    -2 \trace{AB} &\leq 8 \dim_S(|T_{\gamma}| + 2 |S_{\gamma}|) \frac{\alpha^2 t^2}{\dim + 1} \epsilon_{\sinc} + 2 \sum_{(k_1, k_2) \in S_{\gamma}} r \tau(k_1, 1 | k_2, 0) ( A(k_2) - A(k_1)) \\
    &= 8 \dim_S^2 \frac{\alpha^2 t^2}{\dim + 1} \epsilon_{\sinc} + 2 \sum_{(k_1, k_2) \in S_{\gamma}} r \tau(k_1, 1 | k_2, 0) ( A(k_2) - A(k_1)).
\end{align}
Combining with the bound for $\trace{B^2}$ we get the following
\begin{align}
    \trace{B^2} - 2 \trace{AB} &\leq 16 \dim_S^3 \frac{\alpha^4 t^4}{(\dim + 1)^2} \epsilon_{\sinc}^2 + 8 \dim_S^2 \frac{\alpha^2 t^2}{\dim + 1} \epsilon_{\sinc} + 2 \sum_{(k_1, k_2) \in S_{\gamma}} \parens{r^2 \tau(k_1, 1 |k_2, 0)^2 + r \tau(k_1, 1| k_2, 0) (A(k_2) - A(k_1))}.
\end{align}
To prove the required bound, we will require the following two inequalities:
\begin{align}
    16 \dim_S^3 \frac{\alpha^4 t^4}{(\dim + 1)^2}\epsilon_{\sinc}^2 + 8 \dim_S^2 \frac{\alpha^2 t^2}{\dim + 1} \epsilon_{\sinc} &\leq \frac{\epsilon}{2} \\
    2 \sum_{(k_1, k_2) \in S_{\gamma}} r \tau(k_1, 1 | k_2, 0) (r \tau(k_1, 1 | k_2, 0)  + A(k_2) - A(k_1)) &\leq - \frac{3\epsilon}{2}.
\end{align}

Everything below this line has not been updated with the fix to $\tau(i,j|k,l) \leq \frac{\alpha^2 t^2}{\dim + 1} \epsilon_{\sinc}$. 

\noindent\rule{\textwidth}{1pt}

The first inequality is rather straightforward. We use the intuition that the left term (ignoring constant factors) is nearly the square of the right term, and since we want to bound their sum with something small we can then deduce that the right term should be dominant. This yields the intuition that
\begin{align}
    16 \dim_S^3 \frac{\alpha^4 t^4}{(\dim + 1)^2} \epsilon_{\sinc}^2 &\leq 8 \dim_S^2 \frac{\alpha^2 t^2}{\dim + 1} \epsilon_{\sinc} \\
    \epsilon_{\sinc} &\leq \frac{\dim + 1}{2 \dim_S} \frac{1}{(\alpha^2 t^2)} \\
    &= \parens{1 + \frac{1}{\dim}} \frac{1}{\alpha^2 t^2}.
\end{align}
We then note that because the bound $\epsilon_{\sinc} \geq 1/(\Delta_{\min}^2 t^2)$ has to be satisfied, this yields a ``sandwich'' of acceptable ranges for $\epsilon_{\sinc}$ when combined with the upper bound above. To make this sandwich reasonable we require
\begin{align}
    \frac{1}{\Delta_{\min}^2 t^2} &\leq \epsilon_{\sinc} \leq \parens{1 + \frac{1}{\dim}} \frac{1}{\alpha^2 t^2} \\
    \alpha^2 &\leq \Delta_{\min}^2 \parens{1 + \frac{1}{\dim}}.
\end{align}
Curiously, there are no requirements on the value of $t$. To complete this argument, we now simplify the lower bound on $\epsilon$ by noting,
\begin{align}
    16 \dim_S^3 \frac{\alpha^4 t^4}{(\dim + 1)^2} \epsilon_{\sinc}^2 + 8 \dim_S^2 \frac{\alpha^2 t^2}{\dim + 1}\epsilon_{\sinc} &\leq 16 \dim_S^2 \frac{\alpha^2 t^2}{\dim + 1}\epsilon_{\sinc},
\end{align}
which yields $\epsilon \geq 32 \dim_S^2 \epsilon_{\sinc} \frac{\alpha^2 t^2}{\dim + 1}$.

The second inequality is a lot trickier. We will work with this as a quadratic expression in $\tau$ and find values of $\tau$ such that it is satisfied. We plug in for $\tau$ and rewrite the inequality as 
\begin{align}
    &\frac{(\alpha t)^4}{(\dim + 1)^2} \parens{2  \sum_{(k_1, k_2) \in S_{\gamma}} r^2 \sinc((\Delta_S(k_1, k_2) - \gamma) t/2)^4} \nonumber \\
    &+ \frac{(\alpha t)^2}{\dim + 1} \parens{2  \sum_{(k_1, k_2) \in S_{\gamma}}r \sinc((\Delta_S(k_1, k_2) - \gamma) t/2)^2 (A(k_2) - A(k_1))} \nonumber \\
    &+ \parens{\frac{\alpha t}{\dim + 1}}^0 \frac{3 \epsilon}{2} \leq 0
\end{align}
We first simplify the coefficient for $\alpha^4$ as 
\begin{align}
    2 \sum_{(k_1, k_2) \in S_{\gamma}} r^2 \sinc((\Delta_S(k_1, k_2) - \gamma) t/2)^4 &\leq 2 |S_{\gamma}|.
\end{align}
Since everything is positive, if this simplified inequality holds then so does the original. 

For simplicity we denote the summation in the coefficient for $\alpha^2$ as
$$d \coloneqq 2 \sum_{(k_1, k_2) \in S_{\gamma}} r \sinc((\Delta_S(k_1, k_2) - \gamma) t/2)^2 (A(k_2) - A(k_1)).$$
Letting $x$ play the role of $(\alpha t)^2 / (\dim + 1)$ gives us a simple quadratic expression:
\begin{equation}
    2 x^2 |S_{\gamma}| + x d + \frac{3 \epsilon}{2} \leq 0.
\end{equation}
As $(\alpha t)^2 / (\dim + 1) \geq 0$ we have $x$ must be positive as well. This requires for not only $d$ to be negative, but that the roots of the quadratic must both be real and at least one of them must be positive. Denote the roots $x_{\pm}$, which can be seen as:
\begin{align}
    x_{\pm} &= \frac{-d}{4 |S_{\gamma}|} \pm \frac{1}{4 |S_{\gamma}|} \sqrt{d^2 - 12 |S_{\gamma}| \epsilon} \\
    &= \frac{-d}{4 |S_{\gamma}|} \parens{1 \mp \sqrt{1 - \frac{12 |S_{\gamma}| \epsilon}{d^2}}}.
\end{align}
We want to guarantee that these roots are real, so we require 
\begin{equation}
    d^2 \geq 12 |S_{\gamma}| \epsilon. \label{eq:d_squared_epsilon_bound}
\end{equation}
Note that inside the radical, $12  |S_{\gamma}| \epsilon / d^2$ is always positive, implying that the radical is always going to be less than 1. This leads to both roots being positive so long as $d \leq 0$ and $d^2$ is lower bounded as mentioned.

If these two conditions are met, then we can set $\alpha^2$ to be the average of the two roots and the distance reduction claim is satisfied. Ideally we would like the lower root to be as close to 0 as possible, so that way we can simply reduce $\alpha$ to get our desired distance reduction. Regardless, our last objective still is to produce bounds on $d$ that satisfy the given inequalities, we therefore turn our attention to the sum in question. To simplify $d$ we first investigate the sign of $r$, assuming it to be positive we see what conditions result
\begin{align}
    r = a(k_2) b(0) - a(k_1) b(1) &\geq 0 \\
    \bra{k_2} \rho_S \ket{k_2} \frac{e^{-\beta_E \lambda_E(0)}}{\partfun_E(\beta_E)} - \bra{k_1} \rho_S \ket{k_1} \frac{e^{-\beta_E \lambda_E(1)}}{\partfun_E(\beta_E)} &\geq 0 \\
    \frac{\bra{k_2} \rho_S \ket{k_2}}{\bra{k_1} \rho_S \ket{k_1}} &\geq e^{-\beta_E \gamma}. \label{eq:bound_on_rho_s_for_r}
\end{align}
If we were using thermal states, $\rho_S = e^{-\beta H_S} / \partfun_S(\beta)$, the left hand side of the above would be $e^{-\beta \Delta_S(k_2, k_1)}$. Simplifying would result in $\beta \leq \frac{\gamma}{\Delta_S(k_2, k_1)} \beta_E$. Since we expect the distance moved by the thermalizing channel to be greatest as $\beta_E \to \infty$ and $\beta \to 0$, requiring this inequality to be true, and therefore $r \geq 0$, seems to be a reasonable condition to impose.

With the sign of $r$ sorted out, we now have to bound the summation for $d$. Since we require $d \leq 0$ in order to get positive values for $(\alpha t)^2 $, we require
\begin{align}
    \sum_{(k_1, k_2) \in S_{\gamma}} \frac{\sinc((\Delta_S(k_2, k_1) - \gamma)t/2)^2}{\dim + 1} (A(k_2) - A(k_1)) &\leq 0.
\end{align}
To prove this we bound the following, note we surpress the arguments to $\sinc$ to save space, and we make use of the bound required in Eq. \eqref{eq:bound_on_rho_s_for_r}
\begin{align}
    \sum_{(k_1, k_2) \in S_{\gamma}} \frac{\sinc^2}{\dim + 1} A(k_2) &= \sum_{(k_1, k_2) \in S_{\gamma}} \frac{\sinc^2}{\dim + 1} \parens{\frac{e^{-\beta_E \lambda_S(k_2)}}{\partfun_S(\beta_E)}  - \bra{k_2} \rho_S \ket{k_2}} \\
    &\leq \sum_{(k_1, k_2) \in S_{\gamma}} \frac{\sinc^2}{\dim + 1} \parens{\frac{e^{-\beta_E \lambda_S(k_2)}}{\partfun_S(\beta_E)}  - e^{-\beta_E \gamma} \bra{k_1} \rho_S \ket{k_1}} \\
    &= \sum_{(k_1, k_2) \in S_{\gamma}} \frac{\sinc^2}{\dim + 1} \parens{\frac{e^{-\beta_E (\lambda_S(k_1) - \lambda_S(k_1) + \lambda_S(k_2) )}}{\partfun_S(\beta_E)}  - e^{-\beta_E \gamma} \bra{k_1} \rho_S \ket{k_1}} \\
    &= \sum_{(k_1, k_2) \in S_{\gamma}} \frac{\sinc^2}{\dim + 1} \parens{\frac{e^{-\beta_E \lambda_S(k_1)}}{\partfun_S(\beta_E)} e^{-\beta_E \Delta_S(k_2, k_1)}  - e^{-\beta_E \gamma} \bra{k_1} \rho_S \ket{k_1}}.
\end{align}
We note right away that as $\beta_E \to 0$ this yields $\sum_{(k_1, k_2) \in S_{\gamma}} \frac{\sinc^2}{\dim + 1} (A(k_2) - A(k_1)) \leq 0$. 

Now subtracting the sum with $A(k_1)$ and simplifying yields
\begin{align}
    \sum_{(k_1, k_2) \in S_{\gamma}} \frac{\sinc^2}{\dim + 1} (A(k_2) - A(k_1)) &\leq \sum_{(k_1, k_2) \in S_{\gamma}} \frac{\sinc^2}{\dim + 1} \parens{\frac{e^{-\beta_E \lambda_S(k_1)}}{\partfun_S(\beta_E)} (e^{-\beta_E \Delta_S(k_2, k_1)} - 1) - \bra{k_1} \rho_S \ket{k_1} (e^{-\beta_E \gamma} - 1) } .
\end{align}
Since we have to upper bound this summation by 0, a good first step would be to understand when a given term is positive or negative. Taking a generic term, disregarding the prefactor of $\frac{\sinc^2}{\dim + 1}$ and simplifying leads to
\begin{align}
    \bra{k_1} \rho_S \ket{k_1}& \parens{1 - e^{-\beta_E \gamma}} -\frac{e^{-\beta_E \lambda_S(k_1)}}{\partfun_S(\beta_E)}\parens{1 - e^{-\beta_E \Delta_S}} \leq 0 \\
    \bra{k_1} \rho_S \ket{k_1}&  \leq \frac{e^{-\beta_E \lambda_S(k_1)}}{ \partfun_S(\beta_E)}\frac{1 - e^{-\beta_E \Delta_S}}{1 - e^{-\beta_E \gamma}}.
\end{align}
In order to proceed with the analysis we must impose some kind of structure onto $\rho_S$. We will investigate a few limits, one in which $\rho_S$ is a thermal state with $\beta$ close enough to $\beta_E$, one in which $\beta_E \to \infty$, and another in which we bound the operator distance of $\rho_S$ from $\rho_S(\beta_E)$. 

The condition in which we expect the most rapid thermalization is one in which the environment is in it's ground state (temperature of 0 or $\beta_E \to \infty$) and the system is in the maximally mixed state (temperature to infinity or $\beta \to 0$). In this situation, the factors $e^{-\beta_E \Delta_S} \to 0$ and $e^{-\beta_E \gamma} \to 0$. Further, the Boltzmann factors approach 0 if the eigenvalue is not a minimal eigenvalue or 1 if it is a minimal eigenvalue (ground state energy). In this situation, if $(0, i) \notin S_{\gamma}$, for all $i$, then we have that the Boltzmann factors $e^{-\beta_E \lambda_S(k_1)} = 0$. In this case we can also directly evaluate the summation $A(k_2) - A(k_1)$, as this is 
\begin{equation}
    A(k_2) - A(k_1) = \frac{e^{-\beta_E \lambda_S(k_2)}}{\partfun_S(\beta_E)} - \frac{1}{\dim_S} - \frac{e^{-\beta_E \lambda_S(k_1)}}{\partfun_S(\beta_E)} + \frac{1}{\dim_S} = 0.
\end{equation}
From Eq. \eqref{eq:d_squared_epsilon_bound} we see that this implies that $\epsilon = 0$ and we do not have any distance reduction to the ground state possible. This makes intuitive sense, as any probability mass that gets shuffled from high energy states to lower energy states does not get us any closer to the ground state. 
However, whenever the 0 temperature environment and infinite temperature system are coupled when $\gamma$ is close enough to a transition between a system ground state and an excited state we can get distance reduction. We will analyze this situation now. In the case in which there exists a pair $(0, i) \in S_{\gamma}$, meaning $e^{-\beta_E \lambda_S(k)} / \partfun_S(\beta_E) = \delta_{k,0}$ and $|\Delta_S(i, 0) - \gamma| \leq \Delta_{\min}$. In this case, if $(k_1, k_2) \in S_{\gamma}$ and $k_1 \neq 0$, then $A(k_2) - A(k_1) = 0$. However, for $(0, k)$ $A(k) - A(0) = 0 - 1/\dim_S - 1 + 1/\dim_S = -1$. Therefore, the total sum is then 
\begin{align}
    \sum_{(k_1, k_2) \in S_{\gamma}} \frac{\sinc^2}{\dim + 1} (A(k_2) - A(k_1)) &= - \sum_{(0, k) \in S_{\gamma}} \frac{\sinc^2}{\dim + 1} \\
    &\leq \frac{- \epsilon_{\sinc}}{\dim + 1}.
\end{align}
In addition, we can provide the simplistic bound 
\begin{equation}
    \sum_{(k_1, k_2) \in S_{\gamma}} \frac{\sinc^2}{\dim + 1}(A(k_2) - A(k_1)) \geq \frac{-1}{\dim + 1}.
\end{equation}
This allows us to argue that $\frac{\epsilon_{\sinc}^2}{(\dim + 1)^2} \leq d^2 \leq \frac{1}{(\dim + 1)^2}$. Propagating this through to $\epsilon$ via Eq. \eqref{eq:d_squared_epsilon_bound} yields 
\begin{align}
    d^2 &\geq \frac{12 t^4 |S_{\gamma}| \epsilon}{(\dim + 1)^2} \\
    \epsilon &\leq \frac{1}{12 t^4 |S_{\gamma}|}.
\end{align}
Now we look at the lower bound for $\epsilon$, which is given by 
\begin{align}
    \epsilon \geq 32 \epsilon_{\sinc} \alpha^2 t^2 \frac{\dim_S^2}{\dim + 1}
\end{align}


Without adding any structure to $\rho_S$ this is essentially a requirement we must impose on the state. We will go on to investigate for what ranges of $\beta$ this holds in the case that $\rho_S$ is a thermal state. We see as $\beta_E \to \infty$ that this inequality is trivially satisfied for $k_1$ being the ground state, the RHS approaches 1.

As we can see that this bound is pretty much dependent on the structure of the state, we now move on to bounding the value of $\epsilon$ that can be achieved. This comes from the bound in Eq. \eqref{eq:d_squared_epsilon_bound}, where we note that since $d$ is negative (from $r$ being positive) this amounts to an upper bound on $d$ (or a lower bound on $\epsilon$).



\section{Scratch work for single shot capture}

Want to find the ``single shot" capture radius. First should study the single-shot capture radius as $\beta_E \to \infty$, as this should simplify a lot of things. 
\begin{claim}
    We want to characterize the single-shot capture radius, or the $\delta = \beta_E - \beta$ such that 
    \begin{equation}
        \norm{\rho_S(\beta_E) - \mathbb{E}_{\gamma} \Phi_{\gamma}(\rho_S(\beta))}_1 \leq  \norm{\rho_S(\beta_E) - \rho_S(\beta)}_1 .
    \end{equation}
\end{claim}
\begin{proof}
See scratch work below.
\end{proof}
We first expand the channel using the Taylors series given by Eq. \eqref{idk}
\begin{align}
    \norm{\rho_S(\beta_E) - \mathbb{E}_{\gamma} \Phi_{\gamma}(\rho_S(\beta))}_1 &= \norm{\rho_S(\beta_E) - \rho_S(\beta) - \mathbb{E}_{\gamma} T_{\gamma}^{(2)}(\rho_S(\beta)) - \mathbb{E}_{\gamma} R_{\Phi}}_1 \\
    &\leq \norm{\rho_S(\beta_E) - \rho_S(\beta) - \mathbb{E}_{\gamma} T_{\gamma}^{(2)}(\rho_S(\beta))}_1 + \norm{\mathbb{E}_{\gamma} R_{\Phi}}_1.
\end{align}
As $T_{\gamma}^{(2)}$ is diagonal in the $H_S$ basis we can expand the one norm of the leftmost term as
\begin{align}
    \norm{\rho_S(\beta_E) - \rho_S(\beta) - \mathbb{E}_{\gamma} T_{\gamma}^{(2)}(\rho_S(\beta))}_1 &= \sum_k \abs{p_E(k) - p_{\beta}(k) - \mathbb{E}_{\gamma} \bra{k}  T_{\gamma}^{(2)}(\rho_S(\beta)) \ket{k}} .
\end{align}
We will investigate a generic term in the above expression by substituting in the Boltzmann remainder $R_0$ used in Lemma \ref{lem:thermal_state_diff_bound} and the characterization of the expected second order term from Lemma \ref{lem:expected_second_order_term}. Substituting in the values for $T_{\gamma}^{(2)}$ for a specific $k$ gives
\begin{align}
    &\abs{p_E(k) - p_{\beta}(k) - \mathbb{E}_{\gamma} \bra{k} T_{\gamma}^{(2)}(\rho_S(\beta)) \ket{k}} \\
    &\leq \parens{\frac{4 \alpha}{\Delta_{\min}}}^2 + \abs{-R_0(\beta, \beta_E, k) - \frac{\alpha^2 t^2}{\dim + 1} \frac{2}{\dim_S (\dim_S - 1)}\parens{\sum_{i \neq k} \frac{p_{\beta}(i)}{1 + e^{-\beta_E |\Delta_S(i,k)|}} - \frac{p_{\beta}(k)}{1 + e^{+ \beta_E |\Delta_S(i,k)|}}}} \\
    &= \parens{\frac{4 \alpha}{\Delta_{\min}}}^2 + \abs{R_0(\beta, \beta_E, k) + \frac{\alpha^2 t^2}{\dim + 1} \frac{2}{\dim_S (\dim_S - 1)} \parens{\sum_{i \neq k} \frac{p_{\beta}(i)}{1 + e^{-\beta_E |\Delta_S(i,k)|}} - \frac{p_{\beta}(k)}{1 + e^{+ \beta_E |\Delta_S(i,k)|}}}}.
\end{align}
This is pretty cumbersome to write and analyze, so we use the variable $S_k$ to capture the summation
\begin{align}
    S_k &\coloneqq \frac{\alpha^2 t^2}{\dim + 1} \frac{2}{\dim_S (\dim_S - 1)} \sum_{i \neq k} \frac{p_{\beta}(i)}{1 + e^{- \beta_E |\Delta_S(i,k)|}} - \frac{p_{\beta}(k)}{1 + e^{+ \beta_E |\Delta_S(i,k)|}}. \label{eq:single_shot_sum_def}
\end{align}

This allows us to write our total norm as
\begin{align}
    &\norm{\rho_S(\beta_E) - \mathbb{E}_{\gamma} \Phi(\rho_S(\beta)) }_1 \leq \norm{R_{\Phi}}_1 + \dim_S \parens{\frac{4 \alpha}{\Delta_{\min}}}^2 + \sum_k \abs{ R_0(\beta, \beta_E, k) + S_k }. \label{eq:single_shot_capture_sum}
\end{align}

What we do next is a small $\beta_E$ and $\delta \coloneqq \beta_E - \beta$ expansion of $R_0(\beta, \beta_E, k)$ and $S_k$. 

The path forward becomes clear once we realize that $\norm{\rho_S(\beta_E) - \rho_S(\beta)}_1 = \sum_k \abs{p_{\beta_E}(k) - p_{\beta}(k)} = \sum_k \abs{R_0(\beta, \beta_E, k)}$. Looking at the identity we would like to prove $\norm{\rho_S(\beta_E) - \mathbb{E}_{\gamma} \Phi_{\gamma}(\rho_S(\beta))}_1 \leq \norm{\rho_S(\beta_E) - \rho_S(\beta)}_1$, we see that the goal is to have the $S_k$ cancel out parts of the $R_0(k)$ values. We then need that the amount ``cancelled" by the $S_k$ is smaller than the positive contributions of the $S_k$ combined with the approximation errors $\norm{R_{\Phi}}_1$ and $\dim_S \parens{\frac{4 \alpha}{\Delta_{\min}}}^2$. To do this we need to know when $R_0(k)$ is negative and the structure of the $S_k$. 

As $R_0(\beta, \beta_E, k)$ can change sign based on $k$, we need to isolate when each $R_0$ contribution is positive or negative, and then break down the sum over absolute values. Before we approach $R_0$ though, we approach the simpler problem of bounding $S_k$. 

We first will show that each term in $S_k$ is non-negative. This boils down to a simple condition on $i$ compared to $k$: 
\begin{align}
    \frac{p_{\beta}(i)}{1 + e^{- \beta_E |\Delta_S(i,k)|}} - \frac{p_{\beta}(k)}{1 + e^{+ \beta_E |\Delta_S(i,k)|}} &\geq 0 \\
    \frac{p_{\beta}(i)}{1 + e^{-\beta_E |\Delta_S(i,k)|}} &\geq \frac{p_{\beta}(k)}{1 + e^{+\beta_E |\Delta_S(i,k)|}} \\
    e^{-\beta \Delta_S(i,k)} &\geq \frac{1 + e^{-\beta_E |\Delta_S(i,k)|}}{1 + e^{+ \beta_E |\Delta_S(i,k)|}} = e^{-\beta_E |\Delta_S(i,k)|} \\
    -\beta \Delta_S(i,k) &\geq -\beta_E |\Delta_S(i,k)|.
\end{align}
We see if $\Delta_S(i,k) > 0$ the condition reduces to $\beta \leq \beta_E$, which is always true for a cooling channel. If $\Delta_S(i,k) < 0$, then the condition reduces to $\beta \geq - \beta_E$, which for $\beta_E \geq 0$ and $\beta \geq 0$ also always holds. Therefore we can claim that for a cooling channel with $\beta \in [0, \beta_E)$ that $S_k \geq 0$ holds.

Now we want to do a small $\beta_E$ (aka a high temperature limit) and small $\delta \coloneqq \beta_E - \beta$ expansion of $S_k$. First we do a small $\delta$ expansion of $p_{\beta}(k)$. This is pretty straightforward without doing rigorous error bounds:
\begin{align}
    e^{-\beta \lambda_S(k)} &= e^{-\beta_E \lambda_S(k)} (1 + \delta \lambda_S(k) + \bigo{\delta^2} ) \\
    \trace{e^{-\beta H}} &= \trace{e^{-\beta_E H}} + \delta \trace{e^{-\beta_E H} H} + \bigo{\delta^2} \\
    \implies \frac{e^{-\beta \lambda_S(k)}}{\trace{e^{-\beta H}}} &= \frac{e^{-\beta_E \lambda_S(k)}}{\trace{e^{-\beta_E H}}}\parens{1 + \delta (\lambda_S(k) - \trace{\rho_S(\beta_E) H})} + \bigo{\delta^2}.
\end{align}
Now we do a small $\beta_E$ expansion of the boltzmann factors, which is similar work to the above:
\begin{align}
    \frac{e^{-\beta_E \lambda_S(k)}}{\trace{e^{-\beta_E H}}} &= \frac{1}{\dim_S}\parens{1 + \beta_E (\frac{1}{\dim} \trace{H} - \lambda_S(k))} + \bigo{\beta_E^2}.
\end{align}
Plugging this into the expansion about $\delta$ for $p_{\beta}(k)$, with the reminder that $\delta = \beta_E - \beta \implies \delta \in [0, \beta_E] \implies \delta \in \bigo{\beta_E}$ for a cooling channel, we get
\begin{align}
    \frac{e^{-\beta \lambda_S(k)}}{\trace{e^{-\beta H}}} &= \frac{1}{\dim_S}\parens{1 + \beta (\trace{\rho_S(0) H_S} - \lambda_S(k))} + \bigo{\beta_E^2}
\end{align}
We also have the expansion
\begin{align}
    \frac{1}{1 + e^{\pm \beta_E |\Delta_S(i,k)|}} &= \frac{1}{2} \mp \beta_E \frac{|\Delta_S(i,k)|}{4} + \bigo{\beta_E^2}.
\end{align}

Plugging in the above expansions we get the following simplification
\begin{align}
    \frac{p_{\beta}(i)}{1 + e^{-\beta_E |\Delta_S(i,k)|}} &= \frac{1}{2\dim_S}\parens{1 + \beta (\trace{\rho_S(0) H_S} - \lambda_S(i))}\parens{1 + \frac{\beta_E}{2}|\Delta_S(i,k)|} + \bigo{\beta_E^2} \\
    &= \frac{1}{2\dim_S}\parens{1 + \beta (\trace{\rho_S(0) H_S} - \lambda_S(i)) + \frac{\beta_E}{2}|\Delta_S(i,k)|}
\end{align}
Computing the same thing for the negative term in \ref{eq:single_shot_sum_def} and plugging in yields the final subtraction
\begin{align}
    \frac{p_{\beta}(i)}{1 + e^{-\beta_E |\Delta_S(i,k)|}} - \frac{p_{\beta}(k)}{1 + e^{+\beta_E |\Delta_S(i,k)|}} &= \frac{1}{2\dim_S}\parens{1 + \beta (\trace{\rho_S(0) H_S} - \lambda_S(i)) + \frac{\beta_E}{2}|\Delta_S(i,k)|} \\
    &\quad - \frac{1}{2\dim_S}\parens{1 + \beta (\trace{\rho_S(0) H_S} - \lambda_S(k)) - \frac{\beta_E}{2}|\Delta_S(i,k)|} + \bigo{\beta_E^2} \\
    &= \beta (\lambda_S(k) - \lambda_S(i)) + \beta_E |\Delta_S(i,k)| +\bigo{\beta_E^2}\\
    &= |\Delta_S(i,k)| \parens{\beta_E - \text{sign} \left[\Delta_S(i,k)\right] \beta} + \bigo{\beta_E^2}.
\end{align}
We see if $i > k$ this reduces to $|\Delta_S(i,k)| \delta$ and $i < k$ it reduces to $|\Delta_S(i,k)| (\beta_E + \beta)$. 

Now tackling the $R_0(\beta, \beta_E, k)$ term we get
\begin{align}
    R_0(\beta, \beta_E, k) &= \frac{e^{-\beta \lambda_S(k)}}{\partfun_S(\beta)} - \frac{e^{-\beta_E \lambda_S(k)}}{\partfun_S(\beta_E)} \\
    &=\frac{1}{\dim_S} \parens{\beta (\trace{\rho_S(0) H_S} - \lambda_S(k)) - \beta_E (\trace{\rho_S(0) H_S} - \lambda_S(k))} + \bigo{\beta_E^2} \\
    &= \frac{- \delta}{\dim_S} \parens{\trace{\rho_S(0) H_S} - \lambda_S(k)} + \bigo{\beta_E^2}
\end{align}
Plugging these into these into the final norm (and letting $\widetilde{\alpha} = \frac{\alpha^2 t^2}{\dim+ 1} \frac{2}{\dim_S(\dim_S - 1)}$ )we get
\begin{align}
    &\sum_k \abs{R_0(\beta, \beta_E, k) + S_k} \\
    &= \sum_k \abs{- \frac{\delta}{\dim_S}(\trace{\rho_S(0) H_S} - \lambda_S(k)) + \widetilde{\alpha} \sum_{i \neq k} |\Delta_S(i,k)|(\beta_E - \beta \text{sign} \left[ \Delta_S(i,k) \right])} + \bigo{\beta_E^2}\\
    &= \sum_k \abs{- \frac{\delta}{\dim_S}(\trace{\rho_S(0) H_S} - \lambda_S(k)) + \widetilde{\alpha} \sum_{i < k} (\beta + \beta_E)|\Delta_S(i,k)| + \widetilde{\alpha} \sum_{i > k} \delta |\Delta_S(i,k)|} + \bigo{\beta_E^2}
\end{align}
Now I want to plug in $\lambda_S(i) = \log (i)$ to see when this would work. If this is the case we can clearly see that it is well separated, so this expression holds. We can also write
\begin{align}
    \trace{\rho_S(0) H_S} &= \sum_i \frac{1}{\dim_S} \lambda_S(i) \\
    &= \frac{\log (\dim_S!)}{\dim_S}
\end{align}

Further we have
\begin{align}
    \sum_{i < k} \abs{\Delta_S(i,k)} &= \sum_{i < k} \lambda_S(k) - \lambda_S(i) \\
    &= (k-1) \log (k) - \sum_{i < k} \log (i) \\
    &= (k-1) \log (k) - \log((k-1)!) \\
\end{align}
and
\begin{align}
    \sum_{i > k} |\Delta_S(i,k)| &= \sum_{i > k} \lambda_S(i) - \lambda_S(k) \\
    &= \sum_{i > k} \log(i) - (\dim_S - k) \log (k) \\
    &= \log\parens{\frac{\dim_S!}{k!}} - (\dim_S - k) \log(k) \\
    &= \log (\dim_S !) - \log(k!) - (\dim_S - k) \log(k).
\end{align}
Plugging this in, along with $\beta = \beta_E - \delta$:
\begin{align}
    &\sum_k \abs{- \frac{\delta}{\dim_S}(\trace{\rho_S(0) H_S} - \lambda_S(k)) + \widetilde{\alpha} \sum_{i < k} (\beta + \beta_E)|\Delta_S(i,k)| + \widetilde{\alpha} \sum_{i > k} \delta |\Delta_S(i,k)|} \\
    &= \sum_k \bigg| - \frac{\delta}{\dim_S}\parens{\frac{\log(\dim_S!)}{\dim_S} - \log(k)} + 2 \widetilde{\alpha} \beta_E ( (k-1) \log(k) - \log((k-1)!)) \\
    &\quad + \widetilde{\alpha} \delta \parens{ (\log(\dim_S!) - \log(k!) - (\dim_S - k) \log(k)) - ((k-1)\log(k) - \log((k-1)!))} \bigg| \\
    &= \sum_k \bigg| - \frac{\delta}{\dim_S}\parens{\frac{\log(\dim_S!)}{\dim_S} - \log(k)} + 2 \widetilde{\alpha} \beta_E ( (k \log(k) - \log(k!) \\
    &\quad + \widetilde{\alpha} \delta \parens{ \log(\dim_S!)  - \dim_S\log(k)} \bigg| \\
    &= \sum_k \abs{\delta \parens{\widetilde{\alpha} - \frac{1}{\dim_S^2}} \log(\dim_S!) + \delta \parens{\frac{1}{\dim_S} - \widetilde{\alpha} \dim_S} \log (k) + 2 \widetilde{\alpha} \beta_E(k \log(k) -\log(k!))} \\
    &= \sum_k \abs{\delta \parens{\widetilde{\alpha} - 1/\dim_S^2}\parens{\log(\dim_S!) - \dim_S \log(k)} + 2 \widetilde{\alpha} \beta_E (k \log(k) - \log(k!))} \\
    &= \sum_k \abs{\delta ~\frac{\dim_S \log(k) - \log (\dim_S !)}{\dim_S^2} + \widetilde{\alpha} \parens{\delta (\log(\dim_S!) - \dim_S \log(k)) + 2 \beta_E (k \log (k) - \log(k!))}}
\end{align}
Or what I should do is instead isolate the $\widetilde{\alpha}$ version. No, instead figure out if $\widetilde{\alpha} - 1 / \dim_S^2$ is always negative (I think it is)
\begin{align}
    \frac{\alpha^2 t^2}{2 \dim_S + 1} \frac{2}{\dim_S(\dim_S -1)} &\leq \frac{1}{\dim_S^2} \\
    \frac{\alpha^2 t^2}{2 \dim_S + 1} &\leq \frac{\dim_S - 1}{2 \dim_S} \\
    &= 1 - \frac{1}{\dim}.
\end{align}
Which holds whenever $\dim > 2$ as we required that $\frac{\alpha^2 t^2}{\dim + 1} \leq \frac{1}{\dim}$ in order to guarantee that the transition probabilities were valid probabilities. 

Since we have determined that $\widetilde{\alpha} - 1/\dim_S^2$ is always negative and $k\log(k) - \log(k!)$ is always positive, the last remaining factor of interest is
\begin{align}
    \log(\dim_S!) &\geq \dim_S \log(k) \\
    \frac{\log(\dim_S!)}{\dim_S} &\geq \log(k) \\
    (\dim_S!)^{1/\dim_S} &\geq k.
\end{align}
Here we see $k$ just has to be less than the geometric mean of the index set in order for the overall term to be negative. The biggest decision now is what value of $\widetilde{\alpha}$ to use. As we can see, you can think of $\widetilde{\alpha}$ as being increased from 0 to whatever value we decide. This is unfortunately a non-linear optimization problem, due to the absolute values. So we will simply give an ansatze based off some numerics and try to show that it works. The ansatz is that we don't want to set $\widetilde{\alpha}$ so high that it causes the sign of one of the terms in the absolute value to change, we want to set it just large enough to nearly cancel out the negative term that is the closest to zero. As the terms we are trying to cancel are of the form $\delta (\dim_S \log(k) - \log(\dim_S!))/\dim_S^2$, which are monotonically increasing with $k$, we see that the smallest (in absolute value) negative term is when $k = \lfloor (\dim_S!)^{1/\dim_S} \rfloor$.

In order to choose a value of $\widetilde{\alpha}$ that can be analyzed we choose $\widetilde{\alpha}$ such that the following inequality is satisfied, note we use $\Bar{k} \coloneqq \lfloor (\dim_S!)^{1/\dim_S} \rfloor$,
\begin{align}
\abs{\delta ~\frac{\dim_S \log(\Bar{k}) - \log (\dim_S !)}{\dim_S^2} + \widetilde{\alpha} \parens{\delta (\log(\dim_S!) - \dim_S \log(\Bar{k})) + 2 \beta_E (\Bar{k} \log (\Bar{k}) - \log(\Bar{k}!))}} \leq \epsilon_\alpha.
\end{align}
Now we use the fact that $\widetilde{\alpha}$ is user defined to make the input of the absolute value on the positive branch of solutions to the inequality. Further, we remind the reader that 
\begin{align}
    \delta ~\frac{\dim_S \log(\Bar{k}) - \log (\dim_S !)}{\dim_S^2} + \widetilde{\alpha} \parens{\delta (\log(\dim_S!) - \dim_S \log(\Bar{k})) + 2 \beta_E (\Bar{k} \log (\Bar{k}) - \log(\Bar{k}!))} \leq \epsilon_\alpha \\
    \widetilde{\alpha} \leq \frac{ \epsilon_{\alpha} + \delta (\log (\dim_S!) - \dim_S \log(\Bar{k})) / \dim_S^2}{\delta (\log(\dim_S!) - \dim_S \log(\Bar{k})) + 2 \beta_E (\Bar{k} \log (\Bar{k}) - \log(\Bar{k}!))}.
\end{align}
For simplicity we may set $\widetilde{\alpha}$ to saturate this upper bound, which for now (sorry Nathan) I denote $\alpha_{UB}$. Note that we can always set $\epsilon_{\alpha}$ such that every other term in the sum is not affected, meaning we can break up the sum as
\begin{align}
    &\sum_k |R_0(\beta, \beta_E, k) + S_k| \\
    &= \sum_{k \leq \Bar{k}} |R_0(\beta, \beta_E, k) + S_k | + \sum_{k > \Bar{k}} |R_0(\beta, \beta_E, k) + S_k | \\
    &= \sum_{k \leq \Bar{k}} -(R_0(\beta, \beta_E, k) + S_k) + \sum_{k > \Bar{k}} R_0(\beta, \beta_E, k) + S_k \\
    &= \sum_k |R_0(\beta, \beta_E, k)| + \sum_{k > \Bar{k}} S_k - \sum_{k \leq \Bar{k}} S_k \\ 
    &= \norm{\rho_S(\beta) - \rho_S(\beta_E)}_1 + \sum_{k > \Bar{k}} S_k - \sum_{k \leq \Bar{k}} S_k.
\end{align}
Now we notice that we can achieve our goal ($\norm{\rho_S(\beta_E) - \mathbb{E}_{\gamma} \Phi(\rho_S(\beta))}_1 \leq \norm{\rho_S(\beta_E) - \rho_S(\beta)}_1$) if we can prove the following:
\begin{align}
    \norm{R_\Phi}_1 + \dim_S \parens{\frac{4\alpha}{\Delta_{\min}}}^2 + \sum_{k > \Bar{k}} S_k \leq \sum_{k \leq \Bar{k}} S_k.
\end{align}
So now we just have to lower bound the sum on the right hand side and upper bound the sum on the left.

We start with the lower bound as this is more straightforward. For the lower bound we are allowed to assume that $k \leq \Bar{k}$. Also we will introduce the $\bigo{\beta_E^2}$ terms and then drop them from further lines. We start by computing a single term $S_k$ of $\sum_{k \leq \Bar{k}} S_k$ as 
\begin{align}
    S_k &= \alpha_{UB} \sum_{i \neq k} \frac{p_{\beta}(i)}{1 + e^{-\beta_E |\Delta_S(i,k)|}} - \frac{p_{\beta}(k)}{1 + e^{+\beta_E |\Delta_S(i, k)}} \\
    &= \alpha_{UB} \sum_{i \neq k} |\Delta_S(i,k)| \parens{\beta_E - \text{sign} \left[\Delta_S(i,k)\right] \beta} + \bigo{\beta_E^2} \\
    &= \alpha_{UB} \sum_{i < k} |\Delta_S(i,k)| (\beta_E + \beta) + \alpha_{UB} \sum_{i > k} |\Delta_S(i,k)| (\beta_E - \beta) + \bigo{\beta_E^2} \\
    &= \alpha_{UB} (2 \beta_E - \delta) \sum_{i < k} \log \left( \frac{k}{i} \right) + \alpha_{UB} \delta \sum_{i > k} \log \left( \frac{i}{k} \right) + \bigo{\beta_E^2} \\
    &= 2 \alpha_{UB} \beta_E \parens{(k-1) \log(k) - \sum_{i < k} \log(i)} + \alpha_{UB} \delta \sum_{i} (\log (i) - \log(k)) + \bigo{\beta_E^2} \\ 
    &= 2 \alpha_{UB} \beta_E \parens{k \log (k) - \log(k!)} + \alpha_{UB} \delta \parens{\log (\dim_S !) - \dim_S \log(k)} + \bigo{\beta_E^2}. \label{eq:s_k_sum_equality}
\end{align}
We will now drop the $\bigo{\beta_E^2}$ terms. If we wanted to, we could introduce the lower bound given by 
\begin{align}
    k &\leq \Bar{k} \\
    &= \lfloor \dim_S!^{1/\dim_S} \rfloor \\
    &\leq \dim_S!^{1/\dim_S} \\
    \implies -\dim_S \log (k) &\geq \log(\dim_S!).
\end{align}
So we could lower bound the $\delta$ term by zero if we wanted to. That may be what we have to do, but for now we can let it float around. The real lower bound we want to use is Stirling's approximation to lower bound $k \log k - \log k!$. Since the $\log(k!)$ has a minus sign we use the exact upper bound
$$
k! \leq \sqrt{2 \pi k} \parens{\frac{k}{e}}^k e^{\frac{1}{12 k}}. 
$$
Plugging this in to the above yields, after taking logs,
\begin{align}
    S_k &\geq 2 \alpha_{UB} \beta_E \parens{k -\frac{1}{2} \log(2 \pi k) - \frac{1}{12 k}}  + \alpha_{UB} \delta \log \parens{\frac{\dim_S!}{k^{\dim_S}}}.
\end{align}
Now we can compute the sum of these terms over $k \leq \Bar{k}$
\begin{align}
    \sum_{k \leq \Bar{k}} S_K & \geq \sum_{k \leq \Bar{k}} \left[ 2 \alpha_{UB} \beta_E \parens{k -\frac{1}{2} \log(2 \pi k) - \frac{1}{12 k}}  + \alpha_{UB} \delta \log \parens{\frac{\dim_S!}{k^{\dim_S}}} \right] \\
&= \alpha_{UB} \beta_E \parens{\Bar{k} (\Bar{k} - 1) -\log (\Bar{k}!) - \Bar{k} \log(2 \pi) - \frac{1}{6} \sum_{k \leq \Bar{k}} \frac{1}{k}} + \alpha_{UB} \delta \sum_{k \leq \Bar{k}} \log \parens{\frac{\dim_S!}{k^{\dim_S}}}
\end{align}

Now we will repeat the same calculation but for the upper bound. To do this, we use a lower bound version of Stirling's Approximation given by $\sqrt{2 \pi k} \parens{\frac{k}{e}}^k e^{\frac{1}{12k} - \frac{1}{360k^3}} \leq k!$. Since the equality Eq. \eqref{eq:s_k_sum_equality} holds for all $k$ we can plug in the lower bound given by Stirling's to get
\begin{align}
    S_k & \leq 2 \alpha_{UB} \beta_E \parens{k -\frac{1}{2}\log (2 \pi k) - \frac{1}{12k} + \frac{1}{360 k^3}} + \alpha_{UB} \delta \log \parens{\frac{\dim_S!}{k^{\dim_S}}}.
\end{align}
Now upper bounding the sum over $k > \Bar{k}$ yields 
\begin{align}
     \sum_{k > \Bar{k}} S_k &\leq \sum_{k > \Bar{k}} \left[ 2 \alpha_{UB} \beta_E \parens{k -\frac{1}{2}\log (2 \pi k) - \frac{1}{12k} + \frac{1}{360 k^3}} + \alpha_{UB} \delta \log \parens{\frac{\dim_S!}{k^{\dim_S}}} \right] \\
     &= \alpha_{UB} \beta_E \parens{ \dim_S (\dim_S + 1 - \log(2 \pi)) - \Bar{k} (\Bar{k} +1 + \log(2 \pi)) - \log \parens{\frac{\dim_S!  }{\Bar{k}!}} - \sum_{k > \Bar{k}} \frac{30 k^2 + 1}{180 k^3}} + \alpha_{UB} \delta \log \sum_{k > \Bar{k}} \parens{\frac{\dim_S!}{k^{\dim_S}}}
\end{align}

We note that in order to prove our goal, we only have to lower bound the difference of the first sum with the second. This is shown below
\begin{align}
    \sum_{k \leq \Bar{k}} S_k - \sum_{k > \Bar{k}} S_k &\geq \alpha_{UB} \beta_E \parens{\Bar{k} (\Bar{k} - 1) -\log (\Bar{k}!) - \Bar{k} \log(2 \pi) - \frac{1}{6} \sum_{k \leq \Bar{k}} \frac{1}{k}} \nonumber \\
    &\quad + \alpha_{UB} \delta \sum_{k \leq \Bar{k}} \log \parens{\frac{\dim_S!}{k^{\dim_S}}} \nonumber \\
    &\quad - \alpha_{UB} \beta_E \parens{ \dim_S (\dim_S + 1 - \log(2 \pi)) - \Bar{k} (\Bar{k} +1 + \log(2 \pi)) - \log \parens{\frac{\dim_S!  }{\Bar{k}!}} - \sum_{k > \Bar{k}} \frac{30 k^2 + 1}{180 k^3}} \nonumber \\
    &\quad - \alpha_{UB} \delta \log \sum_{k > \Bar{k}} \parens{\frac{\dim_S!}{k^{\dim_S}}} \\
    &= \alpha_{UB} \beta_E \parens{2 \Bar{k}^2 - 2 \log (\Bar{k}!) - \frac{1}{6} \sum_{k \leq \Bar{k}} \frac{1}{k} + \sum_{k > \Bar{k}} \frac{1 - 30 k^2}{180 k^3} - \dim_S (\dim_S + 1 - \log (2 \pi) )} \nonumber \\
    &\quad + \alpha_{UB} \delta \parens{\sum_{k \leq \Bar{k}} \log \parens{\frac{\dim_S!}{k^{\dim_S}}} -  \sum_{k > \Bar{k}} \log \parens{\frac{\dim_S!}{k^{\dim_S}}}} \\
    &= \alpha_{UB} \beta_E \parens{2 \Bar{k}^2 - 2 \log (\Bar{k}!) - \frac{1}{6} \sum_{k \leq \Bar{k}} \frac{1}{k} + \sum_{k > \Bar{k}} \frac{1 - 30 k^2}{180 k^3} - \dim_S (\dim_S + 1 - \log (2 \pi) )} \nonumber \\
    &\quad + 2 \alpha_{UB} \delta \parens{\Bar{k} \log(\dim_S!) - \dim_S \log(\Bar{k}!)}  \\
\end{align}

\newpage

We now aim on bounding the denominators of $1 + e^{\pm \beta_E |\Delta_S(i,k)|}$. To do so, we introduce upper and lower bound on $\beta_E$. The upper bound is given by requiring 
\begin{align}
    \frac{1}{1 + e^{+\beta_E |\Delta_S(i,k)|}} &\geq \epsilon_2 \\
    \frac{1}{\epsilon_2} - 1 &\geq e^{+\beta_E |\Delta_S(i,k)|} \\
    \frac{1}{|\Delta_S(i,k)|} \ln (\epsilon_2^{-1} - 1) &\geq \beta_E,
\end{align}
where we note this bound is implied by $\frac{1}{2 \norm{H}} \ln(\epsilon_2^{-1} - 1 ) \geq \beta_E$. Now we work on the lower bound
\begin{align}
    \frac{1}{1 + e^{+\beta_E |\Delta_S(i,k)|}} &\leq \epsilon_1 \\
    \frac{1}{\epsilon_1} - 1 &\leq e^{+\beta_E |\Delta_S(i,k)|} \\
    \frac{1}{|\Delta_S(i,k)|} \ln (\epsilon_1^{-1} - 1) &\leq \beta_E,
\end{align}
where we note this bound is implied by $\frac{1}{\Delta_{\min}} \ln (\epsilon_1^{-1} - 1) \leq \beta_E$. We also want to route these bounds through to the $e^{-\beta_E |\Delta_S|}$ terms
\begin{align}
    \epsilon_1 \leq \frac{1}{1 + e^{-\beta_E |\Delta_S(i,k)|}} \leq \epsilon_2
\end{align}

Now we return to the $S_k$ terms. We state the upper bound (ignoring the prefactors for now)
\begin{align}
    S_k &\leq \sum_{i \neq k} \epsilon_{2} p_{\beta}(i) - \epsilon_{2} p_{\beta}(k) \\
    &= \epsilon_{2} (1 - p_{\beta}(k) - (\dim_S - 1)p_{\beta}(k)) \\
    &= \epsilon_{2}(1 - \dim_S p_{\beta}(k)).
\end{align}
The lower bound follows similarly
\begin{align}
    S_{k} &\geq \sum_{i \neq k} \epsilon_{1} p_{\beta}(i) - \epsilon_{1} p_{\beta}(k) \\
    &= \epsilon_{1}(1 - p_{\beta}(k) - (\dim_S - 1) p_{\beta}(k)) \\
    &= \epsilon_{1}(1 - \dim_S p_{\beta}(k)).
\end{align}


We need to profit by substituting these bounds into our summation in Eq. \eqref{eq:single_shot_capture_sum}. Isolating the sum over $k$ in that expression we have
\begin{align}
    \sum_{k} \abs{R_0(\beta, \beta_E, k) + S_k}.
\end{align}
It is tempting to apply the triangle inequality and see what can be pulled out of here, but as before we noted $\sum_k \abs{R_0(k)} = \norm{\rho_S(\beta_E) - \rho_S(\beta)}_1$, which means as soon as we apply the triangle inequality we lose. Therefore we need to unpack the absolute values. To do so, we note that $S_k \in \bigo{\alpha^2 t^2 / \dim}$, so we can make it arbitrarily small so that the sign of $R_0(k) + S_k$ is the same as the sign of $R_0$. Then we can think about increasing $S_k$. Also note that $R_0(k) = 0$ implies that $\beta_E = \beta$, which we do not have. So we can always set $\alpha^2 t^2$ to be small enough such that every $S_k$ is smaller than the smallest $R_0$. 

Now our goal is to reduce this sum by controlling values of $S_k$. We know that $R_0(\beta, \beta_E, 0) < 0$ for $\beta < \beta_E$ and that $S_k \geq 0$ for all $k$, so we will target our choice of $\frac{\alpha^2 t^2}{\dim + 1}$ to ensure that $S_0$ is close to cancelling out the error due to $R_0(\beta, \beta_E, 0)$ (remember the subscript 0 denotes the order of the remainder and we will drop the implicit arguments of $\beta, \beta_E$). Then we will use the triangle inequality to deal with the remaining terms. Now when controlling $\alpha^2 t^2 / \dim + 1$ to cancel out $R_0(0)$ we should aim to make $S_0$ as small as we can, as our only knob to adjust $S_0$ (the scalar $\alpha^2 t^2 / \dim + 1$) affects the other $S_k$'s directly and we would like those to be small. 

\begin{align}
    \abs{R_0(0) + S_0} &\leq \epsilon_S \\
    \implies \abs{R_0(0)} - \epsilon_S &\leq S_0 \leq \abs{R_0(0)} + \epsilon_S
\end{align}
To attack this we will plug in for $S_0 - \abs{R_0(0)}$ and start simplifying.
\begin{align}
    S_0 - \abs{R_0(0)} &= \sum_{i > 0} \left( \frac{p_{\beta}(i)}{1 + e^{- \beta_E \abs{\Delta_S(i,0)}}} - \frac{p_{\beta}(0)}{1 + e^{+ \beta_E \abs{\Delta_S(i,0)}}} \right) - |\beta - \beta_E| p_{\beta_{\star}}(0) \parens{\anglebrackets{H}_{\beta_{\star}} - \lambda_S(0)} \label{tmp:ugly_sum_1}
\end{align}
Now the crucial point to this argument is the sum over $i > 0$. If this sum is bounded through simple bounds on $\beta_E$ we lose all of our gains. Further, after preliminary numerical investigations this sum does not seem to be sufficient to prove thermalization for various well separated hamiltonians, such as the number operator squared $N^2 = (a^\dagger a)^2$ analagous to the harmonic oscillator. For this proof, we will investigate a truncated square root of the number operator, or $\lambda_S(k) \propto \sqrt{k}$. To give us one extra knob of control, we will explicitly use $\lambda_S(k) = h \sqrt{k}$ with little $h$ capturing the energy scale. This Hamiltonian is well separated, first off the smallest difference is $\lambda_S(2) - \lambda_S(1) = \sqrt{2} - 1$. For the well-separated condition we need $|\lambda_S(i) - \lambda_S(j)| - |\lambda_S(k) - \lambda_S(l)| \geq \Delta_{\min}$ to be bounded away from zero for nonequal $(i,j)$ and $(k,l)$. This seems to be clear but is busy work I shouldn't do unless it leads to something useful.

The well-separated Hamiltonian we will investigate analytically is the natural logarithm hamiltonian, where $\lambda_S(k) = \ln k$. Should include a proof that it is well-separated. For now we need to determine if it yields any kind of useful bound. First we tackle the term containing $p_{\beta}(1)$. \matt{Need to go through and make sure everything in the paper is one-indexed. Cry.} We have the sum
\begin{align}
    \sum_{i = 2}^{\dim_S} \frac{1}{1 + e^{\beta_E \lambda_S(i)}} &= \sum_{i = 2}^{\dim_S} \frac{1}{1 + i^{\beta_E}}
\end{align}
Now we need to show that this is monotonically decreasing. 
\begin{align}
    \frac{\partial}{\partial x} \frac{1}{1 + x^{\beta_E}} &= \frac{-1}{(1 + x^{\beta_E})^2} \cdot \beta_E x^{\beta_E - 1}.
\end{align}
For $x, \beta_E \geq 0$ this is clearly always negative and therefore each term in the summation is monotonically decreasing. Now we use integral bounds to get good stuff. 

\begin{center}
\textbf{Below is bad}
\end{center}

Now we assume that $\lambda_S(k) = h \sqrt{k}$ is a well-separated Hamiltonian. Our goal is to bound the sums present in Eq. \eqref{tmp:ugly_sum_1}. We start with the easier one on the negative term for $S_0$. To do so we will linearize the sum about $\beta_E \to \infty$, or in other words as the environment temperature approaches zero. 
\begin{align}
    \frac{\partial}{\partial \beta_E^{-1}} \frac{1}{1 + e^{+ \beta_E h \sqrt{i}}} &= \frac{1}{\parens{1 + e^{+ \beta_E h \sqrt{i}}}^2 } \cdot e^{\beta_E h \sqrt{i}} (-1)\beta_E^2 h \sqrt{i}.
\end{align}
It is easy to see that as $\beta_E \to \infty$ the denominator containing the $\left(e^{\beta_E h \sqrt{i}}\right)^2$ term will dominate and the overall derivative approaches zero. When is this regime of approximation valid?
Anyways, this means we \emph{should} be able to use the $\beta_E \to \infty$ behavior as a good approximation. This gives
\begin{align}
    S_0 &\approx \sum_{i > 0} p_{\beta}(i) - p_{\beta}(0) \\
    &= 1 - p_{\beta}(0) - (\dim_S - 1) p_{\beta}(0) \\
    &= 1 - \dim_S p_{\beta}(0).
\end{align}
Unfortunately, this bound is useless for $\beta > 0$. This is because for $\beta > 0$, $p_{\beta}(0) > 1 / \dim_S$, but we already know from before that $S_k > 0$ for all $k$. This means the bound is bad, and if we want to continue we have to analytically compute this sum.
\newpage
Now we have to determine the sign of $R_0$. We will not determine the sign of each term exactly, but we will give conditions on when the sign of $R_0$ only changes once. To start off, we remind the reader of that the remainder is written (using the Mean-Value version)
\begin{align}
    R_0(\beta, \beta_E, k) &= (\beta - \beta_E) p_{\beta_{\star}}(k) \parens{\parens{\sum_i p_{\beta_{\star}}(i) \lambda_S(i)} - \lambda_S(k)}.
\end{align}
Now we know that $\beta_{\star} \in (\beta, \beta_E)$. What we want to do is squeeze this range so small such that the difference in the mean energies $\anglebrackets{H}_{\beta_{\star}} \coloneqq \sum_i p_{\beta_{\star}}(i) \lambda_S(i)$ at $\beta$ and $\beta_E$ are very close. Below, let $\beta_i$ denote the mean value temperature guaranteed for $R_0(\beta, \beta_E, i)$. Also, it will prove sufficient to upper bound the absolute value
\begin{align}
    \abs{\anglebrackets{H}_{\beta} - \anglebrackets{H}_{\beta_E}} &= \left| \sum_i (p_{\beta}(i) - p_{\beta_E}(i)) \lambda_S(i) \right| \\
    &\leq \sum_i \abs{R_0(\beta, \beta_E, i)} \abs{\lambda_S(i)} \\
    &\leq \norm{H} \sum_i |\beta - \beta_E| p_{\beta_i}(i) \abs{\sum_j p_{\beta_i}(j) \lambda_S(j) - \lambda_S(i)} \\
    &\leq \norm{H} \delta \sum_i p_{\beta_i}(i) \parens{\sum_{j} p_{\beta_i}(j) \abs{\lambda_S(j)} + \abs{\lambda_S(i)}} \\
    &\leq 2 \norm{H}^2 \delta \sum_i p_{\beta_i}(i) \\
    &\leq 2 \dim_S \delta \norm{H}^2.
\end{align}
Now we can require $\delta \leq \frac{\Delta_{\min}}{2 \dim_S \norm{H}^2}$ to get $\abs{\anglebrackets{H}_{\beta} - \anglebrackets{H}_{\beta_E}} \leq \Delta_{\min}$.

Our next goal is to show that if some index $k_1$ has a positive zeroth order Boltzmann remainder $R_0(\beta, \beta_E, k_1)$, then every index $k_2 > k_1$ also has a positive zeroth order Boltzmann remainder. Note that $R_0(\beta, \beta_E, j) \geq 0 $ if and only if $\anglebrackets{H}_{\beta_j} - \lambda_S(j) \leq 0$. This means that we only need to show that For below, let $\beta_1$ denote the mean value $\beta$ for $k_1$ and similarly for $\beta_2$
\begin{align}
    \anglebrackets{H}_{\beta_2} - \lambda_S(k_2) &= \anglebrackets{H}_{\beta_2} - \anglebrackets{H}_{\beta_1} + \anglebrackets{H}_{\beta_1} - \lambda_S(k_1) + \lambda_S(k_1) - \lambda_S(k_2) \\
    &\leq \anglebrackets{H}_{\beta_2} - \anglebrackets{H}_{\beta_1} + \lambda_S(k_1) - \lambda_S(k_2) \\
    &\leq \anglebrackets{H}_{\beta} - \anglebrackets{H}_{\beta_E} - \Delta_{\min} \\
    &\leq \abs{\anglebrackets{H}_{\beta} - \anglebrackets{H}_{\beta_E}} - \Delta_{\min} \\
    &\leq 0.
\end{align}
This shows that $R_0(\beta, \beta_E, k_2) \geq 0$ if $R_0(\beta, \beta_E, k_1) \geq 0$ and $k_2 > k_1$. This result, combined with the observation that $R_0(\beta, \beta_E, 0) \leq 0$ yields the proposition that there exists a value $\widetilde{k}$ such that $k < \widetilde{k}$ implies $R_0(\beta, \beta_E, k) \leq 0$ and $k \geq \widetilde{k}$ implies $R_0(\beta, \beta_E, k) \geq 0$. Note we do not really care what this particular value of $\widetilde{k}$ is, just that it exists.

Now we can break down our sum into the following two sums
\begin{align}
    \sum_k \abs{R_0(\beta, \beta_E, k) + S_k} &= \sum_{k < \widetilde{k}} \abs{R_0(\beta, \beta_E, k) + S_k} + \sum_{k \geq \widetilde{k}} \abs{R_0(\beta, \beta_E, k) + S_k} \\
    &= \sum_{k < \widetilde{k}} -(R_0(\beta, \beta_E, k) + S_k) + \sum_{k \geq \widetilde{k}} R_0(\beta, \beta_E, k) + S_k \\ 
    &= \sum_{k < \widetilde{k}} -R_0(\beta, \beta_E, k) + \sum_{k \geq \widetilde{k}} R_0(k) - \sum_{k < \widetilde{k}} S_k + \sum_{k \geq \widetilde{k}} S_k \\
    &= \sum_k \abs{R_0(k)} - \sum_{k < \widetilde{k}} S_k + \sum_{k \geq \widetilde{k}} S_k
\end{align}
This yields
\begin{align}
    \norm{\rho_S(\beta_E) - \mathbb{E}_{\gamma} \Phi_{\gamma}(\rho_S(\beta))}_1 &\leq \norm{\rho_S(\beta_E) - \rho_S(\beta)}_1 + \norm{R_{\Phi}}_1 + \dim_S \parens{\frac{4 \alpha}{\Delta_{\min}}}^2 + \sum_{k \geq \widetilde{k}} S_k - \sum_{k < \widetilde{k}} S_k.
\end{align}

If our only goal is to show contraction, this reduces our problem to proving the following inequality
\begin{align}
    \norm{\rho_S(\beta_E) - \rho_S(\beta)}_1 + \norm{R_{\Phi}}_1 + \dim_S \parens{\frac{4 \alpha}{\Delta_{\min}}}^2 + \sum_{k \geq \widetilde{k}} S_k - \sum_{k < \widetilde{k}} S_k &\leq \norm{\rho_S(\beta_E) - \rho_S(\beta)}_1 \\
    \iff \norm{R_{\Phi}}_1 + \dim_S \parens{\frac{4 \alpha}{\Delta_{\min}}}^2 &\leq \sum_{k < \widetilde{k}} S_k - \sum_{k \geq \widetilde{k}} S_k 
\end{align}
To prove this we will produce a lower bound on the RHS, and show that this lower bound is greater than the LHS. Our lower bounds from before give
\begin{align}
    \sum_{k < \widetilde{k}} S_k - \sum_{k \geq \widetilde{k}} S_k &\geq \epsilon_{1} \sum_{k < \widetilde{k}} (1 - \dim_S p_{\beta}(k)) - \epsilon_{2} \sum_{k \geq \widetilde{k}} (1 - \dim_S p_{\beta}(k)) \\
    &= \epsilon_{1}(\widetilde{k} - 1 - \dim_S) + \epsilon_{1} \dim_S \sum_{k \geq \widetilde{k}} p_{\beta}(k) - \epsilon_{2}(\dim_S - \widetilde{k} + 1) + \epsilon_{2} \dim_S \sum_{k \geq \widetilde{k}} p_{\beta}(k) \\
    &= \parens{\dim_S \sum_{k \geq \widetilde{k}} p_{\beta}(k) - (\dim_S - \widetilde{k} + 1)}(\epsilon_{1} + \epsilon_{2}) \\
    &= \dim_S (\epsilon_{1} + \epsilon_{2}) \parens{\sum_{k \geq \widetilde{k}} p_{\beta}(k) - \parens{1 - \frac{\widetilde{k} - 1}{\dim_S}}} \\
    &= \dim_S (\epsilon_{1} + \epsilon_{2}) \parens{\frac{\widetilde{k} - 1}{\dim_S} - \sum_{k < \widetilde{k}} p_{\beta}(k)}.
\end{align}
We see right off the bat that this fails for very low temperatures, or when $\widetilde{k} = 2$, as the ground state energy $p_{\beta}(1)$ is always greater than $1 / \dim_S$ where as the positive term is $1 / \dim_S$ exactly. In the other limit, say $\widetilde{k} = \dim_S$, we have that the positive term is $1 - 1 /\dim_S$ and the negative term is $\sum_{k < \dim_S} p_{\beta}(k) = 1 - p_{\beta}(\dim_S)$, yielding a subtraction of $p_{\beta}(\dim_S) - 1 / \dim_S$, which is always negative. Again this lower bound does not work. Could it work somewhere in the middle? Evaluating this sum numerically for a harmonic oscillator (linear spectrum), square root spectrum, and squared spectrum seem to show that this bound is always negative. Intuitively this also makes sense, it says that in order to be positive the low-energy states need have less average probability mass than the infinite temperature limit of $1 / \dim_S$, which does not make sense as the low-energy states is where the probability mass of the thermal states increasingly concentrate as the temperature decreases.


We can write out the second order channel term for a thermal state density input with environment at $\beta_E$ as
\begin{equation}
    T^{(2)}(\rho_S(\beta), \alpha) = \sum_{i,j,k,l} \ketbra{k}{k} \frac{e^{-\beta \lambda_S(i)}}{\partfun_S(\beta)} \frac{e^{-\beta_E(j)}}{\partfun_E(\beta_E)} \tau(i,j | k,l) \label{eq:second_order_channel_with_tau}
\end{equation}
\end{theorem}
\begin{proof}
    We focus exclusively on diagonal inputs and outputs for density matrices, $\ketbra{a}{a}$ and $\ketbra{b}{b}$. We use Eq. \eqref{eq:second_order_output}, starting with transitions within the degenerate subspace of $a$. For the following we assume $a \neq b$ and $\Delta_{ab} = 0$: 
    \begin{align}
        &\prob{a \to b | a \neq b, \Delta_{ab} = 0} = \trace{\ketbra{b}{b} \int \Phi_G(\ketbra{a}{a}) dG} \\
        &= \braket{b}{a} \braket{a}{b} + \frac{\alpha^2}{2} \bra{b} \int \frac{\partial^2}{\partial \alpha^2} \Phi_G(\ketbra{a}{a})\bigg|_{\alpha = 0} dG \ket{b} \\
        &= -\frac{\sigma^2 \alpha^2}{\dim + 1} \bigg(\sum_{j: \Delta_{aj} \neq 0} \frac{1 - i \Delta_{aj}t - e^{-i \Delta_{aj} t}}{\Delta_{aj}^2} + \sum_{j: \Delta_{aj} \neq 0} \frac{1 + i \Delta_{aj} t - e^{i \Delta_{aj} t}}{\Delta_{aj}^2} + t^2 \eta(a) \bigg) \braket{b}{a} \braket{a}{b} \nonumber \\
        &~+ \frac{\alpha^2 \sigma^2}{\dim + 1} \parens{\sum_{i: \Delta_{ai} \neq 0} \frac{2(1- \cos(\Delta_{ai} t)}{\Delta_{ai}^2} \braket{b}{i}\braket{i}{b} + t^2 \sum_{i: \Delta_{ai} = 0 } \braket{b}{i} \braket{i}{b}} \label{eq:transition_intermediate} \\
        &= \frac{\sigma^2 \alpha^2 t^2}{\dim + 1}. 
    \end{align}
    We now proceed to the case that $b \neq a$ and $\Delta_{ab} \neq 0$, where we can start from Eq. \eqref{eq:transition_intermediate} and we get
    \begin{equation}
        \prob{a \to b | a \neq b, \Delta_{ab} \neq 0} = \frac{2 \sigma^2 \alpha^2 }{\dim + 1} \frac{1 - \cos (\Delta_{ab} t)}{\Delta_{ab}^2}.
    \end{equation}

    The remaining case to consider is when $b = a$. This involves simplifying the summations in Eq. \eqref{eq:second_order_output} and including the constant $\bigo{\alpha^0}$ term which only contributes a 1. 
    \begin{align}
        \prob{a \to a} &= 1 -\frac{\sigma^2 \alpha^2 }{\dim + 1} \bigg(\sum_{j: \Delta_{aj} \neq 0} \frac{1 - i \Delta_{aj}t - e^{-i \Delta_{aj} t}}{\Delta_{aj}^2} + \sum_{j: \Delta_{aj} \neq 0} \frac{1 + i \Delta_{aj} t - e^{i \Delta_{aj} t}}{\Delta_{aj}^2} + t^2 \eta(a) \bigg) \nonumber \\
    &~ +\frac{\sigma^2 \alpha^2 }{\dim+1} \parens{ \sum_{i: \Delta_{ai} \neq 0 } \frac{2(1- \cos (\Delta_{ai}t))}{\Delta_{ai}^2} \braket{a}{i} \braket{i}{a} + t^2  \sum_{i : \Delta_{ai} = 0} \braket{a}{i} \braket{i}{a}} \\
    &= 1 - \frac{\sigma^2 \alpha^2}{\dim + 1} \parens{\sum_{i: \Delta_{ai} \neq 0} \frac{2(1 - \cos(\Delta_{ai}t))}{\Delta_{ai}^2} + t^2 (\eta(a) - 1)}.
    \end{align}
    To simplify further we reduce the cosine terms to a sinc function as follows
    \begin{align}
        \frac{2(1 - \cos(\Delta t)}{\Delta^2} &= \frac{2(1 - \cos^2(\Delta t / 2) + \sin^2(\Delta t /2)}{\Delta^2} \\
        &= \frac{t^2}{2} \frac{\sin^2 (\Delta t / 2)}{ (\Delta t/ 2)^2} \\
        &= \frac{t^2}{2} \sinc^2(\Delta t /2).
    \end{align}
    This completes the transition probability computation. It is straightforward to see that $\sum_{b} \prob{a \to b} = 1$ and that $0 \leq \prob{a \to b} \leq 1$ given that $\alpha t \leq \sqrt{\dim + 1}$.
\end{proof}

Our last objective in this section is to yield a (relatively) concise statement for transition probabilities among diagonal matrix elements for density matrices. As we will be interested in computing trace distances 
\todo{Update the proof to include the expression for $T^{(2)}$}
\begin{theorem} \label{thm:second_order_transition_coeffs}
Given the inputs to Lemma \ref{lem:big_one}, let 
$$\tau(i,j | k,l) \coloneqq \frac{\alpha^2 }{2} \bra{k,l} \int \frac{\partial^2}{\partial \alpha^2} \Phi_G(\ketbra{i,j}{i,j})\bigg|_{\alpha = 0} dG \ket{k,l},$$ denote the second order transition coefficients in $\alpha$. These are expressed in terms of the eigenvalue differences $\Delta(i',j'|k',l')$ and other constants as 
$$\tau(i,j | k,l) = \begin{cases}
    - \frac{\alpha^2 t^2 }{\dim + 1} \parens{\sum_{(a,b) : \Delta(i,j | a,b) \neq 0} \frac{\sinc^2(\Delta(a,b|i,j) t}{2} + (\eta(i,j) - 1)} & (i,j) = (k,l) \\
    \frac{\alpha^2 t^2}{\dim + 1} & (i,j) \neq (k,l), \Delta(i,j | k,l) = 0 \\
    \frac{\alpha^2 t^2 }{\dim + 1} \sinc^2(\Delta t /2) & \Delta(i,j| k,l) \neq 0.
\end{cases}$$
A summation shows that $\tau(i,j|i,j) = -\sum_{(k,l) \neq (i,j)} \tau(i,j|k,l)$, which as a byproduct shows that the mapping $\Phi$ is trace preserving to order $\bigo{\alpha^2}$.


The goal for the remainder of this section is to simplify the expression for the channel output given by the above lemma. The first thing to note is that our channel does not appear to have large off-diagonal contributions to second order in $\alpha$. 
\begin{corollary}
    Given the inputs to Lemma \ref{lem:big_one} and a diagonal input state $\rho = \sum_{i,j} \rho_{i,j} \ketbra{i,j}{i,j}$, then we have $$\int \bra{k,l} \Phi_G(\rho)  \ket{m,n} ~dG~ \in \bigo{\alpha^3},$$ for $(k,l) \neq (m,n)$.
\end{corollary}
\begin{proof}
    We start from the Taylor's series of $\Phi_G$ with respect to $\alpha$:
    \begin{equation}
        \Phi_G(\rho) = \rho + \frac{\alpha^2}{2!} \int \frac{\partial^2}{\partial \alpha^2} \Phi_G(\rho) dG + \bigo{\alpha^3}. \label{eq:off_diagonal_taylors}
    \end{equation}
    First we note that $\bra{k,l}\rho \ket{m,n} =0$, as $\rho$ is diagonal and we assumed that $(k,l) \neq (m,n)$. Substituting Lemma \ref{lem:big_one} for the second order derivative and taking matrix elements yields:
    \begin{align}
        &\bra{k,l} \int \frac{\partial^2}{\partial \alpha^2} \Phi_G(\rho)\bigg|_{\alpha = 0} dG ~dG ~\ket{m,n}  \\
     &= -\frac{2 }{\dim + 1} \bigg(\sum_{(a,b): \Delta(i,j|a,b) \neq 0} \frac{1 - i \Delta(i,j|a,b)t - e^{-i \Delta(i,j|a,b) t}}{\Delta(i,j|a,b)^2} \nonumber \\
     &~+ \sum_{(a,b): \Delta(k,l|a,b) \neq 0} \frac{1 + i \Delta(k,l|a,b) t - e^{i \Delta(k,l|a,b) t}}{\Delta(k,l|a,b)^2} + \frac{t^2}{2}(\eta(i,j) + \eta(k,l)) \bigg) \bra{k,l} \rho \ket{m,n} \nonumber \\
    &~ + \frac{2}{\dim+1}\sum_{i,j} \rho_{i,j} \bigg(\sum_{(a,b): \Delta(i,j|a,b) \neq 0 } \frac{2(1- \cos (\Delta(i,j|a,b)t))}{\Delta(i,j|a,b)^2} \braket{k,l}{a,b} \braket{a,b}{m,n} \nonumber \\
    &~ + t^2 \sum_{(a,b) : \Delta(i,j|a,b) = 0} \braket{k,l}{a,b} \braket{a,b}{m,n} \bigg).
    \end{align}
    We see that the factors $\bra{k,l}\rho \ket{m,n}$ and $\braket{k,l}{i,j} \braket{i,j}{m,n}$ can never be non-zero due to the assumption $(k,l) \neq (m,n)$. The only remaining term in Eq. \ref{eq:off_diagonal_taylors} is of $\bigo{\alpha^3}$, yielding the stated result.
\end{proof}

